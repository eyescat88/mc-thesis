\documentclass[12pt,it,a4paper,]{report}
\usepackage{XCharter}

% Overwrite \begin{figure}[htbp] with \begin{figure}[H]
\usepackage{float}
\let\origfigure=\figure
\let\endorigfigure=\endfigure
\renewenvironment{figure}[1][]{%
\origfigure[b]
}{%
\endorigfigure
}

% fix for pandoc 1.14
\providecommand{\tightlist}{%
  \setlength{\itemsep}{0pt}\setlength{\parskip}{0pt}}

% TP: hack to truncate list of figures/tables.
\usepackage{truncate}
\usepackage{caption}
\usepackage{tocloft}
% TP: end hack

\usepackage{amssymb,amsmath}
\usepackage{ifxetex,ifluatex}

% Only use fixltx2e if using pre-2015 kernels
\begingroup\expandafter\expandafter\expandafter\endgroup
\expandafter\ifx\csname IncludeInRelease\endcsname\relax
  \usepackage{fixltx2e}
\fi

\ifnum 0\ifxetex 1\fi\ifluatex 1\fi=0 % if pdftex
  \usepackage[T1]{fontenc}
  \usepackage[utf8]{inputenc}
\else % if luatex or xelatex
  \ifxetex
    \usepackage{mathspec}
    \usepackage{xltxtra,xunicode}
  \else
    \usepackage{fontspec}
  \fi
  \defaultfontfeatures{Mapping=tex-text,Scale=MatchLowercase}
  \newcommand{\euro}{€}
    \setmainfont{XCharter}
\fi
% use upquote if available, for straight quotes in verbatim environments
\IfFileExists{upquote.sty}{\usepackage{upquote}}{}
% use microtype if available
\IfFileExists{microtype.sty}{%
\usepackage{microtype}
\UseMicrotypeSet[protrusion]{basicmath} % disable protrusion for tt fonts
}{}
\ifxetex
  \usepackage[setpagesize=false, % page size defined by xetex
              unicode=false, % unicode breaks when used with xetex
              xetex]{hyperref}
\else
  \usepackage[unicode=true]{hyperref}
\fi
\hypersetup{breaklinks=true,
            bookmarks=true,
            pdfauthor={Marina Chiapello},
            pdftitle={I rimedi all'inottemperanza della PA in alcuni ordinamenti europei},
            colorlinks=true,
            citecolor=blue,
            urlcolor=blue,
            linkcolor=magenta,
            pdfborder={0 0 0}}
\urlstyle{same}  % don't use monospace font for urls
\setlength{\parindent}{0pt}
\setlength{\parskip}{6pt plus 2pt minus 1pt}
\setlength{\emergencystretch}{3em}  % prevent overfull lines
\setcounter{secnumdepth}{5}
\ifxetex
  \usepackage{polyglossia}
  \setmainlanguage{italian}
\else
  \usepackage[it]{babel}
\fi

% % \newlength{\cslhangindent}
% \setlength{\cslhangindent}{1.5em}
% \newenvironment{cslreferences}%
%   {}%
%   {\par}
% 
\newlength{\cslhangindent}
\setlength{\cslhangindent}{1.5em}
\newlength{\csllabelwidth}
\setlength{\csllabelwidth}{3em}
\newenvironment{CSLReferences}[2] % #1 hanging-ident, #2 entry spacing
 {% don't indent paragraphs
  \setlength{\parindent}{0pt}
  % turn on hanging indent if param 1 is 1
  \ifodd #1 \everypar{\setlength{\hangindent}{\cslhangindent}}\ignorespaces\fi
  % set entry spacing
  \ifnum #2 > 0
  \setlength{\parskip}{#2\baselineskip}
  \fi
 }%
 {}
\usepackage{calc}
\newcommand{\CSLBlock}[1]{#1\hfill\break}
\newcommand{\CSLLeftMargin}[1]{\parbox[t]{\csllabelwidth}{#1}}
\newcommand{\CSLRightInline}[1]{\parbox[t]{\linewidth - \csllabelwidth}{#1}\break}
\newcommand{\CSLIndent}[1]{\hspace{\cslhangindent}#1}



% Table of contents formatting
\renewcommand{\contentsname}{Table of Contents}
\setcounter{tocdepth}{3}

% Headers and page numbering
\usepackage{fancyhdr}
\pagestyle{plain}

% Following package is used to add background image to front page
\usepackage{wallpaper}

% Table package
\usepackage{ctable}% http://ctan.org/pkg/ctable

% Deal with 'LaTeX Error: Too many unprocessed floats.'
\usepackage{morefloats}
% or use \extrafloats{100}
% add some \clearpage

% % Chapter header
% \usepackage{titlesec, blindtext, color}
% \definecolor{gray75}{gray}{0.75}
% \newcommand{\hsp}{\hspace{20pt}}
% \titleformat{\chapter}[hang]{\Huge\bfseries}{\thechapter\hsp\textcolor{gray75}{|}\hsp}{0pt}{\Huge\bfseries}

% % Fonts and typesetting
% \setmainfont[Scale=1.1]{Helvetica}
% \setsansfont[Scale=1.1]{Verdana}

% FONTS
\usepackage{xunicode}
\usepackage{xltxtra}
\defaultfontfeatures{Mapping=tex-text} % converts LaTeX specials (``quotes'' --- dashes etc.) to unicode
% \setromanfont[Scale=1.01,Ligatures={Common},Numbers={OldStyle}]{Palatino}
% \setromanfont[Scale=1.01,Ligatures={Common},Numbers={OldStyle}]{Adobe Caslon Pro}
%Following line controls size of code chunks
% \setmonofont[Scale=0.9]{Monaco}
%Following line controls size of figure legends
% \setsansfont[Scale=1.2]{Optima Regular}

% CODE BLOCKS
\usepackage[utf8]{inputenc}
\usepackage{listings}
\usepackage{color}

% JAVA CODE BLOCKS
%\definecolor{backcolour}{RGB}{242,242,242}
%\definecolor{javared}{rgb}{0.6,0,0}
%\definecolor{javagreen}{rgb}{0.25,0.5,0.35}
%\definecolor{javapurple}{rgb}{0.5,0,0.35}
%\definecolor{javadocblue}{rgb}{0.25,0.35,0.75}

\lstdefinestyle{javaCodeStyle}{
  language=Java,                         % the language of the code
  backgroundcolor=\color{backcolour},    % choose the background color; you must add \usepackage{color} or \usepackage{xcolor}
  basicstyle=\fontsize{10}{8}\sffamily,
  breakatwhitespace=false,
  breaklines=true,
  keywordstyle=\color{javapurple}\bfseries,
  stringstyle=\color{javared},
  commentstyle=\color{javagreen},
  morecomment=[s][\color{javadocblue}]{/**}{*/},
  captionpos=t,                          % sets the caption-position to bottom
  frame=single,                          % adds a frame around the code
  numbers=left,
  numbersep=10pt,                         % margin between number and code block
  keepspaces=true,                       % keeps spaces in text, useful for keeping indentation of code (possibly needs columns=flexible)
  columns=fullflexible,
  showspaces=false,                      % show spaces everywhere adding particular underscores; it overrides 'showstringspaces'
  showstringspaces=false,                % underline spaces within strings only
  showtabs=false,                        % show tabs within strings adding particular underscores
  tabsize=2                              % sets default tabsize to 2 spaces
}

%Attempt to set math size
%First size must match the text size in the document or command will not work
%\DeclareMathSizes{display size}{text size}{script size}{scriptscript size}.
%\DeclareMathSizes{12}{13}{7}{7}

% ---- CUSTOM AMPERSAND
% \newcommand{\amper}{{\fontspec[Scale=.95]{Adobe Caslon Pro}\selectfont\itshape\&}}

% HEADINGS
\usepackage{sectsty}
\usepackage[normalem]{ulem}
\sectionfont{\rmfamily\mdseries\Large}
\subsectionfont{\rmfamily\mdseries\scshape\large}
\subsubsectionfont{\rmfamily\bfseries\upshape\large}
% \sectionfont{\rmfamily\mdseries\Large}
% \subsectionfont{\rmfamily\mdseries\scshape\normalsize}
% \subsubsectionfont{\rmfamily\bfseries\upshape\normalsize}

% Set figure legends and captions to be smaller sized sans serif font
\usepackage[font={footnotesize,sf}]{caption}

\usepackage{siunitx}

% Adjust spacing between lines to 1.5
\usepackage{setspace}
% \onehalfspacing
\doublespacing
\raggedbottom

% Set margins
\usepackage[top=1.5in,bottom=1.5in,left=1.5in,right=1.4in]{geometry}
\setlength\parindent{0.5in} % indent at start of paragraphs (set to 0.3?)
\setlength{\parskip}{9pt}
\usepackage{indentfirst}

% Add space between pararaphs
% http://texblog.org/2012/11/07/correctly-typesetting-paragraphs-in-latex/
% \usepackage{parskip}
% \setlength{\parskip}{\baselineskip}

% Set colour of links to black so that they don't show up when printed
\usepackage{hyperref}
\hypersetup{colorlinks=false, linkcolor=black}

% Tables
\usepackage{booktabs}
\usepackage{threeparttable}
\usepackage{array}
\usepackage{makecell}
\newcolumntype{x}[1]{%
>{\centering\arraybackslash}m{#1}}%

% Allow for long captions and float captions on opposite page of figures
% \usepackage[rightFloats, CaptionBefore]{fltpage}

% Don't let floats cross subsections
% \usepackage[section,subsection]{extraplaceins}

% Rotate images and tables
\usepackage{float}
\usepackage{pdfpages}
\usepackage{pdflscape}
\usepackage{graphicx}
\usepackage{rotating}

% Custom math
\usepackage{bbold}
\DeclareMathOperator*{\argmin}{\arg\!\min}

% For use of \cref and \Cref used by pandoc secnos
\usepackage{cleveref}

\begin{document}


    \begin{titlepage}
        
        \noindent
        \begin{minipage}[t]{0.19\textwidth}{
            \vspace{-4mm}
            {\includegraphics[scale=1.15]{style/univ_logo.pdf}}
          }
        \end{minipage}
        \hspace{0.5cm}
        \begin{minipage}[t]{1.81\textwidth}
        {
          \setstretch{1.42}
          {\textsc{Università degli Studi di Milano - Bicocca}} \\
          \textbf{} \\
          \textbf{Dipartimento di Giurisprudenza} \\
          \textbf{Corso di Laurea in Scienze dei servizi giuridici} \\
          \par
        }
        \end{minipage}
        
	\vspace{20mm}
        
	\begin{center}
            {\LARGE{
                    \setstretch{1.2}
                    \textbf{I rimedi all'inottemperanza della PA in
alcuni ordinamenti europei}
                    \par
            }}
        \end{center}
        
        \vspace{30mm}

        \noindent
        {\large \textbf{Relatore:} \\ Prof.~Alessandro Squazzoni } \\

        %         % \noindent
        % {\large \textbf{Correlatore:} } \\
        %         
        \vspace{15mm}

        \begin{flushright}
            {\large \textbf{Relazione della prova finale di:}} \\
            \large{Marina Chiapello} \\
            \large{Matricola 112233} 
        \end{flushright}
        
        \vspace{15mm}
        \begin{center}
            {\large{\bf Anno Accademico 2022-2023}}
        \end{center}

        \restoregeometry
        
    \end{titlepage}




% This is where the converted markdown files will go 
\vspace*{\fill}

\noindent \textit{
Io, AUTHORNAME Aliquam pellentesque lorem sem, molestie molestie ipsum sodales sit amet.
} \vspace*{\fill} \pagenumbering{gobble} \newpage

\hypertarget{abstract}{%
\chapter*{Abstract}\label{abstract}}
\addcontentsline{toc}{chapter}{Abstract}

Lorem ipsum dolor sit amet, consectetur adipiscing elit. Nam et turpis
gravida, lacinia ante sit amet, sollicitudin erat. Aliquam efficitur
vehicula leo sed condimentum. Phasellus lobortis eros vitae rutrum
egestas. Vestibulum ante ipsum primis in faucibus orci luctus et
ultrices posuere cubilia Curae; Donec at urna imperdiet, vulputate orci
eu, sollicitudin leo. Donec nec dui sagittis, malesuada erat eget,
vulputate tellus. Nam ullamcorper efficitur iaculis. Mauris eu vehicula
nibh. In lectus turpis, tempor at felis a, egestas fermentum massa.

\pagenumbering{roman}
\setcounter{page}{1}

\hypertarget{ringraziamenti}{%
\chapter*{Ringraziamenti}\label{ringraziamenti}}
\addcontentsline{toc}{chapter}{Ringraziamenti}

Interdum et malesuada fames ac ante ipsum primis in faucibus. Aliquam
congue fermentum ante, semper porta nisl consectetur ut. Duis ornare sit
amet dui ac faucibus. Phasellus ullamcorper leo vitae arcu ultricies
cursus. Duis tristique lacus eget metus bibendum, at dapibus ante
malesuada. In dictum nulla nec porta varius. Fusce et elit eget sapien
fringilla maximus in sit amet dui.

Mauris eget blandit nisi, faucibus imperdiet odio. Suspendisse blandit
dolor sed tellus venenatis, venenatis fringilla turpis pretium. Donec
pharetra arcu vitae euismod tincidunt. Morbi ut turpis volutpat,
ultrices felis non, finibus justo. Proin convallis accumsan sem ac
vulputate. Sed rhoncus ipsum eu urna placerat, sed rhoncus erat
facilisis. Praesent vitae vestibulum dui. Proin interdum tellus ac velit
varius, sed finibus turpis placerat.

\newpage

\pagenumbering{gobble}

\tableofcontents

\newpage

\listoffigures

\newpage

\listoftables

\newpage

\hypertarget{abbreviazioni}{%
\chapter*{Abbreviazioni}\label{abbreviazioni}}
\addcontentsline{toc}{chapter}{Abbreviazioni}

\begin{tabbing}
\hspace{12em} \= \hspace{60em} \= \kill
\textbf{c.p.a.} \> \textbf{c}odice della \textbf{p}ubblica \textbf{a}mministrazione \\
\textbf{P. A.} \> \textbf{P}ubblica \textbf{A}mministrazione \\
\end{tabbing}

\newpage

\setcounter{page}{1}
\pagenumbering{arabic}
\doublespacing
\setlength{\parindent}{0.5in}

\hypertarget{la-giustizia-amministrativa-in-italia}{%
\chapter{La giustizia amministrativa in
Italia}\label{la-giustizia-amministrativa-in-italia}}

\hypertarget{lattuazione-del-giudicato-il-giudizio-di-ottemperanza}{%
\section{L'attuazione del giudicato: il giudizio di
ottemperanza}\label{lattuazione-del-giudicato-il-giudizio-di-ottemperanza}}

Il giudizio di ottemperanza rappresenta uno strumento di particolare
incisività per garantire nei confronti dell'Amministrazione l'attuazione
delle decisioni giudiziali, come stabilito all'art. 112 c.p.a. e in
risposta ai principi di di effettività ed efficacia della tutela
giurisdizionale sanciti dagli artt. 24 e 113 Cost.\footnote{Cfr. art.
  113 Cost: ``Contro gli atti della pubblica amministrazione è sempre
  ammessa la tutela giurisdizionale dei diritti e degli interessi
  legittimi dinanzi agli organi di giurisdizione ordinaria e
  amministrativa''.}, nonché dall'art. 47 della Carta dei diritti
fondamentali dell'UE e dall'art. 13 della CEDU. In base alla legge di
abolizione del contenzioso amministrativo del 1865, dell'atto
amministrativo lesivo di un diritto si poteva chiedere la modifica o
l'annullamento esclusivamente con ricorso gerarchico all'Autorità
amministrativa competente e l'Amministrazione che aveva emesso l'atto
aveva semplicemente l'obbligo di conformarsi al giudicato del tribunale
civile, ma tale obbligo rimaneva incoercibile, in quanto non
accompagnato da un meccanismo volto a garantirne l'effettiva
osservanza\footnote{Cfr. L. 2248 del 1865, art. 4, c.~2: ``L'atto
  amministrativo non potrà essere rivocato o modificato se non sovra
  ricorso alle competenti Autorità amministrative, le quali si
  conformeranno al giudicato dei Tribunali in quanto riguarda il caso
  deciso''.}.

In origine il giudizio di ottemperanza, così come introdotto dall'art. 4
n.~4 della legge 31 marzo 1889, n.~5992, era ammesso solo per le
sentenze passate in giudicato dell'Autorità giudiziaria ordinaria,
aventi per oggetto diritti civili e politici. E' a partire dagli anni
venti del secolo scorso che la giurisprudenza del Consiglio di Stato
estende analogicamente l'applicabilità dell'istituto anche
all'esecuzione del giudicato amministrativo, ma esso trova un
riconoscimento normativo solo con l'art. 37 della legge 6 dicembre 1971,
n.~1034, istitutiva dei tribunali amministrativi regionali. Infine,
viene compiutamente disciplinato con il decreto legislativo 2 luglio
2010, n.~104, in attuazione della legge delega 18 giugno 2009, n.~69,
per il riordino del processo amministrativo. Presupposto per
l'attivazione del giudizio di ottemperanza è l'inosservanza da parte
dell'Amministrazione del dovere di esecuzione della sentenza e l'oggetto
del giudizio è costituito dalla verifica se l'Amministrazione abbia o
meno adempiuto l'obbligo nascente dal giudicato, ovvero se abbia o meno
attribuito all'interessato quell'utilità che la sentenza ha riconosciuto
come dovuta. Mentre nella fase esecutiva della sentenza di condanna del
giudice civile che ha per oggetto diritti soggettivi e stabilisce cosa
deve fare l'Amministrazione soccombente nello specifico ci si trova di
fronte ad una sentenza molto chiara nello stabilire cosa si pretende dal
``debitore'', nel caso della sentenza del giudice amministrativo la
condotta successiva non è sempre segnata con certezza: il vincolo
conformativo ha un'intensità diversa a seconda del vizio accolto e
l'Amministrazione può non essere tenuta solo ad un comportamento
specifico. Il giudizio di ottemperanza non è la mera attuazione di un
giudicato già preciso e sicuro della fase di cognizione, ma deve
ricostruirne il significato. E' un giudizio c.d. ``misto'',
necessariamente di esecuzione ed eventualmente di cognizione,
assoggettato al termine di prescrizione ordinario di dieci anni,
decorrente dalla data del passaggio in giudicato della
sentenza\footnote{Art. 114, c.~1, c.p.a.}. La fase di cognizione non è
necessaria quando l'attività amministrativa successiva al giudicato
abbia carattere vincolato, ovvero quando le statuizioni della sentenza
impartiscano all'amministrazione comandi tassativi e talmente puntuali
da non lasciare spazio alcuno all'esercizio dei suoi poteri
discrezionali. Per converso, gli spazi liberi che possono residuare al
giudicato rendono la \emph{regola iuris} dallo stesso dettata
``implicita, elastica, condizionata ed incompleta'' e, come tale,
suscettibile di essere chiarita nel contesto del giudizio di
ottemperanza\footnote{M. NIGRO, \emph{Giustizia amministrativa,
  332,333.}} . Sempre riguardo alla natura del rito ed alla
compenetrazione di momenti cognitivi con momenti esecutivi, la Corte
costituzionale ha chiarito che \emph{``il giudizio di ottemperanza
assume diversi modi di essere in relazione alla situazione concreta,
alla statuizione giudiziale da attuare, alla natura dell'atto censurato.
Il particolare il giudizio di ottemperanza può costituire semplice
giudizio esecutivo che si aggiunge al procedimento espropriativo,
disciplinato dal codice di procedura civile; lo stesso giudizio può
essere preordinato al compimento di operazioni materiali o (\ldots) alla
sollecitazione di attività provvedimentale amministrativa (\ldots) può
essere utilizzato anche in difetto di completa individuazione del
contenuto della prestazione o attività oggetto del dovere
dell'Amministrazione (\ldots) non deve modellarsi necessariamente anche
nei presupposti sul processo esecutivo ordinario, tenuto conto delle
peculiarità funzionali del giudizio amministrativo, con potenzialità
sostitutive e intromissive nell'azione amministrativa incomparabili ai
poteri del giudice dell'esecuzione del processo civile''\footnote{Cfr.
  Corte Cost., ord. 10 dicembre 1998, n.~406, in \emph{Foro amm.}, 2000,
  751.}.} Il ricorso per l'ottemperanza va proposto nelle forme
ordinarie, quindi notificato all'Amministrazione e a tutte le altre
parti del giudizio di merito. Il ricorrente deve depositare una copia
autentica della sentenza di cui si chiede l'esecuzione, con l'eventuale
prova del passaggio in giudicato\footnote{Art. 114, c.~2, c.p.a.}. In
passato il ricorso doveva essere preceduto dalla notifica
all'Amministrazione di una diffida a provvedere, ma oggi il codice,
all'art. 114, c.~1, stabilisce che tale adempimento non è più
necessario. Il riparto di competenza ha carattere funzionale, ai sensi
dell'art. 14, c.~3, c.p.a. Per l'esecuzione della sentenza
amministrativa, competente è il giudice che ha pronunciato la sentenza.
Nel caso si tratti di sentenza emessa dal Consiglio di Stato, esso può
essere competente in unico grado, ma se la sentenza del Tar è stata
confermata in appello, la competenza spetta sempre al Tar. Qualora
invece si tratti dell'esecuzione della sentenza di un giudice ordinario
o di un altro giudice speciale diverso dal giudice amministrativo, la
competenza spetta sempre al Tar nella cui circoscrizione ha sede il
giudice che ha emesso la sentenza da eseguire\footnote{Art. 113 c.p.a.}.

Per quanto riguarda l'esecuzione delle sentenze del giudice
amministrativo, il ricorso per l'ottemperanza è esperibile
indipendentemente dal fatto che esse siano passate in giudicato o
solamente esecutive e, ai fini del ricorso, non rileva se rispetto a
queste sentenze inadempiente sia l'Amministrazione o una parte privata.
Nel caso di una sentenza non ancora passata in giudicato, l'esecuzione
riguarda una statuizione che non ha ancora carattere di definitività.
Con la sentenza n.~5352/2002 il Consiglio di Stato ha sostenuto che
l'esecuzione della sentenza non ancora passata in giudicato non dovrebbe
mai determinare un assetto \emph{``definito ed immutabile''}, perché
altrimenti verrebbe frustrato l'esito pratico di un eventuale appello
contro la sentenza\footnote{Cfr. Cons. Stato, sez. IV, 9 ottobre 2002,
  n.~5352.}. In generale, la stessa giurisprudenza che ha orientato
anche la redazione del codice del processo amministrativo equipara la
sentenza esecutiva alla sentenza passata in giudicato ai fini
dell'ammissibilità del giudizio di ottemperanza, ma precisa che il
giudice dell'ottemperanza, se la sentenza non sia passata in giudicato,
ne determina le modalità esecutive\footnote{Art. 114, c.~4, lett. c.},
motivo per cui sembra riconosciuta la necessità che l'esecuzione di tale
sentenza non pregiudichi le ragioni di un eventuale appello. In base
all'art. 114, c.~2, lett. \emph{c} ed \emph{e}, il ricorso per
l'ottemperanza è esperibile anche per l'esecuzione delle sentenze
passate in giudicato del giudice ordinario e dei giudici speciali avanti
ai quali non sia previsto un giudizio di ottemperanza, nonché per
l'esecuzione dei lodi arbitrali esecutivi divenuti inoppugnabili. In
questi casi però il giudizio di ottemperanza si caratterizza sul piano
soggettivo come strumento di esecuzione specifica nei confronti di
un'Amministrazione, in quanto non è ammesso per soggetti diversi.

L'elemento decisamente caratteristico del giudizio di ottemperanza è
individuato dall'art. 134, c.~1, lett. \emph{a}, c.p.a., laddove si
prevede che il giudice amministrativo, nello stesso giudizio, esercita
una giurisdizione estesa al merito. Tale previsione comporta che il
giudice amministrativo possa sostituirsi, direttamente o attraverso un
commissario da esso eventualmente nominato, all'Amministrazione
inadempiente. Questa possibilità di sostituzione comporta che nel
giudizio di ottemperanza non possa opporsi al giudice alcuna riserva di
potere all'Amministrazione, in quanto la necessità di dare esecuzione
alla sentenza prevale anche su ogni esigenza di salvaguardia delle
prerogative dell'Amministrazione stessa. Inoltre, la giurisprudenza
largamente prevalente ammette che il giudice dell'ottemperanza possa
compiere anche attività discrezionali, disattendendo l'assunto secondo
cui il medesimo giudice potrebbe sostituirsi all'Amministrazione solo
nei limiti delle statuizioni puntali del giudicato, in quanto le
ulteriori scelte discrezionali dell'Amministrazione non sarebbero di
pertinenza dell'autorità giurisdizionale. L'attività del giudice
dell'ottemperanza o del commissario \emph{ad acta} da lui nominato
infatti non costituisce manifestazione in senso stretto di
discrezionalità amministrativa, poiché essa è essenzialmente preordinata
al conseguimento dell'interesse del ricorrente e non già all'interesse
primario perseguito dall'Amministrazione. Vi sono due ipotesi in cui
l'Amministrazione viola il giudicato del giudice amministrativo: una si
verifica quando la sentenza stabilisce che essa non deve adottare un
provvedimento e la seconda quando l'Amministrazione è inadempiente,
quindi rispetto ad una condotta omissiva, con un'inerzia elusiva del
giudicato. Con l'art. 21 \emph{septies} della legge 241/1990, introdotto
dalla legge 15 del 2005 di riforma del procedimento amministrativo, gli
atti elusivi sono stati assimilati a quelli assunti in violazione del
giudicato, ammettendosi anche nei loro confronti il ricorso per
l'ottemperanza\footnote{Cfr. Cons. St., sez. IV, 10 aprile 1998, n.~565,
  in \emph{Foro amm.}, 1998, 1021: \emph{Il ricorso per l'ottemperanza è
  ammissibile non solo quando l'amministrazione mantiene un
  comportamento inerte di fronte al decisium del giudice, ma anche
  quando il provvedimento da essa adottato, in affermata ottemperanza al
  giudicato stesso, è invece palesemente elusivo dei principi e delle
  regole in esso enunciati. Ciò in quanto il giudicato amministrativo ha
  un contenuto complesso, non limitato agli effetti demolitori e
  ripristinatori rivolti al passato, ma comprensivo anche degli effetti
  confermativi rivolti al futuro e consistenti nei vincoli imposti
  all'autorità amministrativa nella rinnovazione del provvedimento
  annullato, in relazione ai vizi di legittimità riconosciuti
  esistenti}.}.

L'ampia gamma di poteri spendibili dal giudice dell'ottemperanza ammanta
lo stesso istituto di originalità, laddove nella maggior parte delle
principali esperienze continentali domina, quale strumento a presidio
dell'esecuzione del giudicato da parte dell'amministrazione, il rimedio
delle misure patrimoniali di tipo compulsorio, quali lo
\emph{Zwangsgeld} o l'\emph{astrainte}, dove i sistemi sono improntati,
in punto di esecuzione della sentenza, ad una rigida separazione tra i
poteri dell'amministrazione e quelli della giurisdizione, essendo
inibita al giudice qualsiasi ingerenza nell'attività esecutiva del
giudicato amministrativo che rimane appannaggio dell'amministrazione.
Un'eccezione è rappresentata dal modello austriaco della
\emph{Säumnisbeschwerde} quale rimedio avverso il silenzio in
inadempimento dell'amministrazione, prevedendo il legislatore austriaco
al \emph{§ 63/2 VwGG} la possibilità per il giudice amministrativo di
surrogarsi all'amministrazione inadempiente designando l'amministrazione
o il tribunale chiamato ad eseguire la sua decisione e
\emph{``consacrando così una possibile sostituzione del potere
giudiziario all'amministrazione attiva (\ldots) sul fronte
dell'esecuzione''}\footnote{C. FRAENKEL, \_Giurisdizione sul silenzio e
  discrezionalità amministrativa,} .

\hypertarget{i-poteri-sostitutivi-indiretti-il-commissario-ad-acta}{%
\section{\texorpdfstring{I poteri sostitutivi indiretti: il commissario
\emph{ad
acta}}{I poteri sostitutivi indiretti: il commissario ad acta}}\label{i-poteri-sostitutivi-indiretti-il-commissario-ad-acta}}

Il giudice può adottare direttamente i provvedimenti necessari ad
un'integrale esecuzione del giudicato quando essi siano vincolati,
altrimenti si deve limitare a dichiarare l'obbligo di provvedere
assegnando all'amministrazione un termine, nonché disponendo che si
nomini un commissario il quale agisca al posto dell'amministrazione, se
questa non ottemperi entro il termine assegnato. Il commissario \emph{ad
acta} è chiamato ad esercitare quei poteri che il giudice
dell'ottemperanza potrebbe esercitare anche in via diretta, attraverso
un intervento nel merito volto a sostituire l'amministrazione e
finalizzato a rendere effettiva la tutela sostanziale dell'interesse
protetto. Di regola il giudice assegna all'amministrazione un termine e
contestualmente designa un'autorità amministrativa che alla scadenza del
termine assegnato si sostituirà all'amministrazione inadempiente ed
emanerà il provvedimento o terrà il comportamento necessario per
l'attuazione del giudicato. In sede di ottemperanza al giudicato, il
giudice amministrativo, direttamente o per mezzo del commissario da lui
nominato, può emanare provvedimenti di vario tipo, costitutivi,
certificatori, declaratori di obblighi a carico dell'amministrazione e
tutti quegli adempimenti strumentalmente necessari per l'esecuzione
della sentenza. In pratica, si sostituisce all'amministrazione
inadempiente ponendo in essere l'attività che questa avrebbe dovuto
compiere per realizzare concretamente gli effetti scaturenti dalla
sentenza da eseguire, conformando la realtà alle sue statuizioni. Poiché
la discrezionalità amministrativa implica sovente decisioni di matrice
politica, la nomina di un commissario \emph{ad acta} viene ritenuta
preferibile rispetto all'adozione diretta da parte del giudice delle
misure di competenza dell'amministrazione riottosa. Di regola, egli è
scelto fra funzionari di altre amministrazioni e, spesso nella persona
del Prefetto, rappresenta con la sua attività \emph{``il punto di sutura
e saldatura''} tra attività giurisdizionale ed
amministrativa\footnote{M. CLARICH, \emph{Il giudizio amministrativo di
  esecuzione}, \emph{cit}., 344.}. In particolare, \emph{``in quanto
delegato dal giudice amministrativo, ha il potere di emanare i necessari
provvedimenti amministrativi anche in deroga alle vigenti competenze.
Allo stesso è altresì demandato l'onere di porre in essere ogni attività
idonea a dare esecuzione alla decisione''}\footnote{S. PELILLO, \emph{Il
  giudizio di ottemperanza alle sentenze del giudice amministrativo,
  cit.,} 313 ss.}. Una ormai risalente pronuncia della Corte
costituzionale configura il commissario \emph{ad acta} come ausiliario
del giudice e riconduce i suoi atti all'esercizio della giurisdizione
esecutiva del giudice dell'ottemperanza\footnote{Cfr. Corte Cost., 12
  maggio 1977, n.~75, in \emph{Giur. it.,} 1978, I, 980.}. Autorevole
dottrina ha sostanzialmente qualificato l'attività commissariale come
\emph{``proiezione nel mondo esterno di un comando del giudice e,
quindi, della traduzione nel concreto della attribuzione della} potestas
decidendi \emph{che non sempre ha o può avere contenuti rigidamente
predeterminati}, tali da consentire al giudice di portarli direttamente
ad attuazione\footnote{S. PELILLO, \emph{Il giudizio di ottemperanza
  alle sentenze del giudice amministrativo, cit.,} 314, secondo il quale
  sarebbero in realtà poche le situazioni che si prestano ad essere
  compiutamente disciplinate in via immediata con la sentenza: nella
  maggior parte dei casi si renderebbe necessario \emph{``gestire
  nell'ambito naturale (con l'osservanza delle regole) il procedimento e
  pervenire all'adozione finale del provvedimento''}.}. L'ampiezza dei
poteri commissariali dipenderà dal contenuto del giudicato inadempiuto:
essi potranno estrinsecarsi, a seconda delle situazioni dedotte in
giudizio, in attività sia vincolata, come ad esempio la restituzione di
beni illegittimamente espropriati, sia discrezionale, quindi comportante
un potere di scelta\footnote{A. Sandulli, \emph{L'effettività delle
  decisioni giurisdizionali amministrative, cit., 308, 309.}}. Una volta
nominato il commissario, il giudice mantiene comunque un incisivo potere
di vigilanza sul suo operato, nonché il potere di risolvere eventuali
contestazioni, dal momento che le determinazioni del commissario,
laddove esorbitanti dalle specifiche indicazioni del giudice, possono
essere oggetto di un ricorso dinanzi allo stesso giudice, esperibile
anche dall'amministrazione sostituita\footnote{Cfr. Cons. St., sez. V,
  28 febbraio 1995, n.~298, in \emph{Cons. Stato}, 1995, I, 232 ss.}. Da
tempo, sia in dottrina che in giurisprudenza, si dibatte sulla questione
riguardante la misura del potere di adempiere che conserverebbe
l'amministrazione, una volta che sia stato nominato il commissario o sia
scaduto il nuovo termine imposto alla stessa amministrazione. La
giurisprudenza ritiene, per lo più, che l'amministrazione verrebbe
privata del suo potere nel momento in cui viene assunta direttamente dal
giudice la decisione contenente il provvedimento concreto reso in
ottemperanza al giudicato, ovvero in quello in cui viene nominato il
commissario\footnote{Cfr. Cons. St., Ad. Plen., 14 luglio 1978 e Cons.
  St., Ad. Plen., 9 marzo 1973, n.~1, in G. PASQUINI, A. SANDULLI (a
  cura di), \emph{Le grandi decisioni del Consiglio di Stato}, Milano,
  2001, 398 e ss.}. Ove invece venga fissato all'amministrazione un
termine per adempiere, questo assume carattere perentorio, risultando
evidente nel caso in cui la prefissione del termine sia accompagnata
dalla nomina del commissario, poiché, una volta scaduto il termine, il
potere provvedimentale si trasferirebbe in automatico a tale soggetto,
ancorché il contenuto degli atti, eventualmente adottati
dall'amministrazione dopo tale scadenza e sadisfattivi del giudicato,
potrebbe essere confermato dal giudice dell'ottemperanza, il quale in
tal modo avvalorerebbe una legittimazione a provvedere tardivamente in
capo all'amministrazione\footnote{Cfr. Cons. St., sez. VI, 19 gennaio
  1995, n.~41, in \emph{Foro amm., CS}, 1995, 1, 78 ss.}.

Molti dei casi di inosservanza delle sentenze del giudice amministrativo
non sono dettati da una volontà deliberata di disconoscere l'autorità
della cosa giudicata, bensì alle oggettive difficoltà che
l'amministrazione incontra nell'eseguire le sentenze, soprattutto
qualora gli obblighi in esse enunciati appaiano indeterminati, vaghi o
imprecisi e intercorra più tempo fra l'emissione del provvedimento
impugnato e il suo definitivo annullamento. Il c.~5 dell'art. 112 c.p.a.
e il c.~7 dell'art. 114 ammettono il ricorso al giudice
dell'ottemperanza anche soltanto per \emph{``ottenere chiarimenti''} in
merito alle modalità di esecuzione che non presuppone un'inottemperanza,
ma semplicemente un'incertezza sull'interpretazione o sugli effetti
della sentenza da eseguire. In questi casi quindi il ricorso può essere
proposto anche dall'amministrazione tenuta a darvi esecuzione, quando
abbia esigenza di chiarimenti. Per evitare abusi, la giurisprudenza ha
comunque affermato che tale rimedio non deve rappresentare un espediente
per mettere in discussione la sentenza da eseguire o per introdurre
questioni estranee all'ottemperanza\footnote{Cfr. Cons. Stato, sez. VI,
  19 giugno 2012, n.~3569.}.

\hypertarget{il-risarcimento-del-danno-da-inottemperanza}{%
\section{Il risarcimento del danno da
inottemperanza}\label{il-risarcimento-del-danno-da-inottemperanza}}

Una volta ottenuta soddisfazione attraverso il giudizio promosso ai
sensi del c.~2 dell'art. 112 c.p.a., potrebbe ancora residuare al
ricorrente vittorioso un danno connesso alla tardiva realizzazione di
quell'assetto che sarebbe dovuto scaturire dall'annullamento del
provvedimento illegittimo dell'amministrazione, ma che è venuto in
essere solo a seguito di un notevole lasso di tempo oppure che ormai non
risulta più attuabile, per cui il giudizio di ottemperanza, di per sè,
non sarebbe in grado di garantire al ricorrente una tutela piena ed
effettiva. In quest'ultimo caso, lo strumento dell'ottemperanza si
rivelerebbe inutile, se non vi fosse la possibilità di ottenere
contestualmente un risarcimento per equivalente a seguito della perdita
definitiva del bene spettante dovuta alla inesecuzione del giudicato. Si
pensi al caso del definitivo annullamento di un decreto di esproprio cui
non sia seguita la spontanea restituzione dell'immobile al proprietario,
per cui si è reso necessario instaurare il giudizio di ottemperanza. Ove
l'amministrazione opponesse, in questa sede, una legittima
sopravvenienza impediente l'esecuzione del giudicato, al ricorrente
dovrebbe essere riconosciuto, in funzione surrogatoria, anche il danno
c.d. petitorio, consistente nel controvalore del bene, derivante appunto
dalla perdita definitiva dello stesso, cagionata dall'illecito ritardo
nella conformazione al giudicato \footnote{Cfr. Cons. St., sez. IV, 30
  gennaio 2006, n.~290, in \emph{Dir. giust.}, 2006, 10, 80 ove si
  afferma che il proprietario del fondo illegittimamente occupato dalla
  p.a. in esito a declaratoria di illegittimità dell'occupazione e
  all'annullamento dei relativi provvedimenti può legittimamente
  domandare nel giudizio di ottemperanza sia il risarcimento, sia la
  restituzione del fondo che la sua riduzione in pristino. L'azione
  risarcitoria costituisce strutturalmente attuazione del
  \emph{decisium} e quindi trova la sua naturale allocazione nel
  giudizio di ottemperanza, in quanto consente e determina
  quell'adeguamento dello stato di fatto allo stato di diritto che
  rappresenta la finalità tipica del giudizio di ottemperanza,
  realizzando quell'esigenza di completamento della tutela
  giurisdizionale amministrativa, ribadita anche dalla Consulta con
  sentenza n.~204/2004.}. Da questa situazione, va tenuta distinta
quella in cui, già al momento della pronuncia di annullamento, risulta
chiaramente che non è più utile per il ricorrente la rinnovazione del
potere conformemente alla regola concreta dedotta in sentenza, potendo
il giudice amministrativo in tal caso accogliere immediatamente la
domanda di risarcimento del danno per equivalente. In molti altri casi,
invece, il giudice della cognizione non è in grado di prevedere già
all'atto dell'annullamento se ed in quale misura l'ottemperanza potrà
effettivamente ripristinare la situazione soggettiva lesa. In
particolare, in tutti quei casi in cui la domanda del privato è diretta
a conseguire il bene della vita, molto spesso la possibilità e i limiti
entro cui attribuire il bene dipendono dal momento in cui
l'amministrazione esegue il giudicato. Ad esempio, in materia di
appalti, se l'annullamento dell'aggiudicazione in sede giurisdizionale
interviene nell'immediatezza dei fatti, consente al ricorrente di
stipulare il contratto con l'amministrazione; al contrario, se
interviene quando il contratto con l'originale aggiudicatario è già
stato non solo stipulato, ma anche parzialmente eseguito, l'esecuzione
della pronuncia e quindi l'attribuzione del bene della vita, cioè
l'appalto, è possibile solo parzialmente per la parte residua non
eseguita, mentre per la prima parte la tutela può avvenire solo
attraverso il risarcimento, sempre per equivalente. Spesso, quindi, solo
all'esito dell'ottemperanza di un giudicato di annullamento è possibile
accertare e quantificare il danno risarcibile per equivalente. Laddove
non risulta più satisfattiva la pronuncia di annullamento, supplisce la
tutela risarcitoria e il momento in cui emerge con chiarezza lo spazio
per l'esecuzione del giudicato e per il risarcimento del danno è proprio
quello dell'ottemperanza.

Nell'ipotesi di annullamento di un provvedimento ampliativo della sfera
giuridica del privato, occorre distinguere il caso in cui
l'amministrazione, in esecuzione spontanea del giudicato di
annullamento, renda il provvedimento precedentemente negato, dal caso in
cui il bene della vita agognato dal ricorrente venga conseguito soltanto
in esito al giudizio di ottemperanza. Mentre nella prima situazione la
pretesa risarcitoria azionabile riguarderà esclusivamente un danno da
attività provvedimentale illegittima, non avendo luogo una violazione
del giudicato, in quanto l'amministrazione accorda l'utilità prima
negata, a seguito della rinnovazione del potere discrezionale successivo
al giudicato di annullamento\footnote{Il ritardo produttivo del danno
  deriva dal fatto che l'amministrazione ha prima adottato un
  provvedimento illegittimo, sfavorevole al privato (es. diniego di
  permesso di costruire), ed ha poi emanato altro provvedimento, questa
  volta legittimo e favorevole, a seguito dell'annullamento in sede
  giurisdizionale del primo atto. Cfr. R. CHIEPPA, V. LOPILATO,
  \emph{Studi di diritto amministrativo}, Milano, 2007, 624, 625.},
nella seconda la pretesa risarcitoria sarà duplice e riguarderà un danno
scomponibile in una prima voce, relativa al ritardo antecedente alla
formazione del giudicato e commisurato al pregiudizio patito dal
ricorrente, qualora l'amministrazione si fosse spontaneamente conformata
al giudicato, e una seconda voce di danno, propriamente da inadempimento
dell'obbligo conformativo scaturente dalla pronuncia del giudice
amministrativo, volta a coprire il segmento temporale intercorrente fra
il giudicato e la sua concreta attuazione. In entrambe le ipotesi, il
danno c.d. da ritardo potrà essere compiutamente apprezzato soltanto a
posteriori, ovvero una volta che il privato abbia effettivamente
ottenuto il bene della vita cui aspirava con l'istanza a suo tempo
illegittimamente rigettata dall'amministrazione, a meno che non si
tratti di potere amministrativo vincolato, per cui la spettanza del bene
si cristallizza già in esito al giudizio di cognizione\footnote{Cfr.
  Cons. St., sez. IV, 31 marzo 2006, n.~5323, in \emph{Foro amm. CS},
  2006, 9, 2585 - nelle ipotesi nelle quali l'amministrazione sia
  titolare di un potere discrezionale, solo dal nuovo esercizio del
  potere possono derivare certezze in ordine alla spettanza del bene cui
  il privato aspira. Per converso, allorché si tratti di attività
  vincolata, il giudice, riscontrata la sussistenza dei presupposti di
  legge, potrà stabilire che, data quella situazione, la p.a. avrebbe
  dovuto adottare quella certa determinazione.}. In giurisprudenza
ricorre il principio secondo cui, essendo l'oggetto del giudizio di
ottemperanza costituito dalla verifica se l'amministrazione abbia o meno
adempiuto all'obbligo nascente dal giudicato, ovvero abbia o meno
attribuito all'interessato quell'utilità concreta che la sentenza ha
riconosciuta come dovuta, a prescindere dal fatto che residuino o meno
in capo all'amministrazione stessa poteri discrezionali, l'esecuzione
deve essere esatta, al pari di quanto avviene nell'obbligazione civile,
il cui inesatto adempimento viene sanzionato con la condanna al
risarcimento del danno\footnote{Cons. St., sez. V, 27 maggio 1991,
  n.~874, \emph{cit.}, nota 73, 3723.}. L'utilità concreta potrà
consistere \emph{'' nel diritto alla restitutio in integrum sotto forma
di pretesa alla restituzione del bene in caso di annullamento di
provvedimenti ablatori, sotto forma di annullamento del contratto
stipulato in seguito ad aggiudicazione illegittima, nel caso di
provvedimento incidente su interessi legittimi pretensivi; può
consistere nel diritto alla conformazione alla regola contenuta nel
giudicato in caso di riedizione dell'atto che va dal diritto alla non
riedizione o all'ottenimento dell'atto in caso di effetto vincolante
pieno, al diritto alla riedizione nel rispetto delle regole sostanziali
e formali in caso di effetto vincolante semipieno o
strumentale}\footnote{L. MANCINI, \emph{La responsabilità della pubblica
  amministrazione, cit.}, nota 73, 3723.}. Sul piano dell'accertamento e
della prova, se nel giudizio avente ad oggetto il pregiudizio
conseguente al provvedimento amministrativo illegittimo il privato deve
provare tutti gli elementi costitutivi del fatto illecito, in quello
avente ad oggetto il danno da violazione del giudicato opera, invece, il
principio dell'inversione dell'onere della prova di cui all'art. 1218
c.c. nella misura in cui viene posta a carico del debitore la prova che
l'inadempimento è stato determinato da impossibilità della prestazione
derivante da causa non imputabile. Ne consegue che l'interessato deve
dimostrare esclusivamente il suo diritto e la sussistenza di un
giudicato di accoglimento, mentre spetterà all'amministrazione la prova
di avervi ottemperato.

\hypertarget{lesperienza-tedesca}{%
\chapter{L'esperienza tedesca}\label{lesperienza-tedesca}}

\hypertarget{lazione-di-annullamento-anfechtungsklage-e-lazione-di-adempimento-verplichtungsklage}{%
\section{L'azione di annullamento (Anfechtungsklage) e l'azione di
adempimento
(Verplichtungsklage)}\label{lazione-di-annullamento-anfechtungsklage-e-lazione-di-adempimento-verplichtungsklage}}

Il legislatore tedesco ha affrontato il problema dell'esecuzione della
sentenza amministrativa avente ad oggetto un provvedimento
amministrativo, prima ancora che attraverso la predisposizione di un
meccanismo di coazione in presenza di un inadempimento
dell'amministrazione, con la previsione di un articolato sistema di
misure che consentono di prevenire la mancata spontanea esecuzione delle
pronunce del giudice e cercando di definire già a livello normativo
contenuto ed effetti che debbono assumere le decisioni giurisdizionali
in presenza di determinati presupposti. In questa prospettiva, è
importante in primo luogo che gli obblighi dell'amministrazione
derivanti dalla decisione del giudice amministrativo siano facili da
assolvere e perfettamente determinati\footnote{M. FROMONT,
  \emph{L'esecuzione delle decisioni del giudice amministrativo nel
  diritto francese e tedesco}, in \emph{Problemi di amministrazione
  pubblica}, 3, 1989, 523.}. Con la legge del 21 gennaio 1960
\emph{(VwGO - Verwaltungsgerichtsordnung)} sull'ordinamento processuale
amministrativo sono state introdotte due distinte azioni: una di
impugnazione in senso stretto o di annullamento
\emph{(Anfechtungsklage)} ed un'altra di condanna all'emissione di un
dato provvedimento, altrimenti detta ``di adempimento''
\emph{(Verplichtungsklage)}. Ove l'autorità emetta un provvedimento
incidente negativamente nella sfera giuridica del destinatario, questi
ricorrerrà alla \emph{Anechtungsklage} facendone valere eventuali vizi;
ove invece il privato aspiri ad ottenere un provvedimento ampliativo
della propria posizione giuridica soggettiva e si veda opporre un
rifiuto espresso oppure debba constatare l'inerzia dell'amministrazione,
azionerà il rimedio della \emph{Verplichtungsklage}, chiedendo la
condanna al rilascio del provvedimento rifiutato od omesso. L'azione di
annullamento o di impugnazione, così come quella di adempimento, sono
disciplinate dal \emph{§ 42 \emph{VwGO} che al primo comma prevede che
}''mediante azione può essere richiesto l'annullamento di un atto
amministrativo\_ (azione di impugnazione), \emph{come pure la condanna
all'emanazione di un atto amministrativo rifiutato o omesso''} (azione
di inadempimento) e al secondo comma che \emph{``qualora la legge non
disponga diversamente, l'azione è ammissibile solo quando l'attore fa
valere di essere stato leso nei propri diritti dall'atto amministrativo
o dal suo rifiuto o omissione''}\footnote{G. FALCON, C. FRAENKEL,
  \emph{Ordinamento processuale amministrativo tedesco (VwGO)}, Trento,
  2000.}. L'azione di annullamento è fondata allorquando ricorrano i
requisiti previsti dal \emph{§ 113 VwGO}, e cioè nella misura in cui
l'atto risulti illegittimo e lesivo dei cosiddetti diritti civili
pubblici dell'attore \emph{(Kläger)}, quei diritti cioè che conferiscono
al singolo la facoltà di pretendere dall'amministrazione una prestazione
positiva o negativa. Verificata la sussistenza di questi presupposti, il
tribunale potrà quindi annullare l'atto, ma la sentenza che conclude il
giudizio di impugnazione potrà assumere un contenuto ulteriore e diverso
dal mero annullamento del provvedimento impugnato, strettamente
correlato all'attività esecutiva che l'amministrazione dovrebbe
successivamente porre in essere per adeguarsi al \emph{decisium}.
L'effetto demolitorio del provvedimento illegittimo, previa la
sospensione della sua efficacia esecutiva, potrebbe rendere non
necessaria la successiva attività di adeguamento; diversamente,
nell'ambito della stessa sentenza che definisce il giudizio cassatorio,
è prevista la possibilità per il giudice di guidare l'amministrazione
nella scelta delle modalità di esecuzione della sentenza, per il
ripristino dello \emph{status quo ante} attraverso la cancellazione
degli effetti che si sono nel frattempo prodotti. E' questo l'istituto
del cosiddetto \emph{Folgenbeseitigungsanspruch}\footnote{Letteralmente
  ``pretesa o diritto all'eliminazione delle conseguenze dell'atto''.},
indicato con l'abbreviazione \emph{FBA} e contemplato dal \emph{§ 113/1}
secondo alinea \emph{VwGO}, ove si prevede che \emph{``se l'atto
amministrativo è stato già eseguito, il tribunale può anche dichiarare,
su richiesta, se e come l'autorità amministrativa debba revocare
l'esecuzione\_. Il \emph{FBA} è autonomo rispetto all'azione di
annullamento, inquadrabile fra le cosiddette azioni di prestazione,
ancorché il giudice dell'impugnazione si pronunci sull'eliminazione
degli effetti dell'atto con la medesima sentenza che definisce il
giudizio cassatorio. L'utilizzo del termine''può'' }(Kann)\_ da parte
del legislatore indica la mera facoltà di cumulare la domanda di revoca
dell'esecuzione a quella di annullamento dell'atto eseguito, ma ciò non
esclude la possibilità di proporre separata istanza, instaurando un
autonomo giudizio. Rimane impregiudicata la facoltà per il tribunale, ai
sensi del \emph{§ 93} secondo alinea \emph{VwGO}, di ordinare in ogni
caso che le rispettive domande vengano trattate e decise in separati
processi. Il c.~1, alinea terzo del \emph{§113 VwGO} stabilisce che la
pretesa alla revoca dell'esecuzione è ammissibile solo laddove
l'autorità amministrativa sia in grado di darvi seguito. In altri
termini, l'attività di rimozione degli effetti dell'esecuzione del
provvedimento annullato presuppone una prestazione possibile sotto il
profilo giuridico-fattuale. Qualora l'amministrazione non sia in grado
di ripristinare esattamente la situazione pregressa, dovrebbe
ricostruirne una quantomeno simile a quella precedente l'esecuzione
dell'atto annullato, in modo tale da eliminare al massimo i pregiudizi
per il destinatario del provvedimento\footnote{P. BECKER, H. KUNI,
  \emph{Probleme des verwaltungsgerichtlichen Vergabeverfahrens für
  Studienplätze in der Humanmedizin,} in \emph{DVBl} 1976, 863.}. Si
ritiene inoltre che la revoca dell'esecuzione come disposta dal giudice
possa consistere, oltre che nella rimozione di un'attività materiale
dell'amministrazione, anche nell'adozione di un atto amministrativo,
quale ad esempio l'ordine di sgombero di un appartamento a seguito
dell'annullamento della confisca dell'immobile da parte delle forze di
polizia, con susseguente sua occupazione da parte di un
terzo\footnote{\_VGH Kassel, 11.11.1993, in \emph{NVwZ}, 1995, 301; F.
  O. KOPP, W. R. SCHENKE, \emph{Verwaltungsgerichtsordnung}, cit. sub. §
  113, n.~91.}. Ulteriore presupposto di ammissibilità della pretesa,
oltre al fatto che l'autorità sia in grado di darvi seguito, è che la
questione sia matura per la decisione\footnote{§ 113/2, secondo alinea
  \emph{VwGO}.}. Ciò significa che non deve più esserci necessità di
accertare i fatti e non deve residuare alcuna discrezionalità in capo
all'amministrazione per quanto riguarda le modalità di revoca
dell'intervenuta esecuzione. Il \emph{FBA} sarà escluso laddove la
rimozione delle conseguenze dell'esecuzione sia in contrasto con la
legge al momento della decisione del tribunale\footnote{F. O. KOPP, W.
  R. SCHENKE, \emph{Verwaltungsgerichtsordnung}, cit., sub § 113, n.~87.}.
In definitiva, l'amministrazione che si trovi a dover eseguire la
sentenza di annullamento e, quindi, a ripristinare la situazione
esistente prima del provvedimento caducato, potrà essere guidata dal
giudice nella scelta delle misure necessarie all'esecuzione del
\emph{dictum} giudiziale, almeno per quel che concerne la rimozione
degli effetti strettamente connessi all'esecuzione del provvedimento
annullato. L'inottemperanza alla decisione sotto tale profilo, seppur
non assistita da alcun meccanismo di coazione diretta, potrà tuttavia
essere sanzionata attraverso l'attivazione della peculiare procedura di
coercizione indiretta di cui al \emph{§ 172 VwGO}, consistente
nell'assegnazione da parte del giudice, su richiesta dell'interessato,
di un termine per l'esecuzione della pronuncia e, nel caso di
inosservanza del medesimo, nell'irrogazione di un'ammenda, lo
\emph{Zwangsgeld}.

\hypertarget{il-contenzioso-ingiuntivo-la-c.d.-verpflichtungsklage}{%
\section{\texorpdfstring{Il contenzioso ingiuntivo: la c.d.
\emph{``Verpflichtungsklage''}}{Il contenzioso ingiuntivo: la c.d. ``Verpflichtungsklage''}}\label{il-contenzioso-ingiuntivo-la-c.d.-verpflichtungsklage}}

La disposizione al § 113/5 \emph{VwGO} contempla la sentenza sulla c.d.
\emph{Verplichtungsklage}, azione di prestazione o, secondo definizione
della dottrina italiana, di condanna, con la quale si ingiunge alla
pubblica amministrazione l'emanazione di un atto rifiutato od
omesso\footnote{Il § 113/5 \emph{VwGO}, così come tradotto da G. FALCON,
  C. FRAENKEL, \emph{Ordinamento processuale, cit., 94}, recita:
  \emph{``Nella misura in cui il rifiuto o l'omissione dell'atto
  amministrativo è illegittimo e l'attore ne risulta leso nei propri
  diritti, il tribunale dichiara l'obbligo dell'autorità amministrativa
  di porre in essere la richiesta attività dell'ufficio, se la questione
  è matura per la decisione. Altrimenti esso dichiara l'obbligo di
  decidere nei confronti dell'attore nel rispetto della concezione
  giuridica del tribunale}.}. Essa è diretta sostanzialmente a sindacare
il rifiuto del provvedimento amministrativo richiesto dall'interessato
che consegue all'assunzione di un provvedimento espresso di diniego; in
tal caso l'azione assumerà, seppur in via sussidiaria, anche un
contenuto annullatorio, dal momento che l'eventuale sentenza di
accoglimento ne comporterà l'eliminazione\footnote{Rileva D. DE PRETIS,
  \emph{Il processo amministrativo in Europa, cit., 121} come in questo
  caso oggetto del giudizio non è, se non in un secondo momento, il
  provvedimento di espresso diniego di quanto richiesto dal privato, ma,
  in primo luogo, il provvedimento che l'amministrazione avrebbe dovuto
  assumere per soddisfare la pretesa sostanziale del ricorrente, per cui
  è tale ultimo atto che l'amministrazione sarà tenuta ad adottare a
  seguito dell'eventuale sentenza di accoglimento della
  \emph{Verpflightungsklage}.}, ovvero l'omissione di un provvedimento
non ancora denegato. Presupposti di fondatezza della
\emph{Verplichtungsklage}, così come per l'azione di annullamento, sono
l'illegittimità del diniego o dell'inerzia e la lesione dei diritti del
ricorrente. In presenza di tali condizioni, il giudice affermerà
l'obbligo dell'autorità amministrativa di adottare il provvedimento
richiesto, qualora la questione sottoposta al suo esame consenta una
decisione definitiva. Tuttavia, una condanna dell'amministrazione
all'emissione di un atto dotato di un ben preciso contenuto potrà
intervenire soltanto nel caso in cui la stessa amministrazione non abbia
poteri discrezionali, ovvero qualora l'adozione del provvedimento
richiesto rappresenti l'unica manifestazione della discrezionalità
amministrativa priva di errori, tenuto presente che la sentenza di
condanna, in virtù del principio costituzionale della separazione dei
poteri, non sostituisce la decisione dell'amministrazione, ma la obbliga
ad agire. Laddove invece la controversia non consenta una decisione
definitiva, il tribunale, previo riconoscimento della legittimità della
pretesa del ricorrente ad una decisione dell'amministrazione, si
limiterà a statuire l'obbligo di quest'ultima di decidere secondo la
valutazione da esso espressa sulla questione\footnote{§ 113/5
  \emph{VwGO}. Si parla in questo caso di \emph{Bescheidungsurteil}.}.
Partendo dalla considerazione che spesso l'amministrazione non
ottempera, in quanto non sa come ottemperare, l'istituto della
\emph{Verplichtungsklage} acquista notevole importanza nella misura in
cui coinvolge il giudice sin dalla prima pronuncia nel chiarimento della
portata operativa della decisione, ottenendo così una maggior ``presa''
sul successivo comportamento\footnote{G. FALCON, \emph{Per una migliore
  giustizia amministrativa}, cit., 151.}. Per converso, ove
l'amministrazione sia titolare di un più ampio potere discrezionale, il
giudice incontrerà il limite della insostituibilità delle scelte
discrezionali dell'autorità, quindi la susseguente attività
amministrativa rimarrà pur sempre appannaggio dell'amministrazione, con
l'unico limite costituito dall'obbligo di rispettare la ``concezione
giuridica'' del tribunale (c.d. \emph{Bescheidungsurteil}).

\hypertarget{lapplicazione-generalizzata-della-tutela-cautelare}{%
\section{L'applicazione generalizzata della tutela
cautelare}\label{lapplicazione-generalizzata-della-tutela-cautelare}}

Si è visto come, nei casi in cui l'amministrazione incontri delle
difficoltà nell'ottemperare alle sentenze, dovute al fatto di non sapere
come attuarle, il legislatore tedesco abbia cercato di determinare
puntualmente gli effetti delle sentenze, in modo tale da guidare
l'amministrazione nell'esecuzione del \emph{decisium}. Quanto invece
alla difficolta, se non a volte all'impossibilità, di di rimuovere
completamente la situazione creata da un provvedimento sacrificativo poi
annullato, si è cercato di affrontarla attraverso un'estesa applicazione
della tutela interinale della posizione del ricorrente, la quale si
diversifica a seconda del tipo di provvedimento che viene in rilievo.
Nell'ipotesi di atto amministrativo che incide negativamente sulla sfera
giuridica dell'interessato, cui nel nostro ordinamento corrispondono gli
interessi oppositivi, si ammette, salvo alcune eccezioni, la sospensione
automatica dei suoi effetti in derivazione, prima ancora che della mera
proposizione dell'impugnativa in sede giurisdizionale, della
proposizione del ricorso amministrativo previo che costituisce
condizione di ammissibilità della \emph{Anfechtungsklage} ai sensi del §
68 \emph{VwGO} \footnote{BVerwG, 21.06.1961, in \emph{BVerwGE} 13, 5.}.
Attraverso un impiego esteso e generalizzato della sospensione cautelare
del provvedimento impugnato , ex § 80 \emph{VwGO}, molti dei problemi
legati alla esecuzione delle pronunce del tribunale finiscono per essere
risolti o comunque attenuati in via preventiva, considerata,
diversamente, la necessità di eliminare gli effetti materiale prodotti
dal provvedimento fino al passaggio in giudicato della sentenza di
annullamento\footnote{B. MARCHETTI, \emph{L'esecuzione della sentenza
  amministrativa}, cit., 46.}. La sospensione cautelare non è però
idonea a tutelare il ricorrente che lamenti l'illegittimo diniego di un
provvedimento ampliativo o, meglio, l'omesso rilascio del titolo. In
questi casi, contraddistinti dall'emersione di interessi legittimi
pretensivi, la tutela del privato è garantita dalla possibilità per il
giudice, ai sensi del § 123 \emph{VwGo}, di concedere misure provvisorie
a contenuto positivo anche anteriormente al ricorso, volte ad evitare
quelle modificazioni irreparabili che potrebbero \emph{medio tempore}
investire la situazione di fatto, rendendo alla fine una sentenza
favorevole, vantaggiosa sulla carta, ma nella sostanza inutile.

\hypertarget{le-misure-coercitive-lo-zwangsgeld-172-vwgo}{%
\section{\texorpdfstring{Le misure coercitive: lo \emph{Zwangsgeld (§
172
VwGO)}}{Le misure coercitive: lo Zwangsgeld (§ 172 VwGO)}}\label{le-misure-coercitive-lo-zwangsgeld-172-vwgo}}

Il \emph{VwGO} disciplina l'esecuzione coattiva delle sentenze del
giudice amministrativo nei confronti della pubblica amministrazione ai
§§ 167-172. In particolare, la legge sul processo amministrativo
\emph{(VwGO)} determina il giudice dell'esecuzione (§ 167), i titoli
esecutivi (§ 168), l'esecuzione a favore della mano pubblica (§ 169),
l'esecuzione contro la mano pubblica (§§ 170 e 172), nonché i casi in
cui non è necessaria la formula esecutiva (§ 171). I §§ 170 e 172
\emph{VwGO} rappresentano la base normativa dell'esecuzione forzata
contro la pubblica amministrazione, ancorché i rispettivi ambiti di
applicazione siano da tenere distinti. Il § 170 \emph{VwGO} disciplina
l'esecuzione contro la mano pubblica per crediti pecuniari, compresa la
penale di cui al § 172 (\emph{Zwangsgeld}). Tale norma è modellata sul §
882a \emph{ZPO} relativo all'esecuzione per crediti di denaro nei
confronti delle persone giuridiche di diritto pubblico e le modalità di
esecuzione sono sostanzialmente quelle previste dal codice di procedura
civile, nulla dicendo sul punto il § 170 \emph{VwGO} che tuttavia reca
alcuni correttivi che tengono conto della particolare condizione
giuridica del patrimonio pubblico e della sua tendenziale destinazione
all'assolvimento dei compiti dell'amministrazione. Più nel dettaglio, il
tribunale, da un lato, prima di procedere all'esecuzione forzata, deve
intimare all'autorità amministrativa di eseguire il giudicato entro il
termine massimo di un mese e, dall'altro, essendo l'esecuzione
inammissibile in relazione a beni essenziali per l'adempimento di
pubbliche funzioni o alla cui alienazione si contrapponga un pubblico
interesse, non può ordinare il sequestro di beni destinati all'uso o al
servizio pubblico. Il § 172 \emph{VwGO} attiene, in linea di principio,
all'esecuzione di decisioni dichiarative dell'obbligo
dell'amministrazione di rilasciare un provvedimento nei confronti della
controparte. Esso prevede un mezzo di coercizione meramente indiretto,
assistito dalla minaccia di una sanzione pecuniaria da applicarsi
all'autorità inadempiente senza alcuna limitazione o particolare
privilegio per la stessa, salva la misura massima dell'ammenda, di volta
in volta erogabile, pari a diecimila Euro. Nella misura in cui il
contenuto delle norme predette non dovesse essere esaustivo in relazione
al caso concreto, sarà possibile integrarle con i precetti del codice di
rito (\emph{ZPO}), in nome del principio di effettività della tutela
giurisdizionale\footnote{Sul punto si è pronunciata la Corte
  costituzionale tedesca in una decisione di notevole importanza, con
  cui, nella parte motiva, fa esplicito riferimento alle misure previste
  dai §§ da 885 a 896 \emph{ZPO} la cui scelta, in ordine alle modalità
  ed eventuale successione delle misure da adottare, sarà rimessa alla
  valutazione del giudice competente.}. Molto prima che venisse alla
luce la legge sulla giustizia amministrativa e con essa il § 172
\emph{VwGO}, l'idea di poter eseguire coattivamente le pronunce dei
giudici nei confronti di un soggetto esercente un pubblico potere era
stata decisamente avversata in dottrina. Si sosteneva, in particolare,
che si sarebbe rivelato un non senso che lo Stato, quale fondamento del
diritto (\emph{Hort des Rechts}) potesse essere coartato al rispetto di
quello stesso diritto di cui egli era portatore. Ciò si sarebbe rivelato
inconciliabile con il rispetto che si deve allo Stato medesimo ed
avrebbe leso la sua immagine, mettendone in discussione
l'onore\footnote{\_``L'Etat toujours doit être réputé honnête homme'',
  E. LAFERRIERE, \emph{Traité de la juridiction administrative et des
  recours contentieux}, Paris et Nancy 1896, I, 347 ss.}. Così recita il
§ 172 \emph{VwGO}: ``\emph{Se un'autorità, nei casi di cui al § 113, co.
1, secondo periodo, e co. 5 del § 123, non ottempera all'obbligo
impostole nella sentenza o nel provvedimento provvisorio, il tribunale
di primo grado può, su richiesta, comminare con ordinanza nei suoi
confronti un'ammenda fino a Euro diecimila, previa assegnazione di un
termine, applicarla dopo l'infruttuoso decorso del termine e portarla ad
esecuzione d'ufficio. L'ammenda può essere ripetutamente comminata,
applicata e portata ad esecuzione}''\footnote{G. FALCON, C. FRAENKEL,
  \emph{Ordinamento processuale amministrativo}, cit., § 172, 138.}. La
norma fa espresso riferimento all'esecuzione (indiretta) di una sentenza
di annullamento nella parte in cui contestualmente ingiunge
all'amministrazione la revoca della già introdotta esecuzione del
provvedimento caducato (§ 113/1 secondo alinea \emph{VwGO}), ovvero di
adempimento dell'obbligo di emettere un certo atto o di provvedere nel
rispetto della concezione giuridica del tribunale (§ 113/5 \emph{VwGo})
o, infine, il rilascio di un provvedimento positivo (§ 123 \emph{VwGO}).
Rimedi indiretti ulteriori, prodromici allo \emph{Zwangsgeld}, ma meno
impattanti e di valore per lo più simbolico, sono l'interpello
dell'autorità gerarchicamente superiore rispetto a quella che dovrebbe
eseguire la sentenza, ovvero il ricorso all'opinione pubblica attraverso
una petizione popolare o mediante il coinvolgimento della stampa al fine
di sollecitare o quantomeno esercitare una pressione psicologica
sull'amministrazione inottemperante\footnote{W. BANK,
  \emph{Zwangsvollstreckung gegen Behörde}, cit., 62.}. La limitazione
della procedura esecutiva nei confronti della pubblica amministrazione
ad un rimedio come lo \emph{Zwangsgeld}, perlomeno nei casi previsti dal
§ 172 \emph{VwGO}, la mette al riparo dall'esecuzione diretta, anche se
non è del tutto esclusa la possibilità di accedere alla tutela esecutiva
secondo i dettami del codice di rito nel caso di insuccesso della
procedura di coazione indiretta o, a certe condizioni, in alternativa
alla stessa. Ciò sarà consentito laddove si richieda all'amministrazione
una prestazione fungibile, mentre, per quanto riguarda l'emissione di un
atto amministrativo, rimane ferma la rigida separazione dei poteri tra
giurisdizione ed amministrazione. Quanto alla natura giuridica, lo
\emph{Zwangsgeld} è un semplice mezzo di coercizione e non già un
istituto di matrice sanzionatoria. L'ammenda di cui al § 172
\emph{VwGO}, così come altre disposizioni che fanno riferimento a tale
rimedio, non potrebbe, salvo eccezioni, essere applicata e portata ad
esecuzione laddove il destinatario abbia nel frattempo adempiuto. Lo
scopo precipuo dello \emph{Zwangsgeld} è quello di determinare il
debitore all'adempimento di un obbligo, attraverso la pressione mediata
esercitata sul destinatario dalla minaccia di una penale nei casi in
cui, in linea di massima, non sussiste la possibilità di attivare un
meccanismo di coercizione diretta, oppure perché tale meccanismo è a
forte rischio di insuccesso. La coazione indiretta viene dunque in
rilievo quando debba darsi esecuzione ad obblighi di fare infungibili,
la cui realizzazione è inevitabilmente rimessa alla volontà di un
determinato soggetto, oppure nell'ambito degli obblighi di tollerare o
di astenersi da una certa attività che, per definizione, possono essere
solo indirettamente coercibili\footnote{O. R. REMIEN, \emph{Zwangsgeld},
  cit., 11, 12.}. Il § 172 \emph{VwGO} si occupa della sentenza di
annullamento solo in funzione della accessoria statuizione ingiuntiva
dell'obbligo di ripristinare la situazione antecedente all'esecuzione
dell'atto annullato e l'ambito privilegiato dello \emph{Zwangsgeld}
risulterebbe essere quello dell'inottemperanza al giudicato formatosi
sulle sentenze di adempimento, ex § 113/5 \emph{VwGO}
(\emph{Verpflichtungsurteil}), ma l'esecuzione coattiva indiretta
secondo i dettami del § 172 \emph{VwGO} opera anche in conseguenza di un
\emph{Bescheidungsurteil}, con il quale viene unicamente sancito
l'obbligo dell'autorità di provvedere nel rispetto del quadro giuridico
delineato in sentenza, senza alcuna indicazione in ordine allo specifico
contenuto dell'atto da adottare. In sostanza, l'omesso rilascio del
richiesto provvedimento, così come la riedizione del potere
amministrativo in contrasto con il quadro giuridico delineato nel
\emph{Bescheidungsurteil}, consentono di attivare il rimedio dello
\emph{Zwangsgeld}. In particolare, il ricorrente vittorioso potrà
chiedere al tribunale di primo grado, con apposita istanza, di fissare
un termine entro il quale l'autorità dovrà dare completa esecuzione alla
sentenza, contestualmente determinando una penale nell'ammontare massimo
di Euro diecimila, per il caso di persistente inottemperanza anche oltre
la scadenza del termine predetto. In quest'ultima evenienza, sempre su
richiesta di parte, il tribunale provvederà ad applicare
all'amministrazione renitente l'ammenda stabilita ed a riscuoterla
coattivamente d'ufficio, con l'ulteriore possibilità di reiterare in
ipotesi all'infinito la procedura, fin tanto che permanga
l'inadempimento della pubblica autorità. La definitiva impossibilità di
dare esecuzione alla sentenza per causa imputabile all'amministrazione
potrà dar luogo, in ogni caso, ad una responsabilità risarcitoria della
stessa. Per poter attivare la procedura esecutiva ai sensi del § 172
\emph{VwGO} nei confronti dell'amministrazione inadempiente, è
necessario che ricorrano i seguenti presupposti: il titolo esecutivo che
sarà costituito ad esempio da una sentenza di condanna al ripristino
dello \emph{status quo ante} accessoria ad una sentenza di annullamento
(§ 113/1 secondo alinea \emph{VwGO}), di adempimento (§ 113/5
\emph{VwGO}) o da un provvedimento provvisorio positivo (§ 123
\emph{VwGO}), la notifica del titolo alla controparte, ai sensi del §
167/1 \emph{VwGO} in combinato disposto con i §§ 795, 724, 750/1
\emph{ZPO}, la formula esecutiva da apporre alla decisione del giudice e
l'inottemperanza dell'amministrazione alla statuizione giudiziale che
deve essere definitiva, ove si tratti di sentenza.

\hypertarget{i-mezzi-di-tutela-esperibili-dalle-parti}{%
\section{I mezzi di tutela esperibili dalle
parti}\label{i-mezzi-di-tutela-esperibili-dalle-parti}}

I provvedimenti del giudice dell'esecuzione relativi alla minaccia ed
applicazione dello \emph{Zwangsgeld} possono essere eventualmente
impugnati con ricorso, ai sensi del § 146 \emph{VwGO} entro due
settimane dalla loro notifica, di regola davanti al tribunale
amministrativo di grado intermedio. In tal caso, l'impugnativa proposta
dalla pubblica amministrazione avverso l'ordinanza applicativa
dell'ammenda, comporterà l'automatica sospensione dell'efficacia
esecutiva del provvedimento\footnote{R. PIETZNER, in F. SCHOCH, E.
  SCHMIDT-AßMANN, R. PIETZNER, \emph{VwGO}, cit. sub §172, n.~52.}. Allo
stesso modo, il privato potrà avvalersene nel caso in cui il tribunale
rigetti, sempre con ordinanza, l'istanza volta ad ottenere la minaccia o
l'irrogazione della penale. Con tale rimedio possono essere fatti valere
soltanto i vizi formali della procedura esecutiva, cioè il mancato
rispetto delle regole procedurali disciplinanti la stessa\footnote{R.
  PIETZNER, in F. SCHOCH, E. SCHMIDT-AßMANN, R. PIETZNER, \emph{VwGO},
  cit. sub §167, n.~5.}. Avverso la decisione sul predetto ricorso, non
è dato alcun ulteriore mezzo di tutela, conformemente a quanto stabilito
dal § 152/1 \emph{VwGO}. Diversamente, ove sia in contestazione da parte
dell'amministrazione il diritto di procedere ad esecuzione forzata da
parte del privato, potrà essere introdotto ricorso per opposizione
all'esecuzione, ai sensi del § 167/1 \emph{VwGO}, in combinato disposto
con il § 767 \emph{ZPO}. Scopo di questo rimedio è quello di elidere
l'efficacia esecutiva del titolo, non potendo chiaramente essere rimesso
in discussione il giudicato. In tal senso, il § 767/2 \emph{ZPO}
prescrive che possano essere fatte valere soltanto quelle eccezioni che
si fondino su circostanze sopravvenute all'udienza per la discussione
della causa, nell'ambito del processo di cognizione; inoltre, l'autorità
amministrativa dovrà sollevare tutte le eccezioni deducibili al momento
dell'opposizione, come previsto al § 767/3 \emph{ZPO}, ad esempio la
sopravvenuta modifica della situazione di fatto o di diritto rispetto a
quella sulla quale è sceso il giudicato, così come il sopravvenuto
adempimento dell'obbligo da esso discendente. Si pensi all'entrata in
vigore di un nuovo piano regolatore in contrasto con il permesso di
costruire al cui rilascio l'amministrazione veniva condannata con
sentenza di adempimento (\emph{Verplichtungsurteil}): fin tanto che non
venga rilasciato il titolo autorizzativo, il giudicato avente ad oggetto
il riconoscimento della relativa pretesa non è al riparo da eventuali
sopravvenienze di diritto, a differenza di ciò che accade nel diritto
civile, ove è sempre irrilevante il mutamento del quadro normativo entro
il quale dovrebbe essere eseguito il giudicato\footnote{\_BVerwG
  26.10.1984, in \emph{NVwZ} 1985, 563.}. Competente a giudicare
dell'opposizione è il tribunale di prima istanza.

\hypertarget{il-rapporto-fra-lo-zwangsgeld-ed-il-risarcimento-del-danno-da-giudicato}{%
\section{Il rapporto fra lo Zwangsgeld ed il risarcimento del danno da
giudicato}\label{il-rapporto-fra-lo-zwangsgeld-ed-il-risarcimento-del-danno-da-giudicato}}

Una delle peculiarità del sistema tedesco di coercizione indiretta è
rappresentata dalla integrale devoluzione alle casse dello Stato delle
somme ricavate dalle ammende. La diversa impostazione che vuole che il
creditore sia beneficiario delle somme, abbracciata dai paesi che, come
la Francia, hanno subito le influenze del diritto romano, in cui vi era
una commistione fra l'\emph{astreinte} ed il risarcimento del danno, è
stata fortemente criticata in quanto, avendo il ricorrente vittorioso la
possibilità di agire in via risarcitoria per il caso di ritardata od
omessa esecuzione del giudicato da parte dell'amministrazione,
attribuire al privato anche l'importo della penale, vorrebbe dire
arricchirlo ingiustificatamente. La soluzione seguita in Francia
presenterebbe lo svantaggio di rendere incerti i confini tra lo
\emph{Zwangsgeld} e il risarcimento del danno, poiché l'associazione tra
i due istituti pregiudicherebbe l'effetto di coazione che è proprio
dell'ammenda ex § 172 \emph{VwGO} e, allo stesso tempo, la possibilità
di cumularli porterebbe il creditore a ricevere troppo. Inoltre, la
destinazione dello \emph{Zwangsgeld} alle casse dello Stato sarebbe
ulteriormente funzionale a garantire il rispetto delle decisioni
giurisdizionali e quindi il prestigio dell'amministrazione della
giustizia. Per quanto concerne il rimedio risarcitorio, la disciplina è
quella valevole per tutte le ipotesi di responsabilità della pubblica
amministrazione per i danni cagionati nell'esercizio dei propri doveri
d'ufficio. Le norme fondamentali in materia sono l'art. 34 del
\emph{Grundgesetz} e il § 839 del \emph{Bürgerliches Gesetzbuch}, le
quali vanno lette in combinato disposto\footnote{S. DETTERBECK,
  \emph{Allgemeines Verwaltungsrecht}, cit., 379.}. In base al c.~1 del
§ 839 \emph{BGB} i danni causati intenzionalmente o con negligenza, in
violazione di un dovere del funzionario vengono dallo stesso
integralmente risarciti. Se si tratta di mera negligenza, la
responsabilità del funzionario potrà essere invocata in via sussidiaria,
ovvero soltanto nel caso in cui non vi sia un'altra via per ottenere il
risarcimento, ad esempio per contratto, per legge, o in base al sistema
di assicurazione sociale\footnote{U. KARPEN, \emph{L'esperienza della
  Germania}, in D. SORACE (a cura di) \emph{La responsabilità pubblica
  nell'esperienza giuridica europea}, Bologna, 1994, 140.}. Per
converso, l'art. 34 della \emph{Grundnorm} trasferisce la responsabilità
sulla pubblica autorità da cui il funzionario dipende, salvo il regresso
nei casi di dolo e colpa grave per evitare che i funzionari abusino
dell'immunità della responsabilità personale loro garantita\footnote{\ldots{}}
e radica in capo al giudice ordinario la giurisdizione per le azioni per
le azioni di responsabilità nei confronti del potere pubblico,
stabilendo al terzo alinea, con riferimento al diritto al risarcimento e
al diritto di rivalsa, che non può mai essere esclusa l'azione di fronte
alla giurisdizione ordinaria.

\hypertarget{la-giustizia-amministrativa-francese}{%
\chapter{La giustizia amministrativa
francese}\label{la-giustizia-amministrativa-francese}}

\ldots{}

\hypertarget{la-giustizia-amministrativa-nel-regno-unito}{%
\chapter{La giustizia amministrativa nel Regno
Unito}\label{la-giustizia-amministrativa-nel-regno-unito}}

\ldots{}

\hypertarget{il-sistema-spagnolo}{%
\chapter{Il sistema spagnolo}\label{il-sistema-spagnolo}}

\ldots{}

\newpage

Donec eu risus quis ante fermentum vestibulum in at ex. Nulla tellus
est, accumsan vel iaculis eget, lacinia at sapien. Maecenas sed ligula
dignissim, sagittis lorem at, euismod metus. Vivamus sodales elementum
accumsan. Nullam maximus risus vel nisi semper gravida. \textbf{Esempio
di citazione breve:} rinnovo della regolazione di alcuni tra gli
istituti centrali (Salmerón \& Seira s.d.). Aliquam ut dignissim purus.
Quisque quis arcu dignissim, pellentesque tellus a, faucibus ex.

Duis in leo ut felis porttitor consectetur. Ut luctus ante eget orci
vehicula, vel malesuada sapien placerat. \textbf{Esempio di citazione
dettagliata:} Lorem ipsum dolor sit amet, consectetur adipiscing elit.
Etiam molestie ex nec arcu tristique tempus. Aliquam ut dignissim purus.
Quisque quis arcu dignissim, pellentesque tellus a, faucibus ex. Sed
bibendum vulputate est nec scelerisque. Nulla a est molestie, tincidunt
purus non, volutpat dolor.

\begin{quote}
\emph{``La Legge 4/1999, del 13 gennaio, di modifica della LAP, ha
introdotto un insieme di misure destinate a rinnovare la regolazione di
alcuni tra gli istituti centrali o fondamentali del Diritto
Amministrativo spagnolo''} (Salmerón \& Seira s.d.).
\end{quote}

Sed ornare enim ut lectus dictum semper. Duis in leo ut felis porttitor
consectetur. Ut luctus ante eget orci vehicula, vel malesuada sapien
placerat.

\hypertarget{conclusioni}{%
\chapter{Conclusioni}\label{conclusioni}}

\hypertarget{riepilogo}{%
\section{Riepilogo}\label{riepilogo}}

pellentesque habitant morbi tristique senectus et netus et malesuada
fames ac turpis egestas. Nunc eleifend, ex a luctus porttitor, felis ex
suscipit tellus, ut sollicitudin sapien purus in libero. Nulla blandit
eget urna vel tempus. Praesent fringilla dui sapien, sit amet egestas
leo sollicitudin at.

\hypertarget{approfondimenti}{%
\section{Approfondimenti}\label{approfondimenti}}

Lorem ipsum dolor sit amet, consectetur adipiscing elit. Aliquam gravida
ipsum at tempor tincidunt. Aliquam ligula nisl, blandit et dui eu,
eleifend tempus nibh. Nullam eleifend sapien eget ante hendrerit
commodo. Pellentesque pharetra erat sit amet dapibus scelerisque.

Vestibulum suscipit tellus risus, faucibus vulputate orci lobortis eget.
Nunc varius sem nisi. Nunc tempor magna sapien, euismod blandit elit
pharetra sed. In dapibus magna convallis lectus sodales, a consequat sem
euismod. Curabitur in interdum purus. Integer ultrices laoreet aliquet.
Nulla vel dapibus urna. Nunc efficitur erat ac nisi auctor sodales.

\hypertarget{appendice-1-extra-a}{%
\chapter*{Appendice 1: Extra-A}\label{appendice-1-extra-a}}
\addcontentsline{toc}{chapter}{Appendice 1: Extra-A}

Vivamus hendrerit rhoncus interdum. Sed ullamcorper et augue at porta.
Suspendisse facilisis imperdiet urna, eu pellentesque purus suscipit in.
Integer dignissim mattis ex aliquam blandit. Curabitur lobortis quam
varius turpis ultrices egestas.

\hypertarget{appendice-2-extra-b}{%
\chapter*{Appendice 2: Extra-B}\label{appendice-2-extra-b}}
\addcontentsline{toc}{chapter}{Appendice 2: Extra-B}

Aliquam rhoncus mauris ac neque imperdiet, in mattis eros aliquam. Etiam
sed massa et risus posuere rutrum vel et mauris. Integer id mauris sed
arcu venenatis finibus. Etiam nec hendrerit purus, sed cursus nunc.
Pellentesque ac luctus magna. Aenean non posuere enim, nec hendrerit
lacus. Etiam lacinia facilisis tempor. Aenean dictum nunc id felis
rhoncus aliquam.

\footnotesize
\singlespacing
\setlength{\parindent}{0in}

\hypertarget{riferimenti}{%
\chapter*{Riferimenti}\label{riferimenti}}
\addcontentsline{toc}{chapter}{Riferimenti}

\hypertarget{refs}{}
\begin{CSLReferences}{1}{0}
\leavevmode\vadjust pre{\hypertarget{ref-salmeronRiformaProcedimentoAmministrativo}{}}%
Salmerón, M.F. \& Seira, C.C., Riforma Del Procedimento Amministrativo
in {Spagna}: La {Legge} 4/1999, Del 13 Gennaio, Di Modifica Della
{Legge} 30/1992, Del 26 Novembre, de {Régimen Jurídico} de Las
{Administraciones Públicas} y Del {Procedimiento Administrativo Común}.
Available at:
\url{https://digilander.libero.it/bhilex/studi/artprammsez_I_II1.htm?utm_source=pocket_mylist}
{[}Consultato ottobre 8, 2022{]}.

\end{CSLReferences}



\end{document}
