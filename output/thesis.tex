\documentclass[12pt,it,a4paper,]{report}
\usepackage{XCharter}

% Overwrite \begin{figure}[htbp] with \begin{figure}[H]
\usepackage{float}
\let\origfigure=\figure
\let\endorigfigure=\endfigure
\renewenvironment{figure}[1][]{%
\origfigure[b]
}{%
\endorigfigure
}

% fix for pandoc 1.14
\providecommand{\tightlist}{%
  \setlength{\itemsep}{0pt}\setlength{\parskip}{0pt}}

% TP: hack to truncate list of figures/tables.
\usepackage{truncate}
\usepackage{caption}
\usepackage{tocloft}
% TP: end hack

\usepackage{amssymb,amsmath}
\usepackage{ifxetex,ifluatex}

% Only use fixltx2e if using pre-2015 kernels
\begingroup\expandafter\expandafter\expandafter\endgroup
\expandafter\ifx\csname IncludeInRelease\endcsname\relax
  \usepackage{fixltx2e}
\fi

\ifnum 0\ifxetex 1\fi\ifluatex 1\fi=0 % if pdftex
  \usepackage[T1]{fontenc}
  \usepackage[utf8]{inputenc}
\else % if luatex or xelatex
  \ifxetex
    \usepackage{mathspec}
    \usepackage{xltxtra,xunicode}
  \else
    \usepackage{fontspec}
  \fi
  \defaultfontfeatures{Mapping=tex-text,Scale=MatchLowercase}
  \newcommand{\euro}{€}
    \setmainfont{XCharter}
\fi
% use upquote if available, for straight quotes in verbatim environments
\IfFileExists{upquote.sty}{\usepackage{upquote}}{}
% use microtype if available
\IfFileExists{microtype.sty}{%
\usepackage{microtype}
\UseMicrotypeSet[protrusion]{basicmath} % disable protrusion for tt fonts
}{}
\ifxetex
  \usepackage[setpagesize=false, % page size defined by xetex
              unicode=false, % unicode breaks when used with xetex
              xetex]{hyperref}
\else
  \usepackage[unicode=true]{hyperref}
\fi
\hypersetup{breaklinks=true,
            bookmarks=true,
            pdfauthor={Marina Chiapello},
            pdftitle={I rimedi all'inottemperanza della Pubblica Amministrazione in alcuni ordinamenti europei},
            colorlinks=true,
            citecolor=blue,
            urlcolor=blue,
            linkcolor=magenta,
            pdfborder={0 0 0}}
\urlstyle{same}  % don't use monospace font for urls
\setlength{\parindent}{0pt}
\setlength{\parskip}{6pt plus 2pt minus 1pt}
\setlength{\emergencystretch}{3em}  % prevent overfull lines
\setcounter{secnumdepth}{5}
\ifxetex
  \usepackage{polyglossia}
  \setmainlanguage{italian}
\else
  \usepackage[it]{babel}
\fi

% % \newlength{\cslhangindent}
% \setlength{\cslhangindent}{1.5em}
% \newenvironment{cslreferences}%
%   {}%
%   {\par}
% 
\newlength{\cslhangindent}
\setlength{\cslhangindent}{1.5em}
\newlength{\csllabelwidth}
\setlength{\csllabelwidth}{3em}
\newenvironment{CSLReferences}[2] % #1 hanging-ident, #2 entry spacing
 {% don't indent paragraphs
  \setlength{\parindent}{0pt}
  % turn on hanging indent if param 1 is 1
  \ifodd #1 \everypar{\setlength{\hangindent}{\cslhangindent}}\ignorespaces\fi
  % set entry spacing
  \ifnum #2 > 0
  \setlength{\parskip}{#2\baselineskip}
  \fi
 }%
 {}
\usepackage{calc}
\newcommand{\CSLBlock}[1]{#1\hfill\break}
\newcommand{\CSLLeftMargin}[1]{\parbox[t]{\csllabelwidth}{#1}}
\newcommand{\CSLRightInline}[1]{\parbox[t]{\linewidth - \csllabelwidth}{#1}\break}
\newcommand{\CSLIndent}[1]{\hspace{\cslhangindent}#1}



% Table of contents formatting
\renewcommand{\contentsname}{Table of Contents}
\setcounter{tocdepth}{3}

% Headers and page numbering
\usepackage{fancyhdr}
\pagestyle{plain}

% Following package is used to add background image to front page
\usepackage{wallpaper}

% Table package
\usepackage{ctable}% http://ctan.org/pkg/ctable

% Deal with 'LaTeX Error: Too many unprocessed floats.'
\usepackage{morefloats}
% or use \extrafloats{100}
% add some \clearpage

% % Chapter header
% \usepackage{titlesec, blindtext, color}
% \definecolor{gray75}{gray}{0.75}
% \newcommand{\hsp}{\hspace{20pt}}
% \titleformat{\chapter}[hang]{\Huge\bfseries}{\thechapter\hsp\textcolor{gray75}{|}\hsp}{0pt}{\Huge\bfseries}

% % Fonts and typesetting
% \setmainfont[Scale=1.1]{Helvetica}
% \setsansfont[Scale=1.1]{Verdana}

% FONTS
\usepackage{xunicode}
\usepackage{xltxtra}
\defaultfontfeatures{Mapping=tex-text} % converts LaTeX specials (``quotes'' --- dashes etc.) to unicode
% \setromanfont[Scale=1.01,Ligatures={Common},Numbers={OldStyle}]{Palatino}
% \setromanfont[Scale=1.01,Ligatures={Common},Numbers={OldStyle}]{Adobe Caslon Pro}
%Following line controls size of code chunks
% \setmonofont[Scale=0.9]{Monaco}
%Following line controls size of figure legends
% \setsansfont[Scale=1.2]{Optima Regular}

% CODE BLOCKS
\usepackage[utf8]{inputenc}
\usepackage{listings}
\usepackage{color}

% JAVA CODE BLOCKS
%\definecolor{backcolour}{RGB}{242,242,242}
%\definecolor{javared}{rgb}{0.6,0,0}
%\definecolor{javagreen}{rgb}{0.25,0.5,0.35}
%\definecolor{javapurple}{rgb}{0.5,0,0.35}
%\definecolor{javadocblue}{rgb}{0.25,0.35,0.75}

\lstdefinestyle{javaCodeStyle}{
  language=Java,                         % the language of the code
  backgroundcolor=\color{backcolour},    % choose the background color; you must add \usepackage{color} or \usepackage{xcolor}
  basicstyle=\fontsize{10}{8}\sffamily,
  breakatwhitespace=false,
  breaklines=true,
  keywordstyle=\color{javapurple}\bfseries,
  stringstyle=\color{javared},
  commentstyle=\color{javagreen},
  morecomment=[s][\color{javadocblue}]{/**}{*/},
  captionpos=t,                          % sets the caption-position to bottom
  frame=single,                          % adds a frame around the code
  numbers=left,
  numbersep=10pt,                         % margin between number and code block
  keepspaces=true,                       % keeps spaces in text, useful for keeping indentation of code (possibly needs columns=flexible)
  columns=fullflexible,
  showspaces=false,                      % show spaces everywhere adding particular underscores; it overrides 'showstringspaces'
  showstringspaces=false,                % underline spaces within strings only
  showtabs=false,                        % show tabs within strings adding particular underscores
  tabsize=2                              % sets default tabsize to 2 spaces
}

%Attempt to set math size
%First size must match the text size in the document or command will not work
%\DeclareMathSizes{display size}{text size}{script size}{scriptscript size}.
%\DeclareMathSizes{12}{13}{7}{7}

% ---- CUSTOM AMPERSAND
% \newcommand{\amper}{{\fontspec[Scale=.95]{Adobe Caslon Pro}\selectfont\itshape\&}}

% HEADINGS
\usepackage{sectsty}
\usepackage[normalem]{ulem}
\sectionfont{\rmfamily\mdseries\Large}
\subsectionfont{\rmfamily\mdseries\scshape\large}
\subsubsectionfont{\rmfamily\bfseries\upshape\large}
% \sectionfont{\rmfamily\mdseries\Large}
% \subsectionfont{\rmfamily\mdseries\scshape\normalsize}
% \subsubsectionfont{\rmfamily\bfseries\upshape\normalsize}

% Set figure legends and captions to be smaller sized sans serif font
\usepackage[font={footnotesize,sf}]{caption}

\usepackage{siunitx}

% Adjust spacing between lines to 1.5
\usepackage{setspace}
% \onehalfspacing
\doublespacing
\raggedbottom

% Set margins
\usepackage[top=1.5in,bottom=1.5in,left=1.5in,right=1.4in]{geometry}
\setlength\parindent{0.5in} % indent at start of paragraphs (set to 0.3?)
\setlength{\parskip}{9pt}
\usepackage{indentfirst}

% Add space between pararaphs
% http://texblog.org/2012/11/07/correctly-typesetting-paragraphs-in-latex/
% \usepackage{parskip}
% \setlength{\parskip}{\baselineskip}

% Set colour of links to black so that they don't show up when printed
\usepackage{hyperref}
\hypersetup{colorlinks=false, linkcolor=black}

% Tables
\usepackage{booktabs}
\usepackage{threeparttable}
\usepackage{array}
\usepackage{makecell}
\newcolumntype{x}[1]{%
>{\centering\arraybackslash}m{#1}}%

% Allow for long captions and float captions on opposite page of figures
% \usepackage[rightFloats, CaptionBefore]{fltpage}

% Don't let floats cross subsections
% \usepackage[section,subsection]{extraplaceins}

% Rotate images and tables
\usepackage{float}
\usepackage{pdfpages}
\usepackage{pdflscape}
\usepackage{graphicx}
\usepackage{rotating}

% Custom math
\usepackage{bbold}
\DeclareMathOperator*{\argmin}{\arg\!\min}

% For use of \cref and \Cref used by pandoc secnos
\usepackage{cleveref}

\begin{document}


    \begin{titlepage}
        
        % \noindent
        % \begin{minipage}[t]{0.19\textwidth}{
        %     \vspace{-4mm}
        %     {\includegraphics[scale=1.15]{style/univ_logo.pdf}}
        %   }
        % \end{minipage}
        % \hspace{0.5cm}
        % \begin{minipage}[t]{1.81\textwidth}
        % {
        %   \setstretch{1.42}
        %   {\textsc{Università degli Studi di Milano - Bicocca}} \\
        %   \textbf{} \\
        %   \textbf{Dipartimento di Giurisprudenza} \\
        %   \textbf{Corso di Laurea in Scienze dei servizi giuridici} \\
        %   \par
        % }
        % \end{minipage}
        
	% \vspace{20mm}

        
        \noindent

	\vspace{1mm}
	\begin{center}
            {
              \setstretch{1.42}
              {\textsc{Università degli Studi di Milano - Bicocca}} \\
              \textbf{} \\
              \textbf{Dipartimento di Giurisprudenza} \\
              \textbf{Corso di Laurea in Scienze dei servizi
giuridici} \\
              \par
            }
        \end{center}
        
	\vspace{4mm}
        
	\begin{center}
          \begin{minipage}[t]{0.19\textwidth}{
              \vspace{-4mm}
              {\includegraphics[scale=1.5]{style/univ_logo.pdf}}
            }
          \end{minipage}
        \end{center}

        \vspace{10mm}
        
      	\begin{center}
            {\LARGE{
                    \setstretch{1.2}
                    \textbf{I rimedi all'inottemperanza della Pubblica
Amministrazione in alcuni ordinamenti europei}
                    \par
            }}
        \end{center}

        
        \vspace{8mm}

        \noindent
        {\large \textbf{Relatore:}  Prof.~Alessandro Squazzoni } \\

        %         % \noindent
        % {\large \textbf{Correlatore:} } \\
        %         
        \vspace{5mm}

        \begin{flushright}
            {\large \textbf{Relazione della prova finale di:}} \\
            \large{Marina Chiapello} \\
            \large{Matricola 783939} 
        \end{flushright}
        
        \vspace{5mm}
        \begin{center}
            {\large{\bf Anno Accademico 2021-2022}}
        \end{center}

        \restoregeometry
        
    \end{titlepage}




% This is where the converted markdown files will go 
\vspace*{\fill}

\noindent \textit{
Foreword ...
} \vspace*{\fill} \pagenumbering{gobble} \newpage

\hypertarget{abstract}{%
\chapter*{Abstract}\label{abstract}}
\addcontentsline{toc}{chapter}{Abstract}

Lorem ipsum dolor sit amet, consectetur adipiscing elit. Vestibulum ante
ipsum primis in faucibus orci luctus et ultrices posuere cubilia Curae;

Donec at urna imperdiet, vulputate orci eu, sollicitudin leo. Donec nec
dui sagittis, malesuada erat eget, vulputate tellus.

Nam ullamcorper efficitur iaculis. Mauris eu vehicula nibh. In lectus
turpis, tempor at felis a, egestas fermentum massa.

\pagenumbering{roman}
\setcounter{page}{1}

\hypertarget{ringraziamenti}{%
\chapter*{Ringraziamenti}\label{ringraziamenti}}
\addcontentsline{toc}{chapter}{Ringraziamenti}

Interdum et malesuada fames ac ante ipsum primis in faucibus. Aliquam
congue fermentum ante, semper porta nisl consectetur ut. Duis ornare sit
amet dui ac faucibus. Phasellus ullamcorper leo vitae arcu ultricies
cursus. Duis tristique lacus eget metus bibendum, at dapibus ante
malesuada. In dictum nulla nec porta varius.

Donec pharetra arcu vitae euismod tincidunt.

\newpage

\pagenumbering{gobble}

\tableofcontents

\newpage

\hypertarget{abbreviazioni}{%
\chapter*{Abbreviazioni}\label{abbreviazioni}}
\addcontentsline{toc}{chapter}{Abbreviazioni}

\begin{tabbing}
\hspace{12em} \= \hspace{60em} \= \kill
\textbf{c.p.a.} \> \textbf{c}odice della \textbf{p}ubblica \textbf{a}mministrazione \\
\textbf{P. A.} \> \textbf{P}ubblica \textbf{A}mministrazione \\
\end{tabbing}

\newpage

\setcounter{page}{1}
\pagenumbering{arabic}
\doublespacing
\setlength{\parindent}{0.5in}

\hypertarget{la-giustizia-amministrativa-in-italia}{%
\chapter{La giustizia amministrativa in
Italia}\label{la-giustizia-amministrativa-in-italia}}

\hypertarget{lattuazione-del-giudicato-il-giudizio-di-ottemperanza}{%
\section{L'attuazione del giudicato: il giudizio di
ottemperanza}\label{lattuazione-del-giudicato-il-giudizio-di-ottemperanza}}

Il giudizio di ottemperanza rappresenta uno strumento di particolare
incisività per garantire nei confronti dell'amministrazione l'attuazione
delle decisioni giudiziali, come stabilito all'art. 112 c.p.a. e in
risposta ai principi di di effettività ed efficacia della tutela
giurisdizionale sanciti dagli artt. 24 e 113 Cost.\footnote{Termine
  francese, utilizzato sia in Francia che in Italia, che significa
  \emph{corrogare}, nel senso di chiedere una giornata di lavoro. Nel
  Medioevo, la \emph{corvée} indicava un'imposta, poi abolita durante la
  rivoluzione, che veniva richiesta dal signore ai suoi servi, da
  estinguere con un certo numero di giornate di lavoro - Cfr.
  \emph{Portal Français, 100 mots du trésor FR - la corvée} di Agnes
  Dijaux. Nella Francia dell'\emph{Ancien régime}, il termine indicava
  le giornate di lavoro che i sudditi dovevano al re per la manutenzione
  delle strade pubbliche - Cfr. \emph{Treccani.it}, Enciclopedia,
  dizionario di storia (2010).}, nonché dall'art. 47 della Carta dei
diritti fondamentali dell'UE e dall'art. 13 della CEDU. In base alla
legge di abolizione del contenzioso amministrativo del 1865, dell'atto
amministrativo lesivo di un diritto si poteva chiedere la modifica o
l'annullamento esclusivamente con ricorso gerarchico all'autorità
amministrativa competente e l'amministrazione che aveva emesso l'atto
aveva semplicemente l'obbligo di conformarsi al giudicato del tribunale
civile, ma tale obbligo rimaneva incoercibile, in quanto non
accompagnato da un meccanismo volto a garantirne l'effettiva
osservanza\footnote{Cfr. il decreto 16 agosto 1790 sull'ordinamento
  giudiziario, in ALDO TRAVI, \emph{Lezioni di giustizia
  amministrativa}, Torino, 2019, 8, 9: ``Le funzioni giurisdizionali
  sono distinte e rimangono sempre separate dalle funzioni
  amministrative. I giudici non potranno, sotto pena di prevaricazione,
  interferire in qualunque modo sulle operazioni dei corpi
  amministrativi, né citare avanti a sé gli amministratori a motivo
  dell'esercizio delle loro funzioni''.}.

In origine il giudizio di ottemperanza, così come introdotto dall'art. 4
n.~4 della legge 31 marzo 1889, n.~5992, era ammesso solo per le
sentenze passate in giudicato dell'Autorità giudiziaria ordinaria,
aventi per oggetto diritti civili e politici. E' a partire dagli anni
venti del secolo scorso che la giurisprudenza del Consiglio di Stato
estende analogicamente l'applicabilità dell'istituto anche
all'esecuzione del giudicato amministrativo, ma esso trova un
riconoscimento normativo solo con l'art. 37 della legge 6 dicembre 1971,
n.~1034, istitutiva dei tribunali amministrativi regionali. Infine,
viene compiutamente disciplinato con il decreto legislativo 2 luglio
2010, n.~104, in attuazione della legge delega 18 giugno 2009, n.~69,
per il riordino del processo amministrativo. Presupposto per
l'attivazione del giudizio di ottemperanza è l'inosservanza da parte
dell'amministrazione del dovere di esecuzione della sentenza e l'oggetto
del giudizio è costituito dalla verifica se l'amministrazione abbia o
meno adempiuto l'obbligo nascente dal giudicato, ovvero se abbia o meno
attribuito all'interessato quell'utilità che la sentenza ha riconosciuto
come dovuta. Mentre nella fase esecutiva della sentenza di condanna del
giudice civile che ha per oggetto diritti soggettivi e stabilisce cosa
deve fare l'amministrazione soccombente nello specifico ci si trova di
fronte ad una sentenza molto chiara nello stabilire cosa si pretende dal
``debitore'', nel caso della sentenza del giudice amministrativo la
condotta successiva non è sempre segnata con certezza: il vincolo
conformativo ha un'intensità diversa a seconda del vizio accolto e
l'amministrazione può non essere tenuta solo ad un comportamento
specifico. Il giudizio di ottemperanza non è la mera attuazione di un
giudicato già preciso e sicuro della fase di cognizione, ma deve
ricostruirne il significato. E' un giudizio c.d. ``misto'',
necessariamente di esecuzione ed eventualmente di cognizione,
assoggettato al termine di prescrizione ordinario di dieci anni,
decorrente dalla data del passaggio in giudicato della
sentenza\footnote{ALDO TRAVI, \emph{Lezioni di giustizia
  amministrativa}, Torino, 2019, 9.}. La fase di cognizione non è
necessaria quando l'attività amministrativa successiva al giudicato
abbia carattere vincolato, ovvero quando le statuizioni della sentenza
impartiscano all'amministrazione comandi tassativi e talmente puntuali
da non lasciare spazio alcuno all'esercizio dei suoi poteri
discrezionali. Per converso, gli spazi liberi che possono residuare al
giudicato rendono la \emph{regola iuris} dallo stesso dettata
``implicita, elastica, condizionata ed incompleta'' e, come tale,
suscettibile di essere chiarita nel contesto del giudizio di
ottemperanza\footnote{M. HAURIOU, Précis de droit administratif, Paris,
  1933, 13 ss.} . Sempre riguardo alla natura del rito ed alla
compenetrazione di momenti cognitivi con momenti esecutivi, la Corte
costituzionale ha chiarito che \emph{``il giudizio di ottemperanza
assume diversi modi di essere in relazione alla situazione concreta,
alla statuizione giudiziale da attuare, alla natura dell'atto censurato.
Il particolare il giudizio di ottemperanza può costituire semplice
giudizio esecutivo che si aggiunge al procedimento espropriativo,
disciplinato dal codice di procedura civile; lo stesso giudizio può
essere preordinato al compimento di operazioni materiali o (\ldots) alla
sollecitazione di attività provvedimentale amministrativa (\ldots) può
essere utilizzato anche in difetto di completa individuazione del
contenuto della prestazione o attività oggetto del dovere
dell'Amministrazione (\ldots) non deve modellarsi necessariamente anche
nei presupposti sul processo esecutivo ordinario, tenuto conto delle
peculiarità funzionali del giudizio amministrativo, con potenzialità
sostitutive e intromissive nell'azione amministrativa incomparabili ai
poteri del giudice dell'esecuzione del processo civile''\footnote{ALDO
  TRAVI, \emph{Lezioni di giustizia amministrativa}, Torino, 2019, 11.}.}
Il ricorso per l'ottemperanza va proposto nelle forme ordinarie, quindi
notificato all'amministrazione e a tutte le altre parti del giudizio di
merito. Il ricorrente deve depositare una copia autentica della sentenza
di cui si chiede l'esecuzione, con l'eventuale prova del passaggio in
giudicato\footnote{Cfr.
  https://www.vie-publique.fr/fiches/20284-justice-administrative-origines-role-et-specificites}.
In passato il ricorso doveva essere preceduto dalla notifica
all'amministrazione di una diffida a provvedere, ma oggi il codice,
all'art. 114, c.~1, stabilisce che tale adempimento non è più
necessario. Il riparto di competenza ha carattere funzionale, ai sensi
dell'art. 14, c.~3, c.p.a. Per l'esecuzione della sentenza
amministrativa, competente è il giudice che ha pronunciato la sentenza.
Nel caso si tratti di sentenza emessa dal Consiglio di Stato, esso può
essere competente in unico grado, ma se la sentenza del Tar è stata
confermata in appello, la competenza spetta sempre al Tar. Qualora
invece si tratti dell'esecuzione della sentenza di un giudice ordinario
o di un altro giudice speciale diverso dal giudice amministrativo, la
competenza spetta sempre al Tar nella cui circoscrizione ha sede il
giudice che ha emesso la sentenza da eseguire\footnote{Tre sono i gradi
  della giurisdizione amministrativa francese: i tribunali
  amministrativi (\emph{tribunaux administratifs}), le corti
  amministrative d'Appello (\emph{courts administratives d'Appel}) e il
  Consiglio di Stato, Cfr.
  https://www.justice.gouv.fr/organisation-de-la-justice-10031/lordre-administratif-10034/}.

Per quanto riguarda l'esecuzione delle sentenze del giudice
amministrativo, il ricorso per l'ottemperanza è esperibile
indipendentemente dal fatto che esse siano passate in giudicato o
solamente esecutive e, ai fini del ricorso, non rileva se rispetto a
queste sentenze inadempiente sia l'amministrazione o una parte privata.
Nel caso di una sentenza non ancora passata in giudicato, l'esecuzione
riguarda una statuizione che non ha ancora carattere di definitività.
Con la sentenza n.~5352/2002 il Consiglio di Stato ha sostenuto che
l'esecuzione della sentenza non ancora passata in giudicato non dovrebbe
mai determinare un assetto \emph{``definito ed immutabile''}, perché
altrimenti verrebbe frustrato l'esito pratico di un eventuale appello
contro la sentenza\footnote{Cfr. \emph{``juridictions placées au sommet
  de chacun des deux ordres de juridiction reconnus par la
  Constitution''}, \emph{Décision du Conseil constitutionnel n° 2009-595
  DC~du 3 décembre 2009}, § 3, in
  https://www.conseil-constitutionnel.fr/actualites/communique/decision-n-2009-594-dc-du-3-decembre-2009-communique-de-presse\#:\textasciitilde:text=Le\%203\%20d\%C3\%A9cembre\%202009\%2C\%20par,portant\%20diverses\%20dispositions\%20relatives\%20aux}.
In generale, la stessa giurisprudenza che ha orientato anche la
redazione del codice del processo amministrativo equipara la sentenza
esecutiva alla sentenza passata in giudicato ai fini dell'ammissibilità
del giudizio di ottemperanza, ma precisa che il giudice
dell'ottemperanza, se la sentenza non sia passata in giudicato, ne
determina le modalità esecutive\footnote{Cfr. art. L. 8-3 del \emph{code
  des tribunaux administratifs et des cours administratives d'appel} in
  https://www.senat.fr/rap/l98-380/l98-3805.html}, motivo per cui sembra
riconosciuta la necessità che l'esecuzione di tale sentenza non
pregiudichi le ragioni di un eventuale appello. In base all'art. 114,
c.~2, lett. \emph{c} ed \emph{e}, il ricorso per l'ottemperanza è
esperibile anche per l'esecuzione delle sentenze passate in giudicato
del giudice ordinario e dei giudici speciali avanti ai quali non sia
previsto un giudizio di ottemperanza, nonché per l'esecuzione dei lodi
arbitrali esecutivi divenuti inoppugnabili. In questi casi però il
giudizio di ottemperanza si caratterizza sul piano soggettivo come
strumento di esecuzione specifica nei confronti di un'amministrazione,
in quanto non è ammesso per soggetti diversi.

L'elemento decisamente caratteristico del giudizio di ottemperanza è
individuato dall'art. 134, c.~1, lett. \emph{a}, c.p.a., laddove si
prevede che il giudice amministrativo, nello stesso giudizio, esercita
una giurisdizione estesa al merito. Tale previsione comporta che il
giudice amministrativo possa sostituirsi, direttamente o attraverso un
commissario da esso eventualmente nominato, all'amministrazione
inadempiente. Questa possibilità di sostituzione comporta che nel
giudizio di ottemperanza non possa opporsi al giudice alcuna riserva di
potere all'amministrazione, in quanto la necessità di dare esecuzione
alla sentenza prevale anche su ogni esigenza di salvaguardia delle
prerogative dell'amministrazione stessa. Inoltre, la giurisprudenza
largamente prevalente ammette che il giudice dell'ottemperanza possa
compiere anche attività discrezionali, disattendendo l'assunto secondo
cui il medesimo giudice potrebbe sostituirsi all'amministrazione solo
nei limiti delle statuizioni puntali del giudicato, in quanto le
ulteriori scelte discrezionali dell'amministrazione non sarebbero di
pertinenza dell'autorità giurisdizionale. L'attività del giudice
dell'ottemperanza o del commissario \emph{ad acta} da lui nominato
infatti non costituisce manifestazione in senso stretto di
discrezionalità amministrativa, poiché essa è essenzialmente preordinata
al conseguimento dell'interesse del ricorrente e non già all'interesse
primario perseguito dall'amministrazione. Vi sono due ipotesi in cui
l'amministrazione viola il giudicato del giudice amministrativo: una si
verifica quando la sentenza stabilisce che essa non deve adottare un
provvedimento e la seconda quando l'amministrazione è inadempiente,
quindi rispetto ad una condotta omissiva, con un'inerzia elusiva del
giudicato. Con l'art. 21 \emph{septies} della legge 241/1990, introdotto
dalla legge 15 del 2005 di riforma del procedimento amministrativo, gli
atti elusivi sono stati assimilati a quelli assunti in violazione del
giudicato, ammettendosi anche nei loro confronti il ricorso per
l'ottemperanza\footnote{Cfr. art. L. 8-4 del \emph{code des tribunaux
  administratifs et des cours administratives d'appel} in
  https://www.senat.fr/rap/l98-380/l98-3805.html}.

L'ampia gamma di poteri spendibili dal giudice dell'ottemperanza ammanta
lo stesso istituto di originalità, laddove nella maggior parte delle
principali esperienze continentali domina, quale strumento a presidio
dell'esecuzione del giudicato da parte dell'amministrazione, il rimedio
delle misure patrimoniali di tipo compulsorio, quali lo
\emph{Zwangsgeld} o l'\emph{astrainte}, dove i sistemi sono improntati,
in punto di esecuzione della sentenza, ad una rigida separazione tra i
poteri dell'amministrazione e quelli della giurisdizione, essendo
inibita al giudice qualsiasi ingerenza nell'attività esecutiva del
giudicato amministrativo che rimane appannaggio dell'amministrazione.
Un'eccezione è rappresentata dal modello austriaco della
\emph{Säumnisbeschwerde} quale rimedio avverso il silenzio in
inadempimento dell'amministrazione, prevedendo il legislatore austriaco
al \emph{§ 63/2 VwGG} la possibilità per il giudice amministrativo di
surrogarsi all'amministrazione inadempiente designando l'amministrazione
o il tribunale chiamato ad eseguire la sua decisione e
\emph{``consacrando così una possibile sostituzione del potere
giudiziario all'amministrazione attiva (\ldots) sul fronte
dell'esecuzione''}\footnote{Cfr. Dupre de Boulois~X, \emph{«~Le
  référé-liberté pour autrui~»}, AJDA~2013, p.~2137} .

\hypertarget{i-poteri-sostitutivi-indiretti-il-commissario-ad-acta}{%
\section{\texorpdfstring{I poteri sostitutivi indiretti: il commissario
\emph{ad
acta}}{I poteri sostitutivi indiretti: il commissario ad acta}}\label{i-poteri-sostitutivi-indiretti-il-commissario-ad-acta}}

Il giudice può adottare direttamente i provvedimenti necessari ad
un'integrale esecuzione del giudicato quando essi siano vincolati,
altrimenti si deve limitare a dichiarare l'obbligo di provvedere
assegnando all'amministrazione un termine, nonché disponendo che si
nomini un commissario il quale agisca al posto dell'amministrazione, se
questa non ottemperi entro il termine assegnato. Il commissario \emph{ad
acta} è chiamato ad esercitare quei poteri che il giudice
dell'ottemperanza potrebbe esercitare anche in via diretta, attraverso
un intervento nel merito volto a sostituire l'amministrazione e
finalizzato a rendere effettiva la tutela sostanziale dell'interesse
protetto. Di regola il giudice assegna all'amministrazione un termine e
contestualmente designa un'autorità amministrativa che alla scadenza del
termine assegnato si sostituirà all'amministrazione inadempiente ed
emanerà il provvedimento o terrà il comportamento necessario per
l'attuazione del giudicato. In sede di ottemperanza al giudicato, il
giudice amministrativo, direttamente o per mezzo del commissario da lui
nominato, può emanare provvedimenti di vario tipo, costitutivi,
certificatori, declaratori di obblighi a carico dell'amministrazione e
tutti quegli adempimenti strumentalmente necessari per l'esecuzione
della sentenza. In pratica, si sostituisce all'amministrazione
inadempiente ponendo in essere l'attività che questa avrebbe dovuto
compiere per realizzare concretamente gli effetti scaturenti dalla
sentenza da eseguire, conformando la realtà alle sue statuizioni. Poiché
la discrezionalità amministrativa implica sovente decisioni di matrice
politica, la nomina di un commissario \emph{ad acta} viene ritenuta
preferibile rispetto all'adozione diretta da parte del giudice delle
misure di competenza dell'amministrazione riottosa. Di regola, egli è
scelto fra funzionari di altre amministrazioni e, spesso nella persona
del Prefetto, rappresenta con la sua attività \emph{``il punto di sutura
e saldatura''} tra attività giurisdizionale ed
amministrativa\footnote{Cfr.
  https://www.actu-juridique.fr/administratif/le-pouvoir-dinjonction-du-juge-administratif-revisite-par-les-circonstances-exceptionnelles-de-la-crise-sanitaire-du-covid-19/}.
In particolare, \emph{``in quanto delegato dal giudice amministrativo,
ha il potere di emanare i necessari provvedimenti amministrativi anche
in deroga alle vigenti competenze. Allo stesso è altresì demandato
l'onere di porre in essere ogni attività idonea a dare esecuzione alla
decisione''}\footnote{Cfr. \emph{Qu'est-ce que l'exécution des
  décisions?} in \emph{L'exécution des décisions du Juge administratif}
  http://paris.cour-administrative-appel.fr/Demarches-procedures/Les-fiches-pratiques-de-la-justice-administrative}.
Una ormai risalente pronuncia della Corte costituzionale configura il
commissario \emph{ad acta} come ausiliario del giudice e riconduce i
suoi atti all'esercizio della giurisdizione esecutiva del giudice
dell'ottemperanza\footnote{Tradotto letteralmente come procedura di
  costrizione al pagamento, anche chiamata procedura di pagamento
  forzato.}. Autorevole dottrina ha sostanzialmente qualificato
l'attività commissariale come \emph{``proiezione nel mondo esterno di un
comando del giudice e, quindi, della traduzione nel concreto della
attribuzione della} potestas decidendi \emph{che non sempre ha o può
avere contenuti rigidamente predeterminati}, tali da consentire al
giudice di portarli direttamente ad attuazione\footnote{Cfr.
  \emph{Comment faire exécuter les décisions rendues par le juge
  administratif?} in \emph{L'exécution des décisions du juge
  administratif}
  http://paris.cour-administrative-appel.fr/Demarches-procedures/Les-fiches-pratiques-de-la-justice-administrative}.
L'ampiezza dei poteri commissariali dipenderà dal contenuto del
giudicato inadempiuto: essi potranno estrinsecarsi, a seconda delle
situazioni dedotte in giudizio, in attività sia vincolata, come ad
esempio la restituzione di beni illegittimamente espropriati, sia
discrezionale, quindi comportante un potere di scelta\footnote{Cfr.
  \emph{Comment se déroule l'examen de ma demande d'exécution?} in
  \emph{L'exécution des décisions du juge administratif}
  http://paris.cour-administrative-appel.fr/Demarches-procedures/Les-fiches-pratiques-de-la-justice-administrative}.
Una volta nominato il commissario, il giudice mantiene comunque un
incisivo potere di vigilanza sul suo operato, nonché il potere di
risolvere eventuali contestazioni, dal momento che le determinazioni del
commissario, laddove esorbitanti dalle specifiche indicazioni del
giudice, possono essere oggetto di un ricorso dinanzi allo stesso
giudice, esperibile anche dall'amministrazione sostituita\footnote{Cfr.
  \emph{Que faire lorsque l'Administration n'exécute pas le jugement
  d'un tribunal administratif ou l'arrêt d'une cour d'appel?} in
  http://paris.tribunal-administratif.fr/Demarches-procedures/L-execution-des-decisions-du-juge-administratif}.
Da tempo, sia in dottrina che in giurisprudenza, si dibatte sulla
questione riguardante la misura del potere di adempiere che
conserverebbe l'amministrazione, una volta che sia stato nominato il
commissario o sia scaduto il nuovo termine imposto alla stessa
amministrazione. La giurisprudenza ritiene, per lo più, che
l'amministrazione verrebbe privata del suo potere nel momento in cui
viene assunta direttamente dal giudice la decisione contenente il
provvedimento concreto reso in ottemperanza al giudicato, ovvero in
quello in cui viene nominato il commissario\footnote{Osserva D. DE
  PRETIS, \emph{Il processo amministrativo in Europa}, cit., 84, che il
  problema dell'esecuzione della sentenza amministrativa risiede,
  nell'ordinamento francese così come in altri ordinamenti continentali,
  nella difficile coesistenza tra la riserva amministrativa del potere
  di rinnovare l'attività provvedimentale a seguito della decisione del
  giudice e la garanzia dell'effettività della tutela giurisdizionale e
  quindi dell'esecuzione delle statuizioni giudiziali.}. Ove invece
venga fissato all'amministrazione un termine per adempiere, questo
assume carattere perentorio, risultando evidente nel caso in cui la
prefissione del termine sia accompagnata dalla nomina del commissario,
poiché, una volta scaduto il termine, il potere provvedimentale si
trasferirebbe in automatico a tale soggetto, ancorché il contenuto degli
atti, eventualmente adottati dall'amministrazione dopo tale scadenza e
sadisfattivi del giudicato, potrebbe essere confermato dal giudice
dell'ottemperanza, il quale in tal modo avvalorerebbe una legittimazione
a provvedere tardivamente in capo all'amministrazione\footnote{La prima
  applicazione \emph{ex officio} di un'\emph{astreinte} si deve a CE 28
  maggio 2001 (\emph{Bandesapt}), in \emph{Rec. Lebon 251}, nonché in
  \emph{DA} 2001, n.~176, con la quale lo Stato veniva condannato al
  pagamento di una sanzione pecuniaria di 10.000 franchi al giorno, per
  la mancata ottemperanza da parte del governo francese all'obbligo di
  emanare un regolamento come risultante da un decreto del 30 dicembre
  1998.}.

Molti dei casi di inosservanza delle sentenze del giudice amministrativo
non sono dettati da una volontà deliberata di disconoscere l'autorità
della cosa giudicata, bensì alle oggettive difficoltà che
l'amministrazione incontra nell'eseguire le sentenze, soprattutto
qualora gli obblighi in esse enunciati appaiano indeterminati, vaghi o
imprecisi e intercorra più tempo fra l'emissione del provvedimento
impugnato e il suo definitivo annullamento. Il c.~5 dell'art. 112 c.p.a.
e il c.~7 dell'art. 114 ammettono il ricorso al giudice
dell'ottemperanza anche soltanto per \emph{``ottenere chiarimenti''} in
merito alle modalità di esecuzione che non presuppone un'inottemperanza,
ma semplicemente un'incertezza sull'interpretazione o sugli effetti
della sentenza da eseguire. In questi casi quindi il ricorso può essere
proposto anche dall'amministrazione tenuta a darvi esecuzione, quando
abbia esigenza di chiarimenti. Per evitare abusi, la giurisprudenza ha
comunque affermato che tale rimedio non deve rappresentare un espediente
per mettere in discussione la sentenza da eseguire o per introdurre
questioni estranee all'ottemperanza\footnote{Come da definizione data da
  FRANCESCO M. CARALLI, nell'articolo \emph{Il nuovo giudizio di
  ottemperanza, con particolare riguardo alle astreintes}, in
  Italiappalti.it, 2017, 4.}.

\hypertarget{il-risarcimento-del-danno-da-inottemperanza}{%
\section{Il risarcimento del danno da
inottemperanza}\label{il-risarcimento-del-danno-da-inottemperanza}}

Una volta ottenuta soddisfazione attraverso il giudizio promosso ai
sensi del c.~2 dell'art. 112 c.p.a., potrebbe ancora residuare al
ricorrente vittorioso un danno connesso alla tardiva realizzazione di
quell'assetto che sarebbe dovuto scaturire dall'annullamento del
provvedimento illegittimo dell'amministrazione, ma che è venuto in
essere solo a seguito di un notevole lasso di tempo oppure che ormai non
risulta più attuabile, per cui il giudizio di ottemperanza, di per sè,
non sarebbe in grado di garantire al ricorrente una tutela piena ed
effettiva. In quest'ultimo caso, lo strumento dell'ottemperanza si
rivelerebbe inutile, se non vi fosse la possibilità di ottenere
contestualmente un risarcimento per equivalente a seguito della perdita
definitiva del bene spettante dovuta alla inesecuzione del giudicato. Si
pensi al caso del definitivo annullamento di un decreto di esproprio cui
non sia seguita la spontanea restituzione dell'immobile al proprietario,
per cui si è reso necessario instaurare il giudizio di ottemperanza. Ove
l'amministrazione opponesse, in questa sede, una legittima
sopravvenienza impediente l'esecuzione del giudicato, al ricorrente
dovrebbe essere riconosciuto, in funzione surrogatoria, anche il danno
c.d. petitorio, consistente nel controvalore del bene, derivante appunto
dalla perdita definitiva dello stesso, cagionata dall'illecito ritardo
nella conformazione al giudicato \footnote{FRANCESCO M. CARALLI,
  \emph{Il nuovo giudizio di ottemperanza, con particolare riguardo alle
  astreintes}, in Italiappalti.it, 2017, 11.}. Da questa situazione, va
tenuta distinta quella in cui, già al momento della pronuncia di
annullamento, risulta chiaramente che non è più utile per il ricorrente
la rinnovazione del potere conformemente alla regola concreta dedotta in
sentenza, potendo il giudice amministrativo in tal caso accogliere
immediatamente la domanda di risarcimento del danno per equivalente. In
molti altri casi, invece, il giudice della cognizione non è in grado di
prevedere già all'atto dell'annullamento se ed in quale misura
l'ottemperanza potrà effettivamente ripristinare la situazione
soggettiva lesa. In particolare, in tutti quei casi in cui la domanda
del privato è diretta a conseguire il bene della vita, molto spesso la
possibilità e i limiti entro cui attribuire il bene dipendono dal
momento in cui l'amministrazione esegue il giudicato. Ad esempio, in
materia di appalti, se l'annullamento dell'aggiudicazione in sede
giurisdizionale interviene nell'immediatezza dei fatti, consente al
ricorrente di stipulare il contratto con l'amministrazione; al
contrario, se interviene quando il contratto con l'originale
aggiudicatario è già stato non solo stipulato, ma anche parzialmente
eseguito, l'esecuzione della pronuncia e quindi l'attribuzione del bene
della vita, cioè l'appalto, è possibile solo parzialmente per la parte
residua non eseguita, mentre per la prima parte la tutela può avvenire
solo attraverso il risarcimento, sempre per equivalente. Spesso, quindi,
solo all'esito dell'ottemperanza di un giudicato di annullamento è
possibile accertare e quantificare il danno risarcibile per equivalente.
Laddove non risulta più satisfattiva la pronuncia di annullamento,
supplisce la tutela risarcitoria e il momento in cui emerge con
chiarezza lo spazio per l'esecuzione del giudicato e per il risarcimento
del danno è proprio quello dell'ottemperanza.

Nell'ipotesi di annullamento di un provvedimento ampliativo della sfera
giuridica del privato, occorre distinguere il caso in cui
l'amministrazione, in esecuzione spontanea del giudicato di
annullamento, renda il provvedimento precedentemente negato, dal caso in
cui il bene della vita agognato dal ricorrente venga conseguito soltanto
in esito al giudizio di ottemperanza. Mentre nella prima situazione la
pretesa risarcitoria azionabile riguarderà esclusivamente un danno da
attività provvedimentale illegittima, non avendo luogo una violazione
del giudicato, in quanto l'amministrazione accorda l'utilità prima
negata, a seguito della rinnovazione del potere discrezionale successivo
al giudicato di annullamento\footnote{Cfr. Art. 12, decreto n.~831 del 3
  luglio 1995 - \emph{CJA}, art. R 921-5.}, nella seconda la pretesa
risarcitoria sarà duplice e riguarderà un danno scomponibile in una
prima voce, relativa al ritardo antecedente alla formazione del
giudicato e commisurato al pregiudizio patito dal ricorrente, qualora
l'amministrazione si fosse spontaneamente conformata al giudicato, e una
seconda voce di danno, propriamente da inadempimento dell'obbligo
conformativo scaturente dalla pronuncia del giudice amministrativo,
volta a coprire il segmento temporale intercorrente fra il giudicato e
la sua concreta attuazione. In entrambe le ipotesi, il danno c.d. da
ritardo potrà essere compiutamente apprezzato soltanto a posteriori,
ovvero una volta che il privato abbia effettivamente ottenuto il bene
della vita cui aspirava con l'istanza a suo tempo illegittimamente
rigettata dall'amministrazione, a meno che non si tratti di potere
amministrativo vincolato, per cui la spettanza del bene si cristallizza
già in esito al giudizio di cognizione\footnote{Cfr. Art. 59, decreto
  n.~766 del 30 luglio 1963 e succ. mod. - \emph{CJA}, art. R931-1
  \emph{et} 2.}. In giurisprudenza ricorre il principio secondo cui,
essendo l'oggetto del giudizio di ottemperanza costituito dalla verifica
se l'amministrazione abbia o meno adempiuto all'obbligo nascente dal
giudicato, ovvero abbia o meno attribuito all'interessato quell'utilità
concreta che la sentenza ha riconosciuto come dovuta, a prescindere dal
fatto che residuino o meno in capo all'amministrazione stessa poteri
discrezionali, l'esecuzione deve essere esatta, al pari di quanto
avviene nell'obbligazione civile, il cui inesatto adempimento viene
sanzionato con la condanna al risarcimento del danno\footnote{R. CHAPUS,
  \emph{Droit du contentieux administratif}, 12 ed., Paris, 2006, 1132.}.
L'utilità concreta potrà consistere \emph{'' nel diritto alla restitutio
in integrum sotto forma di pretesa alla restituzione del bene in caso di
annullamento di provvedimenti ablatori, sotto forma di annullamento del
contratto stipulato in seguito ad aggiudicazione illegittima, nel caso
di provvedimento incidente su interessi legittimi pretensivi; può
consistere nel diritto alla conformazione alla regola contenuta nel
giudicato in caso di riedizione dell'atto che va dal diritto alla non
riedizione o all'ottenimento dell'atto in caso di effetto vincolante
pieno, al diritto alla riedizione nel rispetto delle regole sostanziali
e formali in caso di effetto vincolante semipieno o
strumentale}\footnote{R. CHAPUS, \emph{Droit du contentieux
  adminisratif}, 12 ed., Paris, 2006, 1135.}. Sul piano
dell'accertamento e della prova, se nel giudizio avente ad oggetto il
pregiudizio conseguente al provvedimento amministrativo illegittimo il
privato deve provare tutti gli elementi costitutivi del fatto illecito,
in quello avente ad oggetto il danno da violazione del giudicato opera,
invece, il principio dell'inversione dell'onere della prova di cui
all'art. 1218 c.c. nella misura in cui viene posta a carico del debitore
la prova che l'inadempimento è stato determinato da impossibilità della
prestazione derivante da causa non imputabile. Ne consegue che
l'interessato deve dimostrare esclusivamente il suo diritto e la
sussistenza di un giudicato di accoglimento, mentre spetterà
all'amministrazione la prova di avervi ottemperato.

\hypertarget{lesperienza-tedesca}{%
\chapter{L'esperienza tedesca}\label{lesperienza-tedesca}}

\hypertarget{lazione-di-annullamento-anfechtungsklage-e-lazione-di-adempimento-verplichtungsklage}{%
\section{L'azione di annullamento (Anfechtungsklage) e l'azione di
adempimento
(Verplichtungsklage)}\label{lazione-di-annullamento-anfechtungsklage-e-lazione-di-adempimento-verplichtungsklage}}

Il legislatore tedesco ha affrontato il problema dell'esecuzione della
sentenza amministrativa avente ad oggetto un provvedimento
amministrativo, prima ancora che attraverso la predisposizione di un
meccanismo di coazione in presenza di un inadempimento
dell'amministrazione, con la previsione di un articolato sistema di
misure che consentono di prevenire la mancata spontanea esecuzione delle
pronunce del giudice e cercando di definire già a livello normativo
contenuto ed effetti che debbono assumere le decisioni giurisdizionali
in presenza di determinati presupposti. In questa prospettiva, è
importante in primo luogo che gli obblighi dell'amministrazione
derivanti dalla decisione del giudice amministrativo siano facili da
assolvere e perfettamente determinati\footnote{J. VIGUIER, \emph{Le
  contentieux administratif}, Paris, 1998, 43.}. Con la legge del 21
gennaio 1960 \emph{(VwGO - Verwaltungsgerichtsordnung)} sull'ordinamento
processuale amministrativo sono state introdotte due distinte azioni:
una di impugnazione in senso stretto o di annullamento
\emph{(Anfechtungsklage)} ed un'altra di condanna all'emissione di un
dato provvedimento, altrimenti detta ``di adempimento''
\emph{(Verplichtungsklage)}. Ove l'autorità emetta un provvedimento
incidente negativamente nella sfera giuridica del destinatario, questi
ricorrerrà alla \emph{Anfechtungsklage} facendone valere eventuali vizi;
ove invece il privato aspiri ad ottenere un provvedimento ampliativo
della propria posizione giuridica soggettiva e si veda opporre un
rifiuto espresso oppure debba constatare l'inerzia dell'amministrazione,
azionerà il rimedio della \emph{Verplichtungsklage}, chiedendo la
condanna al rilascio del provvedimento rifiutato od omesso. L'azione di
annullamento o di impugnazione, così come quella di adempimento, sono
disciplinate dal \emph{§ 42 VwGO} che al primo comma prevede che
\emph{``mediante azione può essere richiesto l'annullamento di un atto
amministrativo\_ (azione di impugnazione), \emph{come pure la condanna
all'emanazione di un atto amministrativo rifiutato o omesso''} (azione
di inadempimento) e al secondo comma che \emph{''qualora la legge non
disponga diversamente, l'azione è ammissibile solo quando l'attore fa
valere di essere stato leso nei propri diritti dall'atto amministrativo
o dal suo rifiuto o omissione''}\footnote{G. FALCON, C. FRAENKEL,
  \emph{Ordinamento processuale amministrativo tedesco (VwGO)}, Trento,
  2000.}. L'azione di annullamento è fondata allorquando ricorrano i
requisiti previsti dal \emph{§ 113 VwGO}, e cioè nella misura in cui
l'atto risulti illegittimo e lesivo dei cosiddetti diritti civili
pubblici dell'attore \emph{(Kläger)}, quei diritti cioè che conferiscono
al singolo la facoltà di pretendere dall'amministrazione una prestazione
positiva o negativa. Verificata la sussistenza di questi presupposti, il
tribunale potrà quindi annullare l'atto, ma la sentenza che conclude il
giudizio di impugnazione potrà assumere un contenuto ulteriore e diverso
dal mero annullamento del provvedimento impugnato, strettamente
correlato all'attività esecutiva che l'amministrazione dovrebbe
successivamente porre in essere per adeguarsi al \emph{decisium}.
L'effetto demolitorio del provvedimento illegittimo, previa la
sospensione della sua efficacia esecutiva, potrebbe rendere non
necessaria la successiva attività di adeguamento; diversamente,
nell'ambito della stessa sentenza che definisce il giudizio cassatorio,
è prevista la possibilità per il giudice di guidare l'amministrazione
nella scelta delle modalità di esecuzione della sentenza, per il
ripristino dello \emph{status quo ante} attraverso la cancellazione
degli effetti che si sono nel frattempo prodotti. E' questo l'istituto
del cosiddetto \emph{Folgenbeseitigungsanspruch}\footnote{Letteralmente
  ``pretesa o diritto all'eliminazione delle conseguenze dell'atto''.},
indicato con l'abbreviazione \emph{FBA} e contemplato dal \emph{§ 113/1}
secondo alinea \emph{VwGO}, ove si prevede che \emph{''se l'atto
amministrativo è stato già eseguito, il tribunale può anche dichiarare,
su richiesta, se e come l'autorità amministrativa debba revocare
l'esecuzione''}. Il \emph{FBA} è autonomo rispetto all'azione di
annullamento, inquadrabile fra le cosiddette azioni di prestazione,
ancorché il giudice dell'impugnazione si pronunci sull'eliminazione
degli effetti dell'atto con la medesima sentenza che definisce il
giudizio cassatorio. L'utilizzo del termine''può'' }(Kann)\_ da parte
del legislatore indica la mera facoltà di cumulare la domanda di revoca
dell'esecuzione a quella di annullamento dell'atto eseguito, ma ciò non
esclude la possibilità di proporre separata istanza, instaurando un
autonomo giudizio. Rimane impregiudicata la facoltà per il tribunale, ai
sensi del \emph{§ 93} secondo alinea \emph{VwGO}, di ordinare in ogni
caso che le rispettive domande vengano trattate e decise in separati
processi. Il c.~1, alinea terzo del \emph{§113 VwGO} stabilisce che la
pretesa alla revoca dell'esecuzione è ammissibile solo laddove
l'autorità amministrativa sia in grado di darvi seguito. In altri
termini, l'attività di rimozione degli effetti dell'esecuzione del
provvedimento annullato presuppone una prestazione possibile sotto il
profilo giuridico-fattuale. Qualora l'amministrazione non sia in grado
di ripristinare esattamente la situazione pregressa, dovrebbe
ricostruirne una quantomeno simile a quella precedente l'esecuzione
dell'atto annullato, in modo tale da eliminare al massimo i pregiudizi
per il destinatario del provvedimento\footnote{P. BECKER, H. KUNI,
  \emph{Probleme des verwaltungsgerichtlichen Vergabeverfahrens für
  Studienplätze in der Humanmedizin,} in \emph{DVBl} 1976, 863.}. Si
ritiene inoltre che la revoca dell'esecuzione come disposta dal giudice
possa consistere, oltre che nella rimozione di un'attività materiale
dell'amministrazione, anche nell'adozione di un atto amministrativo,
quale ad esempio l'ordine di sgombero di un appartamento a seguito
dell'annullamento della confisca dell'immobile da parte delle forze di
polizia, con susseguente sua occupazione da parte di un
terzo\footnote{\_VGH Kassel, 11.11.1993, in \emph{NVwZ}, 1995, 301; F.
  O. KOPP, W. R. SCHENKE, \emph{Verwaltungsgerichtsordnung}, cit. sub. §
  113, n.~91.}. Ulteriore presupposto di ammissibilità della pretesa,
oltre al fatto che l'autorità sia in grado di darvi seguito, è che la
questione sia matura per la decisione\footnote{§ 113/2, secondo alinea
  \emph{VwGO}.}. Ciò significa che non deve più esserci necessità di
accertare i fatti e non deve residuare alcuna discrezionalità in capo
all'amministrazione per quanto riguarda le modalità di revoca
dell'intervenuta esecuzione. Il \emph{FBA} sarà escluso laddove la
rimozione delle conseguenze dell'esecuzione sia in contrasto con la
legge al momento della decisione del tribunale\footnote{F. O. KOPP, W.
  R. SCHENKE, \emph{Verwaltungsgerichtsordnung}, cit., sub § 113, n.~87.}.
In definitiva, l'amministrazione che si trovi a dover eseguire la
sentenza di annullamento e, quindi, a ripristinare la situazione
esistente prima del provvedimento caducato, potrà essere guidata dal
giudice nella scelta delle misure necessarie all'esecuzione del
\emph{dictum} giudiziale, almeno per quel che concerne la rimozione
degli effetti strettamente connessi all'esecuzione del provvedimento
annullato. L'inottemperanza alla decisione sotto tale profilo, seppur
non assistita da alcun meccanismo di coazione diretta, potrà tuttavia
essere sanzionata attraverso l'attivazione della peculiare procedura di
coercizione indiretta di cui al \emph{§ 172 VwGO}, consistente
nell'assegnazione da parte del giudice, su richiesta dell'interessato,
di un termine per l'esecuzione della pronuncia e, nel caso di
inosservanza del medesimo, nell'irrogazione di un'ammenda, lo
\emph{Zwangsgeld}.

\hypertarget{il-contenzioso-ingiuntivo-la-c.d.-verpflichtungsklage}{%
\section{\texorpdfstring{Il contenzioso ingiuntivo: la c.d.
\emph{``Verpflichtungsklage''}}{Il contenzioso ingiuntivo: la c.d. ``Verpflichtungsklage''}}\label{il-contenzioso-ingiuntivo-la-c.d.-verpflichtungsklage}}

La disposizione al § 113/5 \emph{VwGO} contempla la sentenza sulla c.d.
\emph{Verplichtungsklage}, azione di prestazione o, secondo definizione
della dottrina italiana, di condanna, con la quale si ingiunge alla
pubblica amministrazione l'emanazione di un atto rifiutato od
omesso\footnote{Il § 113/5 \emph{VwGO}, così come tradotto da G. FALCON,
  C. FRAENKEL, \emph{Ordinamento processuale, cit., 94}, recita:
  \emph{``Nella misura in cui il rifiuto o l'omissione dell'atto
  amministrativo è illegittimo e l'attore ne risulta leso nei propri
  diritti, il tribunale dichiara l'obbligo dell'autorità amministrativa
  di porre in essere la richiesta attività dell'ufficio, se la questione
  è matura per la decisione. Altrimenti esso dichiara l'obbligo di
  decidere nei confronti dell'attore nel rispetto della concezione
  giuridica del tribunale}.}. Essa è diretta sostanzialmente a sindacare
il rifiuto del provvedimento amministrativo richiesto dall'interessato
che consegue all'assunzione di un provvedimento espresso di diniego; in
tal caso l'azione assumerà, seppur in via sussidiaria, anche un
contenuto annullatorio, dal momento che l'eventuale sentenza di
accoglimento ne comporterà l'eliminazione\footnote{Rileva D. DE PRETIS,
  \emph{Il processo amministrativo in Europa, cit., 121} come in questo
  caso oggetto del giudizio non è, se non in un secondo momento, il
  provvedimento di espresso diniego di quanto richiesto dal privato, ma,
  in primo luogo, il provvedimento che l'amministrazione avrebbe dovuto
  assumere per soddisfare la pretesa sostanziale del ricorrente, per cui
  è tale ultimo atto che l'amministrazione sarà tenuta ad adottare a
  seguito dell'eventuale sentenza di accoglimento della
  \emph{Verpflightungsklage}.}, ovvero l'omissione di un provvedimento
non ancora denegato. Presupposti di fondatezza della
\emph{Verplichtungsklage}, così come per l'azione di annullamento, sono
l'illegittimità del diniego o dell'inerzia e la lesione dei diritti del
ricorrente. In presenza di tali condizioni, il giudice affermerà
l'obbligo dell'autorità amministrativa di adottare il provvedimento
richiesto, qualora la questione sottoposta al suo esame consenta una
decisione definitiva. Tuttavia, una condanna dell'amministrazione
all'emissione di un atto dotato di un ben preciso contenuto potrà
intervenire soltanto nel caso in cui la stessa amministrazione non abbia
poteri discrezionali, ovvero qualora l'adozione del provvedimento
richiesto rappresenti l'unica manifestazione della discrezionalità
amministrativa priva di errori, tenuto presente che la sentenza di
condanna, in virtù del principio costituzionale della separazione dei
poteri, non sostituisce la decisione dell'amministrazione, ma la obbliga
ad agire. Laddove invece la controversia non consenta una decisione
definitiva, il tribunale, previo riconoscimento della legittimità della
pretesa del ricorrente ad una decisione dell'amministrazione, si
limiterà a statuire l'obbligo di quest'ultima di decidere secondo la
valutazione da esso espressa sulla questione\footnote{§ 113/5
  \emph{VwGO}. Si parla in questo caso di \emph{Bescheidungsurteil}.}.
Partendo dalla considerazione che spesso l'amministrazione non
ottempera, in quanto non sa come ottemperare, l'istituto della
\emph{Verplichtungsklage} acquista notevole importanza nella misura in
cui coinvolge il giudice sin dalla prima pronuncia nel chiarimento della
portata operativa della decisione, ottenendo così una maggior ``presa''
sul successivo comportamento\footnote{G. FALCON, \emph{Per una migliore
  giustizia amministrativa}, cit., 151.}. Per converso, ove
l'amministrazione sia titolare di un più ampio potere discrezionale, il
giudice incontrerà il limite della insostituibilità delle scelte
discrezionali dell'autorità, quindi la susseguente attività
amministrativa rimarrà pur sempre appannaggio dell'amministrazione, con
l'unico limite costituito dall'obbligo di rispettare la ``concezione
giuridica'' del tribunale (c.d. \emph{Bescheidungsurteil}).

\hypertarget{lapplicazione-generalizzata-della-tutela-cautelare}{%
\section{L'applicazione generalizzata della tutela
cautelare}\label{lapplicazione-generalizzata-della-tutela-cautelare}}

Si è visto come, nei casi in cui l'amministrazione incontri delle
difficoltà nell'ottemperare alle sentenze, dovute al fatto di non sapere
come attuarle, il legislatore tedesco abbia cercato di determinare
puntualmente gli effetti delle sentenze, in modo tale da guidare
l'amministrazione nell'esecuzione del \emph{decisium}. Quanto invece
alla difficoltà, se non a volte all'impossibilità, di di rimuovere
completamente la situazione creata da un provvedimento sacrificativo poi
annullato, si è cercato di affrontarla attraverso un'estesa applicazione
della tutela interinale della posizione del ricorrente, la quale si
diversifica a seconda del tipo di provvedimento che viene in rilievo.
Nell'ipotesi di atto amministrativo che incide negativamente sulla sfera
giuridica dell'interessato, cui nel nostro ordinamento corrispondono gli
interessi oppositivi, si ammette, salvo alcune eccezioni, la sospensione
automatica dei suoi effetti in derivazione, prima ancora che della mera
proposizione dell'impugnativa in sede giurisdizionale, della
proposizione del ricorso amministrativo previo che costituisce
condizione di ammissibilità della \emph{Anfechtungsklage} ai sensi del §
68 \emph{VwGO}\footnote{BVerwG, 21.06.1961, in \emph{BVerwGE} 13, 5.}.
Attraverso un impiego esteso e generalizzato della sospensione cautelare
del provvedimento impugnato, ex § 80 \emph{VwGO}, molti dei problemi
legati alla esecuzione delle pronunce del tribunale finiscono per essere
risolti o comunque attenuati in via preventiva, considerata,
diversamente, la necessità di eliminare gli effetti materiale prodotti
dal provvedimento fino al passaggio in giudicato della sentenza di
annullamento\footnote{B. MARCHETTI, \emph{L'esecuzione della sentenza
  amministrativa}, cit., 46.}. La sospensione cautelare non è però
idonea a tutelare il ricorrente che lamenti l'illegittimo diniego di un
provvedimento ampliativo o, meglio, l'omesso rilascio del titolo. In
questi casi, contraddistinti dall'emersione di interessi legittimi
pretensivi, la tutela del privato è garantita dalla possibilità per il
giudice, ai sensi del § 123 \emph{VwGo}, di concedere misure provvisorie
a contenuto positivo anche anteriormente al ricorso, volte ad evitare
quelle modificazioni irreparabili che potrebbero \emph{medio tempore}
investire la situazione di fatto, rendendo alla fine una sentenza
favorevole, vantaggiosa sulla carta, ma nella sostanza inutile.

\hypertarget{le-misure-coercitive-lo-zwangsgeld-172-vwgo}{%
\section{\texorpdfstring{Le misure coercitive: lo \emph{Zwangsgeld (§
172
VwGO)}}{Le misure coercitive: lo Zwangsgeld (§ 172 VwGO)}}\label{le-misure-coercitive-lo-zwangsgeld-172-vwgo}}

Il \emph{VwGO} disciplina l'esecuzione coattiva delle sentenze del
giudice amministrativo nei confronti della pubblica amministrazione ai
§§ 167-172. In particolare, la legge sul processo amministrativo
\emph{(VwGO)} determina il giudice dell'esecuzione (§ 167), i titoli
esecutivi (§ 168), l'esecuzione a favore della mano pubblica (§ 169),
l'esecuzione contro la mano pubblica (§§ 170 e 172), nonché i casi in
cui non è necessaria la formula esecutiva (§ 171). I §§ 170 e 172
\emph{VwGO} rappresentano la base normativa dell'esecuzione forzata
contro la pubblica amministrazione, ancorché i rispettivi ambiti di
applicazione siano da tenere distinti. Il § 170 \emph{VwGO} disciplina
l'esecuzione contro la mano pubblica per crediti pecuniari, compresa la
penale di cui al § 172 (\emph{Zwangsgeld}). Tale norma è modellata sul §
882a \emph{ZPO} relativo all'esecuzione per crediti di denaro nei
confronti delle persone giuridiche di diritto pubblico e le modalità di
esecuzione sono sostanzialmente quelle previste dal codice di procedura
civile, nulla dicendo sul punto il § 170 \emph{VwGO} che tuttavia reca
alcuni correttivi che tengono conto della particolare condizione
giuridica del patrimonio pubblico e della sua tendenziale destinazione
all'assolvimento dei compiti dell'amministrazione. Più nel dettaglio, il
tribunale, da un lato, prima di procedere all'esecuzione forzata, deve
intimare all'autorità amministrativa di eseguire il giudicato entro il
termine massimo di un mese e, dall'altro, essendo l'esecuzione
inammissibile in relazione a beni essenziali per l'adempimento di
pubbliche funzioni o alla cui alienazione si contrapponga un pubblico
interesse, non può ordinare il sequestro di beni destinati all'uso o al
servizio pubblico. Il § 172 \emph{VwGO} attiene, in linea di principio,
all'esecuzione di decisioni dichiarative dell'obbligo
dell'amministrazione di rilasciare un provvedimento nei confronti della
controparte. Esso prevede un mezzo di coercizione meramente indiretto,
assistito dalla minaccia di una sanzione pecuniaria da applicarsi
all'autorità inadempiente senza alcuna limitazione o particolare
privilegio per la stessa, salva la misura massima dell'ammenda, di volta
in volta erogabile, pari a diecimila Euro. Nella misura in cui il
contenuto delle norme predette non dovesse essere esaustivo in relazione
al caso concreto, sarà possibile integrarle con i precetti del codice di
rito (\emph{ZPO}), in nome del principio di effettività della tutela
giurisdizionale\footnote{Sul punto si è pronunciata la Corte
  costituzionale tedesca in una decisione di notevole importanza, con
  cui, nella parte motiva, fa esplicito riferimento alle misure previste
  dai §§ da 885 a 896 \emph{ZPO} la cui scelta, in ordine alle modalità
  ed eventuale successione delle misure da adottare, sarà rimessa alla
  valutazione del giudice competente.}. Molto prima che venisse alla
luce la legge sulla giustizia amministrativa e con essa il § 172
\emph{VwGO}, l'idea di poter eseguire coattivamente le pronunce dei
giudici nei confronti di un soggetto esercente un pubblico potere era
stata decisamente avversata in dottrina. Si sosteneva, in particolare,
che si sarebbe rivelato un non senso che lo Stato, quale fondamento del
diritto (\emph{Hort des Rechts}) potesse essere coartato al rispetto di
quello stesso diritto di cui egli era portatore. Ciò si sarebbe rivelato
inconciliabile con il rispetto che si deve allo Stato medesimo ed
avrebbe leso la sua immagine, mettendone in discussione
l'onore\footnote{\_``L'Etat toujours doit être réputé honnête homme'',
  E. LAFERRIERE, \emph{Traité de la juridiction administrative et des
  recours contentieux}, Paris et Nancy 1896, I, 347 ss.}. Così recita il
§ 172 \emph{VwGO}: ``\emph{Se un'autorità, nei casi di cui al § 113, co.
1, secondo periodo, e co. 5 del § 123, non ottempera all'obbligo
impostole nella sentenza o nel provvedimento provvisorio, il tribunale
di primo grado può, su richiesta, comminare con ordinanza nei suoi
confronti un'ammenda fino a Euro diecimila, previa assegnazione di un
termine, applicarla dopo l'infruttuoso decorso del termine e portarla ad
esecuzione d'ufficio. L'ammenda può essere ripetutamente comminata,
applicata e portata ad esecuzione}''\footnote{G. FALCON, C. FRAENKEL,
  \emph{Ordinamento processuale amministrativo}, cit., § 172, 138.}. La
norma fa espresso riferimento all'esecuzione (indiretta) di una sentenza
di annullamento nella parte in cui contestualmente ingiunge
all'amministrazione la revoca della già introdotta esecuzione del
provvedimento caducato (§ 113/1 secondo alinea \emph{VwGO}), ovvero di
adempimento dell'obbligo di emettere un certo atto o di provvedere nel
rispetto della concezione giuridica del tribunale (§ 113/5 \emph{VwGo})
o, infine, il rilascio di un provvedimento positivo (§ 123 \emph{VwGO}).

Rimedi indiretti ulteriori, prodromici allo \emph{Zwangsgeld}, ma meno
impattanti e di valore per lo più simbolico, sono l'interpello
dell'autorità gerarchicamente superiore rispetto a quella che dovrebbe
eseguire la sentenza, ovvero il ricorso all'opinione pubblica attraverso
una petizione popolare o mediante il coinvolgimento della stampa al fine
di sollecitare o quantomeno esercitare una pressione psicologica
sull'amministrazione inottemperante\footnote{W. BANK,
  \emph{Zwangsvollstreckung gegen Behörde}, cit., 62.}.

La limitazione della procedura esecutiva nei confronti della pubblica
amministrazione ad un rimedio come lo \emph{Zwangsgeld}, perlomeno nei
casi previsti dal § 172 \emph{VwGO}, la mette al riparo dall'esecuzione
diretta, anche se non è del tutto esclusa la possibilità di accedere
alla tutela esecutiva secondo i dettami del codice di rito nel caso di
insuccesso della procedura di coazione indiretta o, a certe condizioni,
in alternativa alla stessa. Ciò sarà consentito laddove si richieda
all'amministrazione una prestazione fungibile, mentre, per quanto
riguarda l'emissione di un atto amministrativo, rimane ferma la rigida
separazione dei poteri tra giurisdizione ed amministrazione.

Quanto alla natura giuridica, lo \emph{Zwangsgeld} è un semplice mezzo
di coercizione e non già un istituto di matrice sanzionatoria. L'ammenda
di cui al § 172 \emph{VwGO}, così come altre disposizioni che fanno
riferimento a tale rimedio, non potrebbe, salvo eccezioni, essere
applicata e portata ad esecuzione laddove il destinatario abbia nel
frattempo adempiuto. Lo scopo precipuo dello \emph{Zwangsgeld} è quello
di determinare il debitore all'adempimento di un obbligo, attraverso la
pressione mediata esercitata sul destinatario dalla minaccia di una
penale nei casi in cui, in linea di massima, non sussiste la possibilità
di attivare un meccanismo di coercizione diretta, oppure perché tale
meccanismo è a forte rischio di insuccesso. La coazione indiretta viene
dunque in rilievo quando debba darsi esecuzione ad obblighi di fare
infungibili, la cui realizzazione è inevitabilmente rimessa alla volontà
di un determinato soggetto, oppure nell'ambito degli obblighi di
tollerare o di astenersi da una certa attività che, per definizione,
possono essere solo indirettamente coercibili\footnote{O. R. REMIEN,
  \emph{Zwangsgeld}, cit., 11, 12.}. Il § 172 \emph{VwGO} si occupa
della sentenza di annullamento solo in funzione della accessoria
statuizione ingiuntiva dell'obbligo di ripristinare la situazione
antecedente all'esecuzione dell'atto annullato e l'ambito privilegiato
dello \emph{Zwangsgeld} risulterebbe essere quello dell'inottemperanza
al giudicato formatosi sulle sentenze di adempimento, ex § 113/5
\emph{VwGO} (\emph{Verpflichtungsurteil}), ma l'esecuzione coattiva
indiretta secondo i dettami del § 172 \emph{VwGO} opera anche in
conseguenza di un \emph{Bescheidungsurteil}, con il quale viene
unicamente sancito l'obbligo dell'autorità di provvedere nel rispetto
del quadro giuridico delineato in sentenza, senza alcuna indicazione in
ordine allo specifico contenuto dell'atto da adottare. In sostanza,
l'omesso rilascio del richiesto provvedimento, così come la riedizione
del potere amministrativo in contrasto con il quadro giuridico delineato
nel \emph{Bescheidungsurteil}, consentono di attivare il rimedio dello
\emph{Zwangsgeld}. In particolare, il ricorrente vittorioso potrà
chiedere al tribunale di primo grado, con apposita istanza, di fissare
un termine entro il quale l'autorità dovrà dare completa esecuzione alla
sentenza, contestualmente determinando una penale nell'ammontare massimo
di Euro diecimila, per il caso di persistente inottemperanza, anche
oltre la scadenza del termine predetto. In quest'ultima evenienza,
sempre su richiesta di parte, il tribunale provvederà ad applicare
all'amministrazione renitente l'ammenda stabilita ed a riscuoterla
coattivamente d'ufficio, con l'ulteriore possibilità di reiterare in
ipotesi all'infinito la procedura, fin tanto che permanga
l'inadempimento della pubblica autorità. La definitiva impossibilità di
dare esecuzione alla sentenza per causa imputabile all'amministrazione
potrà dar luogo, in ogni caso, ad una responsabilità risarcitoria della
stessa. Per poter attivare la procedura esecutiva ai sensi del § 172
\emph{VwGO} nei confronti dell'amministrazione inadempiente, è
necessario che ricorrano i seguenti presupposti: il titolo esecutivo che
sarà costituito ad esempio da una sentenza di condanna al ripristino
dello \emph{status quo ante} accessoria ad una sentenza di annullamento
(§ 113/1 secondo alinea \emph{VwGO}), di adempimento (§ 113/5
\emph{VwGO}) o da un provvedimento provvisorio positivo (§ 123
\emph{VwGO}), la notifica del titolo alla controparte, ai sensi del §
167/1 \emph{VwGO} in combinato disposto con i §§ 795, 724, 750/1
\emph{ZPO}, la formula esecutiva da apporre alla decisione del giudice e
l'inottemperanza dell'amministrazione alla statuizione giudiziale che
deve essere definitiva, ove si tratti di sentenza.

\hypertarget{i-mezzi-di-tutela-esperibili-dalle-parti}{%
\section{I mezzi di tutela esperibili dalle
parti}\label{i-mezzi-di-tutela-esperibili-dalle-parti}}

I provvedimenti del giudice dell'esecuzione relativi alla minaccia ed
applicazione dello \emph{Zwangsgeld} possono essere eventualmente
impugnati con ricorso, ai sensi del § 146 \emph{VwGO} entro due
settimane dalla loro notifica, di regola davanti al tribunale
amministrativo di grado intermedio. In tal caso, l'impugnativa proposta
dalla pubblica amministrazione avverso l'ordinanza applicativa
dell'ammenda, comporterà l'automatica sospensione dell'efficacia
esecutiva del provvedimento\footnote{R. PIETZNER, in F. SCHOCH, E.
  SCHMIDT-AßMANN, R. PIETZNER, \emph{VwGO}, cit. sub §172, n.~52.}. Allo
stesso modo, il privato potrà avvalersene nel caso in cui il tribunale
rigetti, sempre con ordinanza, l'istanza volta ad ottenere la minaccia o
l'irrogazione della penale. Con tale rimedio possono essere fatti valere
soltanto i vizi formali della procedura esecutiva, cioè il mancato
rispetto delle regole procedurali disciplinanti la stessa\footnote{R.
  PIETZNER, in F. SCHOCH, E. SCHMIDT-AßMANN, R. PIETZNER, \emph{VwGO},
  cit. sub §167, n.~5.}. Avverso la decisione sul predetto ricorso, non
è dato alcun ulteriore mezzo di tutela, conformemente a quanto stabilito
dal § 152/1 \emph{VwGO}. Diversamente, ove sia in contestazione da parte
dell'amministrazione il diritto di procedere ad esecuzione forzata da
parte del privato, potrà essere introdotto ricorso per opposizione
all'esecuzione, ai sensi del § 167/1 \emph{VwGO}, in combinato disposto
con il § 767 \emph{ZPO}. Scopo di questo rimedio è quello di elidere
l'efficacia esecutiva del titolo, non potendo chiaramente essere rimesso
in discussione il giudicato. In tal senso, il § 767/2 \emph{ZPO}
prescrive che possano essere fatte valere soltanto quelle eccezioni che
si fondino su circostanze sopravvenute all'udienza per la discussione
della causa, nell'ambito del processo di cognizione; inoltre, l'autorità
amministrativa dovrà sollevare tutte le eccezioni deducibili al momento
dell'opposizione, come previsto al § 767/3 \emph{ZPO}, ad esempio la
sopravvenuta modifica della situazione di fatto o di diritto rispetto a
quella sulla quale è sceso il giudicato, così come il sopravvenuto
adempimento dell'obbligo da esso discendente. Si pensi all'entrata in
vigore di un nuovo piano regolatore in contrasto con il permesso di
costruire al cui rilascio l'amministrazione veniva condannata con
sentenza di adempimento (\emph{Verplichtungsurteil}): fintanto che non
venga rilasciato il titolo autorizzativo, il giudicato avente ad oggetto
il riconoscimento della relativa pretesa non è al riparo da eventuali
sopravvenienze di diritto, a differenza di ciò che accade nel diritto
civile, ove è sempre irrilevante il mutamento del quadro normativo entro
il quale dovrebbe essere eseguito il giudicato\footnote{\_BVerwG
  26.10.1984, in \emph{NVwZ} 1985, 563.}. Competente a giudicare
dell'opposizione è il tribunale di prima istanza.

\hypertarget{il-rapporto-fra-lo-zwangsgeld-ed-il-risarcimento-del-danno-da-giudicato}{%
\section{Il rapporto fra lo Zwangsgeld ed il risarcimento del danno da
giudicato}\label{il-rapporto-fra-lo-zwangsgeld-ed-il-risarcimento-del-danno-da-giudicato}}

Una delle peculiarità del sistema tedesco di coercizione indiretta è
rappresentata dalla integrale devoluzione alle casse dello Stato delle
somme ricavate dalle ammende. La diversa impostazione che vuole che il
creditore sia beneficiario delle somme, abbracciata dai paesi che, come
la Francia, hanno subito le influenze del diritto romano, in cui vi era
una commistione fra l'\emph{astreinte} ed il risarcimento del danno, è
stata fortemente criticata in quanto, avendo il ricorrente vittorioso la
possibilità di agire in via risarcitoria per il caso di ritardata od
omessa esecuzione del giudicato da parte dell'amministrazione,
attribuire al privato anche l'importo della penale, vorrebbe dire
arricchirlo ingiustificatamente. La soluzione seguita in Francia
presenterebbe lo svantaggio di rendere incerti i confini tra lo
\emph{Zwangsgeld} e il risarcimento del danno, poiché l'associazione tra
i due istituti pregiudicherebbe l'effetto di coazione che è proprio
dell'ammenda ex § 172 \emph{VwGO} e, allo stesso tempo, la possibilità
di cumularli porterebbe il creditore a ricevere troppo. Inoltre, la
destinazione dello \emph{Zwangsgeld} alle casse dello Stato sarebbe
ulteriormente funzionale a garantire il rispetto delle decisioni
giurisdizionali e quindi il prestigio dell'amministrazione della
giustizia.

Per quanto concerne il rimedio risarcitorio, la disciplina è quella
valevole per tutte le ipotesi di responsabilità della pubblica
amministrazione per i danni cagionati nell'esercizio dei propri doveri
d'ufficio. Le norme fondamentali in materia sono l'art. 34 del
\emph{Grundgesetz} e il § 839 del \emph{Bürgerliches Gesetzbuch}, le
quali vanno lette in combinato disposto\footnote{S. DETTERBECK,
  \emph{Allgemeines Verwaltungsrecht}, cit., 379.}. In base al c.~1 del
§ 839 \emph{BGB} i danni causati intenzionalmente o con negligenza, in
violazione di un dovere del funzionario, vengono dallo stesso
integralmente risarciti. Se si tratta di mera negligenza, la
responsabilità del funzionario potrà essere invocata in via sussidiaria,
ovvero soltanto nel caso in cui non vi sia un'altra via per ottenere il
risarcimento, ad esempio per contratto, per legge, o in base al sistema
di assicurazione sociale\footnote{U. KARPEN, \emph{L'esperienza della
  Germania}, in D. SORACE (a cura di) \emph{La responsabilità pubblica
  nell'esperienza giuridica europea}, Bologna, 1994, 140.}. Per
converso, l'art. 34 della \emph{Grundnorm} trasferisce la responsabilità
sulla pubblica autorità da cui il funzionario dipende, salvo il regresso
nei casi di dolo e colpa grave per evitare che i funzionari abusino
dell'immunità della responsabilità personale loro garantita\footnote{\ldots{}}
e radica in capo al giudice ordinario la giurisdizione per le azioni per
le azioni di responsabilità nei confronti del potere pubblico,
stabilendo al terzo alinea, con riferimento al diritto al risarcimento e
al diritto di rivalsa, che non può mai essere esclusa l'azione di fronte
alla giurisdizione ordinaria.

\hypertarget{la-giustizia-amministrativa-francese}{%
\chapter{La giustizia amministrativa
francese}\label{la-giustizia-amministrativa-francese}}

\hypertarget{la-genesi-del-droit-administratif}{%
\section{\texorpdfstring{La genesi del \emph{droit
administratif}}{La genesi del droit administratif}}\label{la-genesi-del-droit-administratif}}

In Francia, durante il secolo XVIII, si sviluppa progressivamente una
forte \emph{administration royale}, accentrata e gerarchizzata, dotata
di poteri speciali e sottoposta a giurisdizioni apposite. E'
un'amministrazione presente tanto in centro quanto in periferia,
costituita da funzionari borghesi di medio e basso ceto riuniti nel
\emph{Conseil du Roi}, organo che, oltre a detenere la potestà
legislativa, funge sia da suprema corte di giustizia, in quanto dotato
del potere di annullare i decreti di tutti i tribunali ordinari, sia da
tribunale superiore amministrativo, perché da esso dipendono tutte le
giurisdizioni speciali. Durante la fine dell'\emph{Ancien regime} si
formalizza la specialità del diritto amministrativo, sia in relazione
alla peculiarità e ai maggiori poteri conferiti alle amministrazioni
pubbliche, sia per quanto riguarda la specificità del contenzioso
amministrativo che non era affidato al potere giudiziario. In questo
periodo si generalizzano le \emph{corvées}\footnote{Termine francese,
  utilizzato sia in Francia che in Italia, che significa
  \emph{corrogare}, nel senso di chiedere una giornata di lavoro. Nel
  Medioevo, la \emph{corvée} indicava un'imposta, poi abolita durante la
  rivoluzione, che veniva richiesta dal signore ai suoi servi, da
  estinguere con un certo numero di giornate di lavoro - Cfr.
  \emph{Portal Français, 100 mots du trésor FR - la corvée} di Agnes
  Dijaux. Nella Francia dell'\emph{Ancien régime}, il termine indicava
  le giornate di lavoro che i sudditi dovevano al re per la manutenzione
  delle strade pubbliche - Cfr. \emph{Treccani.it}, Enciclopedia,
  dizionario di storia (2010).}, si sviluppa la polizia dei mestieri a
difesa dell'ordine pubblico e crescono le espropriazioni forzate, così
come l'imposizione fiscale e il potere autoritativo dell'amministrazione
verso i suoi contraenti. Di contro, le armi che gli amministrati hanno
nei confronti del potere amministrativo non sono così forti: a favore
del cittadino vi è il rimedio specifico costituito dal cosiddetto
ricorso gerarchico, a mezzo del quale egli può rivolgersi all'organo
gerarchicamente sovraordinato all'amministrazione che aveva emanato
l'atto lesivo e può richiedere la verifica della legalità dell'atto.

La Rivoluzione francese, pur portando innovazioni rilevanti, lascia
immutata l'ampia discrezionalità delle amministrazioni pubbliche che
conservano e potenziano le loro prerogative. In ossequio al principio
della separazione dei poteri, i giudici sono estromessi dalla conoscenza
degli affari amministrativi e non potranno in alcun modo turbare le
attività dei corpi amministrativi. Le forze rivoluzionarie avevano
dimostrato infatti una certa differenza verso la magistratura,
tradizionalmente formata da elementi vicini alle classi aristocratiche
e, nel 1789-1790, prima l'Assemblea nazionale e poi l'Assemblea
costituente avevano sancito in forma solenne che gli organi
giurisdizionali non avrebbero potuto intervenire sull'amministrazione
\footnote{Cfr. il decreto 16 agosto 1790 sull'ordinamento giudiziario,
  in ALDO TRAVI, \emph{Lezioni di giustizia amministrativa}, Torino,
  2019, 8, 9: ``Le funzioni giurisdizionali sono distinte e rimangono
  sempre separate dalle funzioni amministrative. I giudici non potranno,
  sotto pena di prevaricazione, interferire in qualunque modo sulle
  operazioni dei corpi amministrativi, né citare avanti a sé gli
  amministratori a motivo dell'esercizio delle loro funzioni''.}. Ora,
ad occuparsi delle controversie amministrative saranno i ministri;
scompare il \emph{Conseil du Roi}, viene introdotto il Prefetto e, come
agente dell'Esecutivo in provincia, reintrodotta la figura
dell'intendente.

\hypertarget{dalla-giustizia-ritenuta-alla-giustizia-delegata-il-conseil-detat}{%
\section{\texorpdfstring{Dalla giustizia ``ritenuta'' alla giustizia
delegata: il \emph{Conseil
d'Etat}}{Dalla giustizia ``ritenuta'' alla giustizia delegata: il Conseil d'Etat}}\label{dalla-giustizia-ritenuta-alla-giustizia-delegata-il-conseil-detat}}

La Costituzione dell'anno VIII (dicembre 1799) istituisce il
\emph{Conseil d'Etat} (Consiglio di Stato), concepito inizialmente come
organo consultivo del Governo, al quale, nell'epoca napoleonica,
verranno affidate attribuzioni assai ampie, tra cui principalmente il
potere di redigere i progetti di legge e i regolamenti
dell'amministrazione pubblica, gli atti legislativi primari e secondari,
l'alta amministrazione, il controllo sui ministri e sugli enti pubblici
e la risoluzione di controversie amministrative.

Riguardo ai ricorsi, il Consiglio di Stato formalmente esprimeva un
parere al Capo dello Stato, al quale solo, come rappresentante supremo
del potere esecutivo, spettava assumere la decisione che però, nella
pratica, si uniformava sempre al parere e l'intervento del Capo dello
Stato finiva con l'attribuire ancora maggiore autorevolezza al parere e
all'organo che lo esprimeva. Un decreto di Napoleone del 1806 istituì,
in seno al Consiglio di Stato, un'apposita Commissione del contenzioso
con il compito di istruire i ricorsi proposti contro gli atti
(\emph{décisions}) delle amministrazioni centrali e locali. Per
rafforzarne l'imparzialità, ai consiglieri che componevano la
Commissione non potevano essere affidati compiti di amministrazione
attiva\footnote{ALDO TRAVI, \emph{Lezioni di giustizia amministrativa},
  Torino, 2019, 9.}. Il Consiglio di Stato, mantenuto anche con la
Restaurazione (1814-1815) detiene quindi, insieme al \emph{Conseil de
Prefecture}, la giurisdizione amministrativa, non più affidata ai
ministri e agli intendenti: si passa da un sistema tradizionale di
giustizia ``ritenuta'', in cui il re era giudice supremo, dal quale
``emana ogni giustizia'' (secondo la massima ``\emph{toute justice émane
du roi}'') ad una giustizia ``delegata'', affidata pienamente al
Consiglio di Stato, al quale, prima transitoriamente, con la
Costituzione del 4 novembre 1848 e poi, definitivamente, con una legge
del 24 maggio 1872, fu riconosciuta anche formalmente la competenza a
decidere il ricorso, senza più la necessità di una sanzione da parte del
Capo dello Stato. Quest'ultima riforma del 1872 attribuiva così al
Consiglio di Stato tutti i caratteri di un vero e proprio giudice
amministrativo.

La giurisprudenza del \emph{Conseil d'Etat} consente al \emph{droit
administratif} di passare da un semplice insieme di regole derogatorie
al diritto privato ad un autentico sistema autonomo, costituito da
principi e concetti propri, quali il \emph{service public,}
l'\emph{excés de pouvoir}, la \emph{résponsabilité administrative}. Il
contenzioso dinanzi al \emph{Conseil d'Etat} e le decisioni che da esso
emanano assicurano per la prima volta un equilibrio efficace fra
prerogative pubbliche e garanzie degli amministrati.

Nella prima metà dell'Ottocento però, si assiste ad un ampliamento della
competenza giurisdizionale del giudice ordinario, il quale rivendica
un'attitudine naturale a conoscere non solo delle controversie
concernenti la proprietà, ma anche delle dispute in materia di contratti
e di altri istituti regolati dal codice civile. Pertanto, il criterio
utilizzato per determinare la competenza del giudice diviene quello di
individuare il diritto sostanziale applicabile al caso concreto, cui si
aggiunge successivamente il criterio di competenza fondato sulla
distinzione tra \emph{actes de puissance publique}, affidati al giudice
amministrativo, con i quali l'amministrazione agisce come depositaria
dell'autorità attribuita dall'esercizio del potere esecutivo, e
\emph{actes de gestion} che l'amministrazione pone in essere in qualità
di garante dei servizi pubblici, affidati al giudice ordinario.

Il Novecento vede consolidarsi il criterio di competenza basato sul
concetto di \emph{service public}\footnote{M. HAURIOU, Précis de droit
  administratif, Paris, 1933, 13 ss.} riferito ad attività dirette alla
soddisfazione di interessi comuni e caratterizzate dall'erogazione di
prestazioni e servizi ai consociati. Si fa strada la distinzione tra
controversie relative ai \emph{services publics administratifs} la cui
giurisdizione spetta al Consiglio di Stato e controversie riguardanti i
\emph{services publics industriels et commerciaux} che spettano al
giudice ordinario, ma tale distinzione non si rivela essere così netta
nel momento in cui alcune controversie rientranti nell'ambito dei
\emph{service publics administratifs} sono attratte nella competenza del
giudice ordinario, in quanto il criterio del \emph{service public} è
sempre suscettibile di cedere se il regime dell'atto oggetto della
controversia mostri o che il \emph{service administratif} ha agito
nell'ambito del diritto comune, o che il \emph{service industriel et
commercial} ha adottato strumenti di \emph{droit administratif}. In
sostanza, si torna ad un criterio generale di competenza, in virtù del
quale è la natura delle norme sostanziali e degli atti di gestione del
servizio a determinare la giurisdizione.

La previsione di competenze del giudice ordinario ha comportato la
necessità di istituire, nel 1848, un organo che potesse decidere, nei
casi controversi, se la vertenza spettasse allo stesso giudice ordinario
o al giudice amministrativo: il Tribunale dei conflitti. Tale tribunale,
per assicurare l'equilibrio tra le due giurisdizioni, è composto da uno
stesso numero di magistrati della Cassazione e di consiglieri di
Stato\footnote{ALDO TRAVI, \emph{Lezioni di giustizia amministrativa},
  Torino, 2019, 11.}.

Con l'entrata in vigore, il 4 ottobre 1958, della Costituzione francese
che ha dato origine alla Quinta Repubblica, è stato istituito il
Consiglio costituzionale (\emph{Conseil constitutionnel}), organo
accentrato che svolge anche un controllo di legittimità costituzionale.
Successivamente, con due decisioni del Consiglio costituzionale, la
prima del 22 luglio 1980 e la seconda del 23 gennaio 1987, si è
conferito valore costituzionale all'indipendenza ed alla competenza
della giurisdizione amministrativa\footnote{Cfr.
  https://www.vie-publique.fr/fiches/20284-justice-administrative-origines-role-et-specificites}.
La revisione costituzionale del 23 luglio 2008 ha confermato questo
radicamento costituzionale introducendo la nozione di \emph{ordre
administratif} \footnote{Tre sono i gradi della giurisdizione
  amministrativa francese: i tribunali amministrativi (\emph{tribunaux
  administratifs}), le corti amministrative d'Appello (\emph{courts
  administratives d'Appel}) e il Consiglio di Stato, Cfr.
  https://www.justice.gouv.fr/organisation-de-la-justice-10031/lordre-administratif-10034/}
all'art. 65 della Carta e, con decisione del 3 dicembre 2009, il
Consiglio costituzionale ha qualificato la Corte di cassazione e il
Consiglio di Stato quali organi al vertice delle due giurisdizioni
riconosciute dalla Costituzione\footnote{Cfr. \emph{``juridictions
  placées au sommet de chacun des deux ordres de juridiction reconnus
  par la Constitution''}, \emph{Décision du Conseil constitutionnel n°
  2009-595 DC~du 3 décembre 2009}, § 3, in
  https://www.conseil-constitutionnel.fr/actualites/communique/decision-n-2009-594-dc-du-3-decembre-2009-communique-de-presse\#:\textasciitilde:text=Le\%203\%20d\%C3\%A9cembre\%202009\%2C\%20par,portant\%20diverses\%20dispositions\%20relatives\%20aux}.

\hypertarget{il-pouvoir-dinjonction-del-giudice-amministrativo}{%
\section{\texorpdfstring{Il \emph{pouvoir d'Injonction} del giudice
amministrativo}{Il pouvoir d'Injonction del giudice amministrativo}}\label{il-pouvoir-dinjonction-del-giudice-amministrativo}}

La legge n.~95-125 del 8 febbraio 1995 ha introdotto nel codice dei
tribunali amministrativi e delle corti amministrative di appello
(\emph{code des tribunaux administratifs et des cours administratives
d'appel}) un nuovo titolo, dedicato all'esecuzione del giudicato e
inserito all'interno del libro relativo alle attribuzioni
giurisdizionali dei tribunali e delle corti. L'art. L. 8-2 dello stesso
codice stabilisce le modalità con cui i giudici dei suddetti tribunali e
corti d'appello possono ingiungere all'amministrazione di ottemperare
alle sentenze distinguendo due casi, a seconda del potere, più o meno
vincolato, dell'amministrazione stessa. Al comma 1 si prevede che
qualora un ente pubblico, o un organismo di diritto privato incaricato
della gestione di un pubblico servizio, debba adottare una determinata
misura per l'esecuzione della sentenza, il tribunale amministrativo o la
corte amministrativa di appello, precedentemente chiamati a decidere in
tal senso, prescrivono tale misura, fissando altresì un termine entro il
quale l'esecuzione della stessa debba avvenire. Il comma 2 del medesimo
articolo L. 8-2 aggiunge invece che, qualora un ente pubblico, o un
organismo di diritto privato incaricato della gestione di un servizio
pubblico, debba adottare un nuovo provvedimento, a seguito di una nuova
fase istruttoria, il tribunale amministrativo o la corte amministrativa
di appello, precedentemente chiamati a decidere in tal senso,
stabiliscono che l'adozione del nuovo provvedimento debba avvenire entro
un determinato termine\footnote{Cfr. art. L. 8-3 del \emph{code des
  tribunaux administratifs et des cours administratives d'appel} in
  https://www.senat.fr/rap/l98-380/l98-3805.html}.

L'articolo L. 8-3 del codice dei tribunali amministrativi e delle corti
amministrative di appello permette al giudice chiamato a decidere di
accompagnare al potere ingiuntivo prescritto dall'articolo L. 8-2, la
previsione di una penale, la cosiddetta \emph{astreinte}, nell'ambito
della stessa decisione\footnote{Cfr. art. L. 8-3 del \emph{code des
  tribunaux administratifs et des cours administratives d'appel} in
  https://www.senat.fr/rap/l98-380/l98-3805.html}.

L'art. L. 8-4 prevede che, nel caso di mancata esecuzione della sentenza
emessa dal tribunale amministrativo o della corte amministrativa di
appello, la parte interessata all'ottemperanza può chiedere al giudice
dello stesso tribunale o della stessa corte amministrativa di appello
che ha pronunciato la decisione di assicurarne l'esecuzione. Il giudice
potrà così definire le modalità di esecuzione, fissare un termine per
l'esecuzione e stabilire una penale (\emph{astreinte}), ferma restando
la possibilità di rimettere la richiesta di esecuzione al Consiglio di
Stato\footnote{Cfr. art. L. 8-4 del \emph{code des tribunaux
  administratifs et des cours administratives d'appel} in
  https://www.senat.fr/rap/l98-380/l98-3805.html}.

Tuttavia oggi, venticinque anni dopo aver ottenuto la facoltà di
ingiungere all'amministrazione l'esecuzione delle sentenze, il giudice
amministrativo si dimostra normalmente molto prudente nell'esercizio di
tale potere. Tale constatazione deriva dalla lettura di alcune ordinanze
da esso pronunciate durante la pandemia da Covid-19, dalla cui lettura
emerge che le circostanze eccezionali del momento, parallelamente al
rafforzamento dei poteri normalmente attribuiti alle autorità esecutive,
legittimavano un utilizzo più esteso del \emph{pouvoir d'injonction}. In
generale, si è notata da parte del giudice amministrativo una qualche
reticenza nell'esplicitare il suo pieno potere ingiuntivo: sia esso o
meno giuridicamente fondato, egli dimostra una sorta di autolimitazione
quando si tratta di agire d'imperio\footnote{Cfr. Dupre de Boulois~X,
  \emph{«~Le référé-liberté pour autrui~»}, AJDA~2013, p.~2137}. In
particolare, quando pronuncia un'ingiunzione, si obbliga a non
sostituire il suo apprezzamento a quello dell'amministrazione e cerca di
non interferire con essa in presenza di poteri discrezionali. In questa
prospettiva, il potere ingiuntivo che il giudice indirizza verso
l'amministrazione è costantemente circoscritto, così come sono limitate
le conseguenze giuridiche che ne derivano\footnote{Cfr.
  https://www.actu-juridique.fr/administratif/le-pouvoir-dinjonction-du-juge-administratif-revisite-par-les-circonstances-exceptionnelles-de-la-crise-sanitaire-du-covid-19/}.

\hypertarget{lesecuzione-delle-sentenze-del-giudice-amministrativo}{%
\section{L'esecuzione delle sentenze del giudice
amministrativo}\label{lesecuzione-delle-sentenze-del-giudice-amministrativo}}

L'amministrazione è tenuta a conformarsi spontaneamente alla sentenza
pronunciata dal giudice amministrativo il quale talvolta indica anche
come l'amministrazione stessa debba procedere in tal senso. Tuttavia,
qualora invece l'amministrazione non ottemperi, o alla decisione del
giudice segua solo un'esecuzione parziale, la parte interessata può
agire secondo due modalità, a seconda che l'amministrazione sia stata
condannata a versare una somma di denaro, oppure debba adottare un nuovo
provvedimento a seguito della sentenza di annullamento di un atto
amministrativo\footnote{Cfr. \emph{Qu'est-ce que l'exécution des
  décisions?} in \emph{L'exécution des décisions du Juge administratif}
  http://paris.cour-administrative-appel.fr/Demarches-procedures/Les-fiches-pratiques-de-la-justice-administrative}.

Nel primo caso, se la domanda del privato verte unicamente sul
versamento di una somma di denaro da parte dell'amministrazione, egli
può attivare la \emph{procédure de la contrainte au paiement}, chiamata
anche \emph{procédure du paiment forcé}\footnote{Tradotto letteralmente
  come procedura di costrizione al pagamento, anche chiamata procedura
  di pagamento forzato.}che gli permetterà di ottenere il pagamento
della somma dovuta, a condizione che la decisione del giudice sia
divenuta definitiva, ne stabilisca precisamente l'ammontare e che
l'amministrazione non abbia versato la somma entro il termine stabilito
di due mesi dalla notifica della sentenza. La domanda, se il debitore è
lo Stato, deve essere indirizzata al \emph{Comptable public},
normalmente la Direzione Regionale delle Finanze Pubbliche, per ottenere
il pagamento. Qualora invece il debitore sia un ente territoriale
(regione, dipartimento o comune) o un'altra struttura pubblica
(\emph{établissement public}), occorre rivolgersi al Prefetto o
all'autorità di tutela preposta, sollecitando l'erogazione d'ufficio
della somma dovuta.

In tutti gli altri casi di inottemperanza da parte dell'amministrazione,
il privato può richiedere al giudice l'esecuzione. In generale, tale
richiesta al giudice non può essere presentata prima della scadenza del
termine di tre mesi dalla notificazione della sentenza, ma può essere
inoltrata entro un termine diverso nei casi seguenti:

\begin{enumerate}
\def\labelenumi{\arabic{enumi})}
\tightlist
\item
  qualora la sentenza stabilisca un termine preciso per l'esecuzione, la
  domanda non può che essere presentata entro tale termine;
\item
  se l'amministrazione rifiuta espressamente di conformarsi alla
  sentenza, non sussiste un termine per la richiesta al giudice;
\item
  nel caso di sentenza che stabilisca l'attuazione di misure urgenti,
  l'esecuzione può essere richiesta immediatamente.
\end{enumerate}

Per l'esecuzione di una sentenza pronunciata da un tribunale
amministrativo, la domanda di esecuzione dovrà essere presentata allo
stesso tribunale che ha reso il giudizio e, se il giudizio è stato
emesso da una corte amministrativa d'appello, occorrerà rivolgersi alla
stessa corte. Per l'esecuzione delle decisioni del Consiglio di Stato o
di una giurisdizione amministrativa speciale (in particolare, la
\emph{Cour nationale du droit d'asile}), la domanda dovrà essere
indirizzata alla delegazione all'esecuzione delle decisioni di giustizia
della \emph{section du rapport et des études du Conseil d'Etat}. Nel
caso in cui la domanda sia rivolta per errore ad un giudice non
competente, questi la inoltrerà al giudice competente informando le
parti\footnote{Cfr. \emph{Comment faire exécuter les décisions rendues
  par le juge administratif?} in \emph{L'exécution des décisions du juge
  administratif}
  http://paris.cour-administrative-appel.fr/Demarches-procedures/Les-fiches-pratiques-de-la-justice-administrative}.

Per il privato, al fine di presentare la domanda di esecuzione, non è
obbligatorio ricorrere all'assistenza di un avvocato. La domanda può
essere inoltrata a mezzo dell'applicazione \emph{Télérecours citoyens},
accessibile dal sito www.telerecours.fr , oppure a mezzo posta con
raccomandata A/R alla giurisdizione competente, avendo cura di indicare
la decisione del giudice che si ritiene non ottemperata, le difficoltà
che si riscontrano, le misure che si ritengono dover essere intraprese
per rimediare alla situazione specifica e, contestualmente, richiedere
la pronuncia di una \emph{astreinte} a carico dell'amministrazione
renitente, volta ad indurre la medesima amministrazione ad eseguire
quanto statuito dal giudice.

La procedura di esame della domanda di esecuzione si svolge in due
fasi\footnote{Cfr. \emph{Comment se déroule l'examen de ma demande
  d'exécution?} in \emph{L'exécution des décisions du juge
  administratif}
  http://paris.cour-administrative-appel.fr/Demarches-procedures/Les-fiches-pratiques-de-la-justice-administrative}.
Durante la prima fase, denominata \emph{phase administrative}, entro un
termine massimo di sei mesi, il presidente del tribunale amministrativo,
della corte amministrativa di appello o della sezione di rapporto e di
studi del Consiglio di Stato, promuove tutte le iniziative e le indagini
necessarie ad assicurare l'esecuzione della sentenza da parte
dell'amministrazione. Qualora rilevi che l'amministrazione abbia nel
frattempo ottemperato o che la domanda sia infondata, procederà
all'archiviazione dandone comunicazione alle parti.

La seconda fase, la \emph{phase juridictionelle}, può essere avviata sia
su iniziativa della giurisdizione competente, quando il presidente
ritenga sia necessario prescrivere determinate misure di esecuzione, ad
esempio la pronuncia di una \emph{astreinte}, oppure se la domanda di
esecuzione non è stata soddisfatta entro il termine di sei mesi, sia su
iniziativa di parte, qualora venga impugnata l'archiviazione entro il
termine di un mese dalla sua notifica. Tale impugnazione dovrà essere
proposta alla \emph{section du contentieux} del Consiglio di Stato. La
\emph{phase juridictionelle} può sfociare nella pronuncia di
un'ingiunzione all'amministrazione, accompagnata da una
\emph{astreinte}, se il giudice ritiene che la sentenza sia rimasta
inadempiuta. L'ingiunzione consiste nell'imporre all'amministrazione di
adottare un determinato provvedimento o di riesaminarne la spettanza
entro un termine fissato dal medesimo giudice, mentre l'\emph{astreinte}
consiste in una penale stabilita a carico dell'amministrazione, il cui
ammontare è suscettibile di aumentare fin tanto che l'inottemperanza
persiste. Tuttavia, qualora il presidente della giurisdizione competente
ritenga che le cautele adoperate siano suscettibili di permettere
l'esecuzione della sentenza a breve termine può stabilire, dandone
previa informazione alle parti, che l'avvio della procedura giudiziale
abbia luogo solo una volta scaduto un termine supplementare di quattro
mesi\footnote{Cfr. \emph{Que faire lorsque l'Administration n'exécute
  pas le jugement d'un tribunal administratif ou l'arrêt d'une cour
  d'appel?} in
  http://paris.tribunal-administratif.fr/Demarches-procedures/L-execution-des-decisions-du-juge-administratif}.

\hypertarget{lastreinte}{%
\section{\texorpdfstring{L'\emph{astreinte}}{L'astreinte}}\label{lastreinte}}

Il sistema francese di giustizia amministrativa è quello che più si
avvicina al nostro per tradizione giuridica, ma, sotto il profilo
dell'esecuzione della sentenza amministrativa, esso presenta, nella
sostanza, maggiori punti di convergenza con quello tedesco, sia per
quanto attiene al raffronto tra il \emph{pouvoir d'injonction} e i
poteri direttivi di cui al \emph{Verpflichtungsurteil} ovvero al
\emph{Folgenbeseitigungsurteil}, sia per quanto riguarda le misure di
coazione indiretta volte a piegare la resistenza dell'amministrazione
inottemperante e cioè le \emph{astreintes} rispetto allo
\emph{Zwangsgeld}, con il naturale corollario, comune ad entrambi i
sistemi, di un rigoroso recepimento del principio della separazione dei
poteri\footnote{Osserva D. DE PRETIS, \emph{Il processo amministrativo
  in Europa}, cit., 84, che il problema dell'esecuzione della sentenza
  amministrativa risiede, nell'ordinamento francese così come in altri
  ordinamenti continentali, nella difficile coesistenza tra la riserva
  amministrativa del potere di rinnovare l'attività provvedimentale a
  seguito della decisione del giudice e la garanzia dell'effettività
  della tutela giurisdizionale e quindi dell'esecuzione delle
  statuizioni giudiziali.}.

La legge di riforma del contenzioso amministrativo 95-125 dell'8
febbraio 1995 ridefinisce la disciplina sull'esecuzione del giudicato
amministrativo, ribaltando in primo luogo il principio, sino ad allora
mai scalfito, secondo cui il giudice amministrativo, in virtù dell'art.
13 della legge rivoluzionaria 16-24 agosto 1790 e, in generale, della
separazione dei poteri, non poteva ordinare un \emph{facere} o un
\emph{non facere} alla pubblica autorità e, in secondo luogo, ampliava
il potere di \emph{astreinte}, prima attribuito in via esclusiva al
\emph{Conseil d'Etat} ed esercitabile anche \emph{ex
officio}\footnote{La prima applicazione \emph{ex officio} di
  un'\emph{astreinte} si deve a CE 28 maggio 2001 (\emph{Bandesapt}), in
  \emph{Rec. Lebon 251}, nonché in \emph{DA} 2001, n.~176, con la quale
  lo Stato veniva condannato al pagamento di una sanzione pecuniaria di
  10.000 franchi al giorno, per la mancata ottemperanza da parte del
  governo francese all'obbligo di emanare un regolamento come risultante
  da un decreto del 30 dicembre 1998.}, con la devoluzione dello stesso
anche ai tribunali e alle corti di appello.

L'\emph{astreinte}, quale principale mezzo di coercizione indiretta, è
uno strumento a carattere esclusivamente patrimoniale che ha lo scopo di
incentivare l'esecuzione di una sentenza di condanna, attraverso la
previsione di una sanzione pecuniaria che la parte inadempiente dovrà
versare a favore del creditore vittorioso in giudizio\footnote{Come da
  definizione data da FRANCESCO M. CARALLI, nell'articolo \emph{Il nuovo
  giudizio di ottemperanza, con particolare riguardo alle astreintes},
  in Italiappalti.it, 2017, 4.}. Le prime applicazioni dei mezzi di
esecuzione indiretta si rinvengono nel diritto romano classico secondo
cui, nei casi di condanna a rilasciare un fondo o a realizzare un
\emph{opus}, si stabiliva che il soccombente avrebbe dovuto, in difetto,
pagare una somma pari ad una multa del valore del fondo o dell'opera da
realizzare. Viceversa, in epoca medioevale, l'applicazione degli
strumenti di induzione all'adempimento era circoscritta ai casi in cui
l'interesse del creditore non potesse essere soddisfatto attraverso
l'esperimento dell'esecuzione diretta per cui era necessaria la
partecipazione del debitore. In Francia, l'applicazione
dell'\emph{astreinte} è stata per la prima volta formalizzata da una
sentenza del Tribunale di Cray del 1811, mediante la quale il
soccombente veniva condannato a ``compiere una pubblica ritrattazione
sotto pena di dover pagare tre franchi per ogni giorno di ritardo
nell'adempimento''. L'\emph{astreinte} era di conseguenza strutturata
come una pena privata, non avente il fine di riparare un pregiudizio, ma
quello di riparare una disobbedienza. La legge n.~626 del 5 luglio 1972
ha qualificato l'\emph{astreinte} come sanzione oggetto di condanna
accessoria, il cui adempimento non estingue l'obbligazione principale,
motivo per cui il soccombente può essere condannato a corrispondere un
determinato importo al creditore vittorioso indipendentemente e in
aggiunta al risarcimento del danno, stante la cumulabilità della misura
reintegrativa con quella sanzionatoria.

Nell'ordinamento francese, al pari di quello romano, l'\emph{astreinte}
è comminabile per indurre il soccombente ad eseguire ogni sentenza di
condanna, non rilevando che la condotta ordinata sia infungibile. Da ciò
discende che tale strumento è costruito come mezzo sanzionatorio che
giustifica un trasferimento di ricchezza dalla parte inadempiente al
creditore vittorioso e che, in ossequio alla sua natura meramente
compulsorio-retributiva, mira a ``punire'' l'inosservanza di ogni tipo
di sentenza di condanna, indipendentemente dalla fungibilità della
prestazione ordinata dal giudice. Come ulteriore conseguenza, si ha la
possibilità di concorso tra due distinte procedure esecutive, una per
l'\emph{astreinte} e l'altra per l'esecuzione forzata della prestazione
originaria.

Il d.Lgs. n.~104 del 2 luglio 2010, ha introdotto anche nel processo
amministrativo italiano l'istituto dell'\emph{astreinte}, sia pure con
talune rilevanti differenze rispetto al modello francese. L'art. 114,
comma 4, lett. \emph{e)}, c.p.a. attribuisce al giudice amministrativo
il potere di fissare ``la somma di denaro dovuta dal resistente per ogni
violazione o inosservanza successiva, ovvero per ogni ritardo
nell'esecuzione del giudicato''. A differenza dell'ordinamento francese,
in Italia l'\emph{astreinte} può essere irrogato solo in sede di
ottemperanza e non anche di merito: da ciò discende che nel processo
amministrativo italiano la misura coercitiva non si configura come
sanzione ad esecuzione differita, destinata a divenire attuale se ed in
quanto l'amministrazione non esegua l'ordine contenuto nella sentenza di
merito, ma presuppone che l'inadempimento del debitore sia già stato
accertato dal giudice dell'ottemperanza.

Il comma 4, lett. \emph{e)} contempla inoltre due requisiti negativi,
ovvero che il provvedimento di condanna alla misura coercitiva non sia
``manifestamente iniquo'' e che non ricorrano ``ragioni ostative''.
L'art. 114 c.p.a. però tace sia sui parametri in base ai quali calcolare
il \emph{quantum} della sanzione, sia sui \emph{genera} di condotte che
possono essere assistiti dall'\emph{astreinte}. Dottrina e
giurisprudenza hanno dunque avuto il compito di chiarirne il perimetro
applicativo principalmente con due orientamenti, tra loro distinti in
base alla ricostruzione della \emph{ratio}, nonché in base all'autonomia
annessa all'istituto di diritto amministrativo rispetto alla generale
previsione di \emph{astreinte} prevista nel codice di procedura civile.
L'indirizzo dottrinario accolto dalla giurisprudenza maggioritaria
sostiene che, in termini di \emph{ratio}, nulla osta all'applicazione
delle misure coercitive anche al di fuori del tradizionale perimetro
degli obblighi infungibili. In particolare, finalità
dell'\emph{astreinte} sarebbe quella di sanzionare la mancata
conformazione del soccombente all'ordine del giudice, non rilevando in
chiave strutturale il \emph{genus} della condotta rimasta inadempiuta,
analogamente a quanto è previsto nell'ordinamento francese. L'Adunanza
Plenaria del Consiglio di Stato, con sentenza n.~15 del 25 giugno 2014,
ha condiviso tale interpretazione estensiva, ponendo in rilievo un
argomento di diritto comparato che, prendendo a modello il sistema
francese, rileva che l'\emph{astreinte} si caratterizza per
un'indiscussa funzione sanzionatoria, essendo finalisticamente orientato
a costituire una pena per la disobbedienza alla statuizione giudiziaria,
anziché un risarcimento per il pregiudizio sofferto a causa di tale
inottemperanza. Viene quindi in rilievo l'argomento equitativo, in virtù
del quale, essendo le penalità di mora una pena e non una forma di
risarcimento, non sussiste un'inammissibile doppia riparazione di un
unico danno, ma l'aggiunta di una misura sanzionatoria ad una tutela
risarcitoria. L'Adunanza Plenaria precisa che la funzione deterrente e
general-preventiva delle penalità di mora verrebbe frustrata dalla
mancata erogazione della tutela ove vi sia già stato o possa essere
assicurato un integrale risarcimento. Il legislatore, recependo
l'orientamento fatto proprio dall'Adunanza Plenaria, ha novellato l'art.
114, comma 4, lett. \emph{e)}, c.p.a. aggiungendo con la legge n.~208
del 28 dicembre 2015 un ulteriore periodo che stabilisce che ``nei
giudizi di ottemperanza aventi ad oggetto il pagamento di somme di
denaro, la penalità di mora di cui al primo periodo decorre dal giorno
della comunicazione o notificazione dell'ordine di pagamento disposto
nella sentenza di ottemperanza; detta penalità non può considerarsi
manifestamente iniqua quando è stabilita in misura pari agli interessi
legali''.

Infine, l'art. 114, comma 4, lett. \emph{e)} c.p.a., in considerazione
della specialità del debitore pubblico, con specifico riferimento alle
difficoltà nell'adempimento collegate a vincoli normativi e di bilancio,
allo stato della finanza pubblica e alla rilevanza di specifici
interessi pubblici, ha aggiunto al limite negativo della manifesta
iniquità quello della sussistenza di ``altre ragioni
ostative''\footnote{FRANCESCO M. CARALLI, \emph{Il nuovo giudizio di
  ottemperanza, con particolare riguardo alle astreintes}, in
  Italiappalti.it, 2017, 11.}.

\hypertarget{gli-strumenti-di-prevenzione-dellinottemperanza-della-pubblica-amministrazione}{%
\section{Gli strumenti di prevenzione dell'inottemperanza della Pubblica
Amministrazione}\label{gli-strumenti-di-prevenzione-dellinottemperanza-della-pubblica-amministrazione}}

Accanto alle vere e proprie misure coattive fin qui esaminate, ve ne
sono altre di minore impatto quali il cosiddetto \emph{système d'aide à
l'exécution} e il \emph{mediateur de la Republique}, le cui funzioni,
dal 2011, sono esercitate dal \emph{Défenseur des droits}.

Al fine di prevenire un'inottemperanza o un'esecuzione incompleta od
erronea, nonché di pungolare l'amministrazione ad adempiere gli obblighi
discendenti dal giudicato, il ricorrente vittorioso può, attraverso
un'apposita domanda, chiedere chiarimenti in merito alle modalità con le
quali l'autorità soccombente debba conformarsi al giudicato ai tribunali
amministrativi e alle corti amministrative d'appello per quanto attiene
all'esecuzione delle loro decisioni\footnote{Cfr. Art. 12, decreto
  n.~831 del 3 luglio 1995 - \emph{CJA}, art. R 921-5.}, nonché al
Consiglio di Stato e, segnatamente, alla \emph{Section du rapport et des
études}, relativamente all'esecuzione delle sue decisioni, oltre a
quelle promananti dalle giurisdizioni amministrative
speciali\footnote{Cfr. Art. 59, decreto n.~766 del 30 luglio 1963 e
  succ. mod. - \emph{CJA}, art. R931-1 \emph{et} 2.}. I tribunali e le
corti d'appello vengono interessati della questione, nella persona del
loro presidente o del relatore che, per l'occasione viene designato. Il
Consiglio di Stato invece riceve i reclami nei confronti delle
amministrazioni inadempienti e interviene per mezzo della sezione
\emph{du rapport et des études}, affinché l'autorità competente
ottemperi al giudicato. In quest'ultimo caso, non è né necessario il
ministero di un difensore, né attendere il decorso di un termine
dilatorio, di regola di almeno tre mesi, quando si tratti di dare
esecuzione ad una decisione involgente misure urgenti ovvero nei casi di
rifiuto esplicito di adempiere. In giurisprudenza si propende per
l'inoppugnabilità in sede contenziosa di un eventuale rifiuto opposto
dal presidente della \emph{section du rapport et des études} alla
richiesta \emph{d'aide à l'exécution}\footnote{R. CHAPUS, \emph{Droit du
  contentieux administratif}, 12 ed., Paris, 2006, 1132.}.

Il \emph{mediateur} è una figura istituita con la legge del 3 gennaio
1973 (art. 11, al.~2) e succ. mod., alla quale è attribuito il compito
di invitare l'organo inadempiente a conformarsi al giudicato entro un
termine dal medesimo fissato, pena la diffusione della notizia del
biasimevole comportamento tenuto nella fattispecie dall'amministrazione,
attraverso una relazione da pubblicarsi sul cosiddetto \emph{Journal
officiel (``l'inexécution du jugement fera l'objet de sa part d'un
rapport spécial, publié au journal officiel'')}. In definitiva, si
tratterebbe di un mezzo di pressione morale, volto a far leva sulla
minaccia di pubblicità della vicenda. Nel 1994 il \emph{mediateur} aveva
redatto per la prima volta un rapporto speciale su di un caso di
persistente inottemperanza (\emph{Rapport du 20 septembre 1994, JO 14
octobre, p.~14588}), peraltro ancora in atto al momento del rapporto,
relativamente all'inesecuzione di una sentenza del Tribunale
amministrativo di Versailles del 22 giugno 1993 che aveva condannato il
comune di Mennecy ad erogare lo stipendio dovuto ad un suo
impiegato\footnote{R. CHAPUS, \emph{Droit du contentieux adminisratif},
  12 ed., Paris, 2006, 1135.}. L'istituto del \emph{mediateur} non
sarebbe però peculiare del diritto amministrativo, in quanto
l'ordinamento francese conosce una figura simile, detentrice di analoghi
poteri, il c.d. \emph{Défenseur des enfants} di cui all'rt. 10 della
legge 6 marzo 2000\footnote{J. VIGUIER, \emph{Le contentieux
  administratif}, Paris, 1998, 43.}.

\hypertarget{la-giustizia-amministrativa-nel-regno-unito}{%
\chapter{La giustizia amministrativa nel Regno
Unito}\label{la-giustizia-amministrativa-nel-regno-unito}}

\ldots{}

\hypertarget{il-sistema-spagnolo}{%
\chapter{Il sistema spagnolo}\label{il-sistema-spagnolo}}

\ldots{}

\newpage

Donec eu risus quis ante fermentum vestibulum in at ex. Nulla tellus
est, accumsan vel iaculis eget, lacinia at sapien. Maecenas sed ligula
dignissim, sagittis lorem at, euismod metus. Vivamus sodales elementum
accumsan. Nullam maximus risus vel nisi semper gravida. \textbf{Esempio
di citazione breve:} rinnovo della regolazione di alcuni tra gli
istituti centrali (Salmerón \& Seira s.d.). Aliquam ut dignissim purus.
Quisque quis arcu dignissim, pellentesque tellus a, faucibus ex.

Duis in leo ut felis porttitor consectetur. Ut luctus ante eget orci
vehicula, vel malesuada sapien placerat. \textbf{Esempio di citazione
dettagliata:} Lorem ipsum dolor sit amet, consectetur adipiscing elit.
Etiam molestie ex nec arcu tristique tempus. Aliquam ut dignissim purus.
Quisque quis arcu dignissim, pellentesque tellus a, faucibus ex. Sed
bibendum vulputate est nec scelerisque. Nulla a est molestie, tincidunt
purus non, volutpat dolor.

\begin{quote}
\emph{``La Legge 4/1999, del 13 gennaio, di modifica della LAP, ha
introdotto un insieme di misure destinate a rinnovare la regolazione di
alcuni tra gli istituti centrali o fondamentali del Diritto
Amministrativo spagnolo''} (Salmerón \& Seira s.d.).
\end{quote}

Sed ornare enim ut lectus dictum semper. Duis in leo ut felis porttitor
consectetur. Ut luctus ante eget orci vehicula, vel malesuada sapien
placerat.

\hypertarget{conclusioni}{%
\chapter{Conclusioni}\label{conclusioni}}

\hypertarget{riepilogo}{%
\section{Riepilogo}\label{riepilogo}}

pellentesque habitant morbi tristique senectus et netus et malesuada
fames ac turpis egestas. Nunc eleifend, ex a luctus porttitor, felis ex
suscipit tellus, ut sollicitudin sapien purus in libero. Nulla blandit
eget urna vel tempus. Praesent fringilla dui sapien, sit amet egestas
leo sollicitudin at.

\hypertarget{approfondimenti}{%
\section{Approfondimenti}\label{approfondimenti}}

Lorem ipsum dolor sit amet, consectetur adipiscing elit. Aliquam gravida
ipsum at tempor tincidunt. Aliquam ligula nisl, blandit et dui eu,
eleifend tempus nibh. Nullam eleifend sapien eget ante hendrerit
commodo. Pellentesque pharetra erat sit amet dapibus scelerisque.

Vestibulum suscipit tellus risus, faucibus vulputate orci lobortis eget.
Nunc varius sem nisi. Nunc tempor magna sapien, euismod blandit elit
pharetra sed. In dapibus magna convallis lectus sodales, a consequat sem
euismod. Curabitur in interdum purus. Integer ultrices laoreet aliquet.
Nulla vel dapibus urna. Nunc efficitur erat ac nisi auctor sodales.

\hypertarget{appendice-1-extra-a}{%
\chapter*{Appendice 1: Extra-A}\label{appendice-1-extra-a}}
\addcontentsline{toc}{chapter}{Appendice 1: Extra-A}

Vivamus hendrerit rhoncus interdum. Sed ullamcorper et augue at porta.
Suspendisse facilisis imperdiet urna, eu pellentesque purus suscipit in.
Integer dignissim mattis ex aliquam blandit. Curabitur lobortis quam
varius turpis ultrices egestas.

\footnotesize
\singlespacing
\setlength{\parindent}{0in}

\hypertarget{riferimenti}{%
\chapter*{Riferimenti}\label{riferimenti}}
\addcontentsline{toc}{chapter}{Riferimenti}

\hypertarget{refs}{}
\begin{CSLReferences}{1}{0}
\leavevmode\vadjust pre{\hypertarget{ref-salmeronRiformaProcedimentoAmministrativo}{}}%
Salmerón, M.F. \& Seira, C.C., Riforma Del Procedimento Amministrativo
in {Spagna}: La {Legge} 4/1999, Del 13 Gennaio, Di Modifica Della
{Legge} 30/1992, Del 26 Novembre, de {Régimen Jurídico} de Las
{Administraciones Públicas} y Del {Procedimiento Administrativo Común}.
Available at:
\url{https://digilander.libero.it/bhilex/studi/artprammsez_I_II1.htm?utm_source=pocket_mylist}
{[}Consultato ottobre 8, 2022{]}.

\end{CSLReferences}



\end{document}
