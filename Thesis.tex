\documentclass[12pt,it,a4paper,]{report}
\usepackage{XCharter}
\usepackage[condensed]{roboto}

% Overwrite \begin{figure}[htbp] with \begin{figure}[H]
\usepackage{float}
\let\origfigure=\figure
\let\endorigfigure=\endfigure
\renewenvironment{figure}[1][]{%
\origfigure[b]
}{%
\endorigfigure
}

% fix for pandoc 1.14
\providecommand{\tightlist}{%
  \setlength{\itemsep}{0pt}\setlength{\parskip}{0pt}}

% TP: hack to truncate list of figures/tables.
\usepackage{truncate}
\usepackage{caption}
\usepackage{tocloft}
% TP: end hack

\usepackage{amssymb,amsmath}
\usepackage{ifxetex,ifluatex}

% Only use fixltx2e if using pre-2015 kernels
\begingroup\expandafter\expandafter\expandafter\endgroup
\expandafter\ifx\csname IncludeInRelease\endcsname\relax
  \usepackage{fixltx2e}
\fi

\ifnum 0\ifxetex 1\fi\ifluatex 1\fi=0 % if pdftex
  \usepackage[T1]{fontenc}
  \usepackage[utf8]{inputenc}
\else % if luatex or xelatex
  \ifxetex
    \usepackage{mathspec}
    \usepackage{xltxtra,xunicode}
  \else
    \usepackage{fontspec}
  \fi
  \defaultfontfeatures{Mapping=tex-text,Scale=MatchLowercase}
  \newcommand{\euro}{€}
    \setmainfont{XCharter}
\fi
% use upquote if available, for straight quotes in verbatim environments
\IfFileExists{upquote.sty}{\usepackage{upquote}}{}
% use microtype if available
\IfFileExists{microtype.sty}{%
\usepackage{microtype}
\UseMicrotypeSet[protrusion]{basicmath} % disable protrusion for tt fonts
}{}
\ifxetex
  \usepackage[setpagesize=false, % page size defined by xetex
              unicode=false, % unicode breaks when used with xetex
              xetex]{hyperref}
\else
  \usepackage[unicode=true]{hyperref}
\fi
\hypersetup{breaklinks=true,
            bookmarks=true,
            pdfauthor={Marina Chiapello},
            pdftitle={I rimedi all'inottemperanza della Pubblica Amministrazione: il confronto tra le discipline di alcuni ordinamenti europei},
            colorlinks=true,
            citecolor=blue,
            urlcolor=blue,
            linkcolor=magenta,
            pdfborder={0 0 0}}
\urlstyle{same}  % don't use monospace font for urls
\setlength{\parindent}{0pt}
\setlength{\parskip}{6pt plus 2pt minus 1pt}
\setlength{\emergencystretch}{3em}  % prevent overfull lines
\setcounter{secnumdepth}{5}
\ifxetex
  \usepackage{polyglossia}
  \setmainlanguage{italian}
\else
  \usepackage[it]{babel}
\fi

% % \newlength{\cslhangindent}
% \setlength{\cslhangindent}{1.5em}
% \newenvironment{cslreferences}%
%   {}%
%   {\par}
% 
\newlength{\cslhangindent}
\setlength{\cslhangindent}{1.5em}
\newlength{\csllabelwidth}
\setlength{\csllabelwidth}{3em}
\newenvironment{CSLReferences}[2] % #1 hanging-ident, #2 entry spacing
 {% don't indent paragraphs
  \setlength{\parindent}{0pt}
  % turn on hanging indent if param 1 is 1
  \ifodd #1 \everypar{\setlength{\hangindent}{\cslhangindent}}\ignorespaces\fi
  % set entry spacing
  \ifnum #2 > 0
  \setlength{\parskip}{#2\baselineskip}
  \fi
 }%
 {}
\usepackage{calc}
\newcommand{\CSLBlock}[1]{#1\hfill\break}
\newcommand{\CSLLeftMargin}[1]{\parbox[t]{\csllabelwidth}{#1}}
\newcommand{\CSLRightInline}[1]{\parbox[t]{\linewidth - \csllabelwidth}{#1}\break}
\newcommand{\CSLIndent}[1]{\hspace{\cslhangindent}#1}



% Table of contents formatting
\renewcommand{\contentsname}{Table of Contents}
\setcounter{tocdepth}{3}


% Global Footnote Counter
\counterwithout{footnote}{chapter}


% Headers and page numbering
\usepackage{fancyhdr}
\pagestyle{plain}

% Following package is used to add background image to front page
\usepackage{wallpaper}

% Table package
\usepackage{ctable}% http://ctan.org/pkg/ctable

% Deal with 'LaTeX Error: Too many unprocessed floats.'
\usepackage{morefloats}
% or use \extrafloats{100}
% add some \clearpage

% % Chapter header
% \usepackage{titlesec, blindtext, color}
% \definecolor{gray75}{gray}{0.75}
% \newcommand{\hsp}{\hspace{20pt}}
% \titleformat{\chapter}[hang]{\Huge\bfseries}{\thechapter\hsp\textcolor{gray75}{|}\hsp}{0pt}{\Huge\bfseries}

% % Fonts and typesetting
% \setmainfont[Scale=1.1]{Helvetica}
% \setsansfont[Scale=1.1]{Verdana}

% FONTS
\usepackage{xunicode}
\usepackage{xltxtra}
\defaultfontfeatures{Mapping=tex-text} % converts LaTeX specials (``quotes'' --- dashes etc.) to unicode
% \setromanfont[Scale=1.01,Ligatures={Common},Numbers={OldStyle}]{Palatino}
% \setromanfont[Scale=1.01,Ligatures={Common},Numbers={OldStyle}]{Adobe Caslon Pro}
%Following line controls size of code chunks
% \setmonofont[Scale=0.9]{Monaco}
%Following line controls size of figure legends
% \setsansfont[Scale=1.2]{Optima Regular}

% CODE BLOCKS
\usepackage[utf8]{inputenc}
\usepackage{listings}
\usepackage{color}

% JAVA CODE BLOCKS
%\definecolor{backcolour}{RGB}{242,242,242}
%\definecolor{javared}{rgb}{0.6,0,0}
%\definecolor{javagreen}{rgb}{0.25,0.5,0.35}
%\definecolor{javapurple}{rgb}{0.5,0,0.35}
%\definecolor{javadocblue}{rgb}{0.25,0.35,0.75}

\lstdefinestyle{javaCodeStyle}{
  language=Java,                         % the language of the code
  backgroundcolor=\color{backcolour},    % choose the background color; you must add \usepackage{color} or \usepackage{xcolor}
  basicstyle=\fontsize{10}{8}\sffamily,
  breakatwhitespace=false,
  breaklines=true,
  keywordstyle=\color{javapurple}\bfseries,
  stringstyle=\color{javared},
  commentstyle=\color{javagreen},
  morecomment=[s][\color{javadocblue}]{/**}{*/},
  captionpos=t,                          % sets the caption-position to bottom
  frame=single,                          % adds a frame around the code
  numbers=left,
  numbersep=10pt,                         % margin between number and code block
  keepspaces=true,                       % keeps spaces in text, useful for keeping indentation of code (possibly needs columns=flexible)
  columns=fullflexible,
  showspaces=false,                      % show spaces everywhere adding particular underscores; it overrides 'showstringspaces'
  showstringspaces=false,                % underline spaces within strings only
  showtabs=false,                        % show tabs within strings adding particular underscores
  tabsize=2                              % sets default tabsize to 2 spaces
}

%Attempt to set math size
%First size must match the text size in the document or command will not work
%\DeclareMathSizes{display size}{text size}{script size}{scriptscript size}.
%\DeclareMathSizes{12}{13}{7}{7}

% ---- CUSTOM AMPERSAND
% \newcommand{\amper}{{\fontspec[Scale=.95]{Adobe Caslon Pro}\selectfont\itshape\&}}

% HEADINGS
\usepackage{sectsty}
\usepackage[normalem]{ulem}
\chapterfont{\sffamily\mdseries\huge}
\sectionfont{\sffamily\bfseries\Large}
\subsectionfont{\sffamily\mdseries\scshape\large}
\subsubsectionfont{\sffamily\bfseries\upshape\large}

\usepackage{tocloft}
\renewcommand{\cfttoctitlefont}{\hfill\sffamily\mdseries\huge}
\renewcommand{\cftchapfont}{\normalfont\sffamily\bfseries}   
\renewcommand{\cftsecfont}{\normalfont\sffamily}   
\renewcommand{\cftchappagefont}{\normalfont\sffamily\bfseries}   
\renewcommand{\cftsecpagefont}{\normalfont\sffamily}   

% \chapterfont{\OswaldMedium\huge}
% \sectionfont{\OswaldSemiBold\Large}
% \subsectionfont{\OswaldMedium\scshape\large}
% \subsubsectionfont{\OswaldSemiBold\upshape\large}
% \sectionfont{\rmfamily\mdseries\Large}
% \subsectionfont{\rmfamily\mdseries\scshape\large}
% \subsubsectionfont{\rmfamily\bfseries\upshape\large}

% Set figure legends and captions to be smaller sized sans serif font
\usepackage[font={footnotesize,sf}]{caption}

\usepackage{siunitx}

% Adjust spacing between lines to 1.5
\usepackage{setspace}
% \onehalfspacing
\doublespacing
\raggedbottom

% Set margins
\usepackage[top=1.5in,bottom=1.5in,left=1.5in,right=1.4in]{geometry}
\setlength\parindent{0.5in} % indent at start of paragraphs (set to 0.3?)
\setlength{\parskip}{9pt}
\usepackage{indentfirst}

% Add space between pararaphs
% http://texblog.org/2012/11/07/correctly-typesetting-paragraphs-in-latex/
% \usepackage{parskip}
% \setlength{\parskip}{\baselineskip}

% Set colour of links to black so that they don't show up when printed
\usepackage{hyperref}
\hypersetup{colorlinks=false, linkcolor=black}

% Tables
\usepackage{booktabs}
\usepackage{threeparttable}
\usepackage{array}
\usepackage{makecell}
\newcolumntype{x}[1]{%
>{\centering\arraybackslash}m{#1}}%

% Allow for long captions and float captions on opposite page of figures
% \usepackage[rightFloats, CaptionBefore]{fltpage}

% Don't let floats cross subsections
% \usepackage[section,subsection]{extraplaceins}

% Rotate images and tables
\usepackage{float}
\usepackage{pdfpages}
\usepackage{pdflscape}
\usepackage{graphicx}
\usepackage{rotating}

% Custom math
\usepackage{bbold}
\DeclareMathOperator*{\argmin}{\arg\!\min}

% For use of \cref and \Cref used by pandoc secnos
\usepackage{cleveref}

\begin{document}


    \begin{titlepage}
        
        % \noindent
        % \begin{minipage}[t]{0.19\textwidth}{
        %     \vspace{-4mm}
        %     {\includegraphics[scale=1.15]{style/univ_logo.pdf}}
        %   }
        % \end{minipage}
        % \hspace{0.5cm}
        % \begin{minipage}[t]{1.81\textwidth}
        % {
        %   \setstretch{1.42}
        %   {\textsc{Università degli Studi di Milano - Bicocca}} \\
        %   \textbf{} \\
        %   \textbf{Dipartimento di Giurisprudenza} \\
        %   \textbf{Corso di Laurea in Scienze dei servizi giuridici} \\
        %   \par
        % }
        % \end{minipage}
        
	% \vspace{20mm}

        
        \noindent

	\vspace{1mm}
	\begin{center}
            {
              \setstretch{1.42}
              {\textsc{Università degli Studi di Milano - Bicocca}} \\
              \textbf{} \\
              \textbf{Dipartimento di Giurisprudenza} \\
              \textbf{Corso di Laurea in Scienze dei servizi
giuridici} \\
              \par
            }
        \end{center}
        
	\vspace{4mm}
        
	\begin{center}
          \begin{minipage}[t]{0.19\textwidth}{
              \vspace{-4mm}
              {\includegraphics[scale=1.5]{style/univ_logo.pdf}}
            }
          \end{minipage}
        \end{center}

        \vspace{10mm}
        
      	\begin{center}
            {\LARGE{
                    \setstretch{1.2}
                    \textbf{I rimedi all'inottemperanza della Pubblica
Amministrazione: il confronto tra le discipline di alcuni ordinamenti
europei}
                    \par
            }}
        \end{center}

        
        \vspace{7mm}

        \noindent
        {\large \textbf{Relatore:}  Prof.~Alessandro Squazzoni } \\

        %         % \noindent
        % {\large \textbf{Correlatore:} } \\
        %         
        % \vspace{1mm}

        \begin{flushright}
            {\large \textbf{Relazione della prova finale di:}} \\
            \large{Marina Chiapello} \\
            \large{Matricola 787839} 
        \end{flushright}
        
        \vspace{1mm}
        \begin{center}
            {\large{\bf Anno Accademico 2021-2022}}
        \end{center}

        \restoregeometry
        
    \end{titlepage}




% This is where the converted markdown files will go 
\vspace*{\fill}

\noindent \hfill{
\textit{
A Giovanni e alle mie figlie
}} \vspace*{\fill} \pagenumbering{gobble} \newpage

\pagenumbering{roman}
\setcounter{page}{1}

\hypertarget{ringraziamenti}{%
\chapter*{Ringraziamenti}\label{ringraziamenti}}
\addcontentsline{toc}{chapter}{Ringraziamenti}

Ringrazio il mio relatore, Prof.~Alessandro Squazzoni, per l'attenzione
che mi ha dedicato nella stesura di questo lavoro.

\newpage

\pagenumbering{gobble}

\tableofcontents

\newpage

\hypertarget{abbreviazioni}{%
\chapter*{Abbreviazioni}\label{abbreviazioni}}
\addcontentsline{toc}{chapter}{Abbreviazioni}

\begin{tabbing}
\hspace{12em} \= \hspace{60em} \= \kill
\textsc{Italia} \> \\
\textbf{c.p.a.} \> Codice del Processo Amministrativo \\
\textbf{P. A.} \> Pubblica Amministrazione \\
\textbf{D.L.} \> Decreto legge \\
\textbf{D. lgs.} \> Decreto legislativo \\
\textbf{Cost.} \> Costituzione della Repubblica \\
  \textbf{Corte Cost.} \> Corte Costituzionale \\
  \textbf{Cons. St.} \> Consiglio di Stato \\
\textbf{L.} \> Legge \\
\textbf{L. costituzionale} \> Legge costituzionale \\
\textbf{D.M.} \> Decreto ministeriale \\
\textbf{D.P.R.} \> Decreto del Presidente della Repubblica \\
\textbf{D.Lgs.} \> Decreto legislativo \\

 \> \\
\textsc{Germania} \> \\
\textbf{BVerwG} \> Bundesverwaltungsgericht \\
\textbf{BVerwGE} \> Entscheidungen des Bundesverwaltungsgerichts \\
\textbf{FBA} \> Folgenbeseitigungsanspruch \\
\textbf{NVwZ} \> Neue Zeitschrift für Verwaltungsrecht \\
\textbf{VwGO} \> Verwaltungsgerichtsordnung \\
 \> \\
\textsc{Francia} \> \\
\textbf{C. trib. mar} \> Code des Tribunaux Administratifs \\
\textbf{} \>   et des Cours Administratives d’Appel \\
\textbf{CAA} \> Cour Administrative d'Appel \\
\textbf{CC} \> Conseil Constitutionnel \\
\textbf{CJA} \> Code de Justice Administrative \\
\textbf{CE} \> Conseil d'Etat \\
\textbf{DA} \> Droit Administratif \\
\textbf{TA} \> Tribunal administratif \\
 \> \\
\textsc{Regno Unito} \> \\
\textbf{CPR} \> Civil Procedure Rules \\
\textbf{CJCA} \> Criminal Justice and Courts Act \\
\textbf{HRA} \> Human Rights Act \\
\textbf{QB} \> Queen's Bench \\
 \> \\
\textsc{Spagna} \> \\
\textbf{LJCA} \> Ley de la Jurisdicción Contencioso-Administrativa \\

\end{tabbing}
\newpage

\setcounter{page}{1}
\pagenumbering{arabic}
\doublespacing
\setlength{\parindent}{0.5in}

\hypertarget{introduzione}{%
\chapter*{Introduzione}\label{introduzione}}
\addcontentsline{toc}{chapter}{Introduzione}

Questo studio comparativo si propone di osservare e descrivere gli
strumenti a disposizione di coloro che si trovano a far fronte
all'inottemperanza di una pubblica amministrazione in alcuni ordinamenti
europei. Tale inottemperanza può consistere sia nella mancata esecuzione
o nell'esecuzione solo parziale di una sentenza pronunciata dal giudice
amministrativo, sia nel diniego o nell'omissione di un provvedimento che
la legge stabilisce come dovuto. Sebbene in generale il ruolo della
comparazione giuridica sia ormai ridotto nel quadro di armonizzazione
dell'Unione Europea, si nota che i sistemi dei diversi Stati mantengono
ancora rilevanti peculiarità, il cui sviluppo è frutto di un'autonoma
evoluzione storica. Emerge inoltre come il singolo individuo, sia egli
semplice utente dei servizi offerti al pubblico o parte nel processo
amministrativo, in tutti gli ordinamenti considerati, non venga più
considerato unicamente un soggetto passivo sottoposto al potere
dell'autorità amministrativa, ma quale beneficiario invece di una serie
di garanzie volte alla tutela effettiva dei suoi diritti soggettivi e
interessi legittimi. Nel primo capitolo si tratta di quel particolare
istituto che è il giudizio di ottemperanza, proprio della giustizia
amministrativa italiana, in quanto si prevede che il giudice
amministrativo eserciti anche una giurisdizione estesa al merito, la
quale comporta che egli possa sostituirsi, direttamente o attraverso la
nomina di un commissario \emph{ad acta}, all'amministrazione
inadempiente. Ciò non avviene invece in altri sistemi di giustizia
amministrativa, quali quello tedesco e francese, di cui si tratta
rispettivamente nel secondo e terzo capitolo, improntati ad una
separazione dei poteri tra amministrazione e giurisdizione decisamente
più rigida e nei quali, a presidio dell'esecuzione del giudicato da
parte dell'amministrazione, si predilige l'utilizzo delle misure di
coercizione indiretta consistenti nello \emph{Zwangsgeld} e
nell'\emph{astreinte}.\\
Si procede poi nel capitolo quattro affrontando l'evoluzione nel tempo
degli \emph{administrative tribunals} in Gran Bretagna che, a dispetto
del nome, non sono organi giudiziari, ma amministrativi, sebbene le loro
decisioni siano in molti casi simili alle pronunce delle corti. A queste
ultime spetta invece, sempre nell'ambito delle controversie tra i
cittadini e l'autorità amministrativa secondo la procedura di
\emph{judicial review}, un controllo di legittimità sull'azione delle
stesse autorità. Nel quinto capitolo, infine, si osserva come anche nel
sistema spagnolo, a partire dalla \emph{Ley de la Jurisdicción
Contencioso-Administrativa} del 1956, la giustizia amministrativa
diviene specializzata nel controllo giuridico delle pubbliche
amministrazioni con la costituzione di udienze territoriali, ovvero di
tribunali nelle varie province, nelle quali vi è una sala del
contenzioso amministrativo. Al vertice troviamo il tribunale supremo, le
cui sale dedicate al contenzioso amministrativo conoscono delle
controversie relative ai provvedimenti dell'amministrazione centrale
dello Stato. Si ritiene utile rilevare anche come, in alcuni di questi
ordinamenti esaminati, abbia assunto una notevole importanza l'istituto
del difensore civico, organo preposto alla verifica dei casi di
\emph{maladministration}, quando cioè l'attività dell'amministrazione
non sempre si rivela ispirata ai principi di proporzionalità,
trasparenza, imparzialità e ragionevolezza e ciò diviene causa di
disservizi e malfunzionamenti che spesso impattano anche nella sfera
giuridica dei cittadini.

\hypertarget{la-giustizia-amministrativa-in-italia}{%
\chapter{La giustizia amministrativa in
Italia}\label{la-giustizia-amministrativa-in-italia}}

Questa parte descrive la situazione italiana per quanto riguarda gli
strumenti giuridici di tutela verso l'inottemperanza di una pubblica
amministrazione. In questo capitolo e nel successivo, relativo alla
situazione tedesca, viene fatto riferimento allo studio sull'argomento
di Federico Secchi\footnote{F. \textsc{Secchi}, \emph{L'esecuzione del
  giudicato amministrativo nell'esperienza italiana e tedesca: le
  soluzioni al problema dell'inottemperanza}, PhD thesis, University of
  Trento, 2010.}, rispettivanente alla prima\footnote{Ivi, pp.~23-80.
  ``Parte I - L'esperienza italiana''.} e alla seconda parte\footnote{Ivi,
  pp.~81-188. ``Parte II - L'esperienza tedesca''.}.

\hypertarget{lattuazione-del-giudicato-il-giudizio-di-ottemperanza}{%
\section{L'attuazione del giudicato: il giudizio di
ottemperanza}\label{lattuazione-del-giudicato-il-giudizio-di-ottemperanza}}

Il giudizio di ottemperanza rappresenta uno strumento di particolare
incisività per garantire nei confronti dell'amministrazione l'attuazione
delle decisioni giudiziali, come stabilito all'art. 112 c.p.a. e in
risposta ai principi di di effettività ed efficacia della tutela
giurisdizionale sanciti dagli artt. 24 e 113 Cost.\footnote{Art. 113
  Cost.: ``Contro gli atti della pubblica amministrazione è sempre
  ammessa la tutela giurisdizionale dei diritti e degli interessi
  legittimi dinanzi agli organi di giurisdizione ordinaria e
  amministrativa''.}, nonché dall'art. 47 della Carta dei diritti
fondamentali dell'UE e dall'art. 13 della CEDU. In base alla legge di
abolizione del contenzioso amministrativo del 1865, dell'atto
amministrativo lesivo di un diritto si poteva chiedere la modifica o
l'annullamento esclusivamente con ricorso gerarchico all'autorità
amministrativa competente e l'amministrazione che aveva emesso l'atto
aveva semplicemente l'obbligo di conformarsi al giudicato del tribunale
civile, ma tale obbligo rimaneva incoercibile, in quanto non
accompagnato da un meccanismo volto a garantirne l'effettiva
osservanza\footnote{L. 2248 del 1865, Art. 4, c.~2: \emph{``L'atto
  amministrativo non potrà essere revocato o modificato se non sovra
  ricorso alle competenti Autorità amministrative, le quali si
  conformeranno al giudicato dei Tribunali in quanto riguarda il caso
  deciso''}.}.

In origine il giudizio di ottemperanza, così come introdotto dall'art. 4
n.~4 della legge 31 marzo 1889, n.~5992, era ammesso solo per le
sentenze passate in giudicato dell'Autorità giudiziaria ordinaria,
aventi per oggetto diritti civili e politici. E' a partire dagli anni
venti del secolo scorso che la giurisprudenza del Consiglio di Stato
estende analogicamente l'applicabilità dell'istituto anche
all'esecuzione del giudicato amministrativo, ma esso trova un
riconoscimento normativo solo con l'art. 37 della legge 6 dicembre 1971,
n.~1034, istitutiva dei tribunali amministrativi regionali. Infine,
viene compiutamente disciplinato con il decreto legislativo 2 luglio
2010, n.~104, in attuazione della legge delega 18 giugno 2009, n.~69,
per il riordino del processo amministrativo. Presupposto per
l'attivazione del giudizio di ottemperanza è l'inosservanza da parte
dell'amministrazione del dovere di esecuzione della sentenza e l'oggetto
del giudizio è costituito dalla verifica se l'amministrazione abbia o
meno adempiuto l'obbligo nascente dal giudicato, ovvero se abbia o meno
attribuito all'interessato quell'utilità che la sentenza ha riconosciuto
come dovuta. Mentre nella fase esecutiva della sentenza di condanna del
giudice civile che ha per oggetto diritti soggettivi e stabilisce cosa
deve fare l'amministrazione soccombente nello specifico ci si trova di
fronte a una sentenza molto chiara nello stabilire cosa si pretende dal
``debitore'', nel caso della sentenza del giudice amministrativo la
condotta successiva non è sempre segnata con certezza: il vincolo
conformativo ha un'intensità diversa a seconda del vizio accolto e
l'amministrazione può non essere tenuta solo ad un comportamento
specifico. Il giudizio di ottemperanza non è la mera attuazione di un
giudicato già preciso e sicuro della fase di cognizione, ma deve
ricostruirne il significato. E' un giudizio c.d. ``misto'',
necessariamente di esecuzione ed eventualmente di cognizione,
assoggettato al termine di prescrizione ordinario di dieci anni,
decorrente dalla data del passaggio in giudicato della
sentenza\footnote{Art. 114, c.~1, c.p.a.}. La fase di cognizione non è
necessaria quando l'attività amministrativa successiva al giudicato
abbia carattere vincolato, ovvero quando le statuizioni della sentenza
impartiscano all'amministrazione comandi tassativi e talmente puntuali
da non lasciare spazio alcuno all'esercizio dei suoi poteri
discrezionali. Per converso, gli spazi liberi che possono residuare al
giudicato rendono la \emph{regola iuris} dallo stesso dettata
``implicita, elastica, condizionata e incompleta'' e, come tale,
suscettibile di essere chiarita nel contesto del giudizio di
ottemperanza\footnote{In \textsc{F. Secchi}, \emph{op. cit.}, p.~45, si
  riporta l'osservazione di M. \textsc{Nigro} -- A. \textsc{Nigro} -- E.
  \textsc{Cardi}, \emph{Giustizia amministrativa}, Bologna, Il Mulino,
  4a ed 1995, 332,333.}. Sempre riguardo alla natura del rito e alla
compenetrazione di momenti cognitivi con momenti esecutivi, la Corte
costituzionale ha chiarito che ``il giudizio di ottemperanza assume
diversi modi di essere in relazione alla situazione concreta, alla
statuizione giudiziale da attuare, alla natura dell'atto censurato. Il
particolare il giudizio di ottemperanza può costituire semplice giudizio
esecutivo che si aggiunge al procedimento espropriativo, disciplinato
dal codice di procedura civile; lo stesso giudizio può essere
preordinato al compimento di operazioni materiali o (\ldots) alla
sollecitazione di attività provvedimentale amministrativa (\ldots) può
essere utilizzato anche in difetto di completa individuazione del
contenuto della prestazione o attività oggetto del dovere
dell'Amministrazione (\ldots) non deve modellarsi necessariamente anche
nei presupposti sul processo esecutivo ordinario, tenuto conto delle
peculiarità funzionali del giudizio amministrativo, con potenzialità
sostitutive e intromissive nell'azione amministrativa incomparabili ai
poteri del giudice dell'esecuzione del processo civile''\footnote{In
  \textsc{F. Secchi}, \emph{op. cit.}, p.~46, si fa riferimento a Corte
  Cost., ord. 10 dicembre 1998, n.~406, in \emph{Foro amm.}, 2000, 751.}.
Il ricorso per l'ottemperanza va proposto nelle forme ordinarie, quindi
notificato all'amministrazione e a tutte le altre parti del giudizio di
merito. Il ricorrente deve depositare una copia autentica della sentenza
di cui si chiede l'esecuzione, con l'eventuale prova del passaggio in
giudicato\footnote{Art. 114, c.~2, c.p.a.}. In passato il ricorso doveva
essere preceduto dalla notifica all'amministrazione di una diffida a
provvedere, ma oggi il codice, all'art. 114, c.~1, stabilisce che tale
adempimento non è più necessario. Il riparto di competenza ha carattere
funzionale, ai sensi dell'art. 14, c.~3, c.p.a. Per l'esecuzione della
sentenza amministrativa, competente è il giudice che ha pronunciato la
sentenza. Nel caso si tratti di sentenza emessa dal Consiglio di Stato,
esso può essere competente in unico grado, ma se la sentenza del Tar è
stata confermata in appello, la competenza spetta sempre al Tar. Qualora
invece si tratti dell'esecuzione della sentenza di un giudice ordinario
o di un altro giudice speciale diverso dal giudice amministrativo, la
competenza spetta sempre al Tar nella cui circoscrizione ha sede il
giudice che ha emesso la sentenza da eseguire\footnote{Art. 113 c.p.a.}.

Per quanto riguarda l'esecuzione delle sentenze del giudice
amministrativo, il ricorso per l'ottemperanza è esperibile
indipendentemente dal fatto che esse siano passate in giudicato o
solamente esecutive e, ai fini del ricorso, non rileva se rispetto a
queste sentenze inadempiente sia l'amministrazione o una parte privata.
Nel caso di una sentenza non ancora passata in giudicato, l'esecuzione
riguarda una statuizione che non ha ancora carattere di definitività.
Con la sentenza n.~5352/2002 il Consiglio di Stato ha sostenuto che
l'esecuzione della sentenza non ancora passata in giudicato non dovrebbe
mai determinare un assetto ``definito e immutabile'', perché altrimenti
verrebbe frustrato l'esito pratico di un eventuale appello contro la
sentenza\footnote{Cons. Stato, sez. IV, 9 ottobre 2002, n.~5352.}. In
generale, la stessa giurisprudenza che ha orientato anche la redazione
del codice del processo amministrativo equipara la sentenza esecutiva
alla sentenza passata in giudicato ai fini dell'ammissibilità del
giudizio di ottemperanza, ma precisa che il giudice dell'ottemperanza,
se la sentenza non sia passata in giudicato, ne determina le modalità
esecutive\footnote{Art. 114, c.~4, lett. c.}, motivo per cui sembra
riconosciuta la necessità che l'esecuzione di tale sentenza non
pregiudichi le ragioni di un eventuale appello. In base all'art. 114,
c.~2, lett. \emph{c} ed \emph{e}, il ricorso per l'ottemperanza è
esperibile anche per l'esecuzione delle sentenze passate in giudicato
del giudice ordinario e dei giudici speciali avanti ai quali non sia
previsto un giudizio di ottemperanza, nonché per l'esecuzione dei lodi
arbitrali esecutivi divenuti inoppugnabili. In questi casi però il
giudizio di ottemperanza si caratterizza sul piano soggettivo come
strumento di esecuzione specifica nei confronti di un'amministrazione,
in quanto non è ammesso per soggetti diversi.

L'elemento decisamente caratteristico del giudizio di ottemperanza è
individuato dall'art. 134, c.~1, lett. \emph{a}, c.p.a., laddove si
prevede che il giudice amministrativo, nello stesso giudizio, esercita
una giurisdizione estesa al merito. Tale previsione comporta che il
giudice amministrativo possa sostituirsi, direttamente o attraverso un
commissario da esso eventualmente nominato, all'amministrazione
inadempiente. Questa possibilità di sostituzione comporta che nel
giudizio di ottemperanza non possa opporsi al giudice alcuna riserva di
potere all'amministrazione, in quanto la necessità di dare esecuzione
alla sentenza prevale anche su ogni esigenza di salvaguardia delle
prerogative dell'amministrazione stessa. Inoltre, la giurisprudenza
largamente prevalente ammette che il giudice dell'ottemperanza possa
compiere anche attività discrezionali, disattendendo l'assunto secondo
cui il medesimo giudice potrebbe sostituirsi all'amministrazione solo
nei limiti delle statuizioni puntali del giudicato, in quanto le
ulteriori scelte discrezionali dell'amministrazione non sarebbero di
pertinenza dell'autorità giurisdizionale. L'attività del giudice
dell'ottemperanza o del commissario \emph{ad acta} da lui nominato
infatti non costituisce manifestazione in senso stretto di
discrezionalità amministrativa, poiché essa è essenzialmente preordinata
al conseguimento dell'interesse del ricorrente e non già all'interesse
primario perseguito dall'amministrazione. Vi sono due ipotesi in cui
l'amministrazione viola il giudicato del giudice amministrativo: una si
verifica quando la sentenza stabilisce che essa non deve adottare un
provvedimento e la seconda quando l'amministrazione è inadempiente,
quindi rispetto ad una condotta omissiva, con un'inerzia elusiva del
giudicato. Con l'art. 21 \emph{septies} della legge 241/1990, introdotto
dalla legge 15 del 2005 di riforma del procedimento amministrativo, gli
atti elusivi sono stati assimilati a quelli assunti in violazione del
giudicato, ammettendosi anche nei loro confronti il ricorso per
l'ottemperanza\footnote{In \textsc{F. Secchi}, \emph{op. cit.}, p.~39,
  troviamo Cons. Stato, sez. IV, 10 aprile 1998, n.~565, in \emph{Foro
  amm.}, 1998, 1021 ove si chiarisce che ``Il ricorso per l'ottemperanza
  è ammissibile non solo quando l'amministrazione mantiene un
  comportamento inerte di fronte al decisium del giudice, ma anche
  quando il provvedimento da essa adottato, in affermata ottemperanza al
  giudicato stesso, è invece palesemente elusivo dei principi e delle
  regole in esso enunciati. Ciò in quanto il giudicato amministrativo ha
  un contenuto complesso, non limitato agli effetti demolitori e
  ripristinatori rivolti al passato, ma comprensivo anche degli effetti
  confermativi rivolti al futuro e consistenti nei vincoli imposti
  all'autorità amministrativa nella rinnovazione del provvedimento
  annullato, in relazione ai vizi di legittimità riconosciuti
  esistenti''.}.

L'ampia gamma di poteri spendibili dal giudice dell'ottemperanza ammanta
lo stesso istituto di originalità, laddove nella maggior parte delle
principali esperienze continentali domina, quale strumento a presidio
dell'esecuzione del giudicato da parte dell'amministrazione, il rimedio
delle misure patrimoniali di tipo compulsorio, quali lo
\emph{Zwangsgeld} o l'\emph{astrainte}, dove i sistemi sono improntati,
in punto di esecuzione della sentenza, a una rigida separazione tra i
poteri dell'amministrazione e quelli della giurisdizione, essendo
inibita al giudice qualsiasi ingerenza nell'attività esecutiva del
giudicato amministrativo che rimane appannaggio dell'amministrazione.
Un'eccezione è rappresentata dal modello austriaco della
\emph{Säumnisbeschwerde} quale rimedio avverso il silenzio in
inadempimento dell'amministrazione, prevedendo il legislatore austriaco
al \emph{§ 63/2 VwGG} la possibilità per il giudice amministrativo di
surrogarsi all'amministrazione inadempiente designando l'amministrazione
o il tribunale chiamato a eseguire la sua decisione e ``consacrando così
una possibile sostituzione del potere giudiziario all'amministrazione
attiva (\ldots) sul fronte dell'esecuzione''\footnote{In
  \textsc{F. Secchi}, \emph{op. cit.}, p.~43. si riporta C.
  \textsc{Fraenkel-Haeberle}, \emph{Giurisdizione sul silenzio e
  discrezionalità amministrativa: Germania, Austria, Italia}, Trento,
  Università degli studi di Trento, Dipartimento di scienze giuridiche,
  2004.}.

\hypertarget{i-poteri-sostitutivi-indiretti-il-commissario-ad-acta}{%
\section{\texorpdfstring{I poteri sostitutivi indiretti: il commissario
\emph{ad
acta}}{I poteri sostitutivi indiretti: il commissario ad acta}}\label{i-poteri-sostitutivi-indiretti-il-commissario-ad-acta}}

Il giudice può adottare direttamente i provvedimenti necessari a
un'integrale esecuzione del giudicato quando essi siano vincolati,
altrimenti si deve limitare a dichiarare l'obbligo di provvedere
assegnando all'amministrazione un termine, nonché disponendo che si
nomini un commissario il quale agisca al posto dell'amministrazione, se
questa non ottemperi entro il termine assegnato. Il commissario \emph{ad
acta} è chiamato a esercitare quei poteri che il giudice
dell'ottemperanza potrebbe esercitare anche in via diretta, attraverso
un intervento nel merito volto a sostituire l'amministrazione e
finalizzato a rendere effettiva la tutela sostanziale dell'interesse
protetto. Di regola il giudice assegna all'amministrazione un termine e
contestualmente designa un'autorità amministrativa che alla scadenza del
termine assegnato si sostituirà all'amministrazione inadempiente ed
emanerà il provvedimento o terrà il comportamento necessario per
l'attuazione del giudicato. In sede di ottemperanza al giudicato, il
giudice amministrativo, direttamente o per mezzo del commissario da lui
nominato, può emanare provvedimenti di vario tipo, costitutivi,
certificatori, declaratori di obblighi a carico dell'amministrazione e
tutti quegli adempimenti strumentalmente necessari per l'esecuzione
della sentenza. In pratica, si sostituisce all'amministrazione
inadempiente ponendo in essere l'attività che questa avrebbe dovuto
compiere per realizzare concretamente gli effetti scaturenti dalla
sentenza da eseguire, conformando la realtà alle sue statuizioni. Poiché
la discrezionalità amministrativa implica sovente decisioni di matrice
politica, la nomina di un commissario \emph{ad acta} viene ritenuta
preferibile rispetto all'adozione diretta da parte del giudice delle
misure di competenza dell'amministrazione riottosa. Di regola, egli è
scelto fra funzionari di altre amministrazioni e, spesso nella persona
del Prefetto, rappresenta con la sua attività ``il punto di sutura e
saldatura'' tra attività giurisdizionale e amministrativa\footnote{In
  \textsc{F. Secchi}, \emph{op. cit.}, p.~37, ove si riprende il
  contributo di M. \textsc{Clarich}, \emph{Il giudizio amministrativo di
  esecuzione}, in G. \textsc{Paleologo} -- \textsc{Italy} --
  \textsc{France} (Edd.), \emph{I consigli di Stato di Francia e
  d'Italia}, Milano, Giuffrè, 1998, 344.}. In particolare, ``in quanto
delegato dal giudice amministrativo, ha il potere di emanare i necessari
provvedimenti amministrativi anche in deroga alle vigenti competenze.
Allo stesso è altresì demandato l'onere di porre in essere ogni attività
idonea a dare esecuzione alla decisione''\footnote{In
  \textsc{F. Secchi}, \emph{op. cit.}, p.~61, in argomento si veda anche
  S. \textsc{Pelillo}, \emph{Il giudizio di ottemperanza alle sentenze
  del giudice amministrativo}, Milano, Giuffrè, 1990, 313 ss.}. Una
ormai risalente pronuncia della Corte costituzionale configura il
commissario \emph{ad acta} come ausiliario del giudice e riconduce i
suoi atti all'esercizio della giurisdizione esecutiva del giudice
dell'ottemperanza\footnote{In \textsc{F. Secchi}, \emph{op. cit.},
  p.~62, ove si riporta Corte Cost., 12 maggio 1977, n.~75, in
  \emph{Giur. it.,} 1978, I, 980.}. Autorevole dottrina ha
sostanzialmente qualificato l'attività commissariale come ``proiezione
nel mondo esterno di un comando del giudice e, quindi, della traduzione
nel concreto della attribuzione della potestas decidendi che non sempre
ha o può avere contenuti rigidamente predeterminati, tali da consentire
al giudice di portarli direttamente ad attuazione''\footnote{In
  \textsc{F. Secchi}, \emph{op. cit.}, p.~62, viene richiamato S.
  \textsc{Pelillo}, \emph{Il giudizio di ottemperanza alle sentenze del
  giudice amministrativo}.}. L'ampiezza dei poteri commissariali
dipenderà dal contenuto del giudicato inadempiuto: essi potranno
estrinsecarsi, a seconda delle situazioni dedotte in giudizio, in
attività sia vincolata, come ad esempio la restituzione di beni
illegittimamente espropriati, sia discrezionale, quindi comportante un
potere di scelta\footnote{In \textsc{F. Secchi}, \emph{op. cit.}, p.~63,
  come si distingue in A. M. \textsc{Sandulli}, \emph{L'effettività
  delle decisioni giurisdizionali amministrative}, in \emph{Atti del
  Convegno celebrativo del 150° anniversario del Consiglio di Stato},
  Milano, Giuffrè, 1983, 308, 309.}. Una volta nominato il commissario,
il giudice mantiene comunque un incisivo potere di vigilanza sul suo
operato, nonché il potere di risolvere eventuali contestazioni, dal
momento che le determinazioni del commissario, laddove esorbitanti dalle
specifiche indicazioni del giudice, possono essere oggetto di un ricorso
dinanzi allo stesso giudice, esperibile anche dall'amministrazione
sostituita\footnote{Cons. St., sez. V, 28 febbraio 1995, n.~298, in
  \emph{Cons. Stato}, 1995, I, 232 ss.}. Per lungo tempo, sia in
dottrina che in giurisprudenza, si è dibattuto sulla questione
riguardante la misura del potere di adempiere che conserverebbe
l'amministrazione, una volta che sia stato nominato il commissario
\emph{ad acta} o sia scaduto il nuovo termine imposto alla stessa
amministrazione. Con sentenza del 25 maggio 2021, n.~8, l'Adunanza
Plenaria si è definitivamente pronunciata su una questione rimessa dalla
sez. IV del Consiglio di Stato con ordinanza del 10 novembre 2020,
n.~6925, chiarendo il rapporto intercorrente tra il commissario \emph{ad
acta} e l'amministrazione soccombente. Già con ordinanza del 10 maggio
2011, n.~2764, la sez. IV affermava che ``la nomina del commissario ad
acta non determina di per sé l'esaurimento della competenza della p.a.
sostituita a provvedere all'ottemperanza del giudicato, in quanto il
venir meno dell'inerzia della p.a. stessa, pur dopo la scadenza del
termine assegnatole, rende priva di causa la nomina e la funzione del
commissario, secondo i principi di economicità e buon andamento
dell'azione amministrativa, non smentiti dalla legge o dalla pronuncia
del giudice dell'ottemperanza ed essendo indifferente per il privato che
il giudicato sia eseguito dall'amministrazione, piuttosto che dal
commissario, perché l'attività di entrambi resta comunque egualmente
soggetta al controllo del giudice''. Alla luce di queste considerazioni,
l'Adunanza Plenaria ha stabilito che il potere dell'amministrazione e
quello del commissario \emph{ad acta} sono poteri concorrenti, di
conseguenza ``ciascuno dei due soggetti può dare attuazione a quanto
prescritto dalla sentenza passata in giudicato, o provvisoriamente
esecutiva e non sospesa, o dall'ordinanza cautelare fintanto che l'altro
soggetto non abbia concretamente provveduto''\footnote{Cons. St., Ad.
  Plen., 25 maggio 2021, n.~8.}. Inoltre, gli atti emanati
dall'amministrazione, pur in presenza della nomina e dell'insediamento
del commissario \emph{ad acta}, non possono essere considerati affetti
da nullità, in quanto gli stessi sono adottati da un soggetto nella
pienezza dei propri poteri, a nulla rilevando a tal fine la nomina o
l'insediamento del commissario.

Molti dei casi d'inosservanza delle sentenze del giudice amministrativo
non sono dettati da una volontà deliberata di disconoscere l'autorità
della cosa giudicata, bensì alle oggettive difficoltà che
l'amministrazione incontra nell'eseguire le sentenze, soprattutto
qualora gli obblighi in esse enunciati appaiano indeterminati, vaghi o
imprecisi e intercorra più tempo fra l'emissione del provvedimento
impugnato e il suo definitivo annullamento. Il c.~5 dell'art. 112 c.p.a.
e il c.~7 dell'art. 114 ammettono il ricorso al giudice
dell'ottemperanza anche soltanto per ``ottenere chiarimenti'' in merito
alle modalità di esecuzione che non presuppone un'inottemperanza, ma
semplicemente un'incertezza sull'interpretazione o sugli effetti della
sentenza da eseguire. In questi casi quindi il ricorso può essere
proposto anche dall'amministrazione tenuta a darvi esecuzione, quando
abbia esigenza di chiarimenti. Per evitare abusi, la giurisprudenza ha
comunque affermato che tale rimedio non deve rappresentare un espediente
per mettere in discussione la sentenza da eseguire o per introdurre
questioni estranee all'ottemperanza\footnote{Cons. St., sez. VI, 19
  giugno 2012, n.~3569.}.

\hypertarget{il-risarcimento-del-danno-da-inottemperanza}{%
\section{Il risarcimento del danno da
inottemperanza}\label{il-risarcimento-del-danno-da-inottemperanza}}

Una volta ottenuta soddisfazione attraverso il giudizio promosso ai
sensi del c.~2 dell'art. 112 c.p.a., potrebbe ancora residuare al
ricorrente vittorioso un danno connesso alla tardiva realizzazione di
quell'assetto che sarebbe dovuto scaturire dall'annullamento del
provvedimento illegittimo dell'amministrazione, ma che è venuto in
essere solo a seguito di un notevole lasso di tempo oppure che ormai non
risulta più attuabile, per cui il giudizio di ottemperanza, di per sè,
non sarebbe in grado di garantire al ricorrente una tutela piena ed
effettiva. In quest'ultimo caso, lo strumento dell'ottemperanza si
rivelerebbe inutile, se non vi fosse la possibilità di ottenere
contestualmente un risarcimento per equivalente a seguito della perdita
definitiva del bene spettante dovuta alla inesecuzione del giudicato. Si
pensi al caso del definitivo annullamento di un decreto di esproprio cui
non sia seguita la spontanea restituzione dell'immobile al proprietario,
per cui si è reso necessario instaurare il giudizio di ottemperanza. Ove
l'amministrazione opponesse, in questa sede, una legittima
sopravvenienza impediente l'esecuzione del giudicato, al ricorrente
dovrebbe essere riconosciuto, in funzione surrogatoria, anche il danno
c.d. petitorio, consistente nel controvalore del bene, derivante appunto
dalla perdita definitiva dello stesso, cagionata dall'illecito ritardo
nella conformazione al giudicato \footnote{Cons. St., sez. IV, 30
  gennaio 2006, n.~290, in \emph{Dir. giust.}, 2006, 10, 80 ove si
  afferma che il proprietario del fondo illegittimamente occupato dalla
  p.a. in esito a declaratoria di illegittimità dell'occupazione e
  all'annullamento dei relativi provvedimenti può legittimamente
  domandare nel giudizio di ottemperanza sia il risarcimento, sia la
  restituzione del fondo che la sua riduzione in pristino. L'azione
  risarcitoria costituisce strutturalmente attuazione del
  \emph{decisium} e quindi trova la sua naturale allocazione nel
  giudizio di ottemperanza, in quanto consente e determina
  quell'adeguamento dello stato di fatto allo stato di diritto che
  rappresenta la finalità tipica del giudizio di ottemperanza,
  realizzando quell'esigenza di completamento della tutela
  giurisdizionale amministrativa, ribadita anche dalla Consulta con
  sentenza n.~204/2004.}. Da questa situazione, va tenuta distinta
quella in cui, già al momento della pronuncia di annullamento, risulta
chiaramente che non è più utile per il ricorrente la rinnovazione del
potere conformemente alla regola concreta dedotta in sentenza, potendo
il giudice amministrativo in tal caso accogliere immediatamente la
domanda di risarcimento del danno per equivalente. In molti altri casi,
invece, il giudice della cognizione non è in grado di prevedere già
all'atto dell'annullamento se ed in quale misura l'ottemperanza potrà
effettivamente ripristinare la situazione soggettiva lesa. In
particolare, in tutti quei casi in cui la domanda del privato è diretta
a conseguire il bene della vita, molto spesso la possibilità e i limiti
entro cui attribuire il bene dipendono dal momento in cui
l'amministrazione esegue il giudicato. Ad esempio, in materia di
appalti, se l'annullamento dell'aggiudicazione in sede giurisdizionale
interviene nell'immediatezza dei fatti, consente al ricorrente di
stipulare il contratto con l'amministrazione; al contrario, se
interviene quando il contratto con l'originale aggiudicatario è già
stato non solo stipulato, ma anche parzialmente eseguito, l'esecuzione
della pronuncia e quindi l'attribuzione del bene della vita, cioè
l'appalto, è possibile solo parzialmente per la parte residua non
eseguita, mentre per la prima parte la tutela può avvenire solo
attraverso il risarcimento, sempre per equivalente. Spesso, quindi, solo
all'esito dell'ottemperanza di un giudicato di annullamento è possibile
accertare e quantificare il danno risarcibile per equivalente. Laddove
non risulta più satisfattiva la pronuncia di annullamento, supplisce la
tutela risarcitoria e il momento in cui emerge con chiarezza lo spazio
per l'esecuzione del giudicato e per il risarcimento del danno è proprio
quello dell'ottemperanza.

Nell'ipotesi di annullamento di un provvedimento ampliativo della sfera
giuridica del privato, occorre distinguere il caso in cui
l'amministrazione, in esecuzione spontanea del giudicato di
annullamento, renda il provvedimento precedentemente negato, dal caso in
cui il bene della vita agognato dal ricorrente venga conseguito soltanto
in esito al giudizio di ottemperanza. Mentre nella prima situazione la
pretesa risarcitoria azionabile riguarderà esclusivamente un danno da
attività provvedimentale illegittima, non avendo luogo una violazione
del giudicato, in quanto l'amministrazione accorda l'utilità prima
negata, a seguito della rinnovazione del potere discrezionale successivo
al giudicato di annullamento\footnote{Il ritardo produttivo del danno
  deriva dal fatto che l'amministrazione ha prima adottato un
  provvedimento illegittimo, sfavorevole al privato (es. diniego di
  permesso di costruire), ed ha poi emanato altro provvedimento, questa
  volta legittimo e favorevole, a seguito dell'annullamento in sede
  giurisdizionale del primo atto. Sul punto cfr. R. \textsc{Chieppa} --
  V. \textsc{Lopilato}, \emph{Studi di diritto amministrativo}, Milano,
  Giuffrè, 2007, 624, 625.}, nella seconda la pretesa risarcitoria sarà
duplice e riguarderà un danno scomponibile in una prima voce, relativa
al ritardo antecedente alla formazione del giudicato e commisurato al
pregiudizio patito dal ricorrente, qualora l'amministrazione si fosse
spontaneamente conformata al giudicato, e una seconda voce di danno,
propriamente da inadempimento dell'obbligo conformativo scaturente dalla
pronuncia del giudice amministrativo, volta a coprire il segmento
temporale intercorrente fra il giudicato e la sua concreta attuazione.
In entrambe le ipotesi, il danno c.d. da ritardo potrà essere
compiutamente apprezzato soltanto a posteriori, ovvero una volta che il
privato abbia effettivamente ottenuto il bene della vita cui aspirava
con l'istanza a suo tempo illegittimamente rigettata
dall'amministrazione, a meno che non si tratti di potere amministrativo
vincolato, per cui la spettanza del bene si cristallizza già in esito al
giudizio di cognizione\footnote{Cons. St., sez. IV, 31 marzo 2006,
  n.~5323, in \emph{Foro amm. CS}, 2006, 9, 2585 è spiegato che nelle
  ipotesi nelle quali l'amministrazione sia titolare di un potere
  discrezionale, solo dal nuovo esercizio del potere possono derivare
  certezze in ordine alla spettanza del bene cui il privato aspira. Per
  converso, allorché si tratti di attività vincolata, il giudice,
  riscontrata la sussistenza dei presupposti di legge, potrà stabilire
  che, data quella situazione, la p.a. avrebbe dovuto adottare quella
  certa determinazione.}. In giurisprudenza ricorre il principio secondo
cui, essendo l'oggetto del giudizio di ottemperanza costituito dalla
verifica se l'amministrazione abbia o meno adempiuto all'obbligo
nascente dal giudicato, ovvero abbia o meno attribuito all'interessato
quell'utilità concreta che la sentenza ha riconosciuto come dovuta, a
prescindere dal fatto che residuino o meno in capo all'amministrazione
stessa poteri discrezionali, l'esecuzione deve essere esatta, al pari di
quanto avviene nell'obbligazione civile, il cui inesatto adempimento
viene sanzionato con la condanna al risarcimento del danno\footnote{In
  \textsc{F. Secchi}, \emph{op. cit.}, pp.~68, si riporta quanto
  chiarito da L. \textsc{Mancini}, \emph{La responsabilità della
  pubblica amministrazione per inottemperanza al giudicato
  amministrativo di annullamento(nota a Cons.St.,sez.IV,6 ottobre 2003
  n.5820)}, in {«Il Foro amministrativo - CdS»} 2 (2003) 12, 3700--3724,
  nota 22, 3708.}. L'utilità concreta potrà consistere ``nel diritto
alla restitutio in integrum sotto forma di pretesa alla restituzione del
bene in caso di annullamento di provvedimenti ablatori, sotto forma di
annullamento del contratto stipulato in seguito ad aggiudicazione
illegittima, nel caso di provvedimento incidente su interessi legittimi
pretensivi; può consistere nel diritto alla conformazione alla regola
contenuta nel giudicato in caso di riedizione dell'atto che va dal
diritto alla non riedizione o all'ottenimento dell'atto in caso di
effetto vincolante pieno, al diritto alla riedizione nel rispetto delle
regole sostanziali e formali in caso di effetto vincolante semipieno o
strumentale''\footnote{In \textsc{F. Secchi}, \emph{op. cit.}, pp.~77.
  si riferisce quanto osservato da L. \textsc{Mancini}, \emph{Ibidem},
  nota 73, 3723.}. Sul piano dell'accertamento e della prova, se nel
giudizio avente a oggetto il pregiudizio conseguente al provvedimento
amministrativo illegittimo il privato deve provare tutti gli elementi
costitutivi del fatto illecito, in quello avente a oggetto il danno da
violazione del giudicato opera, invece, il principio dell'inversione
dell'onere della prova di cui all'art. 1218 c.c. nella misura in cui
viene posta a carico del debitore la prova che l'inadempimento è stato
determinato da impossibilità della prestazione derivante da causa non
imputabile. Ne consegue che l'interessato deve dimostrare esclusivamente
il suo diritto e la sussistenza di un giudicato di accoglimento, mentre
spetterà all'amministrazione la prova di avervi ottemperato.

\hypertarget{lesperienza-tedesca}{%
\chapter{L'esperienza tedesca}\label{lesperienza-tedesca}}

\hypertarget{lazione-di-annullamento-anfechtungsklage-e-lazione-di-adempimento-verplichtungsklage}{%
\section{L'azione di annullamento (Anfechtungsklage) e l'azione di
adempimento
(Verplichtungsklage)}\label{lazione-di-annullamento-anfechtungsklage-e-lazione-di-adempimento-verplichtungsklage}}

Il legislatore tedesco ha affrontato il problema dell'esecuzione della
sentenza amministrativa avente ad oggetto un provvedimento
amministrativo, prima ancora che attraverso la predisposizione di un
meccanismo di coazione in presenza di un inadempimento
dell'amministrazione, con la previsione di un articolato sistema di
misure che consentono di prevenire la mancata spontanea esecuzione delle
pronunce del giudice e cercando di definire già a livello normativo
contenuto ed effetti che debbono assumere le decisioni giurisdizionali
in presenza di determinati presupposti. In questa prospettiva, è
importante in primo luogo che gli obblighi dell'amministrazione
derivanti dalla decisione del giudice amministrativo siano facili da
assolvere e perfettamente determinati\footnote{In \textsc{F. Secchi},
  \emph{op. cit.}, p.~82. Cfr. M. \textsc{Fromont}, \emph{L'esecuzione
  delle decisioni del giudice amministrativo nel diritto francese e
  tedesco L'exécution des décisions du juge administratif en droit
  français et allemand}, in {«Problemi di Amministrazione Publica»}
  (1989) 3, 523--540.}. Con la legge del 21 gennaio 1960 \emph{(VwGO -
Verwaltungsgerichtsordnung)} sull'ordinamento processuale amministrativo
sono state introdotte due distinte azioni: una di impugnazione in senso
stretto o di annullamento \emph{(Anfechtungsklage)} ed un'altra di
condanna all'emissione di un dato provvedimento, altrimenti detta ``di
adempimento'' \emph{(Verplichtungsklage)}. Ove l'autorità emetta un
provvedimento incidente negativamente nella sfera giuridica del
destinatario, questi ricorrerrà alla \emph{Anfechtungsklage} facendone
valere eventuali vizi; ove invece il privato aspiri ad ottenere un
provvedimento ampliativo della propria posizione giuridica soggettiva e
si veda opporre un rifiuto espresso oppure debba constatare l'inerzia
dell'amministrazione, azionerà il rimedio della
\emph{Verplichtungsklage}, chiedendo la condanna al rilascio del
provvedimento rifiutato od omesso. L'azione di annullamento o di
impugnazione, così come quella di adempimento, sono disciplinate dal
\emph{§ 42 VwGO} che al primo comma prevede che \emph{``mediante azione
può essere richiesto l'annullamento di un atto amministrativo''} (azione
di impugnazione), \emph{``come pure la condanna all'emanazione di un
atto amministrativo rifiutato od omesso''} (azione di inadempimento) e
al secondo comma che \emph{``qualora la legge non disponga diversamente,
l'azione è ammissibile solo quando l'attore fa valere di essere stato
leso nei propri diritti dall'atto amministrativo o dal suo rifiuto od
omissione''}\footnote{§ 42 \emph{VwGO}. Cfr. traduzione in G.
  \textsc{Falcon} -- C. \textsc{Fraenkel} (Edd.), \emph{Ordinamento
  processuale amministrativo tedesco: (VwGO); versione italiana con
  testo a fronte}, Trento, Università degli Studi di Trento, 2000.}.
L'azione di annullamento è fondata allorquando ricorrano i requisiti
previsti dal \emph{§ 113 VwGO}, e cioè nella misura in cui l'atto
risulti illegittimo e lesivo dei cosiddetti diritti civili pubblici
dell'attore \emph{(Kläger)}, quei diritti cioè che conferiscono al
singolo la facoltà di pretendere dall'amministrazione una prestazione
positiva o negativa. Verificata la sussistenza di questi presupposti, il
tribunale potrà quindi annullare l'atto, ma la sentenza che conclude il
giudizio di impugnazione potrà assumere un contenuto ulteriore e diverso
dal mero annullamento del provvedimento impugnato, strettamente
correlato all'attività esecutiva che l'amministrazione dovrebbe
successivamente porre in essere per adeguarsi al \emph{decisium}.
L'effetto demolitorio del provvedimento illegittimo, previa la
sospensione della sua efficacia esecutiva, potrebbe rendere non
necessaria la successiva attività di adeguamento; diversamente,
nell'ambito della stessa sentenza che definisce il giudizio cassatorio,
è prevista la possibilità per il giudice di guidare l'amministrazione
nella scelta delle modalità di esecuzione della sentenza, per il
ripristino dello \emph{status quo ante} attraverso la cancellazione
degli effetti che si sono nel frattempo prodotti. E' questo l'istituto
del cosiddetto \emph{Folgenbeseitigungsanspruch}\footnote{Letteralmente
  ``pretesa o diritto all'eliminazione delle conseguenze dell'atto''.},
indicato con l'abbreviazione \emph{FBA} e contemplato dal \emph{§ 113/1}
secondo alinea \emph{VwGO}, ove si prevede che \emph{``se l'atto
amministrativo è stato già eseguito, il tribunale può anche dichiarare,
su richiesta, se e come l'autorità amministrativa debba revocare
l'esecuzione''}. Il \emph{FBA} è autonomo rispetto all'azione di
annullamento, inquadrabile fra le cosiddette azioni di prestazione,
ancorché il giudice dell'impugnazione si pronunci sull'eliminazione
degli effetti dell'atto con la medesima sentenza che definisce il
giudizio cassatorio. L'utilizzo del termine ``può'' (Kann) da parte del
legislatore indica la mera facoltà di cumulare la domanda di revoca
dell'esecuzione a quella di annullamento dell'atto eseguito, ma ciò non
esclude la possibilità di proporre separata istanza, instaurando un
autonomo giudizio. Rimane impregiudicata la facoltà per il tribunale, ai
sensi del \emph{§ 93} secondo alinea \emph{VwGO}, di ordinare in ogni
caso che le rispettive domande vengano trattate e decise in separati
processi. Il c.~1, alinea terzo del \emph{§113 VwGO} stabilisce che la
pretesa alla revoca dell'esecuzione è ammissibile solo laddove
l'autorità amministrativa sia in grado di darvi seguito. In altri
termini, l'attività di rimozione degli effetti dell'esecuzione del
provvedimento annullato presuppone una prestazione possibile sotto il
profilo giuridico-fattuale. Qualora l'amministrazione non sia in grado
di ripristinare esattamente la situazione pregressa, dovrebbe
ricostruirne una quantomeno simile a quella precedente l'esecuzione
dell'atto annullato, in modo tale da eliminare al massimo i pregiudizi
per il destinatario del provvedimento\footnote{In \textsc{F. Secchi},
  \emph{op. cit.}, p.~89.}. Si ritiene inoltre che la revoca
dell'esecuzione come disposta dal giudice possa consistere, oltre che
nella rimozione di un'attività materiale dell'amministrazione, anche
nell'adozione di un atto amministrativo, quale ad esempio l'ordine di
sgombero di un appartamento a seguito dell'annullamento della confisca
dell'immobile da parte delle forze di polizia, con susseguente sua
occupazione da parte di un terzo\footnote{§ 113 \emph{VwGO}. Cfr. W.-R.
  \textsc{Schenke} et al., \emph{Verwaltungsgerichtsordnung: Kommentar},
  München, C.H. Beck, 28., neubearbeitete Auflage 2022, n.~91.}.
Ulteriore presupposto di ammissibilità della pretesa, oltre al fatto che
l'autorità sia in grado di darvi seguito, è che la questione sia matura
per la decisione. Ciò significa che non deve più esserci necessità di
accertare i fatti e non deve residuare alcuna discrezionalità in capo
all'amministrazione per quanto riguarda le modalità di revoca
dell'intervenuta esecuzione. Il \emph{FBA} sarà escluso laddove la
rimozione delle conseguenze dell'esecuzione sia in contrasto con la
legge al momento della decisione del tribunale\footnote{§ 113
  \emph{VwGO}. Cfr. W.-R. \textsc{Schenke} et al., \emph{Ibidem}, n.~87.}.
In definitiva, l'amministrazione che si trovi a dover eseguire la
sentenza di annullamento e, quindi, a ripristinare la situazione
esistente prima del provvedimento caducato, potrà essere guidata dal
giudice nella scelta delle misure necessarie all'esecuzione del
\emph{dictum} giudiziale, almeno per quel che concerne la rimozione
degli effetti strettamente connessi all'esecuzione del provvedimento
annullato. L'inottemperanza alla decisione sotto tale profilo, seppur
non assistita da alcun meccanismo di coazione diretta, potrà tuttavia
essere sanzionata attraverso l'attivazione della peculiare procedura di
coercizione indiretta di cui al \emph{§ 172 VwGO}, consistente
nell'assegnazione da parte del giudice, su richiesta dell'interessato,
di un termine per l'esecuzione della pronuncia e, nel caso di
inosservanza del medesimo, nell'irrogazione di un'ammenda, lo
\emph{Zwangsgeld}.

\hypertarget{il-contenzioso-ingiuntivo-la-c.d.-verpflichtungsklage}{%
\section{\texorpdfstring{Il contenzioso ingiuntivo: la c.d.
\emph{``Verpflichtungsklage''}}{Il contenzioso ingiuntivo: la c.d. ``Verpflichtungsklage''}}\label{il-contenzioso-ingiuntivo-la-c.d.-verpflichtungsklage}}

La disposizione al § 113/5 \emph{VwGO} contempla la sentenza sulla c.d.
\emph{Verplichtungsklage}, azione di prestazione o, secondo definizione
della dottrina italiana, di condanna, con la quale si ingiunge alla
pubblica amministrazione l'emanazione di un atto rifiutato od
omesso\footnote{In \textsc{F. Secchi}, \emph{op. cit.}, p.~94. Il §
  113/5 \emph{VwGO}, così come tradotto in G. \textsc{Falcon} -- C.
  \textsc{Fraenkel} (Edd.), \emph{Ordinamento processuale amministrativo
  tedesco}, recita: \emph{``Nella misura in cui il rifiuto o l'omissione
  dell'atto amministrativo è illegittimo e l'attore ne risulta leso nei
  propri diritti, il tribunale dichiara l'obbligo dell'autorità
  amministrativa di porre in essere la richiesta attività dell'ufficio,
  se la questione è matura per la decisione. Altrimenti esso dichiara
  l'obbligo di decidere nei confronti dell'attore nel rispetto della
  concezione giuridica del tribunale''}.}. Essa è diretta
sostanzialmente a sindacare il rifiuto del provvedimento amministrativo
richiesto dall'interessato che consegue all'assunzione di un
provvedimento espresso di diniego; in tal caso l'azione assumerà, seppur
in via sussidiaria, anche un contenuto annullatorio, dal momento che
l'eventuale sentenza di accoglimento ne comporterà
l'eliminazione\footnote{In \textsc{F. Secchi}, \emph{op. cit.}, p.~95.
  Cfr. D. \textsc{De Pretis}, \emph{Il processo amministrativo in
  europa: caratteri e tendenze in Francia, Germania, Gran Bretagna e
  nell'Unione europea}, a cura di R. Paganella, Trento, Ed. provvisoria
  2000.}, ovvero l'omissione di un provvedimento non ancora denegato.
Presupposti di fondatezza della \emph{Verplichtungsklage}, così come per
l'azione di annullamento, sono l'illegittimità del diniego o
dell'inerzia e la lesione dei diritti del ricorrente. In presenza di
tali condizioni, il giudice affermerà l'obbligo dell'autorità
amministrativa di adottare il provvedimento richiesto, qualora la
questione sottoposta al suo esame consenta una decisione definitiva.
Tuttavia, una condanna dell'amministrazione all'emissione di un atto
dotato di un ben preciso contenuto potrà intervenire soltanto nel caso
in cui la stessa amministrazione non abbia poteri discrezionali, ovvero
qualora l'adozione del provvedimento richiesto rappresenti l'unica
manifestazione della discrezionalità amministrativa priva di errori,
tenuto presente che la sentenza di condanna, in virtù del principio
costituzionale della separazione dei poteri, non sostituisce la
decisione dell'amministrazione, ma la obbliga ad agire. Laddove invece
la controversia non consenta una decisione definitiva, il tribunale,
previo riconoscimento della legittimità della pretesa del ricorrente ad
una decisione dell'amministrazione, si limiterà a statuire l'obbligo di
quest'ultima di decidere secondo la valutazione da esso espressa sulla
questione\footnote{§ 113/5 \emph{VwGO}. Si parla in questo caso di
  \emph{Bescheidungsurteil}.}. Partendo dalla considerazione che spesso
l'amministrazione non ottempera, in quanto non sa come ottemperare,
l'istituto della \emph{Verplichtungsklage} acquista notevole importanza
nella misura in cui coinvolge il giudice sin dalla prima pronuncia nel
chiarimento della portata operativa della decisione, ottenendo così una
maggior ``presa'' sul successivo comportamento\footnote{In
  \textsc{F. Secchi}, \emph{op. cit.}, p.~83. Cfr. G.
  \textsc{Bronzetti}, \emph{Per una migliore giustizia amministrativa:
  spunti comparatistici con particolare riferimento all'ordinamento
  della Repubblica federale tedesca}, in {«Informator»} (1996) 1,
  139--162.}. Per converso, ove l'amministrazione sia titolare di un più
ampio potere discrezionale, il giudice incontrerà il limite della
insostituibilità delle scelte discrezionali dell'autorità, quindi la
susseguente attività amministrativa rimarrà pur sempre appannaggio
dell'amministrazione, con l'unico limite costituito dall'obbligo di
rispettare la ``concezione giuridica'' del tribunale (c.d.
\emph{Bescheidungsurteil}).

\hypertarget{lapplicazione-generalizzata-della-tutela-cautelare}{%
\section{L'applicazione generalizzata della tutela
cautelare}\label{lapplicazione-generalizzata-della-tutela-cautelare}}

Si è visto come, nei casi in cui l'amministrazione incontri delle
difficoltà nell'ottemperare alle sentenze, dovute al fatto di non sapere
come attuarle, il legislatore tedesco abbia cercato di determinare
puntualmente gli effetti delle sentenze, in modo tale da guidare
l'amministrazione nell'esecuzione del \emph{decisium}. Quanto invece
alla difficoltà, se non a volte all'impossibilità, di rimuovere
completamente la situazione creata da un provvedimento sacrificativo poi
annullato, si è cercato di affrontarla attraverso un'estesa applicazione
della tutela interinale della posizione del ricorrente, la quale si
diversifica a seconda del tipo di provvedimento che viene in rilievo.
Nell'ipotesi di atto amministrativo che incide negativamente sulla sfera
giuridica dell'interessato, cui nel nostro ordinamento corrispondono gli
interessi oppositivi, si ammette, salvo alcune eccezioni, la sospensione
automatica dei suoi effetti in derivazione, prima ancora che della mera
proposizione dell'impugnativa in sede giurisdizionale, della
proposizione del ricorso amministrativo previo che costituisce
condizione di ammissibilità della \emph{Anfechtungsklage} ai sensi del §
68 \emph{VwGO}\footnote{BVerwG, 21.06.1961, in \emph{BVerwGE} 13, 5.}.
Attraverso un impiego esteso e generalizzato della sospensione cautelare
del provvedimento impugnato, ex § 80 \emph{VwGO}, molti dei problemi
legati alla esecuzione delle pronunce del tribunale finiscono per essere
risolti o comunque attenuati in via preventiva, considerata,
diversamente, la necessità di eliminare gli effetti materiale prodotti
dal provvedimento fino al passaggio in giudicato della sentenza di
annullamento\footnote{In \textsc{F. Secchi}, \emph{op. cit.}, p.~98.
  Cfr. B. \textsc{Marchetti}, \emph{L'esecuzione della sentenza
  amministrativa prima del giudicato}, Padova, CEDAM, 2000, 46.}. La
sospensione cautelare non è però idonea a tutelare il ricorrente che
lamenti l'illegittimo diniego di un provvedimento ampliativo o, meglio,
l'omesso rilascio del titolo. In questi casi, contraddistinti
dall'emersione di interessi legittimi pretensivi, la tutela del privato
è garantita dalla possibilità per il giudice, ai sensi del § 123
\emph{VwGo}, di concedere misure provvisorie a contenuto positivo anche
anteriormente al ricorso, volte a evitare quelle modificazioni
irreparabili che potrebbero \emph{medio tempore} investire la situazione
di fatto, rendendo alla fine una sentenza favorevole, vantaggiosa sulla
carta, ma nella sostanza inutile.

\hypertarget{le-misure-coercitive-lo-zwangsgeld-172-vwgo}{%
\section{\texorpdfstring{Le misure coercitive: lo \emph{Zwangsgeld (§
172
VwGO)}}{Le misure coercitive: lo Zwangsgeld (§ 172 VwGO)}}\label{le-misure-coercitive-lo-zwangsgeld-172-vwgo}}

Il \emph{VwGO} disciplina l'esecuzione coattiva delle sentenze del
giudice amministrativo nei confronti della pubblica amministrazione ai
§§ 167-172. In particolare, la legge sul processo amministrativo
\emph{(VwGO)} determina il giudice dell'esecuzione (§ 167), i titoli
esecutivi (§ 168), l'esecuzione a favore della mano pubblica (§ 169),
l'esecuzione contro la mano pubblica (§§ 170 e 172), nonché i casi in
cui non è necessaria la formula esecutiva (§ 171). I §§ 170 e 172
\emph{VwGO} rappresentano la base normativa dell'esecuzione forzata
contro la pubblica amministrazione, ancorché i rispettivi ambiti di
applicazione siano da tenere distinti. Il § 170 \emph{VwGO} disciplina
l'esecuzione contro la mano pubblica per crediti pecuniari, compresa la
penale di cui al § 172 (\emph{Zwangsgeld}). Tale norma è modellata sul §
882a \emph{ZPO} relativo all'esecuzione per crediti di denaro nei
confronti delle persone giuridiche di diritto pubblico e le modalità di
esecuzione sono sostanzialmente quelle previste dal codice di procedura
civile, nulla dicendo sul punto il § 170 \emph{VwGO} che tuttavia reca
alcuni correttivi che tengono conto della particolare condizione
giuridica del patrimonio pubblico e della sua tendenziale destinazione
all'assolvimento dei compiti dell'amministrazione. Più nel dettaglio, il
tribunale, da un lato, prima di procedere all'esecuzione forzata, deve
intimare all'autorità amministrativa di eseguire il giudicato entro il
termine massimo di un mese e, dall'altro, essendo l'esecuzione
inammissibile in relazione a beni essenziali per l'adempimento di
pubbliche funzioni o alla cui alienazione si contrapponga un pubblico
interesse, non può ordinare il sequestro di beni destinati all'uso o al
servizio pubblico. Il § 172 \emph{VwGO} attiene, in linea di principio,
all'esecuzione di decisioni dichiarative dell'obbligo
dell'amministrazione di rilasciare un provvedimento nei confronti della
controparte. Esso prevede un mezzo di coercizione meramente indiretto,
assistito dalla minaccia di una sanzione pecuniaria da applicarsi
all'autorità inadempiente senza alcuna limitazione o particolare
privilegio per la stessa, salva la misura massima dell'ammenda, di volta
in volta erogabile, pari a diecimila Euro. Nella misura in cui il
contenuto delle norme predette non dovesse essere esaustivo in relazione
al caso concreto, sarà possibile integrarle con i precetti del codice di
rito (\emph{ZPO}), in nome del principio di effettività della tutela
giurisdizionale\footnote{Sul punto si è pronunciata la Corte
  costituzionale tedesca in una decisione di notevole importanza, con
  cui, nella parte motiva, fa esplicito riferimento alle misure previste
  dai §§ da 885 a 896 \emph{ZPO} la cui scelta, in ordine alle modalità
  ed eventuale successione delle misure da adottare, sarà rimessa alla
  valutazione del giudice competente.}. Molto prima che venisse alla
luce la legge sulla giustizia amministrativa e con essa il § 172
\emph{VwGO}, l'idea di poter eseguire coattivamente le pronunce dei
giudici nei confronti di un soggetto esercente un pubblico potere era
stata decisamente avversata in dottrina. Si sosteneva, in particolare,
che si sarebbe rivelato un non senso che lo Stato, quale fondamento del
diritto (\emph{Hort des Rechts}) potesse essere coartato al rispetto di
quello stesso diritto di cui egli era portatore. Ciò si sarebbe rivelato
inconciliabile con il rispetto che si deve allo Stato medesimo e avrebbe
leso la sua immagine, mettendone in discussione l'onore\footnote{In
  \textsc{F. Secchi}, \emph{op. cit.}, p.~114, l'autore riporta
  l'espressione ``L'Etat toujours doit être réputé honnête homme'', da
  E. L. J. \textsc{Laferrière}, \emph{Traité de la juridiction
  administrative et des recours contentieux}, vol. 2, Berger-Levrault et
  cie, 1896.}. Così recita il § 172 \emph{VwGO}: ``\emph{Se un'autorità,
nei casi di cui al § 113, co. 1, secondo periodo, e co. 5 del § 123, non
ottempera all'obbligo impostole nella sentenza o nel provvedimento
provvisorio, il tribunale di primo grado può, su richiesta, comminare
con ordinanza nei suoi confronti un'ammenda fino a Euro diecimila,
previa assegnazione di un termine, applicarla dopo l'infruttuoso decorso
del termine e portarla a esecuzione d'ufficio. L'ammenda può essere
ripetutamente comminata, applicata e portata a esecuzione}''\footnote{§
  172 \emph{VwGO}. Cfr. traduzione in G. \textsc{Falcon} -- C.
  \textsc{Fraenkel} (Edd.), \emph{Ordinamento processuale amministrativo
  tedesco}.}. La norma fa espresso riferimento all'esecuzione
(indiretta) di una sentenza di annullamento nella parte in cui
contestualmente ingiunge all'amministrazione la revoca della già
introdotta esecuzione del provvedimento caducato (§ 113/1 secondo alinea
\emph{VwGO}), ovvero di adempimento dell'obbligo di emettere un certo
atto o di provvedere nel rispetto della concezione giuridica del
tribunale (§ 113/5 \emph{VwGo}) o, infine, il rilascio di un
provvedimento positivo (§ 123 \emph{VwGO}).

Rimedi indiretti ulteriori, prodromici allo \emph{Zwangsgeld}, ma meno
impattanti e di valore per lo più simbolico, sono l'interpello
dell'autorità gerarchicamente superiore rispetto a quella che dovrebbe
eseguire la sentenza, ovvero il ricorso all'opinione pubblica attraverso
una petizione popolare o mediante il coinvolgimento della stampa al fine
di sollecitare o quantomeno esercitare una pressione psicologica
sull'amministrazione inottemperante\footnote{In \textsc{F. Secchi},
  \emph{op. cit.}, p.~115.}.

La limitazione della procedura esecutiva nei confronti della pubblica
amministrazione a un rimedio come lo \emph{Zwangsgeld}, perlomeno nei
casi previsti dal § 172 \emph{VwGO}, la mette al riparo dall'esecuzione
diretta, anche se non è del tutto esclusa la possibilità di accedere
alla tutela esecutiva secondo i dettami del codice di rito nel caso di
insuccesso della procedura di coazione indiretta o, a certe condizioni,
in alternativa alla stessa. Ciò sarà consentito laddove si richieda
all'amministrazione una prestazione fungibile, mentre, per quanto
riguarda l'emissione di un atto amministrativo, rimane ferma la rigida
separazione dei poteri tra giurisdizione e amministrazione.

Quanto alla natura giuridica, lo \emph{Zwangsgeld} è un semplice mezzo
di coercizione e non già un istituto di matrice sanzionatoria. L'ammenda
di cui al § 172 \emph{VwGO}, così come altre disposizioni che fanno
riferimento a tale rimedio, non potrebbe, salvo eccezioni, essere
applicata e portata a esecuzione laddove il destinatario abbia nel
frattempo adempiuto. Lo scopo precipuo dello \emph{Zwangsgeld} è quello
di determinare il debitore all'adempimento di un obbligo, attraverso la
pressione mediata esercitata sul destinatario dalla minaccia di una
penale nei casi in cui, in linea di massima, non sussiste la possibilità
di attivare un meccanismo di coercizione diretta, oppure perché tale
meccanismo è a forte rischio di insuccesso. La coazione indiretta viene
dunque in rilievo quando debba darsi esecuzione ad obblighi di fare
infungibili, la cui realizzazione è inevitabilmente rimessa alla volontà
di un determinato soggetto, oppure nell'ambito degli obblighi di
tollerare o di astenersi da una certa attività che, per definizione,
possono essere solo indirettamente coercibili\footnote{In
  \textsc{F. Secchi}, \emph{op. cit.}, p.~120. Cfr. O. \textsc{Remien},
  \emph{Rechtsverwirklichung durch Zwangsgeld: Vergleich,
  Vereinheitlichung, Kollisionsrecht}, Tübingen, J.C.B. Mohr, 1992, 11,
  12.}. Il § 172 \emph{VwGO} si occupa della sentenza di annullamento
solo in funzione della accessoria statuizione ingiuntiva dell'obbligo di
ripristinare la situazione antecedente all'esecuzione dell'atto
annullato e l'ambito privilegiato dello \emph{Zwangsgeld} risulterebbe
essere quello dell'inottemperanza al giudicato formatosi sulle sentenze
di adempimento, ex § 113/5 \emph{VwGO} (\emph{Verpflichtungsurteil}), ma
l'esecuzione coattiva indiretta secondo i dettami del § 172 \emph{VwGO}
opera anche in conseguenza di un \emph{Bescheidungsurteil}, con il quale
viene unicamente sancito l'obbligo dell'autorità di provvedere nel
rispetto del quadro giuridico delineato in sentenza, senza alcuna
indicazione in ordine allo specifico contenuto dell'atto da adottare. In
sostanza, l'omesso rilascio del richiesto provvedimento, così come la
riedizione del potere amministrativo in contrasto con il quadro
giuridico delineato nel \emph{Bescheidungsurteil}, consentono di
attivare il rimedio dello \emph{Zwangsgeld}. In particolare, il
ricorrente vittorioso potrà chiedere al tribunale di primo grado, con
apposita istanza, di fissare un termine entro il quale l'autorità dovrà
dare completa esecuzione alla sentenza, contestualmente determinando una
penale nell'ammontare massimo di Euro diecimila, per il caso di
persistente inottemperanza, anche oltre la scadenza del termine
predetto. In quest'ultima evenienza, sempre su richiesta di parte, il
tribunale provvederà ad applicare all'amministrazione renitente
l'ammenda stabilita ed a riscuoterla coattivamente d'ufficio, con
l'ulteriore possibilità di reiterare in ipotesi all'infinito la
procedura, fin tanto che permanga l'inadempimento della pubblica
autorità. La definitiva impossibilità di dare esecuzione alla sentenza
per causa imputabile all'amministrazione potrà dar luogo, in ogni caso,
ad una responsabilità risarcitoria della stessa. Per poter attivare la
procedura esecutiva ai sensi del § 172 \emph{VwGO} nei confronti
dell'amministrazione inadempiente, è necessario che ricorrano i seguenti
presupposti: il titolo esecutivo che sarà costituito ad esempio da una
sentenza di condanna al ripristino dello \emph{status quo ante}
accessoria ad una sentenza di annullamento (§ 113/1 secondo alinea
\emph{VwGO}), di adempimento (§ 113/5 \emph{VwGO}) o da un provvedimento
provvisorio positivo (§ 123 \emph{VwGO}), la notifica del titolo alla
controparte, ai sensi del § 167/1 \emph{VwGO} in combinato disposto con
i §§ 795, 724, 750/1 \emph{ZPO}, la formula esecutiva da apporre alla
decisione del giudice e l'inottemperanza dell'amministrazione alla
statuizione giudiziale che deve essere definitiva, ove si tratti di
sentenza.

\hypertarget{i-mezzi-di-tutela-esperibili-dalle-parti}{%
\section{I mezzi di tutela esperibili dalle
parti}\label{i-mezzi-di-tutela-esperibili-dalle-parti}}

I provvedimenti del giudice dell'esecuzione relativi alla minaccia ed
applicazione dello \emph{Zwangsgeld} possono essere eventualmente
impugnati con ricorso, ai sensi del § 146 \emph{VwGO} entro due
settimane dalla loro notifica, di regola davanti al tribunale
amministrativo di grado intermedio. In tal caso, l'impugnativa proposta
dalla pubblica amministrazione avverso l'ordinanza applicativa
dell'ammenda, comporterà l'automatica sospensione dell'efficacia
esecutiva del provvedimento\footnote{§ 172 \emph{VwGO}. Cfr. F.
  \textsc{Schoch} et al., \emph{Verwaltungsrecht - VwGO: Kommentar},
  München, C.H. Beck, 2020, n.~52.}. Allo stesso modo, il privato potrà
avvalersene nel caso in cui il tribunale rigetti, sempre con ordinanza,
l'istanza volta ad ottenere la minaccia o l'irrogazione della penale.
Con tale rimedio possono essere fatti valere soltanto i vizi formali
della procedura esecutiva, cioè il mancato rispetto delle regole
procedurali disciplinanti la stessa\footnote{§ 167 \emph{VwGO}. Cfr. F.
  \textsc{Schoch} et al., \emph{Ibidem}, n.~5.}. Avverso la decisione
sul predetto ricorso, non è dato alcun ulteriore mezzo di tutela,
conformemente a quanto stabilito dal § 152/1 \emph{VwGO}. Diversamente,
ove sia in contestazione da parte dell'amministrazione il diritto di
procedere ad esecuzione forzata da parte del privato, potrà essere
introdotto ricorso per opposizione all'esecuzione, ai sensi del § 167/1
\emph{VwGO}, in combinato disposto con il § 767 \emph{ZPO}. Scopo di
questo rimedio è quello di elidere l'efficacia esecutiva del titolo, non
potendo chiaramente essere rimesso in discussione il giudicato. In tal
senso, il § 767/2 \emph{ZPO} prescrive che possono essere fatte valere
soltanto quelle eccezioni che si fondino su circostanze sopravvenute
all'udienza per la discussione della causa, nell'ambito del processo di
cognizione; inoltre, l'autorità amministrativa dovrà sollevare tutte le
eccezioni deducibili al momento dell'opposizione, come previsto al §
767/3 \emph{ZPO}, ad esempio la sopravvenuta modifica della situazione
di fatto o di diritto rispetto a quella sulla quale è sceso il
giudicato, così come il sopravvenuto adempimento dell'obbligo da esso
discendente. Si pensi all'entrata in vigore di un nuovo piano regolatore
in contrasto con il permesso di costruire al cui rilascio
l'amministrazione veniva condannata con sentenza di adempimento
(\emph{Verplichtungsurteil}): fintanto che non venga rilasciato il
titolo autorizzativo, il giudicato avente a oggetto il riconoscimento
della relativa pretesa non è al riparo da eventuali sopravvenienze di
diritto, a differenza di ciò che accade nel diritto civile, ove è sempre
irrilevante il mutamento del quadro normativo entro il quale dovrebbe
essere eseguito il giudicato\footnote{\emph{BVerwG} 26.10.1984, in
  \emph{NVwZ} 1985, 563.}. Competente a giudicare dell'opposizione è il
tribunale di prima istanza.

\hypertarget{il-rapporto-fra-lo-zwangsgeld-e-il-risarcimento-del-danno-da-giudicato}{%
\section{Il rapporto fra lo Zwangsgeld e il risarcimento del danno da
giudicato}\label{il-rapporto-fra-lo-zwangsgeld-e-il-risarcimento-del-danno-da-giudicato}}

Una delle peculiarità del sistema tedesco di coercizione indiretta è
rappresentata dalla integrale devoluzione alle casse dello Stato delle
somme ricavate dalle ammende. La diversa impostazione che vuole che il
creditore sia beneficiario delle somme, abbracciata dai paesi che, come
la Francia, hanno subito le influenze del diritto romano, in cui vi era
una commistione fra l'\emph{astreinte} ed il risarcimento del danno, è
stata fortemente criticata in quanto, avendo il ricorrente vittorioso la
possibilità di agire in via risarcitoria per il caso di ritardata od
omessa esecuzione del giudicato da parte dell'amministrazione,
attribuire al privato anche l'importo della penale vorrebbe dire
arricchirlo ingiustificatamente. La soluzione seguita in Francia
presenterebbe lo svantaggio di rendere incerti i confini tra lo
\emph{Zwangsgeld} e il risarcimento del danno, poiché l'associazione tra
i due istituti pregiudicherebbe l'effetto di coazione che è proprio
dell'ammenda ex § 172 \emph{VwGO} e, allo stesso tempo, la possibilità
di cumularli porterebbe il creditore a ricevere troppo. Inoltre, la
destinazione dello \emph{Zwangsgeld} alle casse dello Stato sarebbe
ulteriormente funzionale a garantire il rispetto delle decisioni
giurisdizionali e quindi il prestigio dell'amministrazione della
giustizia.

Per quanto concerne il rimedio risarcitorio, la disciplina è quella
valevole per tutte le ipotesi di responsabilità della pubblica
amministrazione per i danni cagionati nell'esercizio dei propri doveri
d'ufficio. Le norme fondamentali in materia sono l'art. 34 del
\emph{Grundgesetz} e il § 839 del \emph{Bürgerliches Gesetzbuch}, le
quali vanno lette in combinato disposto\footnote{In \textsc{F. Secchi},
  \emph{op. cit.}, p.~184. Cfr. S. \textsc{Detterbeck},
  \emph{Allgemeines Verwaltungsrecht: mit Verwaltungsprozessrecht},
  München, C.H. Beck, 20. Auflage 2022, 379.}. In base al c.~1 del § 839
\emph{BGB} i danni causati intenzionalmente o con negligenza, in
violazione di un dovere del funzionario, vengono dallo stesso
integralmente risarciti. Se si tratta di mera negligenza, la
responsabilità del funzionario potrà essere invocata in via sussidiaria,
ovvero soltanto nel caso in cui non vi sia un'altra via per ottenere il
risarcimento, ad esempio per contratto, per legge, o in base al sistema
di assicurazione sociale\footnote{In \textsc{F. Secchi}, \emph{op.
  cit.}, p.~184. Cfr. U. \textsc{Karpen}, \emph{L'esperienza della
  Germania}, in D. \textsc{Sorace} (Ed.), \emph{La responsabilità
  pubblica nell'esperienza giuridica europea}, Bologna, Il Mulino, 1994.}.
Per converso, l'art. 34 della \emph{Grundnorm} trasferisce la
responsabilità sulla pubblica autorità da cui il funzionario dipende,
salvo il regresso nei casi di dolo e colpa grave per evitare che i
funzionari abusino dell'immunità della responsabilità personale loro
garantita e radica in capo al giudice ordinario la giurisdizione per le
azioni per le azioni di responsabilità nei confronti del potere
pubblico, stabilendo al terzo alinea, con riferimento al diritto al
risarcimento e al diritto di rivalsa, che non può mai essere esclusa
l'azione di fronte alla giurisdizione ordinaria.

\hypertarget{la-giustizia-amministrativa-francese}{%
\chapter{La giustizia amministrativa
francese}\label{la-giustizia-amministrativa-francese}}

\hypertarget{la-genesi-del-droit-administratif}{%
\section{\texorpdfstring{La genesi del \emph{droit
administratif}}{La genesi del droit administratif}}\label{la-genesi-del-droit-administratif}}

In Francia, durante il secolo XVIII, si sviluppa progressivamente una
forte \emph{administration royale}, accentrata e gerarchizzata, dotata
di poteri speciali e sottoposta a giurisdizioni apposite. E'
un'amministrazione presente tanto in centro quanto in periferia,
costituita da funzionari borghesi di medio e basso ceto riuniti nel
\emph{Conseil du Roi}, organo che, oltre a detenere la potestà
legislativa, funge sia da suprema corte di giustizia, in quanto dotato
del potere di annullare i decreti di tutti i tribunali ordinari, sia da
tribunale superiore amministrativo, perché da esso dipendono tutte le
giurisdizioni speciali. Durante la fine dell'\emph{Ancien regime} si
formalizza la specialità del diritto amministrativo, sia in relazione
alla peculiarità e ai maggiori poteri conferiti alle amministrazioni
pubbliche, sia per quanto riguarda la specificità del contenzioso
amministrativo che non era affidato al potere giudiziario. In questo
periodo si generalizzano le \emph{corvées}\footnote{Termine francese,
  utilizzato sia in Francia che in Italia, che significa
  \emph{corrogare}, nel senso di chiedere una giornata di lavoro. Nel
  Medioevo, la \emph{corvée} indicava un'imposta, poi abolita durante la
  rivoluzione, che veniva richiesta dal signore ai suoi servi, da
  estinguere con un certo numero di giornate di lavoro - Cfr. A.
  \textsc{Dijaux}, \emph{La corvée}, \emph{Chambra d'Òc, Portal
  Français, 100 mots du trésor FR}, in
  \url{http://www.chambradoc.it/traditions/La-corvee.page} (Consultato:
  22 gennaio 2023). Nella Francia dell'\emph{Ancien régime}, il termine
  indicava le giornate di lavoro che i sudditi dovevano al re per la
  manutenzione delle strade pubbliche - Cfr. \textsc{Treccani},
  \emph{corvée}, \emph{Treccani - Enciclopedia on line}, in
  \url{https://www.treccani.it/enciclopedia/corvee} (Consultato: 22
  gennaio 2023).}, si sviluppa la polizia dei mestieri a difesa
dell'ordine pubblico e crescono le espropriazioni forzate, così come
l'imposizione fiscale e il potere autoritativo dell'amministrazione
verso i suoi contraenti. Di contro, le armi che gli amministrati hanno
nei confronti del potere amministrativo non sono così forti: a favore
del cittadino vi è il rimedio specifico costituito dal cosiddetto
ricorso gerarchico, a mezzo del quale egli può rivolgersi all'organo
gerarchicamente sovraordinato all'amministrazione che aveva emanato
l'atto lesivo e può richiedere la verifica della legalità dell'atto.

La Rivoluzione francese, pur portando innovazioni rilevanti, lascia
immutata l'ampia discrezionalità delle amministrazioni pubbliche che
conservano e potenziano le loro prerogative. In ossequio al principio
della separazione dei poteri, i giudici sono estromessi dalla conoscenza
degli affari amministrativi e non potranno in alcun modo turbare le
attività dei corpi amministrativi. Le forze rivoluzionarie avevano
dimostrato infatti una certa diffidenza verso la magistratura,
tradizionalmente formata da elementi vicini alle classi aristocratiche
e, nel 1789-1790, prima l'Assemblea nazionale e poi l'Assemblea
costituente avevano sancito in forma solenne che gli organi
giurisdizionali non avrebbero potuto intervenire sull'amministrazione
\footnote{Decreto del 16-24 agosto 1790 sull'ordinamento giudiziario:
  \emph{``Le funzioni giurisdizionali sono distinte e rimangono sempre
  separate dalle funzioni amministrative. I giudici non potranno, sotto
  pena di prevaricazione, interferire in qualunque modo sulle operazioni
  dei corpi amministrativi, né citare avanti a sé gli amministratori a
  motivo dell'esercizio delle loro funzioni''}. Cfr. A. \textsc{Travi},
  \emph{Lezioni di giustizia amministrativa}, Torino, G. Giappichelli,
  14. ed 2021, 8, 9}. Ora, ad occuparsi delle controversie
amministrative saranno i ministri; scompare il \emph{Conseil du Roi},
viene introdotto il Prefetto e, come agente dell'Esecutivo in provincia,
reintrodotta la figura dell'intendente.

\hypertarget{dalla-giustizia-ritenuta-alla-giustizia-delegata-il-conseil-detat}{%
\section{\texorpdfstring{Dalla giustizia ``ritenuta'' alla giustizia
delegata: il \emph{Conseil
d'Etat}}{Dalla giustizia ``ritenuta'' alla giustizia delegata: il Conseil d'Etat}}\label{dalla-giustizia-ritenuta-alla-giustizia-delegata-il-conseil-detat}}

La Costituzione dell'anno VIII (dicembre 1799) istituisce il
\emph{Conseil d'Etat} (Consiglio di Stato), concepito inizialmente come
organo consultivo del Governo, al quale, nell'epoca napoleonica,
verranno affidate attribuzioni assai ampie, tra cui principalmente il
potere di redigere i progetti di legge e i regolamenti
dell'amministrazione pubblica, gli atti legislativi primari e secondari,
l'alta amministrazione, il controllo sui ministri e sugli enti pubblici
e la risoluzione di controversie amministrative.

Riguardo ai ricorsi, il Consiglio di Stato formalmente esprimeva un
parere al Capo dello Stato, al quale solo, come rappresentante supremo
del potere esecutivo, spettava assumere la decisione che però, nella
pratica, si uniformava sempre al parere e l'intervento del Capo dello
Stato finiva con l'attribuire ancora maggiore autorevolezza al parere e
all'organo che lo esprimeva. Un decreto di Napoleone del 1806 istituì,
in seno al Consiglio di Stato, un'apposita Commissione del contenzioso
con il compito di istruire i ricorsi proposti contro gli atti
(\emph{décisions}) delle amministrazioni centrali e locali. Per
rafforzarne l'imparzialità, ai consiglieri che componevano la
Commissione non potevano essere affidati compiti di amministrazione
attiva\footnote{Cfr. A. \textsc{Travi}, \emph{Ibidem}, 9.}. Il Consiglio
di Stato, mantenuto anche con la Restaurazione (1814-1815) detiene
quindi, insieme al \emph{Conseil de Prefecture}, la giurisdizione
amministrativa, non più affidata ai ministri e agli intendenti: si passa
da un sistema tradizionale di giustizia ``ritenuta'', in cui il re era
giudice supremo, dal quale ``emana ogni giustizia'' (secondo la massima
``\emph{toute justice émane du roi}'') ad una giustizia ``delegata'',
affidata pienamente al Consiglio di Stato, al quale, prima
transitoriamente, con la Costituzione del 4 novembre 1848 e poi,
definitivamente, con una legge del 24 maggio 1872, fu riconosciuta anche
formalmente la competenza a decidere il ricorso, senza più la necessità
di una sanzione da parte del Capo dello Stato. Quest'ultima riforma del
1872 attribuiva così al Consiglio di Stato tutti i caratteri di un vero
e proprio giudice amministrativo.

La giurisprudenza del \emph{Conseil d'Etat} consente al \emph{droit
administratif} di passare da un semplice insieme di regole derogatorie
al diritto privato ad un autentico sistema autonomo, costituito da
principi e concetti propri, quali il \emph{service public,}
l'\emph{excés de pouvoir}, la \emph{résponsabilité administrative}. Il
contenzioso dinanzi al \emph{Conseil d'Etat} e le decisioni che da esso
emanano assicurano per la prima volta un equilibrio efficace fra
prerogative pubbliche e garanzie degli amministrati.

Nella prima metà dell'Ottocento però, si assiste ad un ampliamento della
competenza giurisdizionale del giudice ordinario, il quale rivendica
un'attitudine naturale a conoscere non solo delle controversie
concernenti la proprietà, ma anche delle dispute in materia di contratti
e di altri istituti regolati dal codice civile. Pertanto, il criterio
utilizzato per determinare la competenza del giudice diviene quello
d'individuare il diritto sostanziale applicabile al caso concreto, cui
si aggiunge successivamente il criterio di competenza fondato sulla
distinzione tra \emph{actes de puissance publique}, affidati al giudice
amministrativo, con i quali l'amministrazione agisce come depositaria
dell'autorità attribuita dall'esercizio del potere esecutivo, e
\emph{actes de gestion} che l'amministrazione pone in essere in qualità
di garante dei servizi pubblici, affidati al giudice ordinario.

Il Novecento vede consolidarsi il criterio di competenza basato sul
concetto di \emph{service public}\footnote{Cfr. M. \textsc{Hauriou} et
  al., \emph{Précis de droit administratif et de droit public}, Paris,
  Dalloz, 12e éd 2002, 13 ss.} riferito ad attività dirette alla
soddisfazione di interessi comuni e caratterizzate dall'erogazione di
prestazioni e servizi ai consociati. Si fa strada la distinzione tra
controversie relative ai \emph{services publics administratifs} la cui
giurisdizione spetta al Consiglio di Stato e controversie riguardanti i
\emph{services publics industriels et commerciaux} che spettano al
giudice ordinario, ma tale distinzione non si rivela essere così netta
nel momento in cui alcune controversie rientranti nell'ambito dei
\emph{service publics administratifs} sono attratte nella competenza del
giudice ordinario, in quanto il criterio del \emph{service public} è
sempre suscettibile di cedere se il regime dell'atto oggetto della
controversia mostri o che il \emph{service administratif} ha agito
nell'ambito del diritto comune, o che il \emph{service industriel et
commercial} ha adottato strumenti di \emph{droit administratif}. In
sostanza, si torna ad un criterio generale di competenza, in virtù del
quale è la natura delle norme sostanziali e degli atti di gestione del
servizio a determinare la giurisdizione.

La previsione di competenze del giudice ordinario ha comportato la
necessità di istituire, nel 1848, un organo che potesse decidere, nei
casi controversi, se la vertenza spettasse allo stesso giudice ordinario
o al giudice amministrativo: il Tribunale dei conflitti. Tale tribunale,
per assicurare l'equilibrio tra le due giurisdizioni, è composto da uno
stesso numero di magistrati della Cassazione e di consiglieri di
Stato\footnote{Cfr. A. \textsc{Travi}, \emph{Lezioni di giustizia
  amministrativa}, 11.}.

Con l'entrata in vigore, il 4 ottobre 1958, della Costituzione francese
che ha dato origine alla Quinta Repubblica, è stato istituito il
Consiglio costituzionale (\emph{Conseil constitutionnel}), organo
accentrato che svolge anche un controllo di legittimità costituzionale.
Successivamente, con due decisioni del Consiglio costituzionale, la
prima del 22 luglio 1980 e la seconda del 23 gennaio 1987, si è
conferito valore costituzionale all'indipendenza ed alla competenza
della giurisdizione amministrativa\footnote{Cfr. \textsc{Vie Publique},
  \emph{Pourquoi existe-t-il une justice administrative?},
  \emph{vie-publique.fr}, in \url{https://tinyurl.com/mw2pxm4z}
  (Consultato: 22 gennaio 2023).}. La revisione costituzionale del 23
luglio 2008 ha confermato questo radicamento costituzionale introducendo
la nozione di \emph{ordre administratif} \footnote{Tre sono i gradi
  della giurisdizione amministrativa francese: i tribunali
  amministrativi (\emph{tribunaux administratifs}), le corti
  amministrative d'Appello (\emph{courts administratives d'Appel}) e il
  Consiglio di Stato, Cfr. \textsc{Ministère de la Justice},
  \emph{L'ordre administratif}, \emph{justice.gouv.fr}, in
  \url{https://www.justice.gouv.fr/organisation-de-la-justice-10031/lordre-administratif-10034/}
  (Consultato: 22 gennaio 2023).} all'art. 65 della Carta e, con
decisione del 3 dicembre 2009, il Consiglio costituzionale ha
qualificato la Corte di cassazione e il Consiglio di Stato quali organi
al vertice delle due giurisdizioni riconosciute dalla
Costituzione\footnote{§ 3 CC 3 décembre 2009, \emph{``juridictions
  placées au sommet de chacun des deux ordres de juridiction reconnus
  par la Constitution''}. Cfr. \textsc{Conseil Constitutionnel},
  \emph{Décision n° 2009-594 DC du 3 décembre 2009},
  \emph{conseil-constitutionnel.fr}, in
  \url{https://tinyurl.com/2p95r35b} (Consultato: 22 gennaio 2023).}.

\hypertarget{il-pouvoir-dinjonction-del-giudice-amministrativo}{%
\section{\texorpdfstring{Il \emph{pouvoir d'Injonction} del giudice
amministrativo}{Il pouvoir d'Injonction del giudice amministrativo}}\label{il-pouvoir-dinjonction-del-giudice-amministrativo}}

La legge n.~95-125 del 8 febbraio 1995 ha introdotto nel codice dei
tribunali amministrativi e delle corti amministrative di appello
(\emph{code des tribunaux administratifs et des cours administratives
d'appel}) un nuovo titolo, dedicato all'esecuzione del giudicato e
inserito all'interno del libro relativo alle attribuzioni
giurisdizionali dei tribunali e delle corti. L'art. L. 8-2 dello stesso
codice\footnote{Art. L. 8-2 C. trib. mar. \textsc{Sénat}, \emph{Le
  pouvoir d'injonction et le prononcé d'astreintes pour l'exécution des
  décisions de justice}, \emph{senat.fr}, in
  \url{https://www.senat.fr/rap/l98-380/l98-3805.html} (Consultato: 22
  gennaio 2023).} stabilisce le modalità con cui i giudici dei suddetti
tribunali e corti d'appello possono ingiungere all'amministrazione di
ottemperare alle sentenze distinguendo due casi, a seconda del potere,
più o meno vincolato, dell'amministrazione stessa. Al comma 1 si prevede
che qualora un ente pubblico, o un organismo di diritto privato
incaricato della gestione di un pubblico servizio, debba adottare una
determinata misura per l'esecuzione della sentenza, il tribunale
amministrativo o la corte amministrativa di appello, precedentemente
chiamati a decidere in tal senso, prescrivono tale misura, fissando
altresì un termine entro il quale l'esecuzione della stessa debba
avvenire. Il comma 2 del medesimo articolo L. 8-2 aggiunge invece che,
qualora un ente pubblico, o un organismo di diritto privato incaricato
della gestione di un servizio pubblico, debba adottare un nuovo
provvedimento, a seguito di una nuova fase istruttoria, il tribunale
amministrativo o la corte amministrativa di appello, precedentemente
chiamati a decidere in tal senso, stabiliscono che l'adozione del nuovo
provvedimento debba avvenire entro un determinato termine\footnote{L'articolo
  L. 8-2 utilizza testualmente i termini di \emph{``jugement''} e di
  \emph{``arrêt''}, con cui rispettivamente si intendono le sentenze
  emesse dai tribunali amministrativi e le sentenze dei gradi successivi
  di giurisdizione, c.d. \emph{Hautes jurisdictions} o
  \emph{jurisdictions souvraines}, Cfr. \textsc{Juri'Predis},
  \emph{Différence entre un arrêt et une décision},
  \emph{juripredis.com}, in \url{https://tinyurl.com/ye22pd59}
  (Consultato: 22 gennaio 2023).}.

L'articolo L. 8-3 del codice dei tribunali amministrativi e delle corti
amministrative di appello permette al giudice chiamato a decidere di
accompagnare al potere ingiuntivo prescritto dall'articolo L. 8-2 la
previsione di una penale, la cosiddetta \emph{astreinte}, nell'ambito
della stessa decisione.

L'art. L. 8-4 prevede che, nel caso di mancata esecuzione della sentenza
emessa dal tribunale amministrativo o della corte amministrativa di
appello, la parte interessata all'ottemperanza può chiedere al giudice
dello stesso tribunale o della stessa corte amministrativa di appello
che ha pronunciato la decisione di assicurarne l'esecuzione. Il giudice
potrà così definire le modalità di esecuzione, fissare un termine per
l'esecuzione e stabilire una penale (\emph{astreinte}), ferma restando
la possibilità di rimettere la richiesta di esecuzione al Consiglio di
Stato\footnote{Art. L. 8-4 C. trib. mar.}.

Tuttavia oggi, venticinque anni dopo aver ottenuto la facoltà di
ingiungere all'amministrazione l'esecuzione delle sentenze, il giudice
amministrativo si dimostra normalmente molto prudente nell'esercizio di
tale potere. Tale constatazione deriva dalla lettura di alcune ordinanze
da esso pronunciate durante la pandemia da Covid-19, dalla cui lettura
emerge che le circostanze eccezionali del momento, parallelamente al
rafforzamento dei poteri normalmente attribuiti alle autorità esecutive,
legittimavano un utilizzo più esteso del \emph{pouvoir d'injonction}. In
generale, si è notata da parte del giudice amministrativo una qualche
reticenza nell'esplicitare il suo pieno potere ingiuntivo: sia esso o
meno giuridicamente fondato, egli dimostra una sorta di autolimitazione
quando si tratta di agire d'imperio\footnote{Cfr. X. D. \textsc{de}
  \textsc{Boulois}, \emph{Le référé-liberté pour autrui. Une société
  commerciale au secours du droit à la vie}, in {«Revue des droits et
  libertés fondamentaux-RDLF»} (2013) 12, p.~2137}. In particolare,
quando pronuncia un'ingiunzione, si obbliga a non sostituire il suo
apprezzamento a quello dell'amministrazione e cerca di non interferire
con essa in presenza di poteri discrezionali. In questa prospettiva, il
potere ingiuntivo che il giudice indirizza verso l'amministrazione è
costantemente circoscritto, così come sono limitate le conseguenze
giuridiche che ne derivano\footnote{Cfr. M. \textsc{Bartolucci},
  \emph{Le pouvoir d'injonction du juge administratif revisité par les
  circonstances exceptionnelles de la crise sanitaire du Covid-19},
  \emph{actu-juridique.fr}, in \url{https://tinyurl.com/4sfhd3ux}
  (Consultato: 22 gennaio 2023).}.

\hypertarget{lesecuzione-delle-sentenze-del-giudice-amministrativo}{%
\section{L'esecuzione delle sentenze del giudice
amministrativo}\label{lesecuzione-delle-sentenze-del-giudice-amministrativo}}

L'amministrazione è tenuta a conformarsi spontaneamente alla sentenza
pronunciata dal giudice amministrativo il quale talvolta indica anche
come l'amministrazione stessa debba procedere in tal senso. Tuttavia,
qualora invece l'amministrazione non ottemperi, o alla decisione del
giudice segua solo un'esecuzione parziale, la parte interessata può
agire secondo due modalità, a seconda che l'amministrazione sia stata
condannata a versare una somma di denaro, oppure debba adottare un nuovo
provvedimento a seguito della sentenza di annullamento di un atto
amministrativo\footnote{Cfr. \emph{``Qu'est-ce que l'exécution des
  décisions?''} in \textsc{Cour administrative d'appel de Paris},
  \emph{L'exécution des décisions du juge administratif},
  \emph{paris.tribunal-administratif.fr}, in
  \url{https://tinyurl.com/5frr43pw} (Consultato: 22 gennaio 2023).}.

Nel primo caso, se la domanda del privato verte unicamente sul
versamento di una somma di denaro da parte dell'amministrazione, egli
può attivare la \emph{procédure de la contrainte au paiement}, chiamata
anche \emph{procédure du paiment forcé}\footnote{Tradotto letteralmente
  come procedura di pagamento forzato.}che gli permetterà di ottenere il
pagamento della somma dovuta, a condizione che la decisione del giudice
sia divenuta definitiva, ne stabilisca precisamente l'ammontare e che
l'amministrazione non abbia versato la somma entro il termine stabilito
di due mesi dalla notifica della sentenza. La domanda, se il debitore è
lo Stato, deve essere indirizzata al \emph{Comptable public},
normalmente la Direzione Regionale delle Finanze Pubbliche, per ottenere
il pagamento. Qualora invece il debitore sia un ente territoriale
(regione, dipartimento o comune) o un'altra struttura pubblica
(\emph{établissement public}), occorre rivolgersi al Prefetto o
all'autorità di tutela preposta, sollecitando l'erogazione d'ufficio
della somma dovuta.

In tutti gli altri casi di inottemperanza da parte dell'amministrazione,
il privato può richiedere al giudice l'esecuzione. In generale, tale
richiesta al giudice non può essere presentata prima della scadenza del
termine di tre mesi dalla notificazione della sentenza, ma può essere
inoltrata entro un termine diverso nei casi seguenti:

\begin{enumerate}
\def\labelenumi{\arabic{enumi})}
\tightlist
\item
  qualora la sentenza stabilisca un termine preciso per l'esecuzione, la
  domanda non può che essere presentata entro tale termine;
\item
  se l'amministrazione rifiuta espressamente di conformarsi alla
  sentenza, non sussiste un termine per la richiesta al giudice;
\item
  nel caso di sentenza che stabilisca l'attuazione di misure urgenti,
  l'esecuzione può essere richiesta immediatamente.
\end{enumerate}

Per l'esecuzione di una sentenza pronunciata da un tribunale
amministrativo, la domanda di esecuzione dovrà essere presentata allo
stesso tribunale che ha reso il giudizio e, se il giudizio è stato
emesso da una corte amministrativa d'appello, occorrerà rivolgersi alla
stessa corte. Per l'esecuzione delle decisioni del Consiglio di Stato o
di una giurisdizione amministrativa speciale (in particolare, la
\emph{Cour nationale du droit d'asile}), la domanda dovrà essere
indirizzata alla delegazione all'esecuzione delle decisioni di giustizia
della \emph{section du rapport et des études du Conseil d'Etat}. Nel
caso in cui la domanda sia rivolta per errore ad un giudice non
competente, questi la inoltrerà al giudice competente informando le
parti\footnote{Cfr. \emph{``Comment faire exécuter les décisions rendues
  par le juge administratif?''} in \textsc{Cour administrative d'appel
  de Paris}, \emph{L'exécution des décisions du juge administratif}.}.

Per il privato, al fine di presentare la domanda di esecuzione, non è
obbligatorio ricorrere all'assistenza di un avvocato. La domanda può
essere inoltrata a mezzo dell'applicazione \emph{Télérecours citoyens},
accessibile dal sito www.telerecours.fr , oppure a mezzo posta con
raccomandata A/R alla giurisdizione competente, avendo cura di indicare
la sentenza che si ritiene non ottemperata, le difficoltà che si
riscontrano, le misure che si ritengono dover essere intraprese per
rimediare alla situazione specifica e, contestualmente, richiedere la
pronuncia di una \emph{astreinte} a carico dell'amministrazione
renitente, volta ad indurre la medesima amministrazione ad eseguire
quanto statuito dal giudice.

La procedura di esame della domanda di esecuzione si svolge in due
fasi\footnote{Cfr. \emph{``Comment se déroule l'examen de ma demande
  d'exécution?''} in \textsc{Cour administrative d'appel de Paris},
  \emph{Ibidem}.}. Durante la prima fase, denominata \emph{phase
administrative}, entro un termine massimo di sei mesi, il presidente del
tribunale amministrativo, della corte amministrativa di appello o della
sezione di rapporto e di studi del Consiglio di Stato, promuove tutte le
iniziative e le indagini necessarie ad assicurare l'esecuzione della
sentenza da parte dell'amministrazione. Qualora rilevi che
l'amministrazione abbia nel frattempo ottemperato o che la domanda sia
infondata, procederà all'archiviazione dandone comunicazione alle parti.

La seconda fase, la \emph{phase jurisdictionelle}, può essere avviata
sia su iniziativa della giurisdizione competente, quando il presidente
ritenga sia necessario prescrivere determinate misure di esecuzione, ad
esempio la pronuncia di una \emph{astreinte}, oppure se la domanda di
esecuzione non è stata soddisfatta entro il termine di sei mesi, sia su
iniziativa di parte, qualora venga impugnata l'archiviazione entro il
termine di un mese dalla sua notifica. Tale impugnazione dovrà essere
proposta alla \emph{section du contentieux} del Consiglio di Stato. La
\emph{phase juridictionelle} può sfociare nella pronuncia di
un'ingiunzione all'amministrazione, accompagnata da una
\emph{astreinte}, se il giudice ritiene che la sentenza sia rimasta
inadempiuta. L'ingiunzione consiste nell'imporre all'amministrazione di
adottare un determinato provvedimento o di riesaminarne la spettanza
entro un termine fissato dal medesimo giudice, mentre l'\emph{astreinte}
consiste in una penale stabilita a carico dell'amministrazione, il cui
ammontare è suscettibile di aumentare fin tanto che l'inottemperanza
persiste. Tuttavia, qualora il presidente della giurisdizione competente
ritenga che le cautele adoperate siano suscettibili di permettere
l'esecuzione della sentenza a breve termine può stabilire, dandone
previa informazione alle parti, che l'avvio della procedura giudiziale
abbia luogo solo una volta scaduto un termine supplementare di quattro
mesi\footnote{Cfr. \emph{``Que faire lorsque l'Administration n'exécute
  pas le jugement d'un tribunal administratif ou l'arrêt d'une cour
  d'appel?''} in \textsc{Cour administrative d'appel de Paris},
  \emph{Les fiches pratiques de la justice administrative},
  \emph{paris.tribunal-administratif.fr}, in
  \url{https://tinyurl.com/3fw67rwb} (Consultato: 22 gennaio 2023).}.

\hypertarget{lastreinte}{%
\section{\texorpdfstring{L'\emph{astreinte}}{L'astreinte}}\label{lastreinte}}

Il sistema francese di giustizia amministrativa è quello che più si
avvicina al nostro per tradizione giuridica, ma, sotto il profilo
dell'esecuzione della sentenza amministrativa, esso presenta, nella
sostanza, maggiori punti di convergenza con quello tedesco, sia per
quanto attiene al raffronto tra il \emph{pouvoir d'injonction} e i
poteri direttivi di cui al \emph{Verpflichtungsurteil} ovvero al
\emph{Folgenbeseitigungsurteil}, sia per quanto riguarda le misure di
coazione indiretta volte a piegare la resistenza dell'amministrazione
inottemperante e cioè le \emph{astreintes} rispetto allo
\emph{Zwangsgeld}, con il naturale corollario, comune ad entrambi i
sistemi, di un rigoroso recepimento del principio della separazione dei
poteri\footnote{In \textsc{F. Secchi}, \emph{op. cit.}, p.~204, l'autore
  riprende D. \textsc{De Pretis}, \emph{Il processo amministrativo in
  europa: caratteri e tendenze in Francia, Germania, Gran Bretagna e
  nell'Unione europea}, che osserva come il problema dell'esecuzione
  della sentenza amministrativa risiede, nell'ordinamento francese così
  come in altri ordinamenti continentali, nella difficile coesistenza
  tra la riserva amministrativa del potere di rinnovare l'attività
  provvedimentale a seguito della decisione del giudice e la garanzia
  dell'effettività della tutela giurisdizionale e quindi dell'esecuzione
  delle statuizioni giudiziali.}.

La legge di riforma del contenzioso amministrativo 95-125 dell'8
febbraio 1995 ridefinisce la disciplina sull'esecuzione del giudicato
amministrativo, ribaltando in primo luogo il principio, sino ad allora
mai scalfito, secondo cui il giudice amministrativo, in virtù dell'art.
13 della legge rivoluzionaria 16-24 agosto 1790 e, in generale, della
separazione dei poteri, non poteva ordinare un \emph{facere} o un
\emph{non facere} alla pubblica autorità e, in secondo luogo, ampliava
il potere di \emph{astreinte}, prima attribuito in via esclusiva al
\emph{Conseil d'Etat} ed esercitabile anche \emph{ex
officio}\footnote{In \textsc{F. Secchi}, \emph{op. cit.}, p.~205,
  l'autore descrive il primo caso di applicazione \emph{ex officio} di
  un'\emph{astreinte} (CE 28 maggio 2001).}, con la devoluzione dello
stesso anche ai tribunali e alle corti di appello.

L'\emph{astreinte}, quale principale mezzo di coercizione indiretta, è
uno strumento a carattere esclusivamente patrimoniale che ha lo scopo di
incentivare l'esecuzione di una sentenza di condanna, attraverso la
previsione di una sanzione pecuniaria che la parte inadempiente dovrà
versare a favore del creditore vittorioso in giudizio\footnote{Cfr.
  definizione in F. M. \textsc{Ciaralli}, \emph{Il nuovo giudizio di
  ottemperanza, con particolare riguardo alle astreintes},
  \emph{italiappalti.it}, in
  \url{https://www.italiappalti.it/leggiarticolo.php?id=3595}
  (Consultato: 22 gennaio 2023).}. Le prime applicazioni dei mezzi di
esecuzione indiretta si rinvengono nel diritto romano classico secondo
cui, nei casi di condanna a rilasciare un fondo o a realizzare un
\emph{opus}, si stabiliva che il soccombente avrebbe dovuto, in difetto,
pagare una somma pari ad una multa del valore del fondo o dell'opera da
realizzare. Viceversa, in epoca medioevale, l'applicazione degli
strumenti di induzione all'adempimento era circoscritta ai casi in cui
l'interesse del creditore non potesse essere soddisfatto attraverso
l'esperimento dell'esecuzione diretta per cui era necessaria la
partecipazione del debitore. In Francia, l'applicazione
dell'\emph{astreinte} è stata per la prima volta formalizzata da una
sentenza del Tribunale di Cray del 1811, mediante la quale il
soccombente veniva condannato a ``compiere una pubblica ritrattazione
sotto pena di dover pagare tre franchi per ogni giorno di ritardo
nell'adempimento''. L'\emph{astreinte} era di conseguenza strutturata
come una pena privata, non avente il fine di riparare un pregiudizio, ma
quello di riparare una disobbedienza. La legge n.~626 del 5 luglio 1972
ha qualificato l'\emph{astreinte} come sanzione oggetto di condanna
accessoria, il cui adempimento non estingue l'obbligazione principale,
motivo per cui il soccombente può essere condannato a corrispondere un
determinato importo al creditore vittorioso indipendentemente e in
aggiunta al risarcimento del danno, stante la cumulabilità della misura
reintegrativa con quella sanzionatoria.

Nell'ordinamento francese, al pari di quello romano, l'\emph{astreinte}
è comminabile per indurre il soccombente ad eseguire ogni sentenza di
condanna, non rilevando che la condotta ordinata sia infungibile. Da ciò
discende che tale strumento è costruito come mezzo sanzionatorio che
giustifica un trasferimento di ricchezza dalla parte inadempiente al
creditore vittorioso e che, in ossequio alla sua natura meramente
compulsorio-retributiva, mira a ``punire'' l'inosservanza di ogni tipo
di sentenza di condanna, indipendentemente dalla fungibilità della
prestazione ordinata dal giudice. Come ulteriore conseguenza, si ha la
possibilità di concorso tra due distinte procedure esecutive, una per
l'\emph{astreinte} e l'altra per l'esecuzione forzata della prestazione
originaria.

Il d.Lgs. n.~104 del 2 luglio 2010 (c.p.a.) ha introdotto anche nel
processo amministrativo italiano l'istituto dell'\emph{astreinte}, sia
pure con talune rilevanti differenze rispetto al modello francese.
L'art. 114, comma 4, lett. \emph{e)}, c.p.a. attribuisce infatti al
giudice amministrativo il potere di fissare \emph{``la somma di denaro
dovuta dal resistente per ogni violazione o inosservanza successiva,
ovvero per ogni ritardo nell'esecuzione del giudicato''}. A differenza
dell'ordinamento francese, in Italia l'\emph{astreinte} può essere
irrogato solo in sede di ottemperanza e non anche di merito: da ciò
discende che nel processo amministrativo italiano la misura coercitiva
non si configura come sanzione ad esecuzione differita, destinata a
divenire attuale se ed in quanto l'amministrazione non esegua l'ordine
contenuto nella sentenza di merito, ma presuppone che l'inadempimento
del debitore sia già stato accertato dal giudice dell'ottemperanza.

Il comma 4, lett. \emph{e)} contempla inoltre due requisiti negativi,
ovvero che il provvedimento di condanna alla misura coercitiva non sia
``\emph{manifestamente iniquo}'' e che non ricorrano ``\emph{ragioni
ostative}''. L'art. 114 c.p.a. però tace sia sui parametri in base ai
quali calcolare il \emph{quantum} della sanzione, sia sui \emph{genera}
di condotte che possono essere assistiti dall'\emph{astreinte}. Dottrina
e giurisprudenza hanno dunque avuto il compito di chiarirne il perimetro
applicativo principalmente con due orientamenti, tra loro distinti in
base alla ricostruzione della \emph{ratio}, nonché in base all'autonomia
annessa all'istituto di diritto amministrativo rispetto alla generale
previsione di \emph{astreinte} prevista nel codice di procedura civile.
L'indirizzo dottrinario accolto dalla giurisprudenza maggioritaria
sostiene che, in termini di \emph{ratio}, nulla osta all'applicazione
delle misure coercitive anche al di fuori del tradizionale perimetro
degli obblighi infungibili. In particolare, finalità
dell'\emph{astreinte} sarebbe quella di sanzionare la mancata
conformazione del soccombente all'ordine del giudice, non rilevando in
chiave strutturale il \emph{genus} della condotta rimasta inadempiuta,
analogamente a quanto è previsto nell'ordinamento francese. L'Adunanza
Plenaria del Consiglio di Stato, con sentenza n.~15 del 25 giugno 2014,
ha condiviso tale interpretazione estensiva, ponendo in rilievo un
argomento di diritto comparato che, prendendo a modello il sistema
francese, rileva che l'\emph{astreinte} si caratterizza per
un'indiscussa funzione sanzionatoria, essendo finalisticamente orientato
a costituire una pena per la disobbedienza alla statuizione giudiziaria,
anziché un risarcimento per il pregiudizio sofferto a causa di tale
inottemperanza. Viene quindi in rilievo l'argomento equitativo, in virtù
del quale, essendo le penalità di mora una pena e non una forma di
risarcimento, non sussiste un'inammissibile doppia riparazione di un
unico danno, ma l'aggiunta di una misura sanzionatoria ad una tutela
risarcitoria. L'Adunanza Plenaria precisa che la funzione deterrente e
general-preventiva delle penalità di mora verrebbe frustrata dalla
mancata erogazione della tutela ove vi sia già stato o possa essere
assicurato un integrale risarcimento. Il legislatore, recependo
l'orientamento fatto proprio dall'Adunanza Plenaria, ha novellato l'art.
114, comma 4, lett. \emph{e)}, c.p.a. aggiungendo con la legge n.~208
del 28 dicembre 2015 un ulteriore periodo che stabilisce che ``\emph{nei
giudizi di ottemperanza aventi ad oggetto il pagamento di somme di
denaro, la penalità di mora di cui al primo periodo decorre dal giorno
della comunicazione o notificazione dell'ordine di pagamento disposto
nella sentenza di ottemperanza; detta penalità non può considerarsi
manifestamente iniqua quando è stabilita in misura pari agli interessi
legali''}.

Infine, l'art. 114, comma 4, lett. \emph{e)} c.p.a., in considerazione
della specialità del debitore pubblico, con specifico riferimento alle
difficoltà nell'adempimento collegate a vincoli normativi e di bilancio,
allo stato della finanza pubblica e alla rilevanza di specifici
interessi pubblici, ha aggiunto al limite negativo della manifesta
iniquità quello della sussistenza di ``altre ragioni
ostative''\footnote{Cfr. F. M. \textsc{Ciaralli}, \emph{Ibidem}.}.

\hypertarget{gli-strumenti-di-prevenzione-dellinottemperanza-della-pubblica-amministrazione}{%
\section{Gli strumenti di prevenzione dell'inottemperanza della Pubblica
Amministrazione}\label{gli-strumenti-di-prevenzione-dellinottemperanza-della-pubblica-amministrazione}}

Accanto alle vere e proprie misure coattive fin qui esaminate, ve ne
sono altre di minore impatto quali il cosiddetto \emph{système d'aide à
l'exécution} e il \emph{mediateur de la Republique}, le cui funzioni,
dal 2011, sono state attribuite al \emph{Défenseur des droits}.

Per quanto riguarda il \emph{système d'aide à l'exécution}, al fine di
prevenire un'inottemperanza o un'esecuzione incompleta od erronea,
nonché di pungolare l'amministrazione ad adempiere gli obblighi
discendenti dal giudicato, il ricorrente vittorioso può, attraverso
un'apposita domanda, chiedere chiarimenti in ordine alla modalità con le
quali le autorità soccombenti devono conformarsi al giudicato ai
tribunali amministrativi e alle corti amministrative d'appello, per quel
che attiene all'esecuzione delle loro decisioni\footnote{Art. 12,
  decreto n.~831 del 3 luglio 1995 - \emph{CJA}, art. R 921-5.}, nonché
al Consiglio di Stato e, segnatamente, alla \emph{Section du rapport et
des études}, relativamente all'esecuzione delle sue pronunce oltre a
quelle promananti dalle giurisdizioni speciali\footnote{Art. 59, decreto
  n.~766 del 30 luglio 1963 e succ. mod. - \emph{CJA}, art. R931-1
  \emph{et} 2.}.

I tribunali e le corti d'appello vengono interessate della questione,
nella persona del loro presidente o del relatore che viene per
l'occasione designato. Il Consiglio di Stato riceve invece i reclami nei
confronti delle amministrazioni inadempienti e interviene per mezzo
della sezione \emph{du rapport et des études}, affinché l'autorità
competente ottemperi al giudicato. In quest'ultimo caso, non è
necessario né il ministero di un difensore, né attendere il decorso di
un termine dilatorio, di regola di almeno tre mesi, quando si tratti di
dare esecuzione ad una decisione involgente misure urgenti, ovvero nei
casi di rifiuto esplicito ad adempiere. Inoltre, in giurisprudenza si
propende per l'inoppugnabilità in sede contenziosa di un eventuale
rifiuto opposto dal presidente della \emph{section du rapport et des
études} alla richiesta di \emph{aide à l'exécution}\footnote{Cfr. R.
  \textsc{Chapus}, \emph{Droit du contentieux administratif}, Paris,
  Montchrestien-Lextenso éd, 13e éd 2008, 1132.}.

Il \emph{mediateur de la Republique} era una figura istituita con la
legge del 3 gennaio 1973 (art. 11, al.~2) e succ. mod., alla quale
veniva attribuito il compito d'invitare l'organo inadempiente a
conformarsi al giudicato entro un termine dal medesimo fissato, pena la
diffusione della notizia del biasimevole comportamento tenuto nella
fattispecie dall'amministrazione, attraverso una relazione da
pubblicarsi sul cosiddetto \emph{Journal officiel (``l'inexécution du
jugement fera l'objet de sa part d'un rapport spécial, publié au journal
officiel'')}. In pratica, si trattava di un mezzo di pressione morale,
volto a far leva sulla minaccia di pubblicità della vicenda. Nel 1994 il
\emph{mediateur} aveva redatto per la prima volta un rapporto speciale
su di un caso di persistente inottemperanza (\emph{Rapport du 20
septembre 1994, JO 14 octobre, p.~14588}), peraltro ancora in atto al
momento del rapporto, relativamente all'inesecuzione di una sentenza del
Tribunale amministrativo di Versailles del 22 giugno 1993 che aveva
condannato il comune di Mennecy ad erogare lo stipendio dovuto a un suo
impiegato\footnote{Cfr. R. \textsc{Chapus}, \emph{Ibidem}, 1135.}. Fino
al 2011, il ruolo del \emph{mediateur de la Republique} è stato quello
di dirimere le controversie tra i privati e la pubblica amministrazione
sorte in seguito a disservizi e malfunzionamenti nell'operato di
quest'ultima, quali lentezza del procedimento, errore, silenzio,
omissione di informazioni e mancata esecuzione di un giudicato. Si
trattava di un'autorità indipendente, la cui nomina, irrevocabile,
avveniva da parte del Consiglio dei ministri con un mandato di sei anni
non rinnovabile. Una volta che il \emph{mediateur} riceveva l'istanza,
se la riteneva fondata, intraprendeva una sorta di dialogo di mediazione
con l'amministrazione, a seguito del quale, se la risposta
dell'amministrazione non era ancora ritenuta soddisfacente, poteva
giungere alla formulazione di raccomandazioni che rendeva
pubbliche\footnote{Cfr. X. \textsc{d'} \textsc{Hellencourt}, \emph{A
  quoi sert le médiateur de la république?},
  \emph{dhellencourt-avocats.fr}, in \url{https://tinyurl.com/2wrsatec}
  (Consultato: 22 gennaio 2023).}.

A seguito della revisione costituzionale del 23 luglio 2008 e della
legge del 29 marzo 2011 che l'ha istituito, le persone fisiche o
giuridiche coinvolte in una controversia con la pubblica amministrazione
che verta sul malfunzionamento di pubblici servizi o sull'inottemperanza
da parte dell'amministrazione a sentenze favorevoli alle stesse persone
fisiche o giuridiche possanopossono rivolgersi al \emph{Défenseur des
droits}. Egli verifica innanzitutto che l'istanza ricevuta rientri nella
sua sfera di competenza, in caso contrario, indirizza i richiedenti alle
istituzioni o agli organismi preposti. Se di sua competenza, svolge una
fase istruttoria, raccogliendo tutti gli elementi e le informazioni che
possano essere utili alla conoscenza del caso. Se l'istanza si riferisce
a un errore di procedimento, un'incomprensione o un errore da parte
dell'amministrazione, il \emph{Défenseur des droit} prova a risolvere in
via amichevole il conflitto, ma se ciò non è possibile, rivolge una
raccomandazione all'amministrazione interessata, ad esempio per
richiedere di rispondere alla richiesta di adozione di un provvedimento
entro un dato termine. L'amministrazione sarà poi tenuta a rendergli
conto delle azioni intraprese a seguito di tale raccomandazione. Le
raccomandazioni del \emph{Défenseur des droits} possono anche essere di
carattere generale, relative a problematiche rilevanti, a mezzo delle
quali può avanzare proposte di modifica della legge. Egli non può
impugnare una sentenza del giudice, ma può presentare osservazioni
innanzi a qualsiasi giurisdizione. Il \emph{Défenseur des droits}
rappresenta un'autorità indipendente ed imparziale, è nominato dal
Presidente della Repubblica per sei anni, previa audizione parlamentare.

Prima di ricorrere al \emph{Défenseur des droits} però, è necessario da
parte delle persone fisiche o giuridiche espletare tutte le procedure
previste nell'ambito della stessa amministrazione per contestarne la
decisione, ad esempio un ricorso gerarchico. Per quanto riguarda i
costi, l'assistenza del \emph{Défenseur des droits} è gratuita, ma
l'istanza a lui rivolta non interrompe i termini di prescrizione
normalmente stabiliti per intraprendere l'azione giudiziaria\footnote{Cfr.
  \textsc{Service Public}, \emph{Litige avec l'administration : saisir
  le Défenseur des droits}, \emph{service-public.fr}, in
  \url{https://www.service-public.fr/particuliers/vosdroits/F13158}
  (Consultato: 22 gennaio 2023).}.

\hypertarget{la-giustizia-amministrativa-nel-regno-unito}{%
\chapter{La giustizia amministrativa nel Regno
Unito}\label{la-giustizia-amministrativa-nel-regno-unito}}

\hypertarget{il-processo-di-riforma-degli-administrative-tribunals}{%
\section{\texorpdfstring{Il processo di riforma degli
\emph{administrative
tribunals}}{Il processo di riforma degli administrative tribunals}}\label{il-processo-di-riforma-degli-administrative-tribunals}}

Gli \emph{administrative tribunals}, le cui radici risalgono al secolo
XVII, non hanno la natura di autentiche corti di giustizia, ma sono
organi a carattere amministrativo, ``aventi funzioni giurisdizionali al
di fuori della magistratura ordinaria''\footnote{Cfr. F. \textsc{De
  Franchis}, \emph{Dizionario Giuridico. v.1: inglese-Italiano = Law
  dictionary : English-italian}, Milano, Giuffrè, 1984, 1457.}. Si
tratta, in origine, di corpi amministrativi ai quali appositi
\emph{statutes} attribuiscono il potere di \emph{adjudication} in
determinati settori: si va dal lavoro nelle fabbriche, ai sussidi di
disoccupazione, alle pensioni, all'istruzione, alla sanità,
all'edilizia, alla pianificazione delle città. Le decisioni dei
\emph{tribunals}, sovente simili alle pronunce giudiziali, ma altre
volte più somiglianti ad atti amministrativi, in una prima fase sono
largamente sottratte al controllo delle corti ordinarie\footnote{Cfr. M.
  \textsc{D'Alberti}, \emph{Diritto amministrativo comparato: mutamenti
  dei sistemi nazionali e contesto globale}, Bologna, Il Mulino, 2019,
  67.}. Nel Regno Unito, la mancanza di una netta divisione tra il
potere esecutivo e quello giudiziario ha rappresentato un terreno
fertile per organi di questo tipo, mentre nei ``paesi a diritto
amministrativo'', in cui tale separazione vi è stata, ciò non è
avvenuto\footnote{Avanzando un parallelo con l'ordinamento italiano, nel
  Regno Unito è in pratica mancato un provvedimento normativo simile
  alla legge 31 marzo 1889, n.~5992, che ha istituito un sistema di
  giudici speciali per l'amministrazione. Cfr. M. \textsc{D'Alberti},
  \emph{Ibidem}, 64.}.

I tribunali non si inseriscono nel sistema di governo e non partecipano
alla realizzazione delle \emph{policies} dell'esecutivo, in quanto
svolgono un compito separato di amministrazione giustiziale. Non a tutti
sono però attribuite funzioni di tipo giudiziale. Alcuni
\emph{tribunals}, a cui spettano funzioni regolatorie dei processi
economici, si avvalgono di una tradizionale potestà amministrativa,
rilasciando provvedimenti individuali come le \emph{licences}. Nella
maggior parte dei casi, i tribunali sono legittimati a vagliare il
merito delle decisioni dell'autorità pubblica, disponendo del potere di
rivedere la scelta amministrativa e di assumere una decisione più
favorevole al privato ricorrente, avvalendosi di una procedura più
snella e spedita nel giungere a termine. Essi risolvono conflitti in
numerose materie, quali quella fiscale, assistenziale, previdenziale,
sanitaria, oppure attinente all'immigrazione. In tal modo, il ricorso di
chi si sente leso dall'operato amministrativo è rivolto ad una autorità
diversa da quella che ha emanato il provvedimento, ma che, prima della
riforma, era legata da un rapporto organizzatorio con quest'ultima. La
diffusione dei \emph{tribunals} che, in ipotesi residuali, si occupano
di conflitti che intervengono fra i privati, dipende anche dal fatto che
comportano costi inferiori rispetto a quelli dei rimedi giurisdizionali
tradizionali e decidono più speditamente. Essi non agiscono nelle forme
del procedimento, ma in quelle del processo, poiché esercitano
\emph{quasi-judicial functions}, ossia funzioni che non sono giudiziarie
in senso stretto, ma che vengono esercitate sotto molti aspetti
\emph{judicially}, applicando proprio le regole del procedimento
giudiziario\footnote{Cfr. S. \textsc{Cassese}, \emph{Il privato e il
  procedimento amministrativo: una analisi della legislazione e della
  giurisprudenza}, STEM Mucchi, 1971, 36 ss.}.

Storicamente, all'indomani della seconda guerra mondiale, è stato
l'avvento del \emph{welfare state} a determinare una significativa
espansione del modello dei tribunali, ma non sono mancate ragioni
prettamente politiche che hanno spinto il legislatore ad optare per
questa soluzione, in quanto è stato sostenuto che il \emph{welfare
state} dovesse avere un suo sistema indipendente di \emph{adjudication},
poiché vi era il pericolo di un \emph{judicial sabotage of socialist
legislation}. In virtù dei \emph{tribunals} infatti, si andava
indebolendo il modello degli \emph{inflexible private rights} e si
passava dagli \emph{absolute rights} ai \emph{qualified rights},
protetti in quanto compatibili con il bene comune, come interpretato
dagli stessi \emph{administrative tribunals}. Ne discende una serie di
``regole speciali elaborate dai \emph{tribunals} che venivano componendo
un diritto progressista contro il conservatorismo della \emph{common
law} costruita dalle corti giudiziarie\footnote{Cfr. M.
  \textsc{D'Alberti}, \emph{Diritto amministrativo comparato}, 76.}.
Allo stesso tempo, sulla scorta di una sostanziale ricerca
dell'efficienza, sembrava valere nel dibattito d'oltre Manica la formula
in base alla quale''il miglior giudice dell'amministrazione è chi ne fa
parte e la conosce e, cioè la stessa amministrazione, perché il giudice
che appartiene all'amministrazione non prova alcun timore reverenziale
di fronte a questa ed è più preparato a cogliere i difetti e le malizie
del comportamento di essa''\footnote{Cfr. M. \textsc{Nigro}, \emph{Il
  Consiglio di Stato giudice e amministratore (aspetti di effettività
  dell'organo)}, in \emph{Riv. trim. dir. proc. civ}, vol. 1470, 1974.}.
Ciò nonostante, nel giro di mezzo secolo, la creazione di nuovi
tribunali non si è arrestata, ma, da questa proliferazione nel numero,
derivavano una serie di conseguenze ulteriori che minavano l'efficacia
del sistema nel suo complesso, perché all'eccessiva frammentazione sul
piano delle competenze, seguiva anche una diversità di regole
procedurali, con conseguenti disparità di trattamento. Alcuni tribunali
erano infatti retti da regole di procedura, altri invece ne risultavano
privi, ovvero vigevano regole di composizione non uniformi o, ancora,
alcuni erano obbligati a emettere decisioni motivate, mentre altri
avevano il potere di derogare a questa regola. Ne risultava quindi un
sistema inefficiente e poco affidabile, non in grado di garantire una
tutela effettiva ai diritti dei cittadini, che doveva essere
razionalizzato e connotato di maggiore uniformità, tenuto conto del
maggior volume di ricorsi da definire rispetto a quello delle corti
giudiziarie. Nel 1957, una commissione propose la riorganizzazione del
sistema dei \emph{tribunals}\footnote{Cfr. G. \textsc{Marshall},
  \emph{The Franks Report on Administrative Tribunals and Enquiries}, in
  {«Public Administration»} 35 (dicembre 1957) 4, 347--358.} e, sulla
base dei lavori della \emph{Franks Committee}, s'intervenne con il
\emph{Tribunals and Enquires Act}, adottato nel 1958. L'idea di fondo
che ispirava questo intervento è basata sull'assunto che questi corpi
amministrativi di decisione contenziosa dovevano ``essere considerati
per le loro funzioni, più che per la loro appartenenza
formale''\footnote{Cfr. S. \textsc{Cattaneo}, \emph{Agencies e
  regulation nel Regno Unito}, in S. \textsc{Labriola} (Ed.), \emph{Le
  autorità indipendenti: da fattori evolutivi ad elementi della
  transizione nel diritto pubblico italiano}, Milano, Giuffrè, 1999.}.
Dovevano essere organizzati non sulla base dell'apparato amministrativo
di cui facevano parte, ma secondo l'insieme dei compiti che essi erano
chiamati a svolgere. Secondo il \emph{Franks Report}, la decisione del
tribunale doveva essere considerata come un controllo esterno
sull'amministrazione: attraverso i suoi organi venivano accertati fatti
e veniva applicato il diritto agli stessi fatti, decidendo su questioni
legali. Da ciò discendeva che il \emph{chairman} doveva avere una
preparazione legale, che dovevano essere previste garanzie di tipo
giurisdizionale attinenti sia alla procedura (osservanza del
contraddittorio, diritto di appello, udienza pubblica), sia al soggetto
giudicante (di cui si sanciva l'indipendenza rispetto all'autorità
governativa); che doveva essere garantita la rappresentanza legale degli
interessi e che dovevano poter essere assunte le prove in giudizio,
compresa la testimonianza. L'indipendenza dei tribunali, grazie a questa
riforma, diviene più puntuale e in sostanza simile a quella delle
\emph{courts of law}. Ciò rappresenta una garanzia del potere dei
tribunali dall'interferenza politica e corrisponde ad una maggiore
autonomia nelle scelte.

Con il \emph{Tribunals and Inquiries Act} del 1992 veniva riformato il
\emph{Council on Tribunals}, organismo già istituito nel 1958 e preposto
al ruolo di sorveglianza sull'attività dei tribunali. Sebbene la
funzione svolta da tale organismo fosse limitata all'esercizio di un
potere consultivo, ciò non ha impedito a quest'ultimo di svolgere un
ruolo importante anche nell'assicurare \emph{fairness, openness} e
\emph{impartiality}, a garanzia di coloro che presentano le loro istanze
a tali strutture. A dispetto dei citati interventi riformatori, tesi a
difendere \emph{``cheapness, accessibility, freedom from technicality,
expedition and expert knowledge of their particular
subject''}\footnote{Cfr. S. \textsc{Cattaneo}, \emph{Ibidem}.}, questi
caratteri sono gradualmente venuti meno, poiché gli aspetti processuali
si sono venuti complicando e la presenza di regole complesse, proprie di
determinate strutture, ha avuto l'effetto di rendere alcuni tribunali
concretamente accessibili solo a quanti avevano l'occasione di
frequentarli con regolarità. Permanevano alcune questioni non risolte,
come la scarsa uniformità, la mancanza di un unico apparato complessivo,
l'assenza di un coordinamento centrale. L'assenza di razionalità
nell'organizzazione e di uniformità nelle strutture e nella procedura
spingono, nel maggio 2000, Lord Irvine, in qualità di Lord Chancellor,
ad affidare una valutazione del sistema a Sir Andrew Leggatt, ``\emph{a
retired Lord Justice of Appeal}''\footnote{Cfr. T. G. \textsc{Buck},
  \emph{Tribunal reform in the UK: Precedent and reporting in the new
  unified structure}, 2007.}. Il rapporto \emph{Leggatt}, intitolato
\emph{Tribunals for Users. One System, One Service}, si propone di
semplificare l'assetto, in modo da rendere i tribunali quanto più
possibile \emph{user friendly}. Le proposte in esso contenute sono
state, per la maggior parte, recepite e fatte proprie dall'autorità
governativa, titolare dell'iniziativa legislativa. Fra di esse, il punto
maggiormente qualificante sta nel tentativo di concentrare le
competenze, prima spalmate su una pluralità di organi a competenza
settoriale, in capo a due soli organismi, coadiuvati da un'unica
struttura preposta alla gestione dei servizi amministrativi.

A questo rapporto fa seguito il \emph{White paper} del 2004, adottato
dal Ministero della Giustizia, intitolato \emph{Transforming Public
Servies: complaints, Redress and Tribunals}, nel quale la posizione dei
tribunali si inserisce in un più ampio processo di revisione dei sistemi
alternativi di risoluzione delle controversie tra cittadino e potere
pubblico. Nell'ambito di questi, i \emph{tribunals} sono individuati
come un modello informale ed economico capace di fornire giustizia ,
nell'accoglimento delle principali indicazioni provenienti dal rapporto
\emph{Leggatt}. Nel \emph{paper} l'oggetto dell'indagine è la
\emph{proportionate dispute resolution}\footnote{Secondo la definizione
  contenuta nel \emph{White paper}: ``\emph{the aim is to develop a
  range of policies and services that, so far as possible, will help
  people to avoid problems and legal dispute in the first place, and
  where they cannot, provides tailored solutions to resolve the dispute
  as quickly and cost effectively as possible. It can be summed up as
  ``proportionate dispute resolution''}.}, ossia il rafforzamento di
tutte le procedure volte a tutelare, in via preventiva, il cittadino
contro i pubblici poteri, in modo da ridurre i casi di ricorso dinanzi
alle \emph{courts of law}, nell'auspicio di ``\emph{to avoid disputes
arising in the first place by improving initial
decision-making}''\footnote{Cfr. G. \textsc{Richardson} -- H.
  \textsc{Genn}, \emph{Tribunals in transition: resolution or
  adjudication?}, in {«PUBLIC LAW»} 01 (2007), Sweet and Maxwell,
  116--141, 117.}.

Sulla base del \emph{Leggatt Report}, nonché del contenuto del
\emph{White paper}, è stato adottato il \emph{Tribunals, Courts and
Enforcement Act 2007}, entrato in vigore il 3 novembre 2008, che
interviene ad ampio spettro su molti istituti dedicati agli
\emph{administrative tribunals}. L'innovazione maggiore della riforma
consiste nell'istituzione di due tribunali a competenza generale: il
\emph{First-tier Tribunal} e l'\emph{Upper Tribunal}, ove si concentrano
le funzioni giustiziali di tipo settoriale, finora esercitate dai
tribunali esistenti. Mentre il primo rappresenta l'organo di prima
istanza, il secondo vanta sia le funzioni di organo di appello, sia di
organo di primo grado in particolari ipotesi. I tribunali a competenza
generale sono divisi al loro interno in \emph{chambers} che
costituiscono sezioni specializzate, con propria \emph{jurisdiction},
organizzate per materia. In particolare, quelle del \emph{First-tier}
sono tre: La \emph{Social Entitlement Chamber} che include \emph{Asylum
Support, Social Security and Child Support, and Criminal Injuries
Compensation}; La \emph{Health, Education and Social Care Chamber} che
include \emph{Care Standards, Mental Health} e \emph{Special Education
Needs and Disability}; ed infine la \emph{War Pensions and Armed Forces
Compensation Chamber} che include la \emph{War Pensions and Armed Forces
Compensation}. Nel 2009 è stata istituita una \emph{Tax jurisdiction}.

Anche le \emph{chambers} dell'\emph{Upper Tribunal}, definito una
``\emph{superior court of records}''\footnote{Tribunals, Courts and
  Enforcement Act, ch.~2, s. 3 (5).}, sono tre. Esse sono
\emph{Administrative Appeals, Finance and Tax, Lands}. L'\emph{Upper
Tribunal} opera, nella maggioranza dei casi, quale sede di appello,
gestendo un contenzioso nettamente inferiore rispetto al tribunale di
primo grado: poco numerose sono infatti le ipotesi in cui esso
rappresenta la sede di prima istanza alla quale il cittadino può
rivolgersi per una revisione della decisione amministrativa.

Con il \emph{Tribunals, Courts and Enforcement Act} si interviene anche
sulla disciplina delle impugnazioni, migliorandola e rendendola più
omogenea. I tribunali valutano il merito delle decisioni dell'autorità
pubblica, riesaminando il provvedimento finale e, se del caso, assumendo
una decisione più favorevole al ricorrente. Gli esiti del procedimento
dinanzi ai \emph{tribunals} possono essere schematizzati in tre
differenti tipologie: a) \emph{adjurned}, che consiste in una decisione
di rinvio; b) \emph{overturned}, quando i \emph{tribunals} accolgono
favorevolmente l'istanza del ricorrente e respingono la determinazione
dell'autorità amministrativa (\emph{decision-maker}); c) \emph{upheld},
che si ha quando invece viene confermata la determinazione del
\emph{decision-maker}. Definito il procedimento contenzioso, l'operato
dei \emph{tribunals} può essere sindacato, su istanza di parte, sia in
via di \emph{appeal}, sia in via di \emph{judicial review}. Una
decisione del \emph{First-tier Tribunal} può essere impugnata, per
motivi di legittimità (\emph{on a point of law}), dinanzi
all'\emph{Upper Tribunal}\footnote{Tribunals, Courts and Enforcement
  Act, ch.~2, s. 11.}, la cui decisione è a sua volta appellabile
dinanzi ad una corte ordinaria. E' previsto un diritto di appello presso
la \emph{Court of Appeal}, circoscritto ``\emph{to points of law of
general importance}''\footnote{Tribunals, Courts and Enforcement Act,
  ch.~2, s. 13.}.

Quanto alla \emph{judicial review of administrative action}, questa non
è toccata dall'intervento riformatore, poiché ``prescinde da quel che
stabiliscono gli \emph{statutes} e rientra in un \emph{inherent power}
del giudiziario secondo la \emph{common law}''\footnote{Cfr. M.
  \textsc{D'Alberti}, \emph{Diritto amministrativo comparato}, 75.}.
Pertanto, in forza della \emph{supervisory jurisdiction}, la
\emph{Queen's Bench Division} della \emph{High Court} può esercitare nei
confronti delle decisioni dei tribunali il suo sindacato mediante i
\emph{prerogative orders}. Fra questi, l'\emph{order of certiorari}
assicura che gli \emph{administrative tribunals} restino nell'ambito
delle loro competenze, che siano osservati i principi di \emph{natural
justice}, che gli errori di diritto siano corretti. L'\emph{order of
prohibition} ha la funzione di prevenire un \emph{tribunal}
dall'eccedere dalla propria giurisdizione, ma non è ammissibile quando
sia stata pronunciata una decisione definitiva. Il \emph{mandamus},
invece, è emesso per ingiungere di esercitare la giurisdizione o le
proprie funzioni.

Sul versante organizzativo discendono due ulteriori aspetti innovativi
dal delineato intervento riformatore. Il primo riguarda l'istituzione
del \emph{Administrative Justice and Tribunal Council} che sostituisce
il \emph{Council on Tribunals} costituito nel 1958, a cui sono assegnati
compiti di sorveglianza sul funzionamento dei tribunali; il secondo
attiene invece alla neo istituita agenzia che si occupa della gestione
del personale che svolge la sua attività al servizio di tali organi
contenziosi. La rinnovata struttura è tenuta a monitorare l'evoluzione
della giustizia amministrativa e a verificare che il sistema risponda
sempre a tre caratteri essenziali: equità, efficienza e facile accesso
alla giustizia.

Nelle regole di composizione dei collegi, ovvero nel sistema delle
impugnazioni, sono sempre meno numerosi gli elementi che differenziano i
\emph{tribunals} dalle corti ordinarie\footnote{Come sottolineato in G.
  \textsc{Richardson} -- H. \textsc{Genn}, \emph{Tribunals in
  transition: resolution or adjudication?}, 117, i \emph{tribunals} sono
  stati istituiti \emph{``specifically to adjudicate disputes, tipically
  between citizens and the state, not to resolve them through mediation,
  conciliation or any other form of non-adjudicative dispute
  resolution''}.}. Di certo, i \emph{tribunals} rimangono organi di
appello in senso devolutivo, in quanto ripetono \emph{ex novo} un
giudizio, di fatto e di diritto, di un'autorità pubblica, oppure
riformano l'atto amministrativo, riproducendo in altra sede il giudizio
di ponderazione degli interessi, mentre le corti giurisdizionali non
riconsiderano il provvedimento in ogni suo aspetto, perché il loro
controllo è circoscritto al sindacato di legittimità. Tuttavia, molte
differenze sono venute a cadere, in quanto il \emph{Tribunals Act}
attribuisce il titolo di ``giudici'' ad alcuni membri dei tribunali,
sebbene la loro composizione sia mista, partecipandovi anche esperti
laici del settore; inoltre, i tribunali sono organismi \emph{super
partes} rispetto agli interessi dell'autorità e dell'amministrato, che
agiscono nelle forme del processo, qualificati sul piano delle
conoscenze tecniche e specialistiche. La formula organizzativa del
giudizio è basata sul contraddittorio, in alcuni casi orale e in altri
scritto: le parti si scambiano gli atti, sono stabiliti termini di
deposito per i documenti e per le memorie, vi è accessibilità delle
parti agli atti del giudizio. E' stata resa omogenea la disciplina delle
impugnative. Al di là della struttura dei tribunali assolutamente
ibrida, il punto è che gli \emph{administrative tribunals}, nello
svolgimento dell'attività di risoluzione delle controversie, sono
inseriti in un contesto più ampio, nel quale operano quasi sullo stesso
piano delle corti ordinarie, per cui, i primi per un verso, e le
seconde, per l'altro, ``\emph{work in partnership with one another to
ensure that the public at large have an opportunity to exercise their
rights and to seek effective redress against Government
decisions}''\footnote{Tale espressione si ritrova nei documenti adottati
  dal \emph{Tribunal Service}.}.

\hypertarget{la-procedura-di-judicial-review}{%
\section{La procedura di judicial
review}\label{la-procedura-di-judicial-review}}

La disciplina della Convenzione Europea dei Diritti dell'Uomo (CEDU)
vede la sua piena operatività nell'ordinamento inglese solo all'indomani
del recepimento delle previsioni convenzionali mediante l'emanazione
dello \emph{Human Rights Act} (HRA). La norma di riferimento è quella di
cui all'art. 6, §1, CEDU, rispetto alla quale è stato osservato come le
garanzie ivi fissate non operino soltanto nel contesto civile e penale,
ma anche in ambito amministrativo e come l'efficacia della disposizione
si manifesti non solo nella fase processuale strettamente intesa, ma
nell'ambito di qualsiasi iter procedimentale genericamente considerato
(\emph{decision making process}), in cui sia coinvolta una pubblica
autorità. Di conseguenza, le garanzie convenzionali sono state ritenute
applicabili anche al procedimento e non solo al processo amministrativo.

Il tema della pienezza di poteri riconosciuti al giudice nella
risoluzione di controversie in cui siano parti la Pubblica
Amministrazione e i cittadini attiene a quel che la giurisprudenza della
Corte EDU ha ormai da tempo qualificato come il canone di ``\emph{full
jiurisdiction}''\footnote{Cfr. F. \textsc{Goisis}, \emph{La full
  jurisdiction sulle sanzioni amministrative: continuità della funzione
  sanzionatoria v. separazione dei poteri}, in (2018), Giuffrè, pp.~546
  ss.}. La stessa Corte EDU ha specificato che, oggetto di valutazione,
ai fini dell'integrazione del canone di \emph{full jurisdiction}, non è
soltanto l'iter processuale, ma l'intera vicenda procedimentale,
concepita come un \emph{unicum}, a partire dal procedimento. Inoltre,
l'accertamento della pienezza della tutela giurisdizionale in capo al
giudice deve sempre essere effettuato in concreto, ovvero in relazione
ai confini verso cui può spingersi l'autorità nel sindacato della
decisione amministrativa e in base agli strumenti che questi ha a
disposizione\footnote{Cfr. P. \textsc{Craig}, \emph{The human rights
  act, article 6 and procedural rights}, in {«Public law»} (2003) 4,
  Sweet \& Maxwell, 753--773, pp.~753 ss.}. Pertanto, qualora emerga che
i fatti siano stati assunti e ponderati dall'amministrazione nel
rispetto dei parametri della \emph{procedural fairness}, le garanzie
convenzionali si presumono integrate e ciò anche se nel successivo
giudizio l'autorità giurisdizionale non possiede formalmente il potere
di sindacare la decisione nel merito, in quanto anche la mera verifica
del rispetto dei principi del contraddittorio e di parità delle armi in
sede procedimentale vale a garantire un sindacato pieno del giudice
amministrativo. Qualora invece sia riscontrabile un mancato rispetto
delle garanzie necessarie in sede procedimentale, ciò comporterà la
violazione dei principi del giusto processo.

Il rapporto tra l'implementazione dei principi CEDU e l'area di
sindacato rimessa al giudice nel \emph{judicial review} ha effetti, da
un lato, sul novero dei motivi di ricorso (\emph{grounds}) e,
dall'altro, nei confronti dei canoni di giudizio (\emph{standard})
impiegati nella valutazione da parte delle corti. Con riferimento ai
\emph{grounds}, si evidenzia come le nuove forme di tutela entrate a far
parte della \emph{domestic legislation} via HRA abbiano avuto l'effetto
di ampliare il raggio d'azione (\emph{scope}) del \emph{judicial
review}, mentre, per quanto attiene agli \emph{standard}, il dibattito
ha interessato principalmente il possibile disallineamento tra il
criterio della \emph{Wednesbury unreasonabless}, tipico dell'ordinamento
inglese, e quello di proporzionalità elaborato in sede europea.

Il \emph{judicial review} si caratterizza per essere ``\emph{a claim to
review the lawfulness of an administrative action}''\footnote{Cfr. P.
  \textsc{Cane}, \emph{Understanding judicial review and its impact}, in
  M. \textsc{Hertogh} -- S. \textsc{Halliday} (Edd.), \emph{Judicial
  Review and Bureaucratic Impact}, Cambridge University Press,
  \textsuperscript{1}2004, 15--42, pp.~17 e ss.}. Per quanto riguarda
l'iter processuale che lo connota, esso è articolato in due fasi,
secondo la logica del \emph{two stage process}: la prima rappresenta una
fase filtro, diretta alla verifica dei presupposti formali necessari per
accedere alla tutela innanzi al giudice competente (c.d.
\emph{permission} o \emph{leave})\footnote{CPR, Art. 54, r. 54.4.},
mentre nella seconda viene celebrata l'udienza (\emph{hearing}) in
contraddittorio tra le parti. Il \emph{judicial review} è stato oggetto
di una recente iniziativa di riforma, realizzata attraverso le modifiche
introdotte con il \emph{Criminal Justice and Courts Act (CJCA)} nel
2015. La \emph{ratio} dell'intervento si individua nell'avvertita
necessità di arginare il diffuso fenomeno di impiego del rimedio
giurisdizionale come mera ``\emph{delaying tactic}'', ossia a scopi
dilatori e solo al fine di evitare il dispiegamento degli effetti
dell'atto amministrativo impugnato, data la sospensione della sua
efficacia che segue all'apertura del contenzioso. Il CJCA ha
rappresentato in parte l'antidoto ai malfunzionamenti della macchina
della giustizia britannica, incidendo soprattutto in punto di accesso
alla tutela giurisdizionale, attraverso l'aumento della somma da versare
per l'apertura della fase filtro, una misura volta a scoraggiare il
ricorso alla tutela per \emph{judicial review} se non ove strettamente
necessario: si prevede infatti la corresponsione di un elevato
contributo alle spese legali per ottenere la pronuncia da parte
dell'autorità giudiziaria\footnote{CJCA, s. 85-86 e 88-90.}. Tuttavia,
le critiche mosse a tale soluzione sono state numerose, fondate
soprattutto sul valido assunto in base al quale le \emph{chances} di
ottenere una tutela piena ed effettiva, davanti ad un giudice terzo ed
imparziale, a fronte di una decisione amministrativa illegittima,
rischiano di essere riservate solo ai pochi in grado di sostenerne i
costi procedurali.

Con riguardo all'accesso alla tutela e ai presupposti del giudizio, il
sindacato nella forma del \emph{judicial review} è sottoposto a limiti
di natura teleologica e funzionale piuttosto definiti che corrispondono
ad altrettanti limiti procedurali e sostanziali. Con riferimento ai
primi, si tratta dei c.d. \emph{grounds for judicial review},
identificati dalla giurisprudenza inglese nei caratteri di
\emph{illegality, irrationality, procedural impropriety} e \emph{breach
of legittimate expectations}. Per quel che concerne la materia degli
\emph{human rights}, il canone che più rileva è quello della
\emph{illegality}, in quanto, così come statuito dallo HRA ``\emph{it is
unlawful for a public authority to act in a way which is incompatible
with a Convention right}''\footnote{HRA s. 6(1).}. Di conseguenza, in
tali casi, l'atto adottato deve essere dichiarato illegittimo dal
giudice, in quanto incompatibile con la normativa convenzionale
(\emph{declaration of incompatibility})\footnote{HRA s. 4(2).}.

Per quel che attiene ai limiti sostanziali, il riferimento è ai criteri
di giudizio (\emph{standard}) in base ai quali il giudice è chiamato a
decidere e, indirettamente, l'incisività del suo sindacato. In
particolare, la questione riguarda l'opzione tra l'applicazione del
principio di proporzionalità di matrice europea, così come formulato
dalla Corte EDU e quello invece tipico della tradizione di \emph{common
law}, riconducibile alla \emph{Wednesbury unreasonabless}\footnote{Cfr.
  C. F. \textsc{Forsyth} -- W. \textsc{Wade} -- W. \textsc{Wade},
  \emph{Administrative law}, Oxford, United Kingdom ; New York, NY,
  Oxford University Press, Eleventh edition 2014, pp.~305 ss.}. Laddove
i giudici europei sono chiamati a soppesare la rilevanza dei relativi
diritti ed interessi in gioco, anche entrando parzialmente nell'area di
valutazione dell'amministrazione se necessario, nell'ambito dei
procedimenti di \emph{judicial review} i giudici inglesi, in genere,
sono chiamati ad astenersi dal giudicare su \emph{questions of fact},
potendo verificare al più solo la correttezza formale dell'iter
procedurale tenuto in sede di decisione della pubblica autorità, al fine
di verificarne la rispondenza con le garanzie del giusto procedimento.
Le eventuali criticità che pure si sono evidenziate in sede applicativa
sono state comunque risolte, anche attraverso le statuizioni della
giurisprudenza di Strasburgo. Da un lato, e per quel che riguarda la
pienezza e l'incisività del sindacato giurisdizionale in ambito di
\emph{judicial review}, una soluzione è data dall'adozione del canone di
\emph{full jurisdiction}; dall'altro e con riferimento alle differenze
tra i diversi standard di giudizio adottati dalle corti nella decisione,
il riferimento è alla c.d. dottrina del ``\emph{margin of
appreciation}''\footnote{Cfr. R. \textsc{Ryssdall}, \emph{Opinion\,: the
  coming of age of the European Convention on Human Rights.}, in
  {«EUROPEAN HUMAN RIGHTS LAW REVIEW»} (1996) 1, Thomson Reuters,
  18--29, pp.~24 e ss.}, in base alla quale si precisa la funzione
armonizzatrice della giurisprudenza euro-convenzionale, la quale si
limita a dettare delle linee di principio a cui ciascuno Stato è libero
di decidere se aderire o meno, fatto comunque sempre salvo il rispetto
degli obblighi e delle garanzie fissate nella CEDU.

\hypertarget{il-parliamentary-ombudsman}{%
\section{\texorpdfstring{Il \emph{Parliamentary
Ombudsman}}{Il Parliamentary Ombudsman}}\label{il-parliamentary-ombudsman}}

Sorta nel 1809 in Svezia, quella dell'ombudsman è divenuta una figura di
notevole rilievo nel panorama amministrativo inglese. L'ombudsman
rappresenta uno strumento di tutela, alternativo sia alla giurisdizione
delle corti, sia a quella dei \emph{tribunals}, di interessi collettivi
e individuali lesi dall'inerzia dell'amministrazione o dai suoi
comportamenti attivi illegittimi o inopportuni. Il \emph{Parliamentary
Commissioner for Administration} (PCA), successivamente definito
\emph{Parliamentary Ombudsman} (PO), è stato istituito con il
\emph{Parliamentary Commissioner Act} del 1967. Esso è nominato dalla
Corona e, nella sua struttura, è organo indipendente ed imparziale,
espressione di un incardinamento parlamentare che si manifesta
attraverso due meccanismi, il primo rappresentato da ``\emph{the MP
filter}'', ovvero la necessità, per qualsiasi cittadino che intenda
adire il PO, di presentare ricorso dinanzi al rappresentante
parlamentare della propria circoscrizione\footnote{Parliamentary
  Commissioner Act, 1967, ss. 5-6: \emph{``Complaints are made through
  Members of the Westminster Parliament and may be made by any member of
  the public, including a corporation''}.}, il quale costituisce un
significativo filtro teso ad evitare il sovraccarico degli uffici del
PO. Dinanzi al ricorso del cittadino infatti il parlamentare ha la
possibilità di occuparsene in prima persona o di trasmetterlo al PO che,
al termine dell'inchiesta, dovrà presentare un rapporto allo stesso
parlamentare che gli ha trasmesso il caso\footnote{I rapporti del PO
  sono pubblicati online in \textsc{Parliamentary and Health Service
  Ombudsman (PHSO)}, \emph{PHSO publications}, \emph{ombudsman.org.uk},
  in \url{http://www.ombudsman.org.uk/publications} (Consultato: 24
  gennaio 2023).}. Il secondo meccanismo di incardinamento parlamentare
concerne la regolare presentazione di un rapporto annuale dinanzi al
\emph{Select Commettee} della Camera dei comuni, deputato alla vigilanza
sull'attività del PO, nonché l'audizione su singole questioni
ogniqualvolta a ciò sia invitato dalla Camera dei comuni o qualora lo
stesso \emph{Select Commettee} lo ritenga necessario.

Il modello del PO è stato utilizzato per altre figure; tra tutte, le più
importanti nel settore pubblico sono il \emph{Local Government
Ombudsman} (LGO), il \emph{Health Service Ombudsman} (HSO), il
\emph{European Ombudsman}, il \emph{Prison and Probation Ombudsman} e
l'\emph{Housing Ombudsman}. Mentre il PO è competente per tutti gli atti
e i comportamenti dei dipartimenti governativi e di numerose altre
attività pubbliche, nonché dei gestori privati di pubblici servizi, il
LGO si occupa invece dei reclami nei confronti dei casi di
\emph{maladministration} all'interno dell'amministrazione locale. Con
questa nozione, pur soggetta a diverse interpretazioni, si vuole fare
riferimento ad atti e comportamenti caratterizzati da contraddittorietà,
irragionevolezza, scarsa trasparenza, disparità di trattamento e
negligenza\footnote{QB 287: Richard Crossman: ``bias, neglect,
  inattention, delay, incompetence, ineptitude, arbitrariness and so
  on''. Cfr. T. \textsc{Denning}, \emph{Regina v Local Commissioner for
  Administration for the North and East Area of England ex parte
  Bradford Metropolitan City Council: CA 1979, QB 287},
  \emph{swarb.co.uk}, in \url{https://tinyurl.com/44swkwxm} (Consultato:
  24 gennaio 2023).} che comportino un'ingiustizia per il cittadino e
nei confronti dei quali non sia possibile ricorrere né alle corti, né ai
\emph{tribunals} per l'onerosità in termini economici e di tempo o,
ancora più semplicemente, per l'assenza di una violazione di legge. Il
percorso prospettato al cittadino si struttura quindi in maniera più
rapida ed immediata rispetto al ricorso giurisdizionale, in quanto la
questione può e vuole essere risolta ad uno stadio precedente,
affiancando alla pretesa del cittadino l'indagine, l'accertamento e
quindi le raccomandazioni di un soggetto autorevole che possa spingere
l'amministrazione a rivedere il suo comportamento. L'esito dell'indagine
è una formale \emph{recommendation} rivolta all'amministrazione che
consiste in un invito a porre riparo al motivo di doglianza o alla
disfunzionalità procedurale. Di fatto, il semplice invito fatto dal PO
all'amministrazione di riesaminare l'istanza del cittadino è spesso
sufficiente per trovare una soluzione soddisfacente per il medesimo.

Prima di ricorrere al PO, il privato deve innanzitutto rivolgersi
all'amministrazione per richiedere la revisione della decisione o del
provvedimento che reputa lesivo nei suoi confronti. Chiamato
principalmente a sindacare i casi di \emph{maladministration}, sempre su
istanza del cittadino e non d'ufficio, l'ombudsman è un soggetto
chiamato a stimolare la pubblica amministrazione verso un continuo
miglioramento e gode di un significativo potere discrezionale
relativamente all'introduzione, allo svolgimento e alla chiusura
dell'indagine. Tuttavia, le sue decisioni non sono vincolanti: qualora
al termine dell'indagine dovesse effettivamente riscontrare un caso di
\emph{maladministration}, l'ombudsman non ha alcun potere per
costringere la pubblica amministrazione a rivedere la sua decisione o
correggere il suo comportamento. La stessa amministrazione potrebbe
infatti disattendere le indicazioni contenute nel report del PO, ad
esempio rifiutandosi di pagare il risarcimento stabilito dal PO quale
mezzo di risoluzione della controversia. In tal caso il PO ha, come
unica soluzione, la possibilità di presentare uno \emph{special report}
al \emph{Select Commettee}, denunciando il caso di
\emph{maladministration} e il successivo comportamento della pubblica
amministrazione. Di fatto comunque, il semplice invito fatto
dall'ombudsman all'amministrazione di esaminare le doglianze del
cittadino è spesso sufficiente per trovare una soluzione soddisfacente
per quest'ultimo, in quanto l'autorevolezza e il buon senso delle sue
misure portano le pubbliche amministrazioni ad adeguarvisi, pur non
essendovi formalmente costrette.

\hypertarget{il-local-government-ombudsman-lgo}{%
\section{\texorpdfstring{Il \emph{Local Government Ombudsman}
(LGO)}{Il Local Government Ombudsman (LGO)}}\label{il-local-government-ombudsman-lgo}}

La \emph{Commission for Local Administration}, normalmente conosciuta
come \emph{Local Government and Social Care Ombudsman} (LGO), è stata
istituita con il \emph{Local Government Act} del 1974 ed è un organo
indipendente, la cui costituzione è basata su sovvenzione annuale del
Governo. L'ombudsman è nominato dalla Corona su raccomandazione del
Segretario di Stato; in seguito al \emph{Local Government and Public
Involvement in Health Act} del 2007, il LGO è ora nominato per un
periodo massimo di sette anni e può essere rimosso dall'ufficio dietro
sua richiesta oppure per incapacità o a causa di comportamenti
scorretti. Il LGO ha sedi a Londra, York e Coventry. Il \emph{Local
Government Act} del 1974 originariamente richiedeva di dividere il
territorio inglese in aree di competenza, ciascuna con separata
giurisdizione per i ricorsi relativi a \emph{maladministration}, ma
l'atto del 2007 ha introdotto maggior flessibilità a riguardo e le
materie oggetto di indagine possono essere classificate come appare più
appropriato e assegnate ad uno o più \emph{commissioners}. Le stesse
funzioni del LGO inglese sono svolte dal \emph{Public Services
Ombudsman} in Galles, dallo \emph{Scottish Public Services Ombudsman} in
Scozia e dal \emph{Northern Ireland Ombudsman} in Irlanda del Nord.

I vari tipi di pubblica autorità la cui attività può essere oggetto di
indagine sono elencati sul sito web del LGO che fornisce anche le
istruzioni su come inoltrare il ricorso\footnote{Cfr. \textsc{Local
  Government and Social Care - Ombudsman}, \emph{Challenging our
  decisions}, \emph{lgo.org.uk}, in \url{https://tinyurl.com/4ez42xd9}
  (Consultato: 24 gennaio 2023).}. Tale ricorso può essere proposto dal
cittadino a seguito di ingiustizia subita, conseguente a
\emph{maladministration} o disservizi (\emph{service failures}), e
connessa all'azione o all'inazione dell'autorità o di chi opera per
conto della stessa. L'obbiettivo del LGO è quello di giungere ad
un'indagine completa entro ventisei settimane, ma ci sono materie di cui
il LGO non si può occupare e problematiche che non può seguire. Ciò
accade ad esempio quando sono trascorsi più di dodici mesi dal
verificarsi di un evento o dall'emissione di un provvedimento, quando la
questione non concerne il ricorrente personalmente o quando questi non
ha effettivamente subito un'ingiustizia, quando ha avuto o ha diritto di
appello o di intraprendere un'azione legale e il LGO ritiene che sia
ragionevole che proceda in tal senso e, infine, qualora la vertenza
riguardi questioni personali quali il rapporto di lavoro e i relativi
provvedimenti disciplinari. Come regola generale, l'Ombudsman prenderà
in considerazione solo le istanze concernenti eventi o circostanze
verificatisi non oltre i dodici mesi dalla data in cui il ricorrente ne
è venuto a conoscenza. Inoltre, lo stesso ricorrente dovrà prima aver
espletato la ``\emph{council's own complaints procedure}'' di cui non
sarà necessario che egli abbia esaurito tutti gli stadi, ma perlomeno
dovrà concedere al consiglio un termine ragionevole per l'esame della
domanda, consistente normalmente in dodici settimane dalla presentazione
del ricorso\footnote{Cfr. M. \textsc{Sandford}, \emph{The Local
  Government Ombudsman}, \emph{House of Commons Library}, in
  \url{https://commonslibrary.parliament.uk/research-briefings/sn04117/}
  (Consultato: 5 febbraio 2023).}.

Laddove il LGO accerti che effettivamente la sfera giuridica del
ricorrente sia stata lesa da malfunzionamenti e disservizi
dell'amministrazione, potrà rivolgere raccomandazioni a quest'ultima al
fine di porgergli delle scuse, di sollecitare la fornitura del
bene/servizio a cui il ricorrente aveva diritto, di emettere il
provvedimento precedentemente richiesto, di rivedere una decisione sulla
quale inizialmente l'amministrazione si era soffermata solo
superficialmente e di effettuare un pagamento. Le decisioni del LGO sono
pubblicate in forma anonima dopo tre mesi, a meno che egli decida che
non sia nell'interesse del ricorrente procedere in tal modo. In alcuni
casi, l'Ombudsman può redigere un \emph{Public interest report} per
rendere noto al pubblico il verificarsi di una particolare casistica,
ma, in questo rapporto, non verranno comunque rese note le generalità
del ricorrente.

La sezione n.~92 del \emph{Local Government Act 2000} ha attribuito alle
autorità locali la facoltà di risarcire o erogare altri benefit ai
privati afflitti da casi di \emph{maladministration}, indipendentemente
dal fatto che il LGO sia stato o meno chiamato in causa\footnote{Local
  Government Act 2000, section 92.}.

Avverso le raccomandazioni del LGO, non sussiste diritto di appello.
Come il \emph{Parliamentary Ombudsman}, il LGO è un organismo istituito
dalla legge, indipendente sia dal Parlamento, sia dal Governo. Sebbene
di istituzione pubblica, non risponde delle sue decisioni né al
Segretario di Stato, né al Parlamento, così come non è possibile
rivolgersi ad altro ombudsman o altro ente per la revisione della sua
valutazione qualora il LGO non abbia ravvisato colpe o mancanze
nell'azione o inazione dell'amministrazione locale. Egli procederà ad
una revisione della sua decisione solo al verificarsi di particolari
circostanze, ad esempio qualora emergano importanti testimonianze circa
il verificarsi di fatti precedentemente non accertati, oppure siano
acquisite nuove informazioni rilevanti ai fini della determinazione del
contenuto della decisione\footnote{Cfr. \textsc{Local Government and
  Social Care - Ombudsman}, \emph{Challenging our decisions}.}.

E' possibile adire le vie giudiziarie contro la decisione di un LGO, ma
non sussiste un diritto automatico alla \emph{Judicial review}: chi
agisce deve prima ottenere l'autorizzazione a procedere dalla \emph{High
Court}, la quale comunque non si pronuncerà nel merito del caso, ma solo
sulla legittimità di quanto stabilito dal LGO.

\hypertarget{la-corretta-gestione-delle-rimostranze-principles-of-good-complaint-handling}{%
\section{\texorpdfstring{La corretta gestione delle rimostranze
(\emph{Principles of good complaint
handling})}{La corretta gestione delle rimostranze (Principles of good complaint handling)}}\label{la-corretta-gestione-delle-rimostranze-principles-of-good-complaint-handling}}

La gestione dei ricorsi indirizzati al \emph{Parliamentary Ombudsman} e
al \emph{Health Service Ombudsman} si basa su alcuni principi che
possono essere identificati nella categoria generale dei
``\emph{Principles of good complaint handling}''\footnote{Cfr.
  \textsc{Parliamentary and Health Service Ombudsman (PHSO)},
  \emph{Principles of Good Complaint Handling}, \emph{ombudsman.org.uk},
  in \url{https://tinyurl.com/v8n6sfsf} (Consultato: 24 gennaio 2023).}.
La corretta gestione delle rimostranze rileva infatti quale modalità
fondamentale per assicurare ai beneficiari il bene/servizio oggetto
della loro pretesa, qualora ne abbiano diritto, e ciò in base alla
regola generale secondo cui ognuno ha diritto di ricevere una buona
prestazione dagli enti che erogano un pubblico servizio, cui consegue il
diritto ad ottenere i giusti rimedi laddove si verifichino disservizi e
malfunzionamenti. Occorre tenere presente inoltre che, nel momento in
cui le lamentele e i ricorsi sono gestiti con rapidità ed efficacia,
possono costituire un'opportunità di miglioramento per gli stessi enti
pubblici, con riguardo alla loro reputazione e alle loro prestazioni,
poiché, dalla trattazione delle varie casistiche, essi possono trarre
insegnamenti e nuove conoscenze che portano alla riduzione delle
rimostranze in futuro.

In ossequio ai principi di buona gestione, l'attività della pubblica
amministrazione deve essere dunque improntata al rispetto di sei
condizioni che si possono riassumere come di seguito:

\begin{enumerate}
\def\labelenumi{\arabic{enumi})}
\tightlist
\item
  approccio adeguato ad ogni vicenda (\emph{getting it right});
\item
  orientamento verso le esigenze dei vari utenti (\emph{being customer
  focused});
\item
  disponibilità e affidabilità (\emph{being open and accountable});
\item
  svolgimento di un'attività equa e ponderata (\emph{acting fairly and
  proportionately});
\item
  proposta delle soluzioni più idonee (\emph{putting things right});
\item
  costante ricerca del miglioramento (\emph{seeking continuous
  improvement}).
\end{enumerate}

Per quanto riguarda l'approccio adeguato di cui al punto 1),
l'amministrazione deve agire in conformità alle norme di legge e a
quanto stabilito nei regolamenti relativi alle procedure di erogazione
dei vari servizi e sempre nel rispetto dei diritti di coloro che vi sono
coinvolti. La corretta gestione dei reclami deve essere parte integrante
dei servizi offerti al pubblico e il personale che presta la sua opera
deve essere abilitato e attrezzato in modo da poter fornire prontamente
le risposte più adeguate qualora si verifichino malfunzionamenti. La
stessa amministrazione deve poi informare chiaramente i ricorrenti circa
i tempi di risposta o di adozione del provvedimento finale e, una volta
giunti a questo stadio, gli stessi devono essere indirizzati verso il
passo successivo da compiere in caso di risposta non satisfattiva, anche
nel caso in cui sia per essi necessario ricorrere a strumenti di appello
alternativi.

Secondo quanto previsto al punto 2) sul \emph{``being customer
focused''}, l'amministrazione deve assicurare che i procedimenti si
svolgano in modo semplice, chiaro e nel più breve tempo possibile e che
le disposizioni che regolano la gestione dei ricorsi siano facilmente
accessibili. Gli utenti devono poi essere messi a conoscenza dei mezzi
di supporto a loro disposizione qualora considerino di presentare
ricorso\footnote{La \emph{Community Legal Advice} offre consulenza
  legale generale e L'\emph{Independent Complaints Advocacy Service}
  (ICAS) offre assistenza legale per i ricorrenti verso il
  \emph{National Health Service} (NHS).}; il linguaggio utilizzato deve
essere di facile comprensione e le comunicazioni fra l'autorità
amministrativa e il ricorrente devono svolgersi in maniera adeguata
rispetto alle condizioni in cui si trova quest'ultimo, ad esempio se non
è di madrelingua inglese o se ha bisogni speciali. Nel caso in cui la
rimostranza sia riferita a provvedimenti emanati da più amministrazioni,
se una di queste non può rispondere, deve poter informare
tempestivamente su quali possano essere le vie alternative di accesso
alla tutela.

In base al punto 3), l'amministrazione deve informare accuratamente e
chiaramente circa la portata dei ricorsi che può prendere in
considerazione, sulle aspettative che i ricorrenti possono avere durante
lo svolgimento delle procedure, sulle tempistiche e sugli esiti delle
stesse ed essere disponibile a rendere conto delle proprie azioni e
decisioni. Deve inoltre poter fornire una documentazione affidabile e
idonea ad essere utilizzata per dimostrare l'attività svolta: tale
documentazione deve ricomprendere le prove considerate e le motivazioni
che hanno portato alla decisione finale dell'autorità che potranno
essere utilizzate ai fini del ricorso, oppure per consentire
all'Ombudsman di istruire la pratica qualora si rendesse necessario il
suo intervento. Infine, se lo svolgimento dell'attività amministrativa
deve svolgersi all'insegna della trasparenza, allo stesso tempo durante
le stesse procedure occorre garantire il rispetto della privacy delle
persone coinvolte e la riservatezza delle informazioni confidenziali che
le riguardano.

Nel rispetto del \emph{``acting fairly and proportionately''} del punto
4, l'amministrazione deve comprendere e rispettare le differenze tra i
vari utenti e assicurare l'accesso ai servizi in maniera equa ed
imparziale, così come deve esaminare approfonditamente i reclami,
basando le sue valutazioni sui fatti e sulle prove oggettivamente
disponibili, evitando indebiti ritardi e rigide condotte prestabilite
nella gestione dei ricorsi. Ciò significa garantire una risposta
adeguata alle circostanze del singolo caso e tenere nella giusta
considerazione la gravità delle questioni sollevate, gli effetti
derivanti nella posizione del ricorrente e la possibilità che altri
soggetti abbiano subito pregiudizi all'esito dell'adozione dello stesso
provvedimento. L'azione dell'amministrazione deve però essere svolta in
modo equo anche nei confronti del personale che vi presta servizio, il
quale deve venire a conoscenza del fatto che le sue prestazioni sono
state oggetto di reclamo per garantirgli così la possibilità di
giustificarsi.

Nel momento in cui da parte dell'amministrazione non vi è stato un
approccio adeguato nella gestione della singola richiesta, occorre
``sistemare le cose'', ovvero, secondo il punto 5), ``\emph{put things
right}''. Questo implica, nei limiti del possibile, il ripristino della
posizione di coloro che hanno subito un pregiudizio dall'emanazione di
un provvedimento illegittimo o un disservizio da parte
dell'amministrazione stessa, così che torni ad essere quella che era
prima che tale pregiudizio o disservizio si verificassero. Qualora
questo non sia più possibile, occorre corrispondere un risarcimento per
equivalente. In molti casi, porgere le scuse e fornire una sollecita
spiegazione è la risposta appropriata e idonea a prevenire l'aggravarsi
di un conflitto.

Infine, in coerenza con il punto 6) ``\emph{seeking continuous
improvement}'', la corretta gestione delle rimostranze non deve
limitarsi ad offrire un rimedio al singolo ricorso, ma l'amministrazione
deve fare proprio ogni contributo che possa derivare dal comportamento
del ricorrente ai fini del miglioramento del servizio fornito agli
utenti. E' buona prassi infatti che essa riferisca pubblicamente sulla
gestione dei reclami, includendo indagini statistiche sul loro numero e
sul loro esito dopo la trattazione: questo perché il resoconto sui vari
tipi di ricorso e lamentele può agevolare il raggiungimento
dell'obbiettivo di una migliore offerta dei servizi richiesti,
attraverso l'individuazione di esigenze comuni nelle istanze presentate
che permettono di implementare modelli di risposta idonei a soddisfare
interessi non solo individuali, ma di un maggior numero di soggetti che
potrebbero essere coinvolti nell'adozione dei vari provvedimenti, così
da accrescere la fiducia del pubblico in generale verso l'attività
dell'autorità amministrativa.

\hypertarget{il-sistema-spagnolo}{%
\chapter{Il sistema spagnolo}\label{il-sistema-spagnolo}}

\hypertarget{levoluzione-dellorganizzazione-della-giustizia-amministrativa}{%
\section{L'evoluzione dell'organizzazione della giustizia
amministrativa}\label{levoluzione-dellorganizzazione-della-giustizia-amministrativa}}

Il sistema di giustizia amministrativa è stato introdotto in Spagna nel
1845, sulla base del modello francese di giustizia ``ritenuta'',
esercitata da organi amministrativi quali i consigli provinciali e la
sezione speciale contenziosa del Consiglio regio, poi divenuto Consiglio
di Stato. Con la legge Santamaria de Paredes del 1888 si determina il
passaggio dal sistema di giustizia ``ritenuta'' al sistema di giustizia
``delegata'', indipendente dall'esecutivo; la giustizia amministrativa
viene attribuita ai tribunali provinciali e al tribunale amministrativo,
costituito ora nel Consiglio di Stato. Con la riforma del 1904, il
tribunale amministrativo passa al tribunale supremo, nella cui
composizione dominano i cinque membri di provenienza giudiziaria, a
fronte di tre membri scelti fra i funzionari dell'alta amministrazione.
I tribunali provinciali conoscono dei ricorsi contro i provvedimenti
dell'amministrazione comunale e provinciale, mentre il tribunale
amministrativo è competente sia per gli appelli delle sentenze di prima
istanza, sia in unica istanza nei giudizi contro l'amministrazione dello
Stato. Nel diritto spagnolo, malgrado l'importazione dalla Francia del
sistema di giustizia amministrativa, non ha mai trovato applicazione la
distinzione fra giurisdizione ordinaria e giurisdizione amministrativa e
concretamente, ai fini della tutela, non si è resa necessaria la
distinzione fra diritto soggettivo e interesse legittimo. Fino alla
Costituzione del 1978, tali situazioni, riferite a controversie con la
pubblica amministrazione, davano luogo a un diverso grado di
legittimazione, ma sempre innanzi ai tribunali amministrativi: erano
eccezionali i casi di competenza della giustizia ordinaria che si aveva
quando la legge stabiliva la sottomissione dell'autorità amministrativa
al diritto privato, ad esempio per i contratti da essa stipulati in
ambito privato o nei contratti di lavoro dei pubblici dipendenti.

Con il movimento di riforma degli anni '50 del Novecento, conseguenza
dei tentativi del franchismo di raggiungere l'appoggio delle grandi
potenze mondiali, si modifica anche la legge relativa alla giustizia
amministrativa. Con la ``\emph{Ley de la Jurisdicción
Contencioso-Administrativa}'' (LJCA) del 27 dicembre 1956, la giustizia
amministrativa diviene definitivamente una giustizia giudiziaria e
specializzata nel controllo giuridico delle pubbliche amministrazioni,
nel cui ambito i giudici operano con le medesime attribuzioni che
spettano ai giudici in materia civile, del lavoro e penale. La LJCA
costituisce nelle udienze territoriali, corrispondenti a tribunali per
le varie province, una sala del contenzioso amministrativo che si
compone di tre giudici, uno dei quali specializzato in diritto
amministrativo. Le udienze territoriali conoscono della legittimità dei
provvedimenti dell'amministrazione locale e dell'amministrazione
periferica dello Stato. Nel tribunale supremo si creano le sale del
contenzioso-amministrativo (terza, quarta e quinta), composte da giudici
la cui provenienza determina l'identificazione di tre gruppi: un terzo
dei componenti provengono dalla carriera giudiziaria ordinaria, un terzo
è designato tra i magistrati specializzati in diritto amministrativo e
l'ultimo terzo fra giuristi di meriti rilevanti. Le sale del tribunale
supremo svolgono la loro competenza in unica istanza contro i
provvedimenti dell'amministrazione dello Stato, a eccezione
dell'amministrazione periferica e in appello contro le sentenze delle
udienze territoriali. Se al momento dell'approvazione della legge del
1956 il numero di cause devolute alla giustizia amministrativa era
relativamente basso, negli anni '70, si assiste a una saturazione:
soprattutto per quanto riguarda il tribunale supremo, i ritardi nella
gestione delle controversie diventano cronici e in costante incremento.
Le riforme saranno quindi d'ora in poi indirizzate a contenere tale
saturazione con l'ampliamento delle competenze delle udienze
territoriali relative ai provvedimenti dell'amministrazione dello Stato,
la creazione dell'udienza nazionale, la soppressione dell'appello contro
le sentenze delle udienze territoriali e dell'udienza nazionale, in modo
tale che il tribunale supremo divenga così una corte di cassazione.

\hypertarget{la-legge-riformatrice-n.-291998}{%
\section{La legge riformatrice
n.~29/1998}\label{la-legge-riformatrice-n.-291998}}

La nuova \emph{Ley de la Jurisdicción Contencioso-Administrativa} (LJCA)
n.~29 del 13 luglio 1998 ha introdotto importanti riforme
nell'organizzazione della giustizia amministrativa. Il sistema spagnolo
risulta ora strutturato sostanzialmente in tre gradi di giudizio ed ha
alla sua base organi monocratici quali i giudici provinciali e i giudici
centrali, costituiti da magistrati della carriera giudiziaria ordinaria.
Le principali attribuzioni dei primi sono riferite alle liti relative
all'amministrazione locale, regionale e periferica dello Stato, mentre
afferiscono ai giudici centrali principalmente le controversie con
l'amministrazione centrale dello Stato in materia di personale. Le
udienze territoriali sono state trasformate nei tribunali superiori di
giustizia delle comunità autonome che sono organi collegiali costituiti
da magistrati professionali. Ogni tribunale ha una sala del contenzioso
amministrativo, salvo quello dell'Andalusia che ne ha tre e quelli delle
Canarie e di Castiglia-Leone che ne hanno due. Ciascuna sala, a sua
volta, può avere una o più sezioni, ad esempio nella sala di Madrid
arrivano a un numero di nove. I tribunali superiori di giustizia
conoscono delle liti che non sono state conferite agli altri organi
della giustizia amministrativa e degli appelli e delle revisioni delle
sentenze dei giudici provinciali. Anche l'udienza nazionale è un organo
collegiale e la sua sala del contenzioso amministrativo è costituita da
otto sezioni. Si occupa delle controversie relative all'amministrazione
dello Stato, fatta eccezione per quelle di competenza dei giudici
centrali, e conosce degli appelli e della revisione delle sentenze dei
giudici centrali. Al momento della sua creazione nel 1977, l'udienza
nazionale era stata vista in dottrina come la trasformazione del vecchio
tribunale di ordine pubblico del franchismo ma, attualmente, è chiara
manifestazione del criterio di parità fra l'organo amministrativo
controllato e l'organo giudiziario di controllo. Vi è infine il
tribunale supremo, anch'esso organo collegiale, costituito da magistrati
scelti sempre attraverso procedure che ne garantiscono l'imparzialità e
l'idoneità. La sala terza è quella competente nella materia contenzioso
amministrativa ed è composta da sette sezioni. Le principali
attribuzioni si riferiscono, in unica istanza, ai litigi relativi
all'attività amministrativa dei più alti livelli dell'amministrazione
dello Stato e degli organi costituzionali. La sala terza conosce inoltre
della cassazione e della revisione delle sentenze dei tribunali
superiori e dell'udienza nazionale.

\hypertarget{la-jurisdicciuxf3n-contencioso-administrativa}{%
\section{La jurisdicción
contencioso-administrativa}\label{la-jurisdicciuxf3n-contencioso-administrativa}}

Con la promulgazione della Costituzione del 1978 viene sancito all'art.
24 il diritto di ogni individuo a ottenere la tutela effettiva dai
giudici e dai tribunali nell'esercizio dei loro diritti e interessi
legittimi, senza che in nessun caso possa prodursi una mancanza di
difesa. All'art. 117, c.~3, si stabilisce che l'esercizio della potestà
giurisdizionale in ogni tipo di processo, giudicando e facendo eseguire
il giudicato, spetta esclusivamente ai giudici e ai tribunali
determinati dalle leggi, mentre l'art. 118 obbliga a ottemperare alle
sentenze e alle altre risoluzioni definitive dei giudici e dei
tribunali, così come di prestare la collaborazione richiesta dagli
stessi nel corso del processo e nell'esecuzione delle decisioni. E' però
all'art. 103, c.1, che si dichiara la piena sottomissione della pubblica
amministrazione alla legge e al diritto, cosa che comporta la
sottomissione al giudice come strumento inevitabile del
diritto\footnote{Cfr. E. \textsc{García de Enterría} -- T. R.
  \textsc{Fernández}, \emph{Curso de derecho administrativo}, Madrid,
  Civitas-Revista de Occidente, 1. ed 1974, 650.}. La giurisprudenza
costituzionale ha presto esteso tutte queste garanzie alla giurisdizione
contenzioso-amministrativa e le ha ribadite in seguito in numerose
occasioni e oggi, l'art. 103 della LJCA, n.~29/1998 contiene nei suoi
primi tre commi la traduzione dei medesimi principi costituzionali,
attribuendo ai tribunali e i giudici della giurisdizione
contenzioso-amministrativa il potere di eseguire le sentenze e le altre
risoluzioni giudiziali\footnote{Art. 103, c.~1, LJCA.}. In tal modo si
chiude normativamente un ciclo storico che vede la sparizione
dell'antico privilegio dell'amministrazione di far eseguire le sentenze
e ciò costituisce una delle novità più rilevanti introdotte dalla legge
processuale amministrativa in vigore.

Il c.~2 dell'art. 103 LJCA impone l'obbligo alle parti di conformarsi a
quanto stabilito nella sentenza. Amministrazione e amministrato nel
processo contenzioso-amministrativo si trovano in condizioni di parità e
tale parità si riverbera anche nel momento dell'esecuzione della
sentenza: certamente la posizione singolare dell'amministrazione
modulerà gli effetti del processo esecutivo, ma ora è mutata in modo
sostanziale, in quanto la stessa non potrà più sottrarsi all'obbligo di
ottemperare al contenuto della sentenza così come stabilito dal giudice.
Il c.~3 dell'art. 103 LJCA stabilisce che tutti i soggetti, pubblici e
privati, debbono prestare la collaborazione richiesta dai giudici e dai
tribunali per la completa esecuzione del giudicato, pertanto,
l'amministrazione che agisce in funzione di tale dovere di
collaborazione non opera nell'ambito delle sue prerogative di autorità
amministrativa, ma i suoi atti sono riconducibili al potere del giudice
di utilizzare strumentalmente l'amministrazione per l'esercizio della
sua funzione giurisdizionale: il ruolo di quest'ultima sarà quindi
quello di collaboratore alla realizzazione del dovere imposto dalla
legge nell'ambito del processo di esecuzione. Allo stesso tempo, la LJCA
prevede la nullità assoluta di tutti gli atti e disposizioni
dell'amministrazione che hanno come obiettivo l'elusione del giudicato;
nullità che sarà dichiarata dal giudice attraverso un processo
incidentale di esecuzione\footnote{Art. 109, c.~1, LJCA.
  L'\emph{Incidente de ejecución} è il processo incidentale di
  trattazione di tutte le questioni che, senza contraddire il contenuto
  della sentenza, sorgono e sono correlate all'esecuzione della
  medesima.}.

Ai fini dell'esecuzione della sentenza da parte dell'amministrazione, è
stabilito un termine di due mesi, decorso il quale potrà darsi avvio
all'esecuzione forzata, ai sensi dell'art. 104, c.~2, LJCA. E' prevista
sia l'esecuzione diretta, attraverso i mezzi propri
dell'amministrazione, sia indiretta, richiedendo l'intervento di altre
autorità o agenti pubblici, il cui costo sarà a carico della stessa
amministrazione condannata. Vi è una procedura di esecuzione più agile e
rapida delle sentenze che condannino l'amministrazione al pagamento di
una somma in denaro e, qualora si renda necessaria una modifica
dell'importo, la legge concede un termine di tre mesi per provvedere. E'
altresì permessa la compensazione dei debiti. Riguardo ai mezzi
esecutivi, la legge non prevede alcuna distinzione a seconda del tipo di
pretesa esercitata, sia che si riferisca al mero annullamento di un
provvedimento, al ripristino della situazione giuridica iniziale,
all'inerzia dell'amministrazione o contro le vie di fatto ad essa
imputabili. Tutte queste pretese possono essere fatte valere allo stesso
modo ai fini dell'esecuzione coattiva in presenza di apposita sentenza
del giudice competente.

A grandi linee, potremmo così riassumere il sistema di esecuzione delle
sentenze implementato dalla LJCA n.~29/1998:

\begin{enumerate}
\def\labelenumi{\alph{enumi})}
\item
  una volta pronunciata la sentenza, se ne comunica il dispositivo entro
  dieci giorni all'amministrazione che ha compiuto l'attività oggetto
  del ricorso, affinché ne porti a compimento gli effetti e metta in
  pratica quanto da esso stabilito. Dal canto suo l'amministrazione,
  entro un ulteriore termine di dieci giorni, dovrà confermarne la
  ricezione e indicare l'organo che vi dovrà ottemperare\footnote{Art.
    104, c.~1, LJCA.};
\item
  qualora siano decorsi due mesi dalla notifica della sentenza o dal
  termine fissato per la sua esecuzione, le parti coinvolte e le persone
  interessate potranno avviarne l'esecuzione forzata\footnote{Art. 104,
    c.~2 e 3, LJCA.};
\item
  se la sentenza condanna l'amministrazione a realizzare una data
  attività o ad adottare un determinato atto o provvedimento, l'organo
  giudiziario potrà far eseguire la sentenza sia direttamente, sia
  indirettamente, nel rispetto delle procedure stabilite, e adotterà le
  misure necessarie affinché il dispositivo in essa contenuto acquisisca
  l'efficacia dovuta in relazione all'atto o provvedimento omesso
  dall'amministrazione, determinandone altresì il contenuto in
  sostituzione di quest'ultima\footnote{Art. 108, c.~1, LJCA.}.
\item
  decorsi i termini stabiliti per l'esecuzione della sentenza senza che
  l'amministrazione abbia provveduto, il giudice o l'organo
  giurisdizionale, sentite le parti, adotta le misure necessarie al
  conseguimento di quanto stabilito, potendo, con apposito preavviso,
  imporre il pagamento di una penale di importo variabile da
  centocinquanta a millecinquecento Euro alle autorità, ai funzionari od
  agenti che non ottemperino alle prescrizioni. Tale penale, a titolo di
  sanzione, può essere reiterata fino alla completa esecuzione della
  sentenza, ferme restando le altre responsabilità patrimoniali che
  dovessero insorgere\footnote{Art. 112, LJCA.}.
\end{enumerate}

\hypertarget{il-defensor-del-pueblo}{%
\section{\texorpdfstring{Il \emph{Defensor del
Pueblo}}{Il Defensor del Pueblo}}\label{il-defensor-del-pueblo}}

L'art. 54 della Costituzione spagnola prevede, nell'ambito delle
garanzie e delle libertà fondamentali di tutti gli individui, la figura
del \emph{Defensor del Pueblo} (Difensore del Popolo), una sorta di
difensore civico le cui attribuzioni però lo rendono più paragonabile
agli ombudsman britannici. Il \emph{Defensor del Pueblo} è l'Alto
Commissario delle \emph{Cortes Generales}, incaricato di difendere i
diritti fondamentali e le libertà pubbliche dei cittadini vigilando
sull'attività delle pubbliche amministrazioni spagnole. E' eletto dal
Congresso dei Deputati e dal Senato, a maggioranza dei tre quinti. Il
suo mandato dura cinque anni, è indipendente dagli ordini e dalle
istruzioni di qualsiasi altra autorità, svolge le proprie funzioni con
imparzialità e autonomia, secondo propri criteri e godendo
\hspace{0pt}\hspace{0pt}dell'inviolabilità e dell'immunità
nell'esercizio della sua carica. Qualsiasi cittadino può rivolgersi al
\emph{Defensor del Pueblo} e richiedere il suo intervento che è gratuito
per indagare su qualsiasi presunta azione irregolare della Pubblica
Amministrazione o dei suoi agenti. Può anche intervenire d'ufficio nei
casi pervenuti alla sua attenzione, anche se non è stata sporta
denuncia. Il \emph{Defensor del Pueblo} riferisce della sua gestione
alle Cortes Generali in una relazione annuale e può presentare relazioni
monografiche su questioni che considera gravi, urgenti o che richiedono
una particolare attenzione\footnote{Cfr. \textsc{Defensor del Pueblo},
  \emph{¿Qué es el Defensor?}, \emph{defensordelpueblo.es}, in
  \url{https://www.defensordelpueblo.es/el-defensor/que-es-el-defensor/}
  (Consultato: 24 gennaio 2023).}.

I reclami possono essere presentati individualmente o collettivamente,
quando i cittadini ritengono che l'azione di un'amministrazione
(centrale, regionale o locale) o di un ente erogatore di pubblico
servizio abbia violato i loro diritti. I cittadini possono anche
chiedere al \emph{Defensor del Pueblo} di presentare ricorso per
incostituzionalità e per ricevere tutela dinanzi alla Corte
costituzionale. Nel caso in cui il \emph{Defensor del Pueblo} non possa
indagare su un ricorso perché non rientrante nella competenza, invia una
lettera al cittadino spiegandone i motivi. Quando invece ritiene
possibile il suo intervento, fornisce l'assistenza adeguata al caso
specifico, monitorando ogni violazione di diritti e può agire d'ufficio,
aprendo procedimenti o consultazioni senza indugi. Sebbene egli non
possa annullare o modificare gli atti o le deliberazioni delle pubbliche
amministrazioni, nel caso in cui accerti che dei diritti fondamentali
siano stati violati, la sua missione sarà quella di convincere
l'amministrazione ad adottare le misure più idonee per porre rimedio
alle violazioni. Qualsiasi individuo o persona giuridica può presentare
istanza al \emph{Defensor del Pueblo} gratuitamente, senza la necessità
di ricorrere al patrocinio di un legale. La denuncia o richiesta di
assistenza può essere presentata via internet tramite l'apposito sito
dedicato, per posta ordinaria o di persona, rivolgendosi agli uffici
dedicati ai servizi per il cittadino. La stessa denuncia o istanza dovrà
essere accuratamente motivata e corredata di qualsiasi documento o prova
pertinente alla questione da trattare.

Il \emph{Defensor del Pueblo} non può intervenire nelle controversie tra
cittadini e società private, laddove cioè la Pubblica Amministrazione
non sia parte in causa, quando sia trascorso più di un anno dal momento
in cui si sia venuti a conoscenza dei fatti o delle circostanze oggetto
d'esame, in presenza di una denuncia anonima o di un reclamo per il
quale non siano specificati i motivi, in caso di accertata malafede
nella presentazione dell'istanza, denuncia o reclamo, quando dalla
trattazione del ricorso possa derivare pregiudizio per i diritti di
terzi e, infine, in caso di disaccordo con il dispositivo di una
sentenza o quando la decisione sulla controversia sia ancora pendente
innanzi ad un tribunale\footnote{Cfr. \textsc{Defensor del Pueblo},
  \emph{¿Cómo te podemos ayudar?}, \emph{defensordelpueblo.es}, in
  \url{https://www.defensordelpueblo.es/el-defensor/como-te-podemos-ayudar/}
  (Consultato: 24 gennaio 2023).}.

\hypertarget{conclusioni}{%
\chapter*{Conclusioni}\label{conclusioni}}
\addcontentsline{toc}{chapter}{Conclusioni}

Ancor prima del ``giusto processo'', l'art. 1 del nostro c.p.a. enuncia
il fondamentale principio secondo il quale ``la giurisdizione
amministrativa assicura una tutela piena ed effettiva secondo i principi
della Costituzione e del diritto europeo''. In passato, negli
ordinamenti europei, la pubblica amministrazione fondava il proprio
\emph{agere} su di una illimitata discrezionalità e questo, spesso, si
traduceva in un concreto pregiudizio patito dal privato, considerato
unicamente quale mero destinatario dell'azione amministrativa e, come
tale, costantemente costretto ad uno stato di soggezione nei confronti
del pubblico potere. A partire dagli anni `90 però (in Italia con la
legge 241/90 sul procedimento amministrativo), il cittadino ha iniziato
ad essere considerato non più quale soggetto passivo della
discrezionalità amministrativa, ma, al contrario, è divenuto
beneficiario di una serie di doveri, ai quali l'\emph{agere publicum}
deve ormai costantemente ispirarsi, tra i quali la pubblicità, la
trasparenza e l'imparzialità. Questo cambio di rotta rispetto al passato
ha portato ad avvertire l'esigenza di delineare un apparato giuridico
che fosse sempre più idoneo ad offrire maggiori garanzie a favore del
privato leso in un proprio diritto soggettivo o interesse legittimo per
effetto dell'azione amministrativa. In questa prospettiva, i vari
ordinamenti giuridici hanno messo a punto un più dettagliato ed
articolato sistema di tutele sul quale, in qualsiasi momento, il
cittadino può fare legittimamente affidamento rispetto agli atti e
comportamenti della pubblica amministrazione che gli recano pregiudizio.

In riferimento ad un tema tradizionale come quello della comparazione
dei modelli di giurisdizione, questa indagine ha permesso di rilevare il
deperimento del modello monistico con prevalenza del giudice ordinario,
a vantaggio della prevalenza di organi e procedure specifiche per le
questioni amministrative: il caso della Gran Bretagna è l'esempio
emblematico di come un ordinamento tipicamente monistico si sia aperto
ad una forma di giustizia specializzata che dell'unitarietà della
giurisdizione mantiene solo la veste formale. La riforma del 1977,
perfezionata poi negli anni successivi con il \emph{Supreme Court Act}
del 1981, ha introdotto nel sistema inglese una specifica procedura per
i ricorsi avverso la pubblica amministrazione, la \emph{application for
judicial review}, improntata a regole assai simili a quelle vigenti
negli ordinamenti in cui da tempo esistono particolari forme di tutela
giurisdizionale.

Un caso apparentemente opposto a quello inglese è quello della Spagna,
ove, partiti da un sistema misto con forti connotazioni dualistiche, si
è giunti ad un modello monistico con prevalenza del contenzioso
amministrativo. Sono stati infatti creati organi specializzati della
\emph{Jurisdicción Contencioso-Administrativa} (le \emph{Salas de lo
Contencioso Administrativo}) con al vertice la \emph{Sala de Revision}
del \emph{Tribunal Supremo}, con giudici dedicati e procedure
particolari. Particolare influenza riformatrice ha avuto la nuova
Costituzione democratica del 1978, ove l'art. 24 ha ripreso l'articolo
omologo della nostra Costituzione, ma con una significativa apertura
all'effettività della tutela. Il sistema spagnolo si segnala inoltre per
alcune peculiarità di interesse per gli altri ordinamenti vicini. Una di
queste è la combinazione del modello dispositivo del processo
amministrativo con un rilevante potere riconosciuto al giudice per
l'eventuale sottoposizione alle parti di ulteriori profili di
legittimità, dalle stesse non rilevati; un'altra peculiarità, almeno
rispetto agli altri ordinamenti latini e non per quello britannico che
conosce l'istituto del \emph{leave}, è l'istituto della \emph{admision}
che consente al giudice di filtrare ricorsi palesemente inammissibili.

Nell'ordinamento tedesco, così come per certi versi in quello francese,
emerge la grande importanza attribuita alla prevenzione del rischio di
inottemperanza da parte dell'amministrazione. In pratica, si è qui
cercato di costruire, attraverso la previsione di incisivi poteri
direttivi spendibili dal giudice amministrativo in un momento
antecedente a quello dell'esecuzione forzata del giudicato, un modello
idoneo a garantire al ricorrente una piena ed effettiva tutela
giurisdizionale e finalizzato, attraverso il chiarimento nella maggior
misura possibile della portata delle statuizioni da eseguire, ad
incentivare la spontanea esecuzione delle sentenze da parte
dell'amministrazione. Da ciò si ricava come i meccanismi di coazione
indiretta, quali lo \emph{Zwangsgeld} tedesco e l'\emph{astreinte}
francese, siano traguardati come \emph{estrema ratio}, deputati a
fronteggiare i rischi di volontaria inesecuzione, peraltro del tutto
trascurabili. Si delinea dunque un sistema in cui, da un lato, riguardo
agli aspetti vincolati del potere in contestazione, vengono anticipate
al giudizio di cognizione quelle pronunce ordinatorie
(\emph{Verplichtungsurteil, Folgenbeseitgungurteil} e \emph{pouvoir
d'injonction}), proprie del nostro giudizio di ottemperanza, aventi ad
oggetto il provvedimento o il comportamento determinato che
l'amministrazione è tenuta a rendere, mentre, dall'altro, in ordine al
problema di come indurre l'amministrazione a riempire i residui spazi di
discrezionalità, ricorre il potere del giudice amministrativo di
irrogare delle penali nei confronti della stessa amministrazione
inadempiente. Anche in Germania si avverte sempre più stretto il
collegamento tra procedimento e processo, con modelli normativi similari
di codificazione e sistematicità, tra loro in piena sinergia.
L'efficacia del processo amministrativo, anche ai fini della
satisfattiva combinazione degli interessi, fa sì che qui i rimedi
alternativi alla tutela giurisdizionale non abbiano un particolare peso.
Nell'ordinamento francese, prototipo del sistema dualistico di
giurisdizione, si avverte l'esigenza di andare oltre la
caratterizzazione del giudice amministrativo quale garante della
giustizia nell'amministrazione, per assicurare la pienezza della tutela
dei singoli avverso la pubblica amministrazione. In Francia, taluni
principi, al di là del possibile rilievo costituzionale, sono stati poi
consolidati in leggi: ciò è avvenuto con le riforme del 1995 e degli
anni successivi in tema di potere generale di ingiunzione, di potere di
irrogare sanzioni (\emph{astreinte}) alle amministrazioni inottemperanti
e di arricchimento delle misure cautelari, combinati con un penetrante
potere inquisitorio che consente al giudice di essere il ``signore del
processo''.

Dalla visione comparativa effettuata si è constatato inoltre lo sviluppo
di forme di tutela diverse dalla giurisdizione nei vari ordinamenti. Da
una parte, forme di tutela di tipo amministrativo, preliminari alla
giurisdizione, nel solco di una tradizione che in passato aveva
accomunato anche l'ordinamento italiano; dall'altra, di carattere
alternativo, perché potenzialmente idonee a prevenire o superare il
contenzioso giurisdizionale. Questi istituti, ove ben funzionanti,
contribuiscono alla deflazione delle controversie giurisdizionali, con
procedimenti che si svolgono in modo più rapido e che sono meno onerosi
dei processi. Innanzitutto, i ricorsi amministrativi sono presenti e
rimangono importanti in tutti gli ordinamenti considerati. Essi si
rivelano utili per gli interessati quando non rappresentano
semplicemente un necessario presupposto per accedere alla tutela
giurisdizionale, ma quando sono trattati da soggetti che si distinguono
per formazione e posizione dai funzionari il cui operato è da essi
scrutinato: ne costituiscono l'esempio gli \emph{Administrative
Tribunals} britannici che sono in posizione indipendente
dall'amministrazione esaminata, hanno particolari garanzie operative che
fanno parlare di organi \emph{quasi-judicial}, seguono procedure di tipo
contenzioso e adottano decisioni vincolanti per le amministrazioni.

Spesso confuso con la tutela amministrativa, ha avuto una grande
espansione negli ordinamenti considerati l'istituto del difensore
civico, o \emph{Ombudsman}, secondo l'originaria denominazione
scandinava. Tale figura che i vari ordinamenti hanno organizzato in modi
assai diversi, a seconda della dimensione locale o nazionale, è
usualmente preposta alla verifica dei casi di ``cattiva
amministrazione'', ovvero quando non si contesta tanto l'illegittimità
dell'azione amministrativa, quanto la sua qualità. Malgrado le sue
segnalazioni non abbiano effetto vincolante, difficilmente
l'amministrazione si sottrae alla richiesta di intervento del difensore
civico, almeno in diretta correlazione con il suo prestigio
istituzionale. La connessione con il ricorso amministrativo è impropria,
principalmente perché il difensore civico ha posizione indipendente e si
pone come censore delle amministrazioni, senza in alcun modo esserne
parte e, in secondo luogo, perché normalmente gli organi decisori dei
ricorsi amministrativi hanno il potere di annullare o riformare i
provvedimenti impugnati. In comune, vi è il largo spazio per la
valutazione del merito del caso, con residuale rilievo dei profili di
legittimità. Nei paesi del Nord Europa, in Gran Bretagna e in Spagna
dove ha con successo attecchito, il difensore civico riscuote un largo
apprezzamento quale completamento della tutela in aree altrimenti non
presidiate dalla legalità, mentre non sono chiare le ragioni per le
quali in altri ordinamenti, come in Italia, il difensore civico sia poco
utilizzato, quasi sempre in relazione a questioni amministrative
marginali e con conseguenze poco efficaci. Nel nostro ordinamento ha
certamente avuto un ruolo negativo la carenza di un difensore civico
nazionale, per paradossale effetto dell'evoluzione autonomista
dell'ordinamento repubblicano, ma la debole esperienza sinora avutasi
sembra più espressiva della perdurante centralità della cultura del
processo che di lacune normative.

\footnotesize
\singlespacing
\setlength{\parindent}{0in}

\hypertarget{riferimenti}{%
\chapter*{Riferimenti}\label{riferimenti}}
\addcontentsline{toc}{chapter}{Riferimenti}

\hypertarget{refs}{}
\begin{CSLReferences}{1}{0}
\leavevmode\vadjust pre{\hypertarget{ref-bartolucciPouvoirInjonctionJuge2020}{}}%
\textsc{Bartolucci} Mattéo, \emph{Le pouvoir d'injonction du juge
administratif revisité par les circonstances exceptionnelles de la crise
sanitaire du Covid-19}, \emph{actu-juridique.fr}, in
\url{https://tinyurl.com/4sfhd3ux} (Consultato: 22 gennaio 2023).

\leavevmode\vadjust pre{\hypertarget{ref-de2013refere}{}}%
\textsc{Boulois} Xavier Dupré \textsc{de}, \emph{Le référé-liberté pour
autrui. Une société commerciale au secours du droit à la vie}, in
{«Revue des droits et libertés fondamentaux-RDLF»} (2013) 12.

\leavevmode\vadjust pre{\hypertarget{ref-bronzettiMiglioreGiustiziaAmministrativa1996}{}}%
\textsc{Bronzetti} Gianfranco, \emph{Per una migliore giustizia
amministrativa: spunti comparatistici con particolare riferimento
all'ordinamento della Repubblica federale tedesca}, in {«Informator»}
(1996) 1, 139--162.

\leavevmode\vadjust pre{\hypertarget{ref-Buck2007TribunalRI}{}}%
\textsc{Buck} Trevor Giles, \emph{Tribunal reform in the UK: Precedent
and reporting in the new unified structure}, 2007.

\leavevmode\vadjust pre{\hypertarget{ref-caneUnderstandingJudicialReview2004}{}}%
\textsc{Cane} Peter, \emph{Understanding judicial review and its
impact}, in M. \textsc{Hertogh} -- S. \textsc{Halliday} (Edd.),
\emph{Judicial Review and Bureaucratic Impact}, Cambridge University
Press, \textsuperscript{1}2004, 15--42, in
\url{https://www.cambridge.org/core/product/identifier/CBO9780511493782A010/type/book_part}
(Consultato: 23 gennaio 2023).

\leavevmode\vadjust pre{\hypertarget{ref-cassese1971privato}{}}%
\textsc{Cassese} Sabino, \emph{Il privato e il procedimento
amministrativo: una analisi della legislazione e della giurisprudenza},
STEM Mucchi, 1971.

\leavevmode\vadjust pre{\hypertarget{ref-cattaneoAgenciesRegulationNel1999}{}}%
\textsc{Cattaneo} Salvatore, \emph{Agencies e regulation nel Regno
Unito}, in S. \textsc{Labriola} (Ed.), \emph{Le autorità indipendenti:
da fattori evolutivi ad elementi della transizione nel diritto pubblico
italiano}, Milano, Giuffrè, 1999.

\leavevmode\vadjust pre{\hypertarget{ref-chapusDroitContentieuxAdministratif2008}{}}%
\textsc{Chapus} René, \emph{Droit du contentieux administratif} (=~Domat
droit public), Paris, Montchrestien-Lextenso éd, 13e éd 2008.

\leavevmode\vadjust pre{\hypertarget{ref-chieppaStudiDiDiritto2007}{}}%
\textsc{Chieppa} Roberto -- \textsc{Lopilato} Vincenzo, \emph{Studi di
diritto amministrativo} (=~Percorsi Giuffrè), Milano, Giuffrè, 2007.

\leavevmode\vadjust pre{\hypertarget{ref-ciaralliNuovoGiudizioDi}{}}%
\textsc{Ciaralli} Francesco Maria, \emph{Il nuovo giudizio di
ottemperanza, con particolare riguardo alle astreintes},
\emph{italiappalti.it}, in
\url{https://www.italiappalti.it/leggiarticolo.php?id=3595} (Consultato:
22 gennaio 2023).

\leavevmode\vadjust pre{\hypertarget{ref-clarichGiudizioAmministrativoDi1998}{}}%
\textsc{Clarich} Marcello, \emph{Il giudizio amministrativo di
esecuzione}, in G. \textsc{Paleologo} -- \textsc{Italy} --
\textsc{France} (Edd.), \emph{I consigli di Stato di Francia e
d'Italia}, Milano, Giuffrè, 1998.

\leavevmode\vadjust pre{\hypertarget{ref-conseilconstitutionnelDecision2009594DC}{}}%
\textsc{Conseil Constitutionnel}, \emph{Décision n° 2009-594 DC du 3
décembre 2009}, \emph{conseil-constitutionnel.fr}, in
\url{https://tinyurl.com/2p95r35b} (Consultato: 22 gennaio 2023).

\leavevmode\vadjust pre{\hypertarget{ref-couradministrativedappeldeparisExecutionDecisionsJuge}{}}%
\textsc{Cour administrative d'appel de Paris}, \emph{L'exécution des
décisions du juge administratif},
\emph{paris.tribunal-administratif.fr}, in
\url{https://tinyurl.com/5frr43pw} (Consultato: 22 gennaio 2023).

\leavevmode\vadjust pre{\hypertarget{ref-couradministrativedappeldeparisFichesPratiquesJustice}{}}%
---------, \emph{Les fiches pratiques de la justice administrative},
\emph{paris.tribunal-administratif.fr}, in
\url{https://tinyurl.com/3fw67rwb} (Consultato: 22 gennaio 2023).

\leavevmode\vadjust pre{\hypertarget{ref-craig2003human}{}}%
\textsc{Craig} Paul, \emph{The human rights act, article 6 and
procedural rights}, in {«Public law»} (2003) 4, Sweet \& Maxwell,
753--773.

\leavevmode\vadjust pre{\hypertarget{ref-dalbertiDirittoAmministrativoComparato2019}{}}%
\textsc{D'Alberti} Marco, \emph{Diritto amministrativo comparato:
mutamenti dei sistemi nazionali e contesto globale} (=~Manuali),
Bologna, Il Mulino, 2019.

\leavevmode\vadjust pre{\hypertarget{ref-defranchisDizionarioGiuridicoIngleseItaliano1984}{}}%
\textsc{De Franchis} Francesco, \emph{Dizionario Giuridico. v.1:
inglese-Italiano = Law dictionary : English-italian}, Milano, Giuffrè,
1984.

\leavevmode\vadjust pre{\hypertarget{ref-de2000processo}{}}%
\textsc{De Pretis} Daria, \emph{Il processo amministrativo in europa:
caratteri e tendenze in Francia, Germania, Gran Bretagna e nell'Unione
europea}, a cura di R. Paganella, Trento, Ed. provvisoria 2000, in
\url{https://tinyurl.com/2p8bjs4u}.

\leavevmode\vadjust pre{\hypertarget{ref-defensordelpuebloComoTePodemos}{}}%
\textsc{Defensor del Pueblo}, \emph{¿Cómo te podemos ayudar?},
\emph{defensordelpueblo.es}, in
\url{https://www.defensordelpueblo.es/el-defensor/como-te-podemos-ayudar/}
(Consultato: 24 gennaio 2023).

\leavevmode\vadjust pre{\hypertarget{ref-defensordelpuebloQueEsDefensor}{}}%
---------, \emph{¿Qué es el Defensor?}, \emph{defensordelpueblo.es}, in
\url{https://www.defensordelpueblo.es/el-defensor/que-es-el-defensor/}
(Consultato: 24 gennaio 2023).

\leavevmode\vadjust pre{\hypertarget{ref-denningReginaLocalCommissioner2022}{}}%
\textsc{Denning} Tom, \emph{Regina v Local Commissioner for
Administration for the North and East Area of England ex parte Bradford
Metropolitan City Council: CA 1979, QB 287}, \emph{swarb.co.uk}, in
\url{https://tinyurl.com/44swkwxm} (Consultato: 24 gennaio 2023).

\leavevmode\vadjust pre{\hypertarget{ref-detterbeckAllgemeinesVerwaltungsrechtMit2022}{}}%
\textsc{Detterbeck} Steffen, \emph{Allgemeines Verwaltungsrecht: mit
Verwaltungsprozessrecht} (=~Lernbücher Jura), München, C.H. Beck, 20.
Auflage 2022.

\leavevmode\vadjust pre{\hypertarget{ref-dijauxCorvee}{}}%
\textsc{Dijaux} Agnes, \emph{La corvée}, \emph{Chambra d'Òc, Portal
Français, 100 mots du trésor FR}, in
\url{http://www.chambradoc.it/traditions/La-corvee.page} (Consultato: 22
gennaio 2023).

\leavevmode\vadjust pre{\hypertarget{ref-falconOrdinamentoProcessualeAmministrativo2000b}{}}%
\textsc{Falcon} Giandomenico -- \textsc{Fraenkel} Cristina (Edd.),
\emph{Ordinamento processuale amministrativo tedesco: (VwGO); versione
italiana con testo a fronte} (=~Quaderni del Dipartimento di Scienze
Giuridiche, 27), Trento, Università degli Studi di Trento, 2000.

\leavevmode\vadjust pre{\hypertarget{ref-forsythAdministrativeLaw2014}{}}%
\textsc{Forsyth} C. F. -- \textsc{Wade} William -- \textsc{Wade}
William, \emph{Administrative law}, Oxford, United Kingdom ; New York,
NY, Oxford University Press, Eleventh edition 2014.

\leavevmode\vadjust pre{\hypertarget{ref-fraenkel-haeberleGiurisdizioneSulSilenzio2004}{}}%
\textsc{Fraenkel-Haeberle} Cristina, \emph{Giurisdizione sul silenzio e
discrezionalità amministrativa: Germania, Austria, Italia} (=~Quaderni
del Dipartimento, 43), Trento, Università degli studi di Trento,
Dipartimento di scienze giuridiche, 2004.

\leavevmode\vadjust pre{\hypertarget{ref-fromont1989esecuzione}{}}%
\textsc{Fromont} Michel, \emph{L'esecuzione delle decisioni del giudice
amministrativo nel diritto francese e tedesco L'exécution des décisions
du juge administratif en droit français et allemand}, in {«Problemi di
Amministrazione Publica»} (1989) 3, 523--540.

\leavevmode\vadjust pre{\hypertarget{ref-garciadeenterriaCursoDerechoAdministrativo1974}{}}%
\textsc{García de Enterría} Eduardo -- \textsc{Fernández} Tomás Ramón,
\emph{Curso de derecho administrativo}, Madrid, Civitas-Revista de
Occidente, 1. ed 1974.

\leavevmode\vadjust pre{\hypertarget{ref-goisis2018full}{}}%
\textsc{Goisis} F, \emph{La full jurisdiction sulle sanzioni
amministrative: continuità della funzione sanzionatoria v. separazione
dei poteri}, in (2018), Giuffrè.

\leavevmode\vadjust pre{\hypertarget{ref-hauriouPrecisDroitAdministratif2002}{}}%
\textsc{Hauriou} Maurice et al., \emph{Précis de droit administratif et
de droit public} (=~Bibliothèque Dalloz), Paris, Dalloz, 12e éd 2002.

\leavevmode\vadjust pre{\hypertarget{ref-dhellencourtQuoiSertMediateur2015}{}}%
\textsc{Hellencourt} Xavier \textsc{d'}, \emph{A quoi sert le médiateur
de la république?}, \emph{dhellencourt-avocats.fr}, in
\url{https://tinyurl.com/2wrsatec} (Consultato: 22 gennaio 2023).

\leavevmode\vadjust pre{\hypertarget{ref-juripredisDifferenceEntreArret}{}}%
\textsc{Juri'Predis}, \emph{Différence entre un arrêt et une décision},
\emph{juripredis.com}, in \url{https://tinyurl.com/ye22pd59}
(Consultato: 22 gennaio 2023).

\leavevmode\vadjust pre{\hypertarget{ref-karpenEsperienzaGermania1994}{}}%
\textsc{Karpen} Ulrich, \emph{L'esperienza della Germania}, in D.
\textsc{Sorace} (Ed.), \emph{La responsabilità pubblica nell'esperienza
giuridica europea} (=~Organizzazione e funzionamento della pubblica
amministrazione, 39), Bologna, Il Mulino, 1994.

\leavevmode\vadjust pre{\hypertarget{ref-laferriere1896traite}{}}%
\textsc{Laferrière} Edouard Louis Julien, \emph{Traité de la juridiction
administrative et des recours contentieux}, vol. 2, Berger-Levrault et
cie, 1896.

\leavevmode\vadjust pre{\hypertarget{ref-localgovernmentandsocialcare-ombudsmanChallengingOurDecisions}{}}%
\textsc{Local Government and Social Care - Ombudsman}, \emph{Challenging
our decisions}, \emph{lgo.org.uk}, in \url{https://tinyurl.com/4ez42xd9}
(Consultato: 24 gennaio 2023).

\leavevmode\vadjust pre{\hypertarget{ref-manciniResponsabilitaPubblicaAmministrazione2003}{}}%
\textsc{Mancini} Laura, \emph{La responsabilità della pubblica
amministrazione per inottemperanza al giudicato amministrativo di
annullamento(nota a Cons.St.,sez.IV,6 ottobre 2003 n.5820)}, in {«Il
Foro amministrativo - CdS»} 2 (2003) 12, 3700--3724.

\leavevmode\vadjust pre{\hypertarget{ref-marchettiEsecuzioneSentenzaAmministrativa2000a}{}}%
\textsc{Marchetti} Barbara, \emph{L'esecuzione della sentenza
amministrativa prima del giudicato} (=~Dipartemento di scienze
giuridiche, 32), Padova, CEDAM, 2000.

\leavevmode\vadjust pre{\hypertarget{ref-marshallFranksReportAdministrative1957}{}}%
\textsc{Marshall} Geoffrey, \emph{The Franks Report on Administrative
Tribunals and Enquiries}, in {«Public Administration»} 35 (dicembre
1957) 4, 347--358, in
\url{https://onlinelibrary.wiley.com/doi/10.1111/j.1467-9299.1957.tb01316.x}
(Consultato: 23 gennaio 2023).

\leavevmode\vadjust pre{\hypertarget{ref-ministeredelajusticeOrdreAdministratif}{}}%
\textsc{Ministère de la Justice}, \emph{L'ordre administratif},
\emph{justice.gouv.fr}, in
\url{https://www.justice.gouv.fr/organisation-de-la-justice-10031/lordre-administratif-10034/}
(Consultato: 22 gennaio 2023).

\leavevmode\vadjust pre{\hypertarget{ref-nigro1974consiglio}{}}%
\textsc{Nigro} M, \emph{Il Consiglio di Stato giudice e amministratore
(aspetti di effettività dell'organo)}, in \emph{Riv. trim. dir. proc.
civ}, vol. 1470, 1974.

\leavevmode\vadjust pre{\hypertarget{ref-nigroGiustiziaAmministrativa1995}{}}%
\textsc{Nigro} Mario -- \textsc{Nigro} Alessandro -- \textsc{Cardi}
Enzo, \emph{Giustizia amministrativa} (=~Strumenti), Bologna, Il Mulino,
4a ed 1995.

\leavevmode\vadjust pre{\hypertarget{ref-parliamentaryandhealthserviceombudsmanphsoPHSOPublications}{}}%
\textsc{Parliamentary and Health Service Ombudsman (PHSO)}, \emph{PHSO
publications}, \emph{ombudsman.org.uk}, in
\url{http://www.ombudsman.org.uk/publications} (Consultato: 24 gennaio
2023).

\leavevmode\vadjust pre{\hypertarget{ref-parliamentaryandhealthserviceombudsmanphsoPrinciplesGoodComplaint}{}}%
---------, \emph{Principles of Good Complaint Handling},
\emph{ombudsman.org.uk}, in \url{https://tinyurl.com/v8n6sfsf}
(Consultato: 24 gennaio 2023).

\leavevmode\vadjust pre{\hypertarget{ref-pelilloGiudizioDiOttemperanza1990}{}}%
\textsc{Pelillo} Sandro, \emph{Il giudizio di ottemperanza alle sentenze
del giudice amministrativo} (=~Collana della Facoltà Serie istituzionale
/ Università G. D'Annunzio, Facoltà di Giurisprudenza, Teramo, 16),
Milano, Giuffrè, 1990.

\leavevmode\vadjust pre{\hypertarget{ref-remienRechtsverwirklichungDurchZwangsgeld1992}{}}%
\textsc{Remien} Oliver, \emph{Rechtsverwirklichung durch Zwangsgeld:
Vergleich, Vereinheitlichung, Kollisionsrecht} (=~Beiträge zum
ausländischen und internationalen Privatrecht, 54), Tübingen, J.C.B.
Mohr, 1992.

\leavevmode\vadjust pre{\hypertarget{ref-ae58aae3e457425485dcebf1e158bb92}{}}%
\textsc{Richardson} Genevra -- \textsc{Genn} Hazel, \emph{Tribunals in
transition: resolution or adjudication?}, in {«PUBLIC LAW»} 01 (2007),
Sweet and Maxwell, 116--141.

\leavevmode\vadjust pre{\hypertarget{ref-ryssdallOpinionComingAge1996}{}}%
\textsc{Ryssdall} Rolv, \emph{Opinion\,: the coming of age of the
European Convention on Human Rights.}, in {«EUROPEAN HUMAN RIGHTS LAW
REVIEW»} (1996) 1, Thomson Reuters, 18--29, in
\url{https://search.informit.org/doi/10.3316/agispt.19961123}.

\leavevmode\vadjust pre{\hypertarget{ref-sandfordLocalGovernmentOmbudsman2023}{}}%
\textsc{Sandford} Mark, \emph{The Local Government Ombudsman},
\emph{House of Commons Library}, in
\url{https://commonslibrary.parliament.uk/research-briefings/sn04117/}
(Consultato: 5 febbraio 2023).

\leavevmode\vadjust pre{\hypertarget{ref-sandulliEffettivitaDecisioniGiurisdizionali1983}{}}%
\textsc{Sandulli} Aldo Mazzini, \emph{L'effettività delle decisioni
giurisdizionali amministrative}, in \emph{Atti del Convegno celebrativo
del 150° anniversario del Consiglio di Stato} (=~Miscellanea
dell'Istituto Giuridico dell'Università di Torino, Miscellanea II della
Serie II), Milano, Giuffrè, 1983.

\leavevmode\vadjust pre{\hypertarget{ref-schenkeVerwaltungsgerichtsordnungKommentar2022}{}}%
\textsc{Schenke} Wolf-Rüdiger et al., \emph{Verwaltungsgerichtsordnung:
Kommentar}, München, C.H. Beck, 28., neubearbeitete Auflage 2022.

\leavevmode\vadjust pre{\hypertarget{ref-schochVerwaltungsrechtVwGOKommentar2020}{}}%
\textsc{Schoch} Friedrich et al., \emph{Verwaltungsrecht - VwGO:
Kommentar} (=~Verwaltungsrecht), München, C.H. Beck, 2020.

\leavevmode\vadjust pre{\hypertarget{ref-secchi2010esecuzione}{}}%
\textsc{Secchi} Federico, \emph{L'esecuzione del giudicato
amministrativo nell'esperienza italiana e tedesca: le soluzioni al
problema dell'inottemperanza}, PhD thesis, University of Trento, 2010,
in
\url{http://eprints-phd.biblio.unitn.it/312/2/Secchi_tesi_dottorato.pdf}.

\leavevmode\vadjust pre{\hypertarget{ref-senatPouvoirInjonctionPrononce}{}}%
\textsc{Sénat}, \emph{Le pouvoir d'injonction et le prononcé
d'astreintes pour l'exécution des décisions de justice},
\emph{senat.fr}, in \url{https://www.senat.fr/rap/l98-380/l98-3805.html}
(Consultato: 22 gennaio 2023).

\leavevmode\vadjust pre{\hypertarget{ref-servicepublicLitigeAvecAdministration}{}}%
\textsc{Service Public}, \emph{Litige avec l'administration : saisir le
Défenseur des droits}, \emph{service-public.fr}, in
\url{https://www.service-public.fr/particuliers/vosdroits/F13158}
(Consultato: 22 gennaio 2023).

\leavevmode\vadjust pre{\hypertarget{ref-traviLezioniDiGiustizia2021}{}}%
\textsc{Travi} Aldo, \emph{Lezioni di giustizia amministrativa}, Torino,
G. Giappichelli, 14. ed 2021.

\leavevmode\vadjust pre{\hypertarget{ref-treccaniCorvee}{}}%
\textsc{Treccani}, \emph{corvée}, \emph{Treccani - Enciclopedia on
line}, in \url{https://www.treccani.it/enciclopedia/corvee} (Consultato:
22 gennaio 2023).

\leavevmode\vadjust pre{\hypertarget{ref-viepubliquePourquoiExistetilJustice}{}}%
\textsc{Vie Publique}, \emph{Pourquoi existe-t-il une justice
administrative?}, \emph{vie-publique.fr}, in
\url{https://tinyurl.com/mw2pxm4z} (Consultato: 22 gennaio 2023).

\end{CSLReferences}



\end{document}
